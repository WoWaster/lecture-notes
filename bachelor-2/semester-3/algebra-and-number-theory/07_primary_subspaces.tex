% !TeX root = ./main.tex
\documentclass[main]{subfiles}
\begin{document}


\chapter{Примарные подпространства}

\begin{definition} [Примарное подпространство]
    Допустим, $p$ -- неприводимый многочлен, $\mca \in \End V, \ W \subset V$ называется $p$-примарным, если $W$ -- $\mca$-инвариантно
    и $\forall v \in W \ \mu_{\mca, v} = p^m, \ m \geq 0$, $p$ -- неприводимый.

    $W_p = \{w \in V | \mu_{\mca, w} = p^m, \ m \geq 0\}$
\end{definition}

\begin{proposition}
    $W_p$ -- максимальное по включению $p$-примарное подпространство.
\end{proposition}

\begin{proof}
    Нужно проверить:
    \begin{enumerate}
        \item $W_p < V$: очевидно, $\lambda \neq 0 \ \mu_{\mca, \lambda w} = \mu_{\mca, w}$.

              $W_1, \ W_2 \in W_p$,

              $\mu_{\mca, w_1} = p^{m_1}, \ \mu_{\mca, w_2} = p^{m_2}, \ m = max(m_1, \ m_2)$.

              $p^m(\mca)(W_1+W_2) = p^{m_1}(\mca)(W_1)+ p^{m_2}(\mca)(W_2) = 0 \Rightarrow
                  \mu_{\mca, W_1 +W_2} | p^m \Rightarrow \mu_{\mca, W_1 +W_2} = p^l$, где $l \leq m \Rightarrow W_1+W_2 \subset W_p$.
        \item $W_p$ -- $\mca$-инвариантно: пусть $w \in W_p \Rightarrow \mu_{\mca, w} = p^m$

              $p^m(\mca)(w) = 0$

              $p^m(\mca)(\mca w) - \mca\underbrace{(p^m(\mca)(w))}_{0} = 0 \Rightarrow \mu_{\mca, \mca_w} | p^m \Rightarrow \mu_{\mca, \mca_w} = p^l, \ l \leq m \Rightarrow \mca w \in W_p$.
    \end{enumerate}
\end{proof}

\begin{proposition}
    Допустим $f(\mca) = 0, \ f = gh, \ (g, h) = 1$, тогда $V = W_1 \bigoplus W_2$, где
    $W_1, W_2$ -- $\mca$-инвариантны  $g(\mca|_{W_1}) = 0, \  h(\mca |_{W_2}) = 0$.
\end{proposition}

\begin{proof}
    Предположим, $W_1 = \Ker g(\mca), \ W_2 = \Ker h(\mca),
        W_1, W_2$ -- $\mca$-инвариантные пространства.
    $(g, h) = 1 \Rightarrow \exists \ a, \ b \in K[X]$ такие, что $ag+bh = 1$

    \begin{enumerate}
        \item Проверим, что $W_1 + W_2 = V$.

              \[g(\mca)a(\mca) + h(\mca)b(\mca) = \epsilon_V\]

              $v \in V, \ v = \underbrace{g(\mca)a(\mca)(v)}_{\in W_2} + \underbrace{h(\mca)b(\mca)(v)}_{\in W_1}$.
              \begin{gather*}
                  h(\mca)(g(\mca)a(\mca)v(\mca)) = \underbrace{(hg)}_{f}(\mca)a(\mca)(v) = 0 \\
                  \Rightarrow g(\mca)a(\mca)(v) \in W_2
              \end{gather*}
              Аналогично, $h(\mca)b(\mca)(v) \in W_1$.

              Таким образом, $\forall v \in W_1 + W_2$.
        \item Проверим, что $W_1 \bigcap W_2 = 0$.

              Пусть $v \in W_1 \bigcap W_2$.
              \begin{gather*}
                  a(\mca)g(\mca) + b(\mca)h(\mca) = \epsilon_V \\
                  a(\mca)\underbrace{g(\mca)(v)}_{0} + b(\mca)\underbrace{h(\mca)(v)}_{0} = v
              \end{gather*}
              Таким образом, $v = 0$.
    \end{enumerate}
\end{proof}


\end{document}