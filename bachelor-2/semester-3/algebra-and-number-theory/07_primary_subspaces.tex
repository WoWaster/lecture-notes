% !TeX root = ./main.tex
\documentclass[main]{subfiles}
\begin{document}


\chapter{Примарные подпространства}

\begin{definition} [Примарное подпространство]
    Допустим, $p$ -- неприводимый многочлен, $\mca \in \End V, \ W \subset V$ называется $p$-примарным, если $W$ -- $\mca$-инвариантно
    и $\forall v \in W \ \mu_{\mca, v} = p^m, \ m \geq 0$, $p$ -- неприводимый.

    $W_p = \{w \in V | \mu_{\mca, w} = p^m, \ m \geq 0\}$
\end{definition}

\begin{proposition}
    $W_p$ -- максимальное по включению $p$-примарное подпространство.
\end{proposition}

\begin{proof}
    Нужно проверить:
    \begin{enumerate}
        \item $W_p < V$: очевидно, $\lambda \neq 0 \ \mu_{\mca, \lambda w} = \mu_{\mca, w}$.

              $W_1, \ W_2 \in W_p$,

              $\mu_{\mca, w_1} = p^{m_1}, \ \mu_{\mca, w_2} = p^{m_2}, \ m = max(m_1, \ m_2)$.

              $p^m(\mca)(W_1+W_2) = p^{m_1}(\mca)(W_1)+ p^{m_2}(\mca)(W_2) = 0 \Rightarrow
                  \mu_{\mca, W_1 +W_2} | p^m \Rightarrow \mu_{\mca, W_1 +W_2} = p^l$, где $l \leq m \Rightarrow W_1+W_2 \subset W_p$.
        \item $W_p$ -- $\mca$-инвариантно: пусть $w \in W_p \Rightarrow \mu_{\mca, w} = p^m$

              $p^m(\mca)(w) = 0$

              $p^m(\mca)(\mca w) - \mca\underbrace{(p^m(\mca)(w))}_{0} = 0 \Rightarrow \mu_{\mca, \mca_w} | p^m \Rightarrow \mu_{\mca, \mca_w} = p^l, \ l \leq m \Rightarrow \mca w \in W_p$.
    \end{enumerate}
\end{proof}

\begin{proposition}
    Допустим $f(\mca) = 0, \ f = gh, \ (g, h) = 1$, тогда $V = W_1 \oplus W_2$, где
    $W_1, W_2$ -- $\mca$-инвариантны  $g(\mca|_{W_1}) = 0, \  h(\mca |_{W_2}) = 0$.
\end{proposition}

\begin{proof}
    Предположим, $W_1 = \Ker g(\mca), \ W_2 = \Ker h(\mca),
        W_1, W_2$ -- $\mca$-инвариантные пространства.
    $(g, h) = 1 \Rightarrow \exists \ a, \ b \in K[X]$ такие, что $ag+bh = 1$

    \begin{enumerate}
        \item Проверим, что $W_1 + W_2 = V$.

              \[g(\mca)a(\mca) + h(\mca)b(\mca) = \epsilon_V\]

              $v \in V, \ v = \underbrace{g(\mca)a(\mca)(v)}_{\in W_2} + \underbrace{h(\mca)b(\mca)(v)}_{\in W_1}$.
              \begin{gather*}
                  h(\mca)(g(\mca)a(\mca)v(\mca)) = \underbrace{(hg)}_{f}(\mca)a(\mca)(v) = 0 \\
                  \Rightarrow g(\mca)a(\mca)(v) \in W_2
              \end{gather*}
              Аналогично, $h(\mca)b(\mca)(v) \in W_1$.

              Таким образом, $\forall v \in W_1 + W_2$.
        \item Проверим, что $W_1 \bigcap W_2 = 0$.

              Пусть $v \in W_1 \bigcap W_2$.
              \begin{gather*}
                  a(\mca)g(\mca) + b(\mca)h(\mca) = \epsilon_V \\
                  a(\mca)\underbrace{g(\mca)(v)}_{0} + b(\mca)\underbrace{h(\mca)(v)}_{0} = v
              \end{gather*}
              Таким образом, $v = 0$.
    \end{enumerate}
\end{proof}

\begin{proposition}
    Пусть $f(\mca) = 0$, $f = \epsilon {p_1}^{r_1} \ldots {p_s}^{r_s}$, $\epsilon \in K^*$,
    $p_i$ поарно неассоциативные неприводимые многочлены. Тогда $V = W_{p_1} \oplus \ldots \oplus W_{p_s}$, т.е.
    \begin{enumerate}
        \item $V = W_{p_1} \ldots + W_{p_s}$
        \item $ 0 = w_1 + \ldots + w_s, \ w_j \in W_{p_j}$
    \end{enumerate}
\end{proposition}

\begin{proof}
    Индукция по $s$.

    $s = 1$: ${p_1}^{r_1}(\mca) = 0 \Rightarrow \mu_{\mca, v} | {p_1}^{r_1} \Rightarrow W_{p_1} = V$.

    Переход: $f = gh, g = \epsilon {p_1}^{r_1} \ldots {p_{s-1}}^{r_{s-1}}$, $h = {p_s}^{r_s}$, $(g, h) = 1 \Rightarrow
        V = V' \oplus V'', \ g(\mca |_{V'}) = 0, h(\mca |_{V''}) = 0$.

    По индукционному предположению, $V' =  \tilde{W}_{p_1} \oplus \ldots \oplus \tilde{W}_{p_{s-1}}$.
    $\tilde{W}_{p_j}$ - максимальной $p_j$-примарное подпространство для $\mca |_{V'}$.
    $\forall v \in \tilde{W}_{p_j}$: $\mu_{\mca |_{V'}, v} = \mu_{\mca, v} \Rightarrow \tilde{W}_{p_j} \subset W_{p_1}$.

    $p^{r_s}(\mca |_{V''}) = 0 \Rightarrow V'' -- p_s$-примарное $ \Rightarrow
        V'' \subset W_{p_s}$

    $V = \tilde{W}_{p_1} \oplus \ldots \oplus \tilde{W}_{p_{s-1}} \oplus V'' \subset W_{p_1} + \oplus + \tilde{W}_{p_{s-1}} \subset W_{p_1} + \ldots + W_{p_s}$.

    Таким образом, $V = W_{p_1} + \ldots + W_{p_s}$. Предположим, $w_1 + \ldots + w_s = 0$, $w_j \in W_{p_j}, \ j = 1, \ldots, s$.
    Проверим: $w_j = 0$ (аналогично $w_j = 0 \forall j)$. $w_s = - w_1 - w_2 - \ldots - w_s$. $w_j \in W_{p_i} \Rightarrow \mu_{\mca, w_j} = {p_j}^{l_j}, \ j = 1, \ldots, s$.
    ${p_1}^{l_1} \ldots {p_{s-1}}^{l_{s-1}} (\mca) (w_j) = 0, \ j = 1, \ldots, s-1 \Rightarrow
        {p_1}^{l_1} \ldots {p_{s-1}}^{l_{s-1}} (\mca) \underbrace{w_1 + \ldots + w_{s-1}}_{-w_s} = 0  \Rightarrow
        {p_s}^{l_s} | {p_1}^{l_1} \ldots {p_{s-1}}^{l_{s-1}} \Rightarrow ls = 0 \Rightarrow w_s = 0$.....
\end{proof}

\begin{corollary}
    Пусть $\mca \in \End V$. Тогда $V = \bigoplus W$.
\end{corollary}

\begin{remark}
    $p| \chi_{\mca} \Rightarrow W_p \neq 0$.
\end{remark}

\begin{remark}
    Пусть $V$ -- $p$-примарное, тогда $V = \bigoplus C_{v_i}, \ \mu_{\mca_i, v_i} = p^{m_i}$.
\end{remark}

В частном подходящем базисе:

\[ [\mca]_{E'} = \begin{pmatrix}

    \end{pmatrix}\]


\end{document}