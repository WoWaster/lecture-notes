% !TeX root = ./main.tex
\documentclass[main]{subfiles}
\begin{document}


\chapter{Характеристический многочлен оператора}

$\mca \in \End V$, $[\mca]_E = A$.
Задача: найти собственное значение $\mca$.
$\lambda$ - собственое значение $\mca \Leftrightarrow
    \Ker(\mca - \lambda\epsilon) \neq 0 \Leftrightarrow
    [\mca - \lambda\epsilon]_E \in \GL_n(K) \Leftrightarrow |\mca - \lambda\epsilon| = 0$. Задача сводится к нахождению таких
$\lambda$, при которых определитель матрицы равен нулю.
$[\mca - \lambda\epsilon]_E = \A - \lambda \E_n \Leftrightarrow
    \A = \begin{pmatrix}
        a_{11} & \ldots & a_{1n} \\
        \vdots & \ddots & \vdots \\
        a_{n1} & \ldots & a_{nn} \\
    \end{pmatrix}$

$|\A - \lambda E_n| = \begin{pmatrix}
        a_{11} - \lambda & a_{12}           & \ldots & a_{1n} \\
        a_{21}           & a_{22} - \lambda & \ldots & a_{2n} \\
        \ddots           & \ddots           & \vdots & a_{1n} \\
        a_{n1} - \lambda & a_{n2}           & \ldots & a_{nn} \\
    \end{pmatrix}$

$\begin{pmatrix}
        a_{11} - \lambda & a_{12}           \\
        a_{21}           & a_{22} - \lambda \\
    \end{pmatrix} = (a_{11} - \lambda)(a_{22} - \lambda) - a_{12} a_{21} =
    \lambda ^2 - \lambda(a_{11}+a_{22}) + a_{11}a_{22} - a_{12} a_{21}$. Определитель обращается в ноль, когда $\lambda$
является корнем этого уравнения.

\begin{definition}
    Пусть $\A \in M_n(K)$. Есть характеристический многочлен называется
    $\chi_\A = \underbrace{|A - X\cdot E_n|}_{\in M_n(K[x])\subset M_n(K(x))} \in K[x]$

    $\begin{pmatrix}
            \ddots &        \\
                   & \ddots \\
        \end{pmatrix} = (a_{11} - x)(a_{22} - x)\ldots(a_{nn} - x) + G =
        (-1)^{n-1}x^n+(-1)^{n-1}\underbrace{(a_{11}+\ldots+a_{nn})}_{\Tr\A}x^{n-1}+\ldots+|\A|$,
    где $\A= (a_{ij}),\ \deg G \leq n-2,\ \Tr\A$ - след матрицы.
\end{definition}

\begin{definition}
    Пусть $\mca \in \End V$. Его характеристический многочлен $\chi_{\mca}$ называют
    $\chi_{[\mca]_E}$, где $E$  — любой базис $\V$.
\end{definition}

Проверка корректности: пусть $\A = [\mca]_E, \A_1 = [\mca]_{E_1}$,
$C = M_{E \rightarrow E_1}$. Нужно: $\chi_{\mca} = \chi_{\mca_1}$.

$\A_1 = C^{-1}\A C$

$\chi_{\mca_1} = |\A_1 - X E_n| =
    |C^{-1}\A C - XC^{-1}C| = |C^{-1}\A C - C^{-1}XE_nC| =
    |C^{-1}(\A - XE_n) C| = \underbrace{|C^{-1}}_{|C|^{-1}}|\A - XE_n||C| =
    |\A - XE_n| = \chi_{\mca}$

У эквивалентных матриц след одинаков.

Таким образом, $\lambda$ - собственное значение $\mca \Leftrightarrow \lambda$
-- корень $\mca$.

\begin{definition}
    Кратная копия $\lambda$ у многочлена $\chi_{\mca}$ называется собственной алгебраической кратностью собственного значения $\lambda$.
\end{definition}

\begin{proposition}
    Пусть $\mca \in \End V$.
    \begin{enumerate}
        \item Пусть $\mca$ - инариантное подпространство $V$; $\mca_1 = \mca|_w \in W$. Тогда $\chi_{\mca_1} | \chi_{\mca}$.
        \item Пусть $V = W_1 \bigoplus W_2; W_1, W_2$ -- $\mca$-инвариант.
    \end{enumerate}

\end{proposition}

\begin{definition}
    \begin{enumerate}
        \item 1
        \item Аналогично: в подходящем $E$, $[\mca]'_E = \left(\begin{array}{c|c}
                          A_1 & 0   \\ \hline
                          0   & A_2 \\
                      \end{array}\right); A_1 = [\mca_1]_{E_1}, A_2 = [\mca_2]_{E_2} \Rightarrow
                  \chi_{\mca} = \chi_{A_1}\chi_{A_2}= \chi_{\mca_1}\chi_{\mca_2}$.
    \end{enumerate}
\end{definition}

\begin{corollary}
    Пусть $\lambda$ - собственное значение $\mca$. Тогда $g_\lambda \leq a_lambda$.
    Применим предложение к $W=V_\lambda$. Очевидно, $W$ -- $\mca$-инвариант $\Rightarrow
        \chi_{\mca|_{V_\lambda} | \chi_{\mca}}$.


\end{corollary}

\begin{theorem}

\end{theorem}

\begin{proof}
    $1 \Rightarrow 2$: Существует базис $E$, такой что:
    $[\mca]_E = diag(\underbrace{\lambda_1, \ldots, \lambda_1}_{g_{\lambda_1}}, \ldots, \underbrace{\lambda_k, \ldots, \lambda_k}_{g_{\lambda_k}})$.
    $c_{\lambda_1} = g_{\lambda_1}; \lambda_1, \ldots, \lambda_k$ - различные значения.
    $\chi_{\mca} = \chi_A = (\lambda_1 - X)^{g_{\lambda_1}(\lambda_2 - X)^{g_{\lambda_2}} \ldots (\lambda_k - X)^{g_{\lambda_k}}$.
\end{proof}

\begin{example}
    \begin{enumerate}
        \item
    \end{enumerate}
\end{example}

\end{document}