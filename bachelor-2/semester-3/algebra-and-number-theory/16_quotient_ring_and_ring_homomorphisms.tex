% !TeX root = ./main.tex
\documentclass[main]{subfiles}

\begin{document}

\part{ Элементы теории полей}

\chapter{Факторкольца и гомоморфизмы колец}

$R$ -- коммутативное (ассоциативность подразумевается) кольцо с 1. 

$I \subset R$ называется идеалом, если $I$ -- подгруппа $R$ относительно сложения и $RI \subset R$.

$(a) = \{ax \ |\ x \in R\}$

$R = (1), \ 0 = (0)$.

\begin{proposition}
    $R$ -- поле $\Leftrightarrow$ в $R$ ровно 2 идеала.
\end{proposition}

\begin{proof}
    $\Rightarrow$: $\underbrace{(1)}_{1 \in} \neq \underbrace{(0)}_{1 \in}$, т.к. $1 \neq 0$.
    Пусть $I$ -- идеал в $R$, $I \neq 0 \Rightarrow \exists c \in I \backslash \left\langle 0\right\rangle
    \Rightarrow 1 = c c^{-1} \Rightarrow 1 \in I \Rightarrow I = R \ (r = r \cdot 1) $.

    $\Leftarrow$: если $0 = 1$, $r = r \cdot 1 = r \cdot 0 = 0 \Leftarrow$ нет двух идеалов.
    
    Проверим обратимость: $c \in R \backslash \left\langle 0\right\rangle, \ (c) \neq 0 \Rightarrow
    (c) = R \Rightarrow 1 \in (c) \Rightarrow c \in R^*$. 
\end{proof}

Пусть $R$ -- коммутативное кольцо, $I$ -- идеал в $R$.

Рассмотрим факторгруппу $R/I$ и введем на ней умножение: \\ $\underset{(a+I, b+I) \mapsto ab+I}{(R/I)\times(R/I) \rightarrow R/I}$.

Проверка корректности:

\begin{gather*}
    \begin{gathered}
        a + I = a' + I \\
        b + I = b' + I 
    \end{gathered} \Rightarrow ab +I = a'b' +I \\
    \begin{gathered}
        a'= a + s, \ s\in I \\
        b'= b + s, \ t\in I \\
        a'b' = ab + \underbrace{at}_{\in I} + \underbrace{bs}_{\in I} + \underbrace{st}_{\in I}  \Rightarrow a'b' - ab \in I
    \end{gathered} \Rightarrow a'b' + I = ab +I
\end{gather*}

\begin{proposition}
    $(R/I, +, \cdot)$ -- коммутативное кольцо с 1.
\end{proposition}

\begin{proof}
    Непосредственная проверка.
\end{proof}

\begin{theorem}
    Пусть $R$ -- ОГИ, $a \in R$. Тогда $R/(a)$ -- поле 
    $\Leftrightarrow a$ неприводимый.
\end{theorem}

\begin{remark}
    $R/(a)$ поле $\Leftrightarrow \forall b \in R, \ a \not\in R^*$: $b \not\in (a) \Rightarrow
    \exists c \in R: \ \overline{b}\overline{c} = \overline{1} \ (\overline{b} = b + (a)) \Leftrightarrow
    \forall b \in R: \ a \not\mid b \Rightarrow \exists c \in R: \ bc = 1 + ax, \ x \in R$. 
\end{remark}

\begin{proof}
    Антипрямо: пусть $a$ неприводим. Тогда 
    $a \not\mid b \Rightarrow (a, b) = 1 \Rightarrow ax + by = 1$ для некоторых $x, \ y \in R \Rightarrow by = 1 - ax$.
    Таким образом, $R/(a)$ -- поле. 

    Антиобратно,  $R/(a)$ -- поле $\Rightarrow R/(a) \neq 0$, т.е. $a \not\in R^*$ и $1 = bc - ax$, т.е. 
    $a \not\mid b \Rightarrow (a, b) = 1 \Rightarrow a$ -- неприводим.
\end{proof}

\begin{example}
    \begin{enumerate}
        \item $\R[x]/(x^2+1) = \C$
        \item $\F_2 = \Z/(2)$, $\F_2[x]/(x^2+x+1) = \{0, 1, \overline{x}, 1 + \overline{x}\}  = \F_4$
    \end{enumerate}
\end{example}

\begin{definition} [Гомоморфизм]
    Пусть $R, \ S$ -- кольца с 1. $\phi: R \rightarrow S$ называется гоморфизмом, если: 
    \begin{enumerate}
        \item $\phi(a+b) = \phi(a) + \phi(b), \ \forall a, \ b \in R$
        \item $\phi(ab) = \phi(a)\phi(b), \ \forall a, \ b \in R$
        \item $\phi(1_R) = 1_S$
    \end{enumerate}
\end{definition}

\begin{example}
    $\underset{a \mapsto a+I}{R \rightarrow R/I}$
\end{example}

\begin{theorem}[О гомоморфизме для колец]
    Пусть $\phi: \ R \rightarrow S$ -- гомоморфизм колец. Тогда:
    \begin{enumerate}
        \item $\Ker \phi$ -- идеал в $R$.
        \item $\Im \phi$ -- подкольцо в $S$.
        \item Существует изоморфизм: $\overline{\phi} = \underset{a+\Ker \phi \mapsto \phi(a)}{R/\Ker \phi \rightarrow \Im \phi}$.
    \end{enumerate}
\end{theorem}

\begin{proof}
    \begin{enumerate}
        \item $\Ker \phi$ -- подгруппа $R$. Пусть $a \in \Ker \phi$, $r \in R \Rightarrow
        \phi(ra) = \phi(r)\underbrace{\phi(a)}_{0} = 0 \Rightarrow ra \in \Ker \phi$.
        \item $a', \ b' \in \Im \phi \Rightarrow a' = \phi(a), \ b' = \phi(b) \\
        a' - b' = \phi (a) - \phi (b) = \phi (a - b) \in \Im \phi \\
        a'b' = \phi (a) \phi (b) = \phi (ab) \\
        1_S = \phi(1_R) \in \Im \phi$
        \item Из ТГ знаем:
        \begin{enumerate}
            \item $\overline{\phi}$ определено корректно.
            \item $\overline{\phi}$ -- изоморфизм групп.
        \end{enumerate}
        $\overline{\phi}((a + \Ker \phi)(b + \Ker \phi)) = 
        \overline{\phi}(ab + \Ker \phi) = \phi(ab) = \phi(a)\phi(b) = \overline{\phi}(a + \Ker \phi) \overline{\phi}(b + \Ker \phi)$ \\
        $\overline{\phi}(1_{R/\Ker \phi}) = \phi(1_R) = 1_S = 1_{\Im \phi}$

    \end{enumerate}
\end{proof}














\end{document}