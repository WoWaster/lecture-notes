% !TeX root = ./main.tex
\documentclass[main]{subfiles}
\begin{document}

\chapter{Жорданова нормальная форма}

Пусть $\mca \in \End V$, $\chi_{\mca}$ раскладывается на линейные множители.

$p| \chi_{\mca}$ неприводимый приведенный $\Rightarrow p = X - \lambda$, $\lambda$ - собственное значение $\mca$.

$W_{X-\lambda} = R_{\lambda} = \{v \ | \ \exists j:(X-\lambda)^j (\mca) (v) = 0\} =
  \{v \ | \ (\mca - \lambda \mse )^j (v) = 0\}$ -- корневые векторы, принадлежащее собственному значению $\lambda$.

\begin{definition} [Корневой вектор и корневое подпространство]
  Корневым вектором, принадлежащим собственному значению $\lambda$ называется $v \in V$ такое, что $(\mca - \lambda \mse)^j (v) = 0$ для некоторого $j$.

  $R_{\lambda}$ -- корневое подпространство, принадлежащее собственному значению $\lambda$.
\end{definition}

\begin{propertylist}
  \begin{enumerate}
    \item $V_{\lambda} \subset R_{\lambda}$, т.к. $V_{\lambda} = \Ker(\mca - \lambda \mse)$
    \item $R_{\lambda} \neq 0 \Leftrightarrow \lambda$ -- собственное значение $\mca$
          \begin{proof}
            $\Rightarrow: \ R_{\lambda} = W_{X-\lambda}$ (по опр.). $R_{\lambda} \neq 0 \Rightarrow
              (X - \lambda) \mid \chi_{\mca} \Rightarrow \lambda$ -- собственное значение.

            $\Leftarrow: \lambda$ -- собственное значение $\Rightarrow V_{\lambda} \neq 0 \Rightarrow R_{\lambda} \neq 0$
          \end{proof}
    \item $V = \bigoplus_{\lambda} R_{\lambda}$ (интерпретация основного определения) $V = \bigoplus_{p | \chi_{\mca}} W_p$
  \end{enumerate}
\end{propertylist}

\begin{definition} [Высота корневого вектора]
  Высота корневого вектора $v$ -- это минимальное $h$ такое, что $(\mca - \lambda \mse)^h (v) = 0$.
\end{definition}

Корневой вектор высоты 0 -- это 0, 1 -- собсвтенные векторы, 2 -- такие $v$, что $(\mca - \lambda \mse)(v)$ - собственный вектор,
$h$ -- $\Ker(\mca - \lambda \mse)^h \setminus \Ker(\mca - \lambda \mse)^{h-1}$.

Очевидно, $R_{\lambda} = \bigcup_{h \in \N} \underbrace{Ker(\mca - \lambda \mse)^h}_{= R_{\lambda, h}}$.

$\underbrace{R_{\lambda, 0}}_{=0} \subset \underbrace{R_{\lambda, 1}}_{= V_{\lambda}} \subset R_{\lambda, 2} \subset \ldots \subsetneq  R_{\lambda, N_{\lambda}} = R_{\lambda, N_{\lambda} + 1} \Rightarrow
  R_{\lambda} = R_{\lambda, N_{\lambda}}$.

\begin{proposition}
  Пусть $N_{\lambda}$ -- минимальное натуральное число такое, что $R_{\lambda} = R_{\lambda, N_{\lambda}}, \ \lambda_1, \ldots, \lambda_s$ -- все собственные значения $\mca$.
  Тогда ${\mu}_{\mca} = \underbrace{\prod_{i = 1}^s(x-\lambda)^{N_{\lambda_i}}}_{=f}$.
\end{proposition}

\begin{proof}
  $R_{\lambda_i} = R_{\lambda_i, N_{\lambda_i}} \Rightarrow (x - \lambda_i)^{N_{\lambda_i}}(\mca|_{R_{\lambda_i}}) = 0 \Rightarrow
    f(\mca|_{R_{\lambda_i}}) = 0, \ i = 1, \ldots, s \Rightarrow f(\mca) = 0 \Rightarrow \mu_{\mca} | f$.

  Докажем: $(x - \lambda)^{\lambda_i} \mid \mu_{\mca}, \ i = 1, \ldots, s$.

  $\exists v \in R_{\lambda_i, N_{\lambda_i}}\setminus R_{\lambda_i, N_{\lambda_i} - 1}$

  $\mu_{\mca, v} = (x - \lambda_i)^{N_{\lambda_i}}, \ \mu_{\mca, v} \mid \mu_{\mca} \Rightarrow f \mid \mu_{\mca}$.
\end{proof}

\begin{corollary}
  Множества корней $\chi_{\mca}$ и $\mu_{\mca}$ совпадают.
\end{corollary}

\begin{proof}
  $\mu_{\mca}$ делит $\chi{\mca}$, и у $\chi{\mca}$ нет других корней, так как $N_{\lambda_i} \geq  1$.
\end{proof}

\begin{definition} [Жорданова клетка порядка $n$]
  Пусть  $\lambda \in K$, $n \in \N$. Жордановой клеткой порядка $n$ с собственным значением $\lambda$ называется
  \begin{gather*} \mathcal{J}_n(\lambda) =
    \begin{pmatrix}
      \lambda & 0       & \ldots & 0       & 0       \\
      1       & \lambda & \ldots & 0       & 0       \\
      0       & 1       & \ldots & 0       & 0       \\
      \vdots  & \vdots  & \ddots & \vdots  & \vdots  \\
      0       & 0       & \ldots & \lambda & 0       \\
      0       & 0       & \ldots & 1       & \lambda \\
    \end{pmatrix}
  \end{gather*}

  Жорданова матрица -- блочно-диагональная матрица, блоки которой -- некоторые жордановы клетки.

  $\chi_{\mathcal{J}_n(\lambda)} = (\lambda - x)^n = \pm (x - \lambda)^n, \ (a_{\lambda} = n,\ g_{\lambda} = 1)$ для $\mathcal{J}_n(\lambda)$.

  Базис $E$ пространство $V$ называется Жордановым базисом (для $\mca$), если $[\mca]_E$ жорданова.
\end{definition}

\begin{definition} [Нильпотентный оператор]
  $\mca$ называется нильпотентным, если $\exists \ N > 0: \ \mca^N = 0$ (т.е. $\mu_{\mca} = x^{...}$).
\end{definition}

$V = \bigoplus R_{\lambda}$

$(x - \lambda)^{N_{\lambda}}(\mca |_{R_{\lambda}}) = 0 \Rightarrow (\mca |_{R_{\lambda}} - \mse |_{R_{\lambda}})^{N_{\lambda}} = 0$.
Таким образом, $\mca |_{R_{\lambda}} - \mse_{R_{\lambda}}$ -- нильпотентный. Минимальное $N$ называется индексом нильпотентности $\mca$.

\begin{remark}
  $E$ -- жорданов базис для $\mca \Rightarrow E$ -- жорданов базис для $\mca + \alpha \mse$.
  $[\mca + \alpha \mse]_E = [\mca]_E + [\alpha \mse]_E = [A]_E + diag(\alpha, \ldots, \alpha)$
  $\mathcal{J}_n(\lambda) + \alpha E_n = \mathcal{J}_n(\lambda + \alpha)$
\end{remark}

Построение жорданового базиса для нильпотентного оператора.

$\mcb$ - нильпотент, $N$ - индекс нильпотентности.

$\mu_{\mcb} = x^N, \ V = R_0 = R_{0, N} \supsetneq R_{0, N-1} \supset \ldots \supset R_{0, 1} \supset R_{0, 0} = \{0\}$

Пусть $e_{N, 1}, e_{N,2}, \ldots, e_{N, q_N}$ -- базис $R_{0, N}$ относительно $R_{0, N-1}$.

\begin{lemma}
  Пусть $v_1, \ldots, v_s \in R_{0, m}$ ЛНЗ относительно $R_{0, m-1}$.
  Тогда: $\mcb v_1, \ldots, \mcb v_s \in R_{0, m-1}$. $\mcb^m v_i = 0 = \mcb^{m-1}(\mcb v_i) = 0 \Rightarrow \mcb v_i \in R_{0, m-1}$.
\end{lemma}

\begin{proof}
  Пусть $\alpha_1 \mcb v_1 + \ldots + \alpha_s \mcb v_s \in R_{0, m-2}. \ \mcb^{m-2}(\alpha_1 \mcb v_1 + \ldots \alpha_s \mcb v_s) =
    \mcb^{m-1}(\alpha_1 \mcb v_1 + \ldots \alpha_s \mcb v_s) = 0 \Rightarrow \alpha_1 \mcb v_1 + \ldots \alpha_s \mcb v_s \in \Ker B^{m-1} = R_{0, m-1}
    \Rightarrow \alpha_1 = \ldots = \alpha_s = 0$.
\end{proof}




\end{document}