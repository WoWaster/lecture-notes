% !TeX root = ./main.tex
\documentclass[main]{subfiles}
\begin{document}
\chapter{Норма линейного отображения}
\begin{definition}[Линейный оператор]
    $A: {\vR}^m \rightarrow {\vR}^n$ -- линейный оператор, если
    $ \forall x_1, x_2 \in {\vR}^m, \forall p, q \in {\vR}^n \Rightarrow$
    \[A(px_1 qx_2) = pA(x_1) + qA(x_2)\]
\end{definition}
\begin{gather*}
    A \Leftrightarrow \tilde{A}_{m \times n} \\
    X =
        \begin{bmatrix}
            x_1 \\
            \vdots \\
            x_m  \\
        \end{bmatrix}
    A(x) = \tilde{A}_{m \times n} X \\
    \tilde{A} = 
    \begin{bmatrix}
        a_{11} & \ldots & a_{1m} \\
        a_{21} & \ldots & a_{2m} \\
        \vdots & \ddots & \vdots \\
        a_{n1} & \ldots & a_{nm} \\
    \end{bmatrix}
\end{gather*}

\begin{definition}[Норма линейного отображения]
    \[|A|| \stackrel{def}{=} \underset{x \in {\vR}^m, ||X|_{{\vR}^m \leq 1}}{\sup}
     {||AX||_{{\vR}^n}} \tag{1} \]
\end{definition}


\[AX = \begin{bmatrix}
    a_{11}x_1 + \ldots a_{1n}x_m \\
    a_{21}x_1 + \ldots a_{2n}x_m \\
    \vdots \\
    a_{n1}x_1 + \ldots a_{nm}x_m
\end{bmatrix}\]



\begin{multline*}
    ||AX||^2_{{\vR}^n} = \sum_{k=1}^n(a_{k1}x_1 + \ldots a_{km}x_n)^2 \leq 
    \sum_{k=1}^n (a^2_{k1} + \ldots + a^2_{km})\underbrace{(x_1^2 + \ldots x_m^2)}_{=||X||^2_{{\vR}^n}} \leq \\
    \sum_{k=1}^n (a^2_{k1} + \ldots + a^2_{km}) =
\end{multline*}
     \[=\sum_{k=1}^n \sum_{l=1}^m a_{kl}^2 =||A||_2  \tag{2}\]


\[(2) \implies ||A|| \leq ||A||_2 \geq 0\]

\subsection*{Свойства нормы линейного отображения}
 \begin{theorem}$||A|| \geq 0, ||A|| = 0 \Leftrightarrow A = \mathbb{0}$ \end{theorem}
    \begin{proof}
        Пусть $||A|| = 0$ 
        $A = \begin{bmatrix}
            a_{11} & \ldots & a_{1m} \\
            \vdots & \ddots & \vdots \\
            a_{1n} & \ldots & a_{nm} \\
        \end{bmatrix}$
        
        $1 \leq i \leq n, 1 \leq j \leq m$ не зависят друг от друга.
        \newline
        $e_i = (0, \ldots, \underbrace{1}_i, \ldots 0)$
        $f_j = \begin{bmatrix}
            0 \\
            \vdots \\
            \underbrace{1}_j \\
            \vdots \\
            0 \\ \end{bmatrix}$
        \newline
        $||A|| = 0 \rightarrow Af_j = \mathbb{0}_{{\vR}^n} $.
        $Af_j =
        \begin{bmatrix}
            a_{1j} \\
            a_{2j}\\
            \vdots \\
            a_{nj}
        \end{bmatrix} $
        Теперь рассмотрим $e_i \underbrace{(Af_j)}_{\mathbb{0}_{{\vR}^n}} =$
        $ (0 \ldots 1 \ldots) \begin{bmatrix}
            0 \\
            \vdots \\
            1 \\
            \vdots \\
            0 \\ \end{bmatrix}$ = $a_{ij} = 0$
    \end{proof}
    $A, c \in {\vR}, (cA(x)) \stackrel{def}{=} c(A(x))$ 
    \begin{theorem}$||cA|| = |c| \cdot ||A||$ \end{theorem}
    \begin{proof}
        \begin{multline*}
        ||cA|| = \underset{X \in {\vR}^m, ||X||_{{\vR}^n}\leq 1}{\sup}||cA(X)||_{{\vR}^n} = 
        \underset{X \in {\vR}^m, ||X||_{{\vR}^n}\leq 1}{\sup} ||c(AX)||_{\mathbb{R^n}} = \\
        =\underset{X \in {\vR}^m, ||X||_{{\vR}^n}\leq 1}{\sup} |c| \cdot ||AX||_{{\vR}^n}
        \end{multline*} 
    \end{proof}
    $A, B : {\vR}^m \rightarrow {\vR}^n$
        \begin{theorem}
           $ ||A+B|| \leq ||A|| + ||B|| $
        \end{theorem}
        \begin{proof}
            \[||A + B|| = \underset{x \in {\vR}^m, ||X||_{{\vR}^m} \leq 1}{\sup} ||(A+B)X||_{{\vR}^n}
            = \underset{\ldots}{\sup} ||AX + BX||_{{\vR}^n} \leq \]
            \[\leq \underbrace{\sup(\underbrace{||AX||}_{\leq ||A||_2} + \underbrace{||BX||}_{\leq ||B||_2})}_{M}
            \leq \underset{\ldots}{\sup} ||AX||_{{\vR}^n} + \underset{\ldots}{\sup} ||BX||_{{\vR}^n}\]
           
           \[ \exists x_1 \in {\vR}^m, ||x_1||_{{\vR}^m} \leq 1 \text{и}
            ||Ax_1|| + ||Bx_1|| > M - \varepsilon
             \tag{3} \]
           
            $(3) \implies M - \varepsilon < ||A|| + ||B| \rightarrow 
            \underset{x \in {\vR}^m, ||X||_{{\vR}^m}\leq1}{\sup}
            (||AX||_{{\vR}^n} + ||BX||_{{\vR}^n}) \leq ||A|| + ||B||$
        \end{proof}
         \begin{theorem}
           $ ||AX||_{{\vR}^n} \leq ||A|| \cdot ||X||_{{\vR}^m}$
        \end{theorem}
        \begin{proof}
            \begin{gather*}
            X \neq \mathbb{0}^m \Leftrightarrow ||X||_{{\vR}^m} \stackrel{def}{=} t > 0 \\
            x_0 = \frac{1}{t}x \\
            ||x_0||_{{\vR}^m} =||\frac{1}{t}x||_{{\vR}^m}= \frac{1}{t}||X||_{{\vR}^m}
            = \frac{1}{t} \cdot t = 1 \implies
        \end{gather*}
        \[ ||Ax_0||_{{\vR}^n} \leq ||A|| \tag{5}\] 
        $ (5) \implies ||Ax_0||_{{\vR}^n}=||A \left(\frac{1}{t}x \right)||_{{\vR}^n} $
        $\leq ||A|| \implies 4.$
        \end{proof}
        \begin{theorem}
            \[c > 0, \forall x \in {\vR}^m\] 
            \[||Ax||_{{\vR}^n} \leq c||X||_{{\vR}^m} \forall x \in {\vR}^m \tag{6} \]
            \[\implies ||A|| \leq c \tag{6\prime} \]
        \end{theorem}
        \begin{proof}
            \begin{gather*}
            (6) \implies \text{при } ||X||_{{\vR}^m} < 1 \text{ имеем} \\
            ||AX||_{{\vR}^n} \leq c \cdot ||X||_{{\vR}^m} \leq c \rightarrow
            \underset{||X|| \leq 1}{\sup} ||AX|| \leq c \implies (6\prime)
            \end{gather*}
        \end{proof}
    \begin{theorem}
        \begin{gather*}
        E = \{ c \in {\vR}, с > 0 : \forall x \in {\vR}^m \text{ имеем }
        ||Ax||_{{\vR}^n} \leq C ||X||_{{\vR}^m} \} \tag{7} \\
        \text{В случае } A \neq 0 \\
        ||A|| = \inf E, ||A||_2 \in E, \inf E \stackrel{def}{=} m \\
        ||A|| = \inf E \tag{8} 
    \end{gather*}
    \end{theorem}
    \begin{proof}
        \begin{enumerate}
            \item $ m = 0$
            \begin{gather*}
                \forall \varepsilon > 0 \exists c_1 \in E : с_1 < \varepsilon \\
                (7) \rightarrow ||Ax||_{{\vR}^n} \leq c_1||x||_{{\vR}^m}
                \forall x \in {\vR}^m \rightarrow ||A|| \leq c_1 < \varepsilon \\
                \rightarrow ||A|| = 0
            \end{gather*}
            \item $m > 0$
            \begin{gather*}
                ||A|| \in E, m \leq ||A|| \\
                \text{пусть } m < ||A|| \implies \exists c_2 : m < c_2 < ||A|| \tag{9} \\
                (7) \implies ||Ax||_{{\vR}^n} \leq c_2||x||_{{\vR}^m} \forall x \in {\vR}^m \\
                [(9) \implies ||A|| \leq c_2 \text{ противоречие} \end{gather*}
        \end{enumerate}
    \end{proof}
    \begin{theorem}
    \begin{gather*}
        A : {\vR}^m \rightarrow {\vR}^n, B: {\vR}^n \rightarrow {\vR}^k \\
        L: {\vR}^m \rightarrow {\vR}^k \\
    \end{gather*}
        \[Lx = B(A(x))\ \tag{10} \]
        Каждая норма соответствует своей паре пространств! 
        \[||L|| \leq ||A|| \cdot ||B||\]

    \end{theorem}
    \begin{proof}
        \begin{gather*}
            \forall x \in {\vR}^m \text{  }||LX||_{{\vR}^k} = ||B(AX)||_{{\vR}^k}
        \leq ||B|| \cdot ||AX||_{{\vR}^n} 
        \end{gather*}
        \[\leq \underbrace{||B|| \cdot ||A||}_{c}
         \cdot ||X||_{{\vR}^m}\   \tag{11} \]
        по свойству 5 $(11) \implies (10)$
    \end{proof}

\subsection*{Дифференцируемость суперпозиции линейных отображений}
    \begin{gather*}
        \Omega \subset {\vR}^m, m \geq 1 \\
        X_o \in \Omega - \text{ внутрення точка}\\
        G \in {\vR}^n, Y_0 \in G, Y_0 - \text{ внутренняя точка в } G \\
        F = \begin{bmatrix}
            f_1 \\
            \vdots \\
            f_n
        \end{bmatrix} : \Omega \rightarrow {\vR}^n \text{ }\forall x \in \Omega 
        \text{ }F(x) \in G, F(X_0) = Y_0 \\
        \Phi = \begin{bmatrix}
            \varphi_1(y) \\
            \vdots \\
            \varphi_k(y)
        \end{bmatrix} : G \rightarrow \mathbb{R^k} 
    \end{gather*} 
    \[\exists P: \Omega \rightarrow {\vR}^k, P(x) \stackrel{def}{=} \Phi(F(x))
     \tag{15} \]


    \begin{theorem}
        $F$ дифференцируема в $X_0$, $\Phi$ дифференцируема в $X_0$ 
        $\Rightarrow P$ дифференцируема в $X_0$ и 
        \[DP(X_0) = D\Phi(X_0) \cdot DF(X_0) \text{( это матрицы Якоби)} \tag{16}\]
    \end{theorem}

    \begin{proof}
        \begin{gather*}
            \Phi(Y_0 + \lambda) - \Phi(Y_0) = B \lambda + \rho(\lambda) \tag{17} \\
            B = D\Phi(Y_0), \rho(\lambda) \in {\vR}^k \text{ и } \frac{||\rho(\lambda)||_{{\vR}^k}}{||\lambda||_{{\vR}^n}} \xrightarrow[\lambda \to \mathbb{0}_n]{}0 \tag{18} \\
            \lambda = \mathbb{0}_n, \Phi(Y_0) - \Phi(Y_0) = \mathbb{0}_k + \rho(\mathbb{0}_n) \rightarrow \rho(\mathbb{0}_n) = \mathbb{0}_k \\
            \forall \eta > 0 \exists \delta_1 > 0 : \\
            (18) \implies\forall \lambda \in {\vR}^n, \lambda \neq \mathbb{0}_n \text{ и } ||\lambda||_{{\vR}_n} <\delta_1 \text{ будет } \frac{||\rho(\lambda)||_{{\vR}_k}}{||\lambda||_{{\vR}^n}} < \eta \tag{19}
            \intertext{и}
            ||\rho(\lambda)||_{{\vR}^k} \leq \eta \cdot ||\lambda||_{{\vR}^n}  \tag{20} \\
            (19) \implies ||\rho(\lambda)||_{{\vR}^k} \leq \eta \cdot ||\lambda||_{{\vR}^n} \tag{19'} \\
            \forall \varepsilon > 0 \exists \delta_2 > 0 : \forall H \in {\vR}^m, ||H||_{{\vR}^m} \delta_2  \text{ имеем } \\
            ||r(H)||_{{\vR}^n} \leq \varepsilon||H||_{{\vR}^m} \tag{21} \\
            H \in {\vR}^m, x_0 + H \in \Omega \text{ возможно т.к. внутрення точка}\\
            F(X_0) = Y_0 \\
            \text{определим } F(x_0 + H) - F(x_0) = \lambda \tag{22} \\
            F(x_0+H) - Y_0 = \lambda \tag{22'}
        \end{gather*}
        \begin{align*}
            P(x_0+H) - P(x_0) = \Phi(F(x_0 + H)) - \Phi(F(x_0)) \stackrel{(22\prime)}{=} \\
            \Phi(Y_0 + \lambda) - \Phi(Y_0) 
             \stackrel{(17)}{=} B\lambda + \rho(\lambda)
              \stackrel{(12),(22)}{=} \\
             B(AH+r(H)) + \rho(AH + r(H)) = 
        \end{align*} 

        \begin{gather*}
         = \overbrace{(BA)}^{P(x_0+H)-P(x_0)}H + \overbrace{Br(H) + \rho(AH+r(H))}^{r_1(H)} \tag{23\prime}\\
         ||H||_{{\vR}^m} < \delta_2 \rightarrow ||r(H)||_{{\vR}^m} \leq \varepsilon ||H||_{{\vR}^m}
        \tag{23} \end{gather*}
        \begin{align*}
            \delta = \varepsilon \text{  } \exists \delta_1 : \text{ выполнено (20) при }
            ||\lambda||_{{\vR}^n} \delta_1 \\
            ||\lambda||_{{\vR}^n} = ||AH + r(H)||_{{\vR}^n} \leq ||AH||_{{\vR}^n}
            + ||r(H)||_{{\vR}^n} \\
            \stackrel{(22)}{\leq} ||A|| \cdot ||H||_{{\vR}^m} + \varepsilon \cdot ||H||_{{\vR}^m}
            =(||A||+\varepsilon)||H||_{{\vR}^m} <\delta_1
        \end{align*}
        то есть 
        \[|H||_{{\vR}^m} \leq \frac{\delta_1}{||A||+\varepsilon}\tag{24}\]
        \[\delta_0 = \min(\delta_2, \frac{\delta_1}{||A|| + \varepsilon}) \tag{25}\]
        и полагаем $||H||_{{\vR}^n} < \delta_0$

    \begin{gather*}
       \text{При } ||H||_{{\vR}^m} < \delta_0 (26) \implies ||r_1(H)||_{{\vR}^k}  
        \leq ||Br(H)||_{{\vR}^k} + \\ 
        + ||\rho(AH+r(H))||_{{\vR}^k} 
        \leq ||B|| \cdot ||r(H)||_{{\vR}^n} + \varepsilon||AH+r(H)||_{{\vR}^n} \\
        \stackrel{(20), (23)}{\leq} ||B|| \cdot \varepsilon||H||_{{\vR}^m} +
        \varepsilon(||AH||_{{\vR}^n}+||r(H)||_{{\vR}^n}) \leq \\
        \leq ||B||\varepsilon||H||_{{\vR}^m} + \varepsilon(||A|| \cdot ||H||_{{\vR}^m}
        +\varepsilon||H||_{{\vR}^m}) = \\
        =\varepsilon(||B|| + ||A|| + \varepsilon)||H||_{{\vR}^m} \tag{22}\\
        P(x_0)+H - P(X_0) = (BA)H + r_1(H) \tag{27\prime} \\
        \text{При } ||H||_{{\vR}^n} < \delta_0 \text{ имеем }
        ||r_1(H)||_{{\vR}^k} \leq \varepsilon(||B|| + ||A|| + \varepsilon)||H||_{{\vR}^m} 
        \tag{27\prime\prime}\\
        (27^{\prime\prime}) \implies \frac{||r_1(H)||_{{\vR}^k}}{||H||_{{\vR}^m}}
        \underset{H \to \mathbb{0}_m}{\longrightarrow} 0 \tag{28} \\
        (23^\prime), (27^\prime), (28) \implies (16) \end{gather*}
    \end{proof}
\end{document}