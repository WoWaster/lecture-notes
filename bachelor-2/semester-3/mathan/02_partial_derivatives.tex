% !TeX root = ./main.tex

\documentclass[main]{subfiles}
\begin{document}
\chapter{Частные производные}
    \section{Частные производные второго порядка}
    \begin{gather*}
        E \in \vR^n \quad n \geq 2 \\
        x_0 \in E \quad x_0 - \text{ внутренняя точка} \\
        f: E \rightarrow \vR \\
        1 \leq i \leq n \quad 1 \leq j \leq n \quad x_0 \in W \subset E \\
        \forall x \in W \quad \exists f^\prime_{x_i}(x) \\
        g(x) = f^\prime_{x_i}(x) \quad g: W \rightarrow \vR
        \exists g^\prime_{x_j}(x_0) \\
        f^{\prime\prime}_{x_ix_j} \stackrel{def}{=} g^\prime_{x_j}(x_0) \\
        i \ne j \quad \forall x \in W \quad \exists f^\prime_{x_j}(x) \\
        h: W \rightarrow \vR \quad h(x) \stackrel{def}{=} f^\prime_{x_j}(x) \\
        \exists h^\prime_{x_i}(x_0) \\
        f^{\prime\prime}_{x_jx_i}(x_0) \stackrel{def}{=} h^\prime_{x_i} (x_0) \\
        f(x_1,x_2) = \begin{cases}
            \frac{x_1^3x_2}{x_1^2 + x_2^2} \quad \text{ если} x_1^2 + x_2^2 > 0\\
            0 \quad \text{ если } x_1 = x_2 = 0
        \end{cases} 
    \end{gather*}
    \begin{gather*}
        |f(\ldots)| \leq \frac{|x_1|^3|x_2|}{x_1^2} = |x_1||x_2| \underset{(x_1,x_2) \to (0,0)}{\longrightarrow} 0 \\
        f^\prime_{x_1} (\ldots) = \frac{3x_1^2x_2}{x_1^2+x_2^2} - 2\frac{x_1^3\cdot x_2 \cdot x_1}{(x_1^2+x_2^2)^2} = 
        \frac{3x_1^2x_2}{x_1^2+x_2^2} - \frac{2x_1^4x_2}{(x_1^2+x_2^2)^2} \end{gather*}
        \begin{gather*}
        f^\prime_{x_1}(0,0) = \underset{x_1 \to 0}{\lim} \left ( \left (\frac{x_1^3 \cdot 0}{x_1^2+0} - 0 \right ) \frac{1}{x_1} \right ) = 0\\
        \left | \frac{3x_1^2x_2}{x_1^2 + x_2^2} \right | = 3 |x_2| \underset{(x_1,x_2) \to 0}{\longrightarrow} 0\\
        x_2 \ne 0 \\
        f^\prime_{x_1}(0,x_2) = 0 \\
        f^{\prime\prime}_{x_1x_2}(0,0) = \underset{x_2 \to 0}{\lim} \frac{f^\prime_{x_1}(0,x_2)-f^\prime_{x_1}(0,0)}{x_2} = 0\\
        f^\prime_{x_2}(x_1,x_2) = \left ( \frac{x_1^3x_2}{x_1^2+x_2^2} \right )^\prime_{x_2} = \frac{x_1^3}{x_1^2+x_2^2} - 2\frac{x_1^3x_2^2}{(x_1^2+x_2^2)^2} \\
        f^\prime_{x_2}(0,0) = \underset{x_2 \to 0}{\lim} \left ( \left ( \frac{0 \cdot x_2}{0 + x_2^2} - 0 \right ) \frac{1}{x_2} \right ) = 0 \\
        \left | \frac{x_1^3}{x_1^2 + x_2^2} \right | \leq |x_1| \underset{(x_1,x_2) \to 0}{\longrightarrow} 0 \\
        -2 \frac{x_1^3}{x_1^2+x_2^2} \cdot \frac {x_2^2}{x_1^2+x_2^2} \\
        f^{\prime\prime}_{x_2x_1}(0,0) = \underset{x_1 \to 0}{ \lim} \frac{f^\prime_{x_2}(x_1,0)-f^\prime_{x_2}(0,0)}{x_1} = \underset{x_1 \to 0}{\lim} \frac{x_1}{x_1} = 1 \\
        f^\prime_{x_2}(x_1,0) = x_1\\
    \end{gather*}

    \section{Теорема о смешанных производных}
        \includegraphics[width=0.1\linewidth]{chikulya_finally.pdf}
        \begin{gather*}
            E \subset \vR^2 - \text{ открытое множество } \\
            f: E \rightarrow \vR \\
            x_0 = (x_1^0, x_2^0) \in E \\
            f \in C(E) \forall x \in E \quad \exists f^\prime_{x_1}(x) \quad f^\prime_{x_2}(x) \\
            f^\prime_{x_1}, f^\prime_{x_2} \in C(E) \\
            \forall x \in E \quad \exists f^{\prime\prime}{x_1x_2}(x) \quad \exists f^{\prime\prime}_{x_2x_1} \\
            f^{\prime\prime}_{x_1x_2} \text{  и } f^{\prime\prime}_{x_2x_1} \text{ непрерывны в } x_0 \\
            \implies f^{\prime\prime}_{x_1x_2} (x_0)  = f^{\prime\prime}_{x_2x_1}(x_0)
        \end{gather*}
        \begin{longProof}
            \begin{gather*}
                \exists r > 0 \quad B_{2r}(x_0) = \{ (x_1,x_2) : (x_1-x_1^0)^2 + (x_2-x_2^0)^2 < 4r^2 \} \\
                B_{2r}(x_0) \in E \quad |x_1^0 - x_1^0| \leq r \quad |x_2-xx2^0| \leq r \implies \\
                \implies (x_1,x_2) \subset B_{2r}(x_0) \\
                0 < h \leq r \\
                g(h) = \frac{f(x_1^0+h,x_2^0+h)-f(x_1^0+h,x_2^0) - f(x_1^0,x_2^0+h) + f(x_1^0,x_2^0)}{h^2} \tag{1}\\
                \phi(x_2) = \frac{f(x_1^0+h,x_2) - f(x_1^0,x_2)}{h} \tag{2}\\
                x_2 \in [x_2^0-r, x_2^0+r]  \quad (1),(2) \implies g(h) = \frac{\phi(x_2^0+h)-\phi(x_2^0)}{h} \tag{3} \\
                (2) \implies \forall x_2 \in [x^0_2 - r, x^0_2 + r] \exists \phi^\prime(x_2) = \frac{f^\prime_{x_2}(x_1^0+h,x_2)-f^\prime_{x_2}(x_1^0,x_2)}{h} \tag{4}\\
                \text{по теореме Лагранжа } \exists h_2 \quad 0 < h_2 < h : \\
                \phi(x_2^0+h) - \phi(x_2^0) = \phi^\prime(x_2^0+h_2) \cdot h \tag{5}
            \end{gather*}
            \begin{gather*}
                (3),(5) \implies g(h) = \phi^\prime(x_2^0 + h_2) \tag{6}\\
                (4),(6) \implies g(h) = \frac{f^\prime_{x_2}(x_1^0+h,x_2^0+h_2)-f^\prime_{x_2}(x_1^0,x^0_2 + h_2)}{h} \tag{7}\\
                \forall x_1 \in [x_1^0-r,x_1^0+r] \exists (f^\prime_{x_2}(x_1,x_2^0+h_2))^\prime_{x_1} = f^{\prime\prime}_{x_2x_1}(x_1,x_2^0+h_2) \tag{8}\\
                \exists 0 < h_1 < h : f^\prime_{x_2}(x_1^0+h,x_2^0+h_2) - f^\prime_{x_2}(x_1^0,x_2^0+h_2) =\\=
                f^{\prime\prime}_{x_2x_1}(x_1^0 + h_1, x_2^0 + h_2) \cdot h \tag{9}\\
                (7),(9) \implies g(h) = f^{\prime\prime}_{x_2x_1}(x_1^0+h_1,x_2^0 + h_2) \tag{10}\\
                \psi(x_1) \ \frac{f(x_1,x_2^0+h)-f(x_1,x_2^0)}{h} \tag{11}
            \end{gather*}
            \begin{gather*}
                x_1 \in [x_1^0-r,x_1^0+r] \\
                g(h) = \frac{\psi(x^0_1+h)-\psi(x_1^0)}{h} \tag{12}\\
                (11) \implies \forall x_1 \in [x_1^0-r,x_1^0+r] \exists \psi^\prime(x_1) =
                \frac{f^\prime_{x_1}(x_1,x_2^0+h)-f^\prime_{x_1}(x_1,x_2^0)}{h} \tag{13}\\
                (13) \implies \psi (x_1^0+h) - \psi(x_1^0) = \psi^\prime(x_1^0+\overline{h})h \tag{14}
            \end{gather*}
            \begin{gather*}
                \exists \overline{h_1} \quad 0 < \overline{h_1} < h \\
                (12),(14) \implies g(h) = \psi^\prime(x_1^0 + \overline{h_1}) = \\ =
                 \frac{f^\prime_{x_1}(x_1^0+\overline{h_1},x_2^0+h)-f^\prime_{x_1}(x_1^0+\overline{h_1},x_2^0)}{h} \tag{15}\\
                f^\prime_{x_1}(x_1^0 + \overline{h}, x_2) \\
                (15) \implies \exists \overline{h_2}, 0 < \overline{h_2} < h : 
                \frac{f^\prime_{x_1}(x_1^0 + \overline{h_1}, x_2^0 + h) - f^\prime_{x_1}(x_1^0 + \overline{h_1}, x_2^0)}{h} = \\
               = f^{\prime\prime}_{x_1 x_2}(x_1^0 + \overline{h_1}, x_2^0 + \overline{h_2}) \tag{16} \\
                (15),(16) \implies g(h) = f^{\prime\prime}_{x_1x_2}(x_1^0+ \overline{h_1}, x_2^0 + \overline{h_2}) \tag{17}\\
                f^{\prime\prime}_{x_2x_1}(x_1^0+ h_1, x_2^0 + h_2) \underset{h \to 0}{\rightarrow} f^{\prime\prime}_{x_2x_1}(x_1^0, x_2^0) \tag{18}\\
            f^{\prime\prime}_{x_1x_2}(x_1^0+ \overline{h_1}, x_2^0 + \overline{h_2}) \underset{h \to 0}{\rightarrow} f^{\prime\prime}_{x_1x_2}(x_1^0x_2^0) \tag{19}\\
            (10), (17), (18), (19) \implies f^{\prime\prime}_{x_1x_2}(x_0) = f^{\prime\prime}_{x_2x_1}(x_0)
            \end{gather*}
        \end{longProof}

        \begin{theorem}
            \begin{gather*}
                E \subset \vR^n \quad n \geq 3 \quad x_0 = (x_1^0 \ldots x_i^0 \ldots x^0_j \ldots x_n^0) \in E \\
                f: E \rightarrow \vR \quad f \in C(E) \quad \forall X \in E \exists f^\prime_{x_i}(X) f^\prime{x_j}(X) \\
                f^\prime_{x_i}(X), f^\prime_{x_j}(X) \in C(E) \quad \exists f^{\prime\prime}_{x_ix_j}(X) f^{\prime\prime}_{x_jx_i}(X) \\
                f^{\prime\prime}_{x_ix_j}(X) \text{ и } f^{\prime\prime}_{x_jx_i}(X) \text{ непрерывны в } X_0 \implies \\
                \implies f^{\prime\prime}_{x_ix_j}(X_0) = f^{\prime\prime}_{x_jx_i}(X_0)
            \end{gather*}
        \end{theorem}
        \begin{longProof}
            \begin{gather*}
                B_r(X_0) \subset E \quad F(x_i, x_j) = f(x_1^0 \ldots x_i \ldots x_j \ldots x^0_n) \\
                K_r(x_i, x_j) = \{ (x_i - x^0_i)^2 + (x_j-x_j^0)^2 < r^2 \} \\
                \forall (x_i, x_j) \in K_2(x_i^0, x_j^0) \quad X = (x_1^0 \ldots x_i \ldots x_j \ldots x_n^0) \\
                F^{\prime\prime}_{x_ix_j}(x_i,x_j) = f^{\prime\prime}_{x_ix_j}(X) \\
                F^{\prime\prime}_{x_jx_i}(x_i, x_j) = f^{\prime\prime}_{x_jx_i}(X) \\
                F^{\prime\prime}(x_i^0, x_j^0) = F^{\prime\prime}_{x_j x_i}(x_i^0, x_j^0)
            \end{gather*}
            \[
                 \left. \begin{gathered} E \subset \vR^n \quad n \geq 2 \quad i \ne j \\
                f: E \rightarrow \vR \\
                f \quad f\prime_{x_i} f^\prime_{x_j} f^{\prime\prime}_{x_jx_j} f^{\prime\prime}_{x_jx_i} \in C(E) \end{gathered} \right\} \implies \forall x \in E \quad f^{\prime\prime}_{x_ix_j}(X) = f^{\prime\prime}_{x_jx_i}(X)
            \]
        \end{longProof}

\section{Частные производные третьего и более порядков}
\begin{gather*}
    E \subset \vR^n \quad n \geq 2 \quad x_0 \in E \quad f: E \rightarrow \vR \\
    1 \leq i,j,k \leq n \quad \forall x \in E \quad \exists f^{\prime\prime}_{x_ix_j}(X) \\
    \exists (f^{\prime\prime}_{x_ix_j})\prime_{x_k}(X_0) \stackrel{def}{=} f^{\prime\prime\prime}_{x_ix_jx_k}(X_0) \\
    1 \leq i_1 \ldots i_m \leq n \quad f^{(m)}_{x_{i_1}\ldots x_{i_m}}(X) \forall X \in E \\
    1 \leq i_{m+1} \leq n \\
    \exists (f^{(m)}_{x_{i_1} \ldots x_{i_m}})^{\prime}_{x_{i_{m+1}}}(X_0) \stackrel{def}{=} f^{(m+1)}_{x_{i_1} \ldots x_{i_{m+1}}}(X_0) \\
    C^r(E) \quad r \geq 1 \quad e \subset \vR^n \quad n \geq 2 
\end{gather*}
    $f \in$ классу $C^1(E)$, если $f \in C(E) \forall x \in E$ и $\forall i \quad 1 \leq i \leq n \quad
    \exists f^{\prime}_{x_i}(X)$ т.ч. $f^\prime_{x_i} \in C(E)$. $f \in C^{r+1}(E)$, если $f^\prime_{x_i} \in C^r(E) \quad 1 \leq i \leq n$.
    $f \in C^2(E) \Leftrightarrow f^\prime_{x_i} \in C^1(E) \quad 1 \leq i \leq n$. $f^\prime_{x_i} \in C^1(E) \Leftrightarrow
    \forall i, j f^{\prime\prime}_{x_i x_j}(X) \in C(E) \implies$ при $i \ne j f^{\prime\prime}_{x_ix_j}(X) = f^{\prime\prime}_{x_jx_i}(X)$

    \begin{theorem}
        \begin{gather*}
            E \subset \vR^n \quad n \geq 2 \quad r \geq 2 \quad 1 \leq i_1 \ldots i_r \leq n \quad 1 \leq j_1 \ldots j_r \leq n \\
            f \in C^r(E) \quad \forall X \in E \quad f^{(r)}_{x_{i_1} \ldots x_{i_r}}(X) = f^{(r)}_{x_{j_1} \ldots x_{j_r}}(X) 
        \end{gather*}
    \end{theorem}
        \begin{longProof}[по индукции]
            \begin{gather*}
                r = 2 \text{ -- уже  рассмотрено.} \\
                i_1, \ldots, i_{r-1}, i_r, i_{r+1} \\
                j_1, \ldots, j_{r-1}, j_{r+1}, j_r\\
                i_r \ne i_{r+1} \\
                f \in  C^{r+1}(E) \implies f \in C^{r-1}(E) \\
                F(X) = f^{(r-1)}_{x_{i_1} \ldots x_i{r-1}}(X) = f^{(r-1)}_{x_{j_1} \ldots x_{j_{r-1}}}(X) \forall X \in E \\
                f^{(r+1)}_{x_{i_1} \ldots x_{i_{r-1}} x_{i_r} x_{i_{r+1}}}(X) = F^{\prime\prime}_{x_{i_r} x_{i_{r+1}}}(X) \\
                f^{(r+1)}_{x_{j_1} \ldots x_{j_{r-1}} x_{i_{r_1}} x_{i_r}}(X) = F^{\prime\prime}_{x_{i_{r+1}} x_{i_r}}(X) \\
                F^{\prime\prime}_{x_{i_r} x_{i_{r+1}}} \in C(E) \quad F^{\prime\prime}_{x_{i_{r+1}}x_{i_r}} \in C(E) \\
                \implies F^{\prime\prime}_{x_{i_r}x_{i_{r+1}}}(X) = F^{\prime\prime}_{x_{i_{r+1}} x_{i_r}}(X) \\
                f^{(r+1)}_{x_{i_1} \ldots x_{i_{r-1}} x_{i_r} x_{i_{r+1}}}(X) =  f^{(r+1)}_{x_{j_1} \ldots x_{j_{r-1}} x_{j_{r+1}} x_{j_r}}(X) \\
                1 \leq k < r \\
                i_{1} i_{k-1} i_k i_{k+1} i_{k+2} \ldots i_{r+1} \\
                j_{1} j_{k-1} j_k j_{k+1} j_{k+2} \ldots j_{r+1} \\
                \Phi(X) = f^{(k+1)}_{x_{i_1} \ldots x_{i_{k-1}} x_{i_k} x_{i_{k+1}}}(X) = f^{(k+1)}_{x_{j_1} \ldots x_{j_{k-1}} x_{j_{k+1}} x_{j_k}}(X)  \\
                f^{(r+1)}_{x_{i_1} \ldots x_{i_{k-1}} x_{i_k} x_{i_{k+1}} \ldots x_{i_{r+1}}}(X) = \Phi^{(r-k)}_{x_{i_{k+2}} \ldots x_{i_{r+1}}}(X) \\ 
                f^{(r+1)}_{x_{j_1} \ldots x_{j_{k-1}} x_{j_k} x_{j_{k+1}} \ldots x_{j_{r+1}}}(X) = \Phi^{(r-k)}_{x_{i_{k+2}} \ldots x_{i_{r+1}}}(X) 
            \end{gather*}
        \end{longProof}
    \begin{gather*}
        E \quad f \in C^r(E) \quad E \subset \vR^n \quad n \geq 2 \\
        \forall x \in E \quad f^{(r)}_{x_{i_1} \ldots x_{i_r}} = f^{(r)}_{\underbrace{x_1 \ldots x_1}_{P_1} \underbrace{x_2 \ldots x_2}_{P_2} \ldots \underbrace{x_n \ldots x_n}_{P_n}} \\
        n \geq 2 \quad \alpha = (\alpha_1 \ldots \alpha_n) \quad \alpha_j \geq 0 \quad \alpha_j \in \mathbb{Z} \quad 1 \leq j \leq n \\
        |\alpha| = \alpha_1 + \ldots + \alpha_n \\
        \alpha! \stackrel{def}{=} \alpha_1! \ldots \alpha_n! \\
        |\alpha| = r > 0 \quad C^\alpha_r \stackrel{def}{=} \frac{r!}{\alpha!} = \frac{r!}{\alpha_1! \ldots \alpha_n!} \\
        X = (X_1 \ldots X_n) \in \vR^n \\
        X^\alpha \stackrel{def}{=} X_1^{\alpha_1} \ldots X_n^{\alpha_n} \\
        0^0 \stackrel{def}{=} 1 \\
        \partial^{\alpha}f(x) \stackrel{def}{=} f^{|\alpha|}_{\underbrace{x_1 \ldots x_1}_{\alpha_1} \ldots \underbrace{x_n \ldots x_n}_{\alpha_n}} 
    \end{gather*}

    \begin{theorem}
    \begin{gather*}
        E \subset \vR^n \quad X_0 \in E \quad H \in \vR^n \quad t_0 \in \vR \\
        X_0 + t_0H \in E \quad f: E \rightarrow \vR \quad f \in C^r(E) \quad r \geq 1 \\
        g(t) = f(X_0 + tH) \quad t \in (t_0 - \delta, t_0 + \delta) \quad g \in C^r((t_0 - \delta, t_0 + \delta)) \\
        g^{(r)}(t_0) = \sum_{\alpha : |\alpha| = r} C^\alpha_r \partial^\alpha f(X_0 + t_0H)H^\alpha  \tag{1}
    \end{gather*}
\end{theorem}
\begin{longProof}[по индукции]
    \begin{gather*}
        \text{ пусть } r=1 \quad f \in C^1(E) \\
        P(t) = t \rightarrow X_0 + tH \quad (t_0 - \delta, t_0 + \delta) \rightarrow \vR^n \\
        f: E \rightarrow \vR^1 \quad H =  \begin{bmatrix}
            h_1 \\
            \vdots \\
            h_n
        \end{bmatrix} \quad \ldots = x_0 + t_0H \\
        Dg(t_0) = \underbrace{Df(X_0 + t_0H)}_{=(f^{\prime}_{x_1}(\ldots), \ldots, f^\prime_{x_n}(\ldots))}\underbrace{DP(t_0)}_{=H}  \tag{2} \\
        (2) \implies g^\prime(t_0) = (f^{\prime}_{x_1}(\ldots), \ldots, f^\prime_{x_n}(\ldots)) \cdot \begin{bmatrix}
            h_1 \\
            \vdots \\
            h_n
        \end{bmatrix}  =\\
         = f^\prime_{x_1}(\ldots)h_1 + \ldots + f^\prime_{x_n}(\ldots)h_n \tag{3} \\
        f^\prime_{x_j}(t_0 + tH) \in C((t_0 - \delta, t_0 + \delta)) \\
        \sum_{\alpha: |\alpha| = 1} C^\alpha_1 \partial^\alpha f(\ldots) H^\alpha = \sum^{n}_{j=1} f^\prime_{x_j}(\ldots)h_j \tag{4}\\
        \tilde{\alpha} = (1,0,\ldots, 0) \\
        C^{\tilde{\alpha}}_1 = \frac{1!}{1!0!\ldots0!} = 1 \\
        H^{\tilde{\alpha}} = h_1^1 h_2^0 \ldots h_n^0 = h_1 
    \end{gather*}
    База доказана. Теперь переход.
    \begin{gather*}
        f \in C^{r+1}(E) \\
        g^{(r)}(t_0) = \sum_{\alpha: |\alpha| = r} C^\alpha_r \partial^\alpha f(X_0 + t_0H) H^\alpha \tag{5} \\
        g^{(r+1)}(t_0) = \sum_{\alpha: |\alpha| = r} C^\alpha_r (\partial^\alpha f(X_0 + t_0H))^\prime_t H^\alpha \stackrel{(3)}{=} \\
     \partial^\alpha f(X_0 + tH) \in C^1((t_0 - \delta, t_0 + \delta)) \\
     \stackrel{(3)}{=} \sum_{\alpha: |\alpha| = 1} C^\alpha_r \left( \sum_{j=1}^n(\partial^\alpha f(X_0 + t_0H)^\prime_{x_j}h_j)\right)H^\alpha \textcolor{red}{ =} \\
     e_j = (0, \ldots, \underbrace{1}_j, \ldots, 0) \quad h_j = H_{e_j} \\
     (\partial^\alpha f(\ldots))^\prime_x = \partial^{\alpha + e_j} f(\ldots) \\
     \textcolor{red}{= } \sum^n_{j=1} \sum_{\alpha: |\alpha|=r}C^\alpha_r \partial^{\alpha + e_j} f(\ldots)H^{\alpha + e_j} = \\
     \{ \beta: \beta = \alpha + e_j, |\alpha| = r, 1 \leq j \leq n \} \quad |\beta| = r+1 \text{ множество всех } \beta : |\beta| = r+1 \\
     \beta = (\beta_1, \ldots, \beta_n) \quad |\beta| =r + 1 
    \end{gather*}
    \begin{gather*}
     \beta_{k_1}, \ldots, \beta_{k_m} \ne 0 \\
     \alpha = (0, 0, \beta_{k_1}-1, \ldots, \beta_{k_m}, \ldots 0) \\
     \beta_j = 0 \text{ если } j \ne k_1, \ldots, k_m \\
     (0,0 \ldots, \beta_{k_1}, \beta_{k_2}-1, \ldots, \beta_{k_m}, 0) \quad (0, 0, \ldots, \beta_{k_1} \ldots, \beta_{k_m}-1) \\
     = \sum_{\beta: |\beta| = r+1} \partial^\beta f(\ldots)H^\beta \sum_{\alpha, e_j: \alpha + e_j=\beta}C^\alpha_r \textcolor{red}{ =} \\
    \end{gather*}
    \includegraphics[width=1\linewidth]{eji.pdf}
    \begin{gather*}
        \beta_{k_1} + \ldots + \beta_{k_m} = |\beta| = r + 1 \\
        \sum_{\alpha, j: \alpha + e_j = \beta} C^\alpha_r = \sum^m_{\nu=1} \frac{r! \cdot \beta_{k_\nu}}{\beta_{k_1}!(\beta_{k_\nu}-1)! \ldots \beta_{k_m}!} = 
        \frac{r!}{\beta!}\sum_{\nu=1}^m \beta_{k\nu} \textcolor{green}{ =} \\
        (0, \beta_{k_1}, \ldots, \beta_{k_\nu}-1, \ldots, \beta_{k_m}, 0) = \gamma_\nu \quad |\gamma_\nu| = r \\
        \textcolor{green}{= } \frac{r!|\beta|}{\beta!} = \frac{(r+1)!}{\beta!} \tag{6}\\
        \textcolor{red}{= } \sum \partial^\beta f(X_0 + t_0H)H^\beta \frac{(r+1)!}{\beta!}
     \end{gather*}
\end{longProof}

\section{Формула Тейлора для функции нескольких переменных с остатком в форме Лагранжа }
\begin{gather*}
    E \subset \vR^n \quad n \geq 2 \quad X_0 \in E \quad H \in \vR^n \quad H \ne \mathbb{0}_n \\
    t \in \vR \quad t \ne 0 \quad  X_0 + tH \in E \quad X_0 + \tau H \in E \text{ при }0 \leq |\tau| \leq |t| \quad t\tau > 0 \\ 
    [X_0, X_0 + tH] \subset E \quad r \geq 1 \quad f \in C^{r+1}(E) \\
    \exists c \quad 0 < c < 1 \end{gather*}
    \begin{multline*}
    f(X_0 + tH) = f(X_0) + \sum^r_{k=1} \sum_{\alpha: |\alpha| = k} \frac{1}{\alpha!} \partial^\alpha f(X_0)(tH)^\alpha + \\
     + \sum_{\alpha: |\alpha| = r+1} \frac{1}{\alpha!} \partial^\alpha f(X_0 + ctH)(tH)^{\alpha} \tag{7}
\end{multline*}
\begin{gather*}
    (tH)^\alpha = t^{|\alpha|}H^\alpha \quad tH = \tilde{H} \quad g(S) = f(X_0 + S\tilde{H}) \quad 0 \leq S \leq 1\\
    g \in C^{r+1}([0,1]) \quad g \in C^{r+1}((-\varepsilon, 1 + \varepsilon)) \\
    g(1) = g(0) = \sum^r_{k=1} \frac{1}{k!} g^{(k)}(0) \cdot 1^k + \frac{1}{(r+1)!} g^{(r+1)}(c) 1^{r+1} \tag{8}\\
    \exists c \quad 0 < c < 1 \\
    g(1) = g(0) + \sum^r_{k=1} \frac{1}{k!} g^{(k)}(0) \cdot  + \frac{1}{(r+1)!}g^{(r+1)}(c)  \tag{8\prime} \\
    (8) \implies (8\prime) \\
    g(0) = f(X_0) \tag{9\prime} \\
    g^{(k)}(0) = \sum_{\alpha: |\alpha| = k} C^\alpha_k \partial^\alpha f(X_0) \tilde{H}^\alpha \tag{9\prime\prime} \\
    g^{(r+1)}(c) = \sum_{\alpha : |\alpha| = r + 1} C^\alpha_{r+1} \partial^\alpha(X_0+c\tilde{H})\tilde{H}^\alpha \tag{9\prime\prime\prime} \\
    g(1) = f(X_0 + \tilde{H}) \tag{9} \\
    \frac{1}{k!} C^\alpha_k = \frac{1}{k!} \cdot \frac{k!}{\alpha!} = \frac{1}{\alpha!} \tag{10} \\
    (8\prime), (9), (9\prime) \ldots (10) \implies (7)
\end{gather*}

\section{Формула Тейлора для функции нескольких переменных с остатком в форме Пеано}

\begin{gather*}
    E \subset \vR^n \quad n \geq 2 \quad f: E \rightarrow \vR \quad f \in C^r(E) \quad r \geq 1 \\
    X_0 \in E \quad X_0 + H \in E \quad H \ne \mathbb{0}_n  \quad X_0 + tH \in E \forall t \in [0,1] \\
    f(X_0 + H) = f(X_0) + \sum^r_{k=1} \sum_{\alpha: |\alpha| = k} \frac{1}{\alpha!} \partial^\alpha f (X_0) H^\alpha + R(H) \tag{11}\\
    \frac{|R(H)|}{||H||_{\vR^n}}^r \underset{H \to \mathbb{0}_n}{\rightarrow} 0 \tag{12} \\
\end{gather*}
$r=1$
\[ f(X_0+H) = f(X_0) + \sum^n_{j=1} f^\prime_{x_j}(X_0)h_j + R(H) \quad H = \begin{bmatrix}
    h_1 \\
    \vdots \\
    h_n
\end{bmatrix} \tag{13\prime}\]
\[\frac{|R(H)|}{||H||_{\vR^n}} \underset{h \to \mathbb{0}_n}{\rightarrow} 0 \tag{13} \]
$r \geq 2 \quad r - 1 \geq 1 $
    \[\exists c \quad 0  < c < 1 \text{ т.ч. } \]
    \begin{multline*}
        f(X_0+H) = f(X_0) + \sum^{r-1}_{k=1}\sum_{\alpha: |\alpha| = k} \frac{1}{\alpha!} \partial^\alpha f(X_0)H^\alpha + \\
        + \sum_{\alpha: |\alpha| = r} \frac{1}{\alpha!} \partial^\alpha f(X_0 + cH) H^\alpha = \\
        = f(X_0) + \sum_{k=1}^r \sum_{\alpha: |\alpha| = k} \frac{1}{\alpha!} \partial^\alpha f(X_0) H^\alpha + 
        \underbrace{\sum_{\alpha: |\alpha|=r} \frac{1}{\alpha!}(\partial^\alpha f(X_0 + cH) - \partial^\alpha f(X_0))H^\alpha}_{R(H)} \tag{14} 
    \end{multline*}
    \[f \in C^r(E) \implies \forall \alpha, |\alpha| = r \quad \partial^\alpha f \in C(E) \implies \] 
    \[ |\partial^\alpha f(X_0 + cH) - \partial^\alpha f(X_0)| \underset{H \to \mathbb{0}_n}{\rightarrow} 0  \tag{15} \]
    \[|H^\alpha| = |h_1^{\alpha_1} \ldots h_n^{\alpha_n}| \leq ||H||^{\alpha_1} \cdot \ldots \cdot ||H||^{\alpha_n} = ||H||^{|\alpha|}_{\vR^n} \tag{16} \]
    \begin{multline*}
        (15), (16) \implies \frac{|(\partial^\alpha f (X_0 + cH) - \partial^\alpha f(X_0))H^\alpha|}{||H||_{\vR^n}^r} \leq \\
    \leq |\partial^\alpha f(X_0 + cH) - \partial^\alpha f(X_0)| \underset{H \to \mathbb{0}_n}{\rightarrow}  0 \tag{17}
    \end{multline*}
    $(14), (17) \implies (11), (12) $

    \section{Дифференциалы второго и последующих порядков}

    \begin{gather*}
        E \subset \vR^n \quad n \geq 1 \quad E \text{ -- открытое } \quad x_0 \in E \quad f: E \rightarrow \vR \\
        f \text{ дифференцируема в } X_0 \\
        df(X_0, H) = f^\prime_{x_1} (X_0)h_1 + \ldots + f^\prime_{x_n}(X_0)h_n \textcolor{red}{ =} \\
        H = \begin{bmatrix}
            h_1 \\
            \vdots \\
            h_n
        \end{bmatrix} \quad H \in \vR^n \quad d^1f(X_0,H) \stackrel{def}{=} df(X_0,H) \quad f \in C^r(E) \quad r \geq 1\\
        \forall x \in E \text{ и } \forall  H \in \vR^n \quad d^2f(X_0,H)  = \sum_{\alpha:|\alpha|=r} A{r,\alpha} \partial^\alpha f(X_0)H^\alpha  \tag{1} \\
        \textcolor{red}{ = } \sum_{\alpha: |\alpha| = 1} 1 \cdot \partial^\alpha f(X_0)H^\alpha \\
        f \in C^{r+1}(E) \\
        d^{r+1}f(X_0, H) \stackrel{def}{=} \sum_{\alpha: |\alpha| = r} A_{r,\alpha}d(\partial^\alpha f(X_0, H))H^\alpha \tag{2} \\
    \end{gather*}
    \begin{theorem}
        \[A_{r, \alpha} = c^\alpha_r \tag{3} \]
        \end{theorem}

        \begin{longProof}[по индукции]
            $r = 1$ справедливо. Предположим, верно для $r \geq 1$ 
            \begin{multline*}
                (2) \implies d^{r+1}f(X_0, H) = \sum_{\alpha: |\alpha| = r} C^\alpha_r d(\partial^\alpha f(X_0,H))H^\alpha = \\
                = \sum_{\alpha: |\alpha| = r} C^\alpha_r \left( \sum^n_{j=1} (\partial^\alpha f(X_0))^\prime_{x_j}h_j \right)H^\alpha = \sum_{\beta: |\beta|
                 = r+1} C_{r+1}^\beta \partial^\beta f(X_0) H^\beta 
            \end{multline*}
            \[ (1), (2), (3) \implies d^r f(X_0,H) = \sum_{\alpha: |\alpha| = r} C^\alpha_r \partial^\alpha f(X_0) H^\alpha =
             \sum_{\alpha: |\alpha|=r} \frac{r!}{\alpha!} \partial^\alpha f(X_0) H^\alpha \tag{4} \] 
            \[ (4) \implies f(X_0 + H) = f(X_0) + \sum^r_{k=1|} \frac{1}{k!} d^k f(X_0, H) + R(H) \tag{5} \]
            \[ \frac{|R(H)|}{||H||^r_{\vR^n}} \underset{H \to \mathbb{0}_n}{\rightarrow} 0 \tag{6} \]
            \[ (4) \implies d^2f(X_0, H) = \sum_{\alpha : |\alpha| = 2} C^\alpha_2 \partial^\alpha f(X_0)H^\alpha \tag{7} \] 
            \begin{gather*}
                H^\alpha = h_i h_j \quad (0 \ldots \underbrace{1}_i \ldots \underbrace{1}_j \ldots 0) \quad C^\alpha_2 = \frac{2!}{0! \ldots 1! \ldots 1! \ldots 0!} = 2 \\
                (0 \ldots \underbrace{2}_k \ldots 0) \quad H^\alpha = h^2k C^\alpha_2 = \frac{2!}{0! \ldots 2! \ldots 0!} = 1 \\
            \end{gather*}
               \[ (7) \implies d^2 f(X_0, H) \sum 2 f^{\prime\prime}_{x_ix_j} h_i h_j + \sum^n_{k=1}f^{\prime\prime}_{x_kx_k}(X_0)h^2 k \textcolor{red}{ =} \] 
               \[ 2f^{\prime\prime}_{x_ix_j}(X_0) h_ih_j = f^{\prime\prime}_{x_ix_j}(X_0) + f^{\prime\prime}_{x_jx_i}(X_0)h_j h_i \tag{8} \]
               \[ \textcolor{red}{ = } \sum^n_{i=1} \sum^n_{j=1} f^{\prime\prime}_{x_ix_j}(X_0) h_i h_j \]
        \end{longProof}

\section{Локальные экстремумы}
\[ n \geq 2 \quad a_{ik} \in \vR \quad 1 \leq i, k \leq n \quad a_{ik} = a_{ki} \quad X = \begin{bmatrix}
    x_1 \\
    \vdots \\
    x_n
\end{bmatrix} \]
$A(X) = \sum^n_{i=1}\sum^n_{k=1} a_{ik} x_i x_k$ называется положительно определённой, если $\forall X \in \vR^n, X \ne \mathbb{0}_n \quad A(X) > 0$.
$B(X) = \sum^n_{i=1}\sum^n_{j=1} b_{ik}x_i x_k$ -- отрицательно определённой, если $B(X) < 0$.
$C(X) = \sum^n_{i=1}\sum^n_{j=1} c_{ik}x_k x_k$ -- неопределённой, если $\exists x_1, x_2$ т.ч. $C(x_1) > 0, C(X_2) < 0$.
$\tilde{A}(X) \geq 0$ -- неотрицательно определённая. $\tilde{B}(X) \leq 0$ -- неположительно определённая. \\
$x^2 + y^2$ -- положительно определена \\
$x^2 - y^2$ -- не определена \\
$-x^2 -y^2$ -- отрицательно определена \\
$x^2 - 2xy y^2$ -- неотрицателна \\
$-x^2 + 2xy - y^2$ -- не положительна \\
\[A(X) > 0 \Leftrightarrow -A(X) < 0 \]
\begin{gather*}
     A(X) \text{положительно определена } \exists 0 < m \leq M : \forall X \in \vR^n \\
    m||X||^2_{\vR^n} \leq A(X) \leq M||X||^2_{\vR^n} \tag{1} \\
    B(X) \text{ отрицательно определена } \exists 0 < m_1 \leq M_1: \forall X \in \vR^n \\
    -M_1||X||^2_{\vR^n} \leq B(X) \leq -m_1||X||^2_{\vR^n} \tag{2} \\
\end{gather*}
\[ S_n = \{ X \in \vR^n : ||X||_{\vR^n} = 1 \} \] 
\[ A(X) \in C(S_n) \]
По второй теореме Вейерштрасса
\begin{gather*}
    \exists  X_- \in S_n \text{ и } \exists X_+ \in S_n: \forall X \in S_n  \\
    A(X_-) \leq A(X) \leq A(X_+) \tag{3\prime} \\
    m = A(X_-) > 0 \quad M = A(X_+) \implies \text{соотношение (1) справедливо } \tag{3}
\end{gather*} 

\begin{gather*}
    X \ne \mathbb{0}_n \quad t = ||X||_{\vR^n} > 0 \quad X_0 = \frac{1}{t}X \\
     ||X_0||_{\vR^n} = \frac{1}{t} ||X||_{\vR^n} = \frac{t}{t} = 1, \text{ то есть } x_0 \in S_n\\
     (3), (3\prime) \implies A(X) = A(tX_0) = t^2A(X_0) \geq m||X||^2_{\vR^n} \\
     A(X) = A(tX_0) \leq M||X||^2_{\vR^n} \\
\end{gather*}

\begin{gather*}
    \text {неопределённая форма }C(X) \quad \exists m_2, M_2 > 0 \quad X_1, X_2 \in \vR^n : t \in \vR \ne 0 \\
    C(tX_1) = -m_2t^2 \\
    C(tX_2) = M_2t^2 \\
    -m_2 = C(X_1) < 0 \\
    M_2 = C(X_2) > 0 \\
    d^2 f(X,H) = \sum_{i=1}^n\sum_{k=1}^n f^{\prime\prime}_{x_i x_k}(X) h_i h_k \text{ квадратичная форма } \tag{4}
\end{gather*}

\section{Достаточное условие наличия или отсутствия локального экстремума}

\begin{theorem}
    \begin{gather*}
        E \subset \vR^n \quad n \geq 2 \quad f : E \rightarrow \vR \quad x_0 \in E \\
        X_0 \in W \subset E \quad f \in C^2(W) \\
        df(X_0,H) \equiv 0 \tag{5} \\
    \end{gather*}
    Если $d^2f(X_0,H)$ положительно определена $\implies X_0$ строгий локальный минимум.
    Если $d^2f(X_0,H)$ отрицательно определена $\implies X_0$ строгий локальный максимум.
    Если не определена $\implies$ в $X_0$ нет локального экстремума.
\end{theorem}

\begin{longProof}
    \begin{gather*}
        H \ne \mathbb{0}_n, X_0 + H \in W, X_0 + tH \in W \quad 0 \leq t \leq 1 \\
        f(X_0 + H) = f(X_0) + df(X_0,H) + \frac{1}{2}d^2f(X_0,H) + r(H) \tag{6}\\
        \frac{|r(H)|}{||H||^2_{\vR^n}} \underset{H \to \mathbb{0}_n}{\rightarrow} 0 \tag{7}\\
        \exists m > 0 : d^2f(X_0,H) \geq m||H||^2_{\vR^n} \tag{8}\\
        (7) \implies \exists \delta > 0 : \text{ при } H \ne \mathbb{0}_n \quad ||H|| < \delta \\
        \text{ выполнено } |r(H)| < \frac{m}{4}||H||^2 \tag{9} \end{gather*}
        \begin{multline*}
            (5),(6),(8), (9) \implies f(X_0 + H) \geq f(X_0) + \frac{m}{2}||H||^2 - |r(H)| \geq \\
            \geq f(X_0) + \frac{m}{2}||H||^2 - \frac{m}{4}||H||^2 = f(X) + \frac{m}{4}||H||^2 > f(X_0)
        \end{multline*}
        \begin{gather*}
        f_1, f_2 \quad a,b \in \vR \quad (af_1 + bf_2)^\prime_{x_i} = a{f_1}^\prime_{x_i} + b{f_2}^\prime_{x_i} \\
            ((af_1 + bf_2)^\prime_{x_i})^\prime_{x_k} = (a{f_1}^\prime_{x_i} + b{f_2}^\prime_{x_i})^\prime_{x_k} =
            a{f_1}^{\prime\prime}_{x_ix_k} + b{f_2}^{\prime\prime}_{x_i x_k} \\
            d^2(-f(X_0,H)) = -d^2f(X_0,H) \tag{10} 
        \end{gather*}

        отрицательно определена 
        \begin{gather*}
            g(x) = -f(x) \Leftrightarrow f(x) = - g(x) \\
            (10) \implies d^2g(X_0,H) = -d^2f(X_0,H) \quad X_0 \text{  -- локальный минимум } g \tag{10\prime}
        \end{gather*}
        не определена
        \begin{gather*}
            \exists H_1 \text{ и } m_1 > 0 : d^2f(X_0,tH_1) = m_1t^2 \tag{11}\\
            \exists H_2 \text{ и } m_2 > 0: d^2f(X_0,tH_2) = -m_2t^2 \tag{12} \\
            m = \min(m_1, m_2) > 0 \\
            f(X_0 + tH) = f(X_0) + \frac{1}{2} t^2 d^2f(X_0,H) + r(tH) \tag{13}\\
            \frac{|r(tH)|}{|t|^2} \underset{t \to 0}{\rightarrow} 0 \quad |t|||H|| = ||tH|| \tag{14} \\
            \varepsilon > 0 \exists \quad  \delta > 0 : |r(tH)| < \varepsilon|t|^2||H||^2 \tag{15} \\
            |t|||H_1|| < \delta \\
            \varepsilon||H_1||^2 \leq \frac{m}{4} \tag{17} \\
        \end{gather*}
        \begin{multline*}
            (5), (11), (15) \implies f(X_0 + tH_1) \geq f(X_0) + \frac{1}{2} m_1t^2 - |r(tH_1)| > f(X_0) + \frac{m_1}{2}t^2 - \\
            - \varepsilon t^2||H_1||^2 \geq f(X_0) + \frac{m}{4}t^2 > f(X_0)  \tag{16} \\
        \end{multline*}

        \begin{multline*}
            (5), (15), (12) \implies f(X_0 + tH) \leq f(X_0) - \frac{1}{2} m_2t^2 + |r(tH_2)| \leq \\
            \leq f(X_0) - \frac{1}{2} m_2t^2 + \varepsilon t^2||H_2||^2 \leq f(X_0) - \frac{1}{2}m_2t^2 + \frac{1}{4}m_2t^2 \leq \\
            \leq f(X_0) - \frac{1}{4}m_2t^2 < f(X_0) \tag{18}
        \end{multline*}
\end{longProof}
\section{Теорема Лагранжа для вектор-функций}
\begin{gather*}
    F: (a,b) \rightarrow \vR^n \quad n \geq 2 \\
    F(X) = \begin{bmatrix}
        f_1(x) \\
        \vdots \\
        f_n(x)
    \end{bmatrix} \tag{1} \\
    x \text{ -- фиксированная точка } \in [a,b] \quad DF(x) = \begin{bmatrix}
        f^\prime_1(x) \\
        \vdots \\
        f^\prime_n(x) 
    \end{bmatrix} = F^\prime(x) \\
    \frac{F(x+h) - F(x)}{h} \underset{h \to 0}{\rightarrow} F^\prime(x) = \begin{bmatrix}
        \frac{f_1(x+h)-f_1(x)}{h} \\
        \vdots \\
        \frac{f_n(x+h) -f_n(x)}{h}
     \end{bmatrix} \tag{2} \end{gather*}
        \[ F: [a,b] \rightarrow \vR^n \]
        \[ \forall k \quad 1 \leq k \leq n \quad \forall x \in (a,b) \exists f^\prime_k(x) \] 
        \[ F \in C([a,b]) \Leftrightarrow f^\prime_k \in C([a,b]) \quad 1 \leq k \leq n \] 
        \[ \implies c \in (a,b) : ||F(b) - F(a)||_{\vR^n} \leq ||F^{\prime}(c)||_{\vR^n}(b-a) \tag{3}\]
\begin{example}[Важный!]
    \begin{gather*}
        x \in [0, 2\pi] \quad
        F(X) = \begin{bmatrix}
            \cos x \\
            \sin x
        \end{bmatrix} \quad F^\prime(x) = \begin{bmatrix}
            -\sin x \\
            \cos x
        \end{bmatrix} \quad ||F^\prime(x)||_{\vR^n} = 1 \\
        F(2\pi) - F(0) = \mathbb{0}_2 \\
        F(b) \ne F(a) \\
        g(x) = \sum^n_{k=1} f_k(x)(f_k(b)-f_k(a)) \tag{4}\\
        g \in C([a,b]) \forall x \in (a,b) \quad \exists g^\prime(x) \\
        g^\prime(x) = \sum^n_{k=1} f^\prime_k(x)(f_k(b) - f_k(a)) \tag{5} \\
        \exists c \in (a,b) : g(b) - g(a) = g^\prime(c)(b-a) \tag{6} 
    \end{gather*}
    \begin{multline*}
        (4) \implies g(b) - g(a) = \sum^n_{k=1} f_k(b)(f_k(b)-f_k(a)) - \sum^n_{k=1}f_k(a)(f_k(b)-f_k(a)) = \\
        = \sum^n_{k=1} (f_k(b) - f_k(a))^2 = ||F(b)-F(a)||^2_{\vR^n} \tag{7}
    \end{multline*}
    \begin{multline*}
        (5) \implies |g^\prime(c)| = \left| \sum^n_{k=1}f^\prime_k(c)(f_k(b)-f_k(a))\right| \leq \\
        \leq \left( \sum^n_{k=1} (f^\prime_k(c))^2 \right)^{\frac{1}{2}} \cdot \left( \sum^n_{k=1}(f_k(b)-f_k(a))^2\right)^{\frac{1}{2}} = \\
        = ||F^\prime(c)||_{\vR^n} \cdot ||F(b) -F(a)||_{\vR^n} \tag{8}
    \end{multline*}
    \[ (6),(7),(8) \implies ||F(b) - F(a)||^2 \leq ||F^\prime(c)|| \cdot ||F(b) - F(a)||(b-a) \implies (3) \] 
\end{example}
\end{document}   
  