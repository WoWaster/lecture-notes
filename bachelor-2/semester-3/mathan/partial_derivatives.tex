% !TeX root = ./main.tex
\documentclass[main]{subfiles}
\begin{document}
\chapter{Частные производные}
\subsection*{Теорема об обратном отображении}
    $E \subset {\vR}^n, n \geq 2$, $x_0 \in E$, $x_0 $ -- внутренняя точка
    $F: E \rightarrow {\vR}^n$
    \[F \in C^{1}(w) \eqno\mathrm{(1)}\]
    \[ \det DF(x_0) \ne 0 \eqno\mathrm{(2)}\]
    Соотношения (1) и (2) влекут:
        \[F(x_0) = Y_0, \exists x_0 \in v \text{ и }\exists y_0 \in V:
        F|_v - \text{гомеоморфизм на } V  \eqno\mathrm{(3)} \]
        \[ \Phi = F^{-1}, \Phi \in C^1(V) \eqno\mathrm{(4)}\]

        \begin{definition}[Якобиан]
        $I(x_0) \stackrel{def}{=} \det DF(x_0)$ -- Якобиан отображения $F$ в точке $x_0$
        \end{definition}

        \begin{gather*}
        \Phi(F(X)) \equiv X \implies D\Phi(Y)DF(X) = DI(X) \implies \\
        Y = F(X), I(X) \equiv X \\
        I_n - \text{ единичная матрица в } {\vR}^n \\
       \implies \det D(\Phi(Y_0)) \cdot \det DF(X_0) = \det I_n = 1
        \end{gather*}
        \begin{longProof}
         \begin{enumerate}
            \item Будем пользоваться определителем матрицы Якоби. Будем обозначать дальше
            $DF(x_0) = A$. Условие (2) влечёт $\exists A^{-1}$
            \[ \text{Будем обозначать }||A^{-1}||  = \frac{1}{4 \lambda}, \lambda> 0
             \tag{5} \]
             Раассмотрим такое линейное отображение
             $x \in w, ||DF(x) - DF(x_0)|| < ||DF(x) - DF(x_0)||_2$
             \begin{gather*}
             (\ldots |f^\prime_{ix_j}(x) - f^\prime_{ix_j}(x_0)|^2 + \ldots)^{\frac{1}{2}} \\
             (1) \implies |f^\prime_{ix_j} - f^\prime_{ix_j}(x_0)|^2 
             \underset{x \to x_0}{\longrightarrow} 0 \tag{6} \\
             (6) \implies ||DF(X) - DF(X_0)||_2 \underset{x \to x_0}{\longrightarrow} 0
             \tag{7} \\
             (7) \implies ||DF(X - DF(X_0)|| \underset{x \to x_0}{\longrightarrow} 0 \tag{8} \\
             (8) \implies \exists r > 0: \forall x, ||X-X_0||_{{\vR}^n} < r
             \text{ имеем } \\
             DF(X) - DF(X_0)|| < 2\lambda \tag{9} 
            \end{gather*}

             \[U = B_r(X_0) = \{ x \in {\vR}^n: ||x-x_0|_{{\vR}^n} < \lambda
             \eqno\mathrm{(10)}\]
             \[DF(X) - A|| < 2 \lambda \eqno\mathrm{(9^\prime)} \]
             \begin{remark}[О внутренности шара]
               \[X_1, X_2 \in B_r(X_0), 0 < t < 1 \implies tX_1 + (1-t)x_2 \in B_r(x_0) \]
               \begin{multline*}
                X_1, X_2 \in B_r(X_0), 0 < t < 1 \implies tX_1 + (1-t)x_2 \in B_r(x_0) \\
                  ||(tX_1 + (1-t)X_2)-X_0||_{{\vR}^n} = \\ 
                  = ||t(X_1-X_0)+ (1-t)(X_2-X_0)||_{{\vR}^n} 
                \leq ||t(X_1 - X_0)||_{{\vR}^n} + \\
                + ||(1-t)(X_2-X_0)||_{{\vR}^n} < 
               < tr + (1-t)r = r
                \end{multline*}
               \end{remark}

             \item Биективность отображения $F$ на $U$
              \[ 0 < t < 1 \]
             \begin{gather*}
            X \in B_r(x_0), H \in {\vR}^n, X + H \in B_r(X_0) \\
             t(X+H) + (1-t)X = X+tH \in B_r(X_0) \\
            g:[0,1] \rightarrow {\vR}^n
             \end{gather*}
            \[g(t) = F(X+tH) - tAH \eqno\mathrm{(11)} \]
            По достаточному условию дифференцируемости все матрицы Якоби существуют
            \begin{align*}
            (11) \implies Dg(t) = g^\prime(t) = D(F(X+tH)) - D(tAH) =\\
            =DF(X+tH)D(t + tH) - AH = \\  
            = DF(x+tH)H-DF(x_0)H = \\
            = (DF(x+tH)-DF(x_0))H \tag{12}
            \end{align*} 
            \begin{gather*}
            \text{Правая часть} (12) \leq ||DF(X+tH)-DF(x_0)|| \cdot ||H||_{{\vR}^n} \\
            \underset{13^\prime}{\leq} \frac{1}{2} ||AH||_{{\vR}^n} \tag{14} \\
            ||A^{-1}|| = \frac{1}{4\lambda} \implies \forall X \in {\vR}^n 
            ||A^{-1}X||_{{\vR}^n} \leq \frac{1}{4\lambda}||X||_{{\vR}^n} 
            \Leftrightarrow \\
            ||X||_{{\vR}^n} \geq 4\lambda||A^{-1}X||_{{\vR}^n} 
            \Leftrightarrow ||AY||_{{\vR}^n} \geq 4\lambda ||Y||_{{\vR}^n} \tag{13} \\
            (13) \Leftrightarrow 2\lambda||Y||_{{\vR}^n} \leq 
            \frac{1}{2}||AY||_{{\vR}^n} \tag{13'} \\
            \intertext{По теореме Лагранжа  ..} \\
            (12), (14) \implies ||g^\prime(t)||_{{\vR}^n} < 
            \frac{1}{2}||AH||_{{\vR}^n} \tag{15} \\
            \exists t_o \in (0,1): ||g(1)-g(0)||_{{\vR}^n} \leq 
            ||g^\prime(t_0)||_{{\vR}^n} \cdot (1-0) = ||g^\prime(t_0)||_{{\vR}^n}\tag{16}\\
             g(1)-g(0) = F(X+H)-AH - F(X) = (F(X+H)-F(X)) - AH \tag{17} \\
            (15), (16), (17) \implies ||(F(X+H)-F(X))-AH||_{{\vR}^n}
            <  \frac{1}{2}||AH||_{{\vR}^n} \tag{18} \\
             (18) \implies ||F(X+H)-F(X)||_{{\vR}^n} \geq ||AH||_{{\vR}^b} -\\
            -||F(X+H)-F(X)-AH||_{{\vR}^n} > ||AH||_{{\vR}^n} -
            \frac{1}{2}||AH||_{{\vR}^n} =  \\
           = \frac{1}{2}||AH||_{{\vR}^n} > 0 \tag{19} \\
            AH = (AH-F(X+H))-F(X) + (F(X_H)-F(X)) \\
            ||F(X+H)-F(X)||_{{\vR}^n} > \frac{1}{2}||AH||_{{\vR}^n} \tag{20} \\
            \intertext{при} 
            X \in B_r(X_0), X+H \in B_r(x_0), H \ne \mathbb{0}_n \\
            (20) \implies F(X+H) \ne F(X) \text{ при } H \ne \mathbb{0}_n \\
            V \stackrel{def}{=} F(U) \tag{21} \\
            \exists \Phi: V \rightarrow U  \tag{22} \\
             \intertext{т.ч.} \Phi = F^{-1} \\
            \end{gather*}
             \item Открытость отображения
             \[(20), (13^\prime) \implies ||F(X+H) - F(X)||_{{\vR}^n} 
             >2\lambda||H||_{{\vR}^n} \tag{23} \]
             \begin{lemma}[]
                $X_1 \in U, Y_1 = F(X_1) $, 
                $0 < \rho < \rho - ||X_1-X_0||_{{\vR}^n} $ 
                Такой выбор влечёт 
                \[\overline{B}_\rho(X_1) \in U Y \in B_{\lambda\rho}(Y_1), Y \neq Y_1 
                \eqno\mathrm{(24)}\]
                \[\implies \exists X \in B_\rho(X_1) : F(X) = Y \eqno\mathrm{(25)} \]
             \end{lemma}
             \begin{longProof} 
                Давайте рассмотрим функцию
                \[ P(X): \overline B_\rho(X_1) \rightarrow {\vR}\]
                \[ P(X) = ||F(X) - Y||_{{\vR}^n}  \tag{26}\]
                Видно, что функция непрерывная,  класса $C^1$, норма это непрерывная 
                функция на замкнутом шаре. Так как замкнутый шар это компакт:
                \begin{gather*}
                \exists X_1 \in \overline{B}_\rho : P(X_) \leq P(X) \forall x \in \overline{B}_\rho(X_1)  
                \tag{27} \\
                X_2 \text{ т.ч. } ||X_2-X_1||_{{\vR}^n} = \rho, H = X_2-X_1 \\
                 (23) \implies ||F(X_2) - F(X_1)||_{{\vR}^n} = ||F(x_1+H)
                 - F(X_1)||_{{\vR}^n} > \\
                 > 2 \lambda ||H||_{{\vR}^n} =2\lambda||X_2-X_1||_{{\vR}^n}  = 2\lambda\rho \tag{28} \\
                  (28), (24) \implies ||F(X_2) -Y||_{{\vR}^n} \geq ||F(X_2) -\\
                 -\underbrace{F(X_1)}_{Y_1}||_{{\vR}^n} - ||\underbrace{F(X_1)}_{Y_1} -
                  Y||_{{\vR}^n} > 2\lambda\rho - \lambda\rho > 2\lambda\rho-\lambda\rho 
                = \lambda\rho \tag{29} \\
                 (29): P(X_2) > \lambda\rho \tag{30} \\
                  P(X_1) = ||F(X_1)-Y||_{{\vR}^n} = ||Y_1-Y||_{|{\vR}^n} < \lambda\rho \tag{31} \\
                 (30), (31) \implies P(X_1) < P(X_2) \tag{32} \\
                 (32) \implies X_- \in B_{\rho}(X_1) \tag{33} \\
                 \intertext{Теперь хотим ввести функцию}  
                 f(X) = P^2(X)  \text{ и получаем, что }
               f(X_-) \leq f(X) \forall X \in \overline{B}\rho(X_1) \tag{34} 
                \end{gather*}
                 \begin{align*}Y = \begin{bmatrix}
                    Y_1 \\
                    \vdots \\
                    Y_n \\
                 \end{bmatrix} &&
               F(X) = 
                 \begin{bmatrix}
                    F_1(X) \\
                    \vdots \\
                    F_n(X) \\
                 \end{bmatrix} \end{align*}
                 $f(X) \geq 0 $
                 ~ обозначили координатные функции ~
                 \begin{gather*}
                 f(X) = \sum_{k=1}^n(F_k(X)-Y_k)^2 \tag{35} \\
               (35) \implies C^1(U)  \tag{36} \\
               (34), (35) \implies f^\prime_{xj} = 0, 1 \leq j \leq n  \tag{37} \\
                 \intertext{(необходимое условие экстремума, согласны ?)}
               (35) \implies f^\prime_{x_j}(X) = 2 \sum_{k=1}^n(F_k(X)-Y_k)F^\prime_{kx_j}(X)
                 \tag{38} \\
                 f_k = F_k(X_-) -Y_k \\
                 (37), (38) \implies \sum_{k=1}^n F^\prime_{kx_j}(X_-)l_k = 0, 1 \leq j \leq n \tag{39} \\
                 L = (e_1, \ldots, e_n) \\
                 (39) \implies LDF(X_-) = \mathbb{0}^T_n \tag{40} \\
                 \intertext{Будем для краткости записи пользоваться обозначеними  из Якобиана}
                 \forall X \in U J_F(X) \neq 0 \tag{41} \\
                 ||A^{-1}|| = \frac{1}{4\lambda} \\
                 \intertext{Хотим обозначить теперь}
                 B = DF(X)  \\
                  \beta = || A - B|| < 2 \lambda
                \intertext{по  теореме из предыдущей лекции 22.09.22 (туть появится ссылка когда лекция от 22 числа появится)}
                ||B^{-1}|| \leq \frac{1}{4\lambda - \beta} < \frac{1}{2\lambda} \tag{42} \\
                \intertext{матрица якоби из (40) обратима, сейчас обратим её}
                 (40), (41) \implies (LDF(X_-))(DF(X_-))^{-1} = \mathbb{0}^T_n(DF(X_-))^{-1} = 
                \mathbb{0}_n \\
                \implies L = \mathbb{0}_n^T \tag{43} \end{gather*}
                \end{longProof}
                $G \subset U$, $G$ -- открытое $\implies F(G)$ открытое.
                 $ \forall Y_1 \in F(G)$, пусть $X_1 \in G, F(X_1) = Y_1$.
                 $ \exists \rho > 0$ т.ч. $B_\rho(X_1) \in G$ и 
                 $\overline{B_\rho(X_1)} \in U$ 
                 по предыдущей лемме получаем соотношение
                 \[ B_{\lambda\rho}(Y_1) \subset F(B_\rho(X_1)) \subset F(G) \]
                 Отображение $F$ действительно является открытым отображением.
                 \[ V =F(U), V - \text{ открытое } , G \subset U, G - \text{ открытое }\]
                 хотим рассмотреть отображение
                 \[ \Phi = F^{-1}; V \rightarrow U \]
                 посмотрим на прообразы открытых множеств $V$.
                 Пусть $\Omega \in V - $ открытое.
                 \[ \Phi^{-1}(G) = F(G) - \text{ открытое}\]
                 Применяем топологическое определение непрерывности
                 \[ \implies \Phi \text{ непрерывна на } V \]
                 Мы выяснили что $F$ биективно, $V$ - открыто, а обратное отображение непрерывно
                   на $V$. Теперь надо проверять что $\Phi$ такой же гладкости как и ...
                   Осталось проверить что обратное отображение класса $C^1$
                  \item \[\forall Y \in V , K \in {\vR}^n , Y + K \in V\] 
                  \begin{gather*}
                  \Phi(Y+K) - \Phi(Y) \stackrel{def}{=} H \text { т.к. отображение } \Phi
                 \text { непрерывно }\\
                  H \underset{K \to \mathbb{0}_n}{\longrightarrow} \mathbb{0}_n 
                  \end{gather*}
                 \begin{gather*}
                     \Phi(Y) = X \text {  } \Phi(Y+K) = X + H \\
                      F(\Phi(Y+K)) = F(X+H) \\
                     Y + K = F(X+H) \text{  } F(\Phi(Y)) = F(X) 
                 \end{gather*}
                 \[   K = (Y+K) - Y = F(X+H) - F(X) \eqno\mathrm{(1)} \]

                 \[DF(X) \text {   } ||(DF(X))^{-1}|| < \frac{1}{2\lambda} \text{ (42) от 29.09} \eqno\mathrm{(2)} \]

                 \[DF(X))^{-1} = B \]
                 \[F(X+H) - F(X) = DF(X)H+ t(H) \eqno\mathrm{(3)} \]

                 \[ \frac{t(H)}{||H||_{{\vR}^n}} \underset{H \to \mathbb{0}_n}{\longrightarrow} 0 \eqno\mathrm{(4)} \]
                  \begin{gather*}
                     (1),(3) \implies \\
                     K = DF(X)H + t(H) \\
                     DF(X)H = K - t(H) \\
                     (BDF(X))H = BK - B(t(H))
                  \end{gather*}

                  \[H = BK-Bt(H) \eqno\mathrm{(5)} \]
                  \[(5) \implies \Phi(Y+K) - \Phi(Y) = BK - Bt(H) \eqno\mathrm{(6)} \]
                  Дифференцируемость почти получилась, потому что есть линейное отображение $BK$...,
                  надо выяснить, что есть соответствующее свойство для дифференирцемости отображения.

                  \[ ||F(X+H)-F(X)||_{{\vR}^n} > 2 \lambda ||H||_{{\vR}^n} \text{ (23) из прошлой лекции } \eqno\mathrm{(7)} \]

                  \[ ||K||_{{\vR}^n} > 2\lambda||H||_{{\vR}^n} \eqno\mathrm{(7^\prime)} \] 
                  \begin{multline*}
                     \frac{||t(H)||_{{\vR}^n}}{||K||_{{\vR}^n}} \leq
                      \frac{||t(H)||_{{\vR}^n}}{||K||_{{\vR}^n}} \stackrel{<}{(2)} 
                      \frac{1}{2\lambda} \frac{||t(H)||_{{\vR}^n}}{||K||_{{\vR}^n}} = \\ 
                      \frac{1}{2\lambda} \frac{||t(H)||_{{\vR}^n}}{||H||_{{\vR}^n}} \cdot
                     \frac{||H||_{{\vR}^n}}{||K||_{{\vR}^n}} \stackrel{<}{(7^\prime)}
                  \end{multline*}

                  \[ < \frac{1}{4\lambda^2} \frac{||t(H)||_{{\vR}^n}}{||H||_{{\vR}^n}}
                  \underset{K \to \mathbb{0}_n}{\longrightarrow} 0 \tag{8} \] 
                  \[ (6), (8) \implies \Phi \text{ дифференцируема в } Y \] 
                  Получаем следующие равенства:

                  \[ D\Phi(Y) = (DF(X))^{-1} \tag{9} \] 
                  \[ \text{ где } Y = F(X) \Leftrightarrow X = \Phi(Y) \] 
                  то, что мы доказали, влечёт следующее: если мы рассмотрим координатные функции $F$,
                  то получится, что существуют все частные производные. осталось проверить их непрерывность
                  \item 
                  \begin{align*}
                     F(X) =
                  \begin{bmatrix*}
                     F_1(X) \\
                     \vdots \\
                     F_n(X) 
                  \end{bmatrix*} &
                  \Phi(Y) = \begin{bmatrix*}
                     \phi_1(Y) \\
                     \vdots \\
                     \phi_n(Y)                     
                  \end{bmatrix*} 
               \end{align*}
               \begin{gather*}
               \begin{bmatrix*}
                  \phi^\prime_{1y_1}(Y) & \ldots & \phi^\prime_{1y_n}(Y) \\
                  \ldots & \ddots & \ldots \\
                   \phi^\prime_{ny_1} & \ldots & \phi^\prime_{ny_n}(Y) \\
               \end{bmatrix*} =
               \left( \begin{bmatrix*}
                  F^\prime_{1x_1}(X) & \ldots & F^\prime_{1x_n} \\
                  \ldots & \ddots & \ldots \\
                  F^\prime_{1x_1}(X) & \ldots & F^\prime_{nx_n}(X)
               \end{bmatrix*} \right)^{-1} \\
               (9) \implies \phi^\prime_{ky_l}(Y) = \frac{\sum \pm F^\prime_{ix_i}(x)
               \cdots F^\prime_{sx_t}(X)}{\underbrace{\sum + \cdots F^\prime_{px_q}(X) \cdots F^\prime{ux_v}(X)}_{\ne 0}}
               \tag{10}\\
                F^\prime_{ix_j}(X) = F^\prime_{ix_j}(\Phi(Y)) \in C(V) \tag{11} \\
               (10), (11) \implies \phi^\prime_{ky_l} \in C(V) \forall k, l \tag{12} \\
               \text{По определению класса } C^r \text{ получаем } 
               (12) \implies \Phi \in C^1(V) \end{gather*}
               \end{enumerate}
        \end{longProof}

        \begin{corollary} %мне кажется, что лучше так, чем с gather
            \begin{align*}
               r \geq 1, F \in C^r(E)\\
                E \in {\vR}^n, n \geq 2, E - \text { открытое множество } \\
                X_0 \in E,  F (X_0) = Y_0,  d_F(X_0) \ne 0 \\
                \implies \exists x_0 \in U \text{ и } \exists Y_0 \in V \text{ т.ч. } 
               F|_{U} - \text{ гомеоморфизм } \\
               \Phi =F^{-1} \\
                \Phi \in C^r(E)
            \end{align*}
        \end{corollary}
        \begin{proof}
            % Доказывать будем по индукции. База верна.
            % \[F \in C^{r+1}(E) \] 
            % Посмотрим на соотношение (10)
            % \[ F^\prime_{ix_j}(\Phi(Y)) \]
            Доказательство было проговорено устно и я ничего не расслышал :(
        \end{proof}
        \begin{corollary}[Теорема об открытом отображении]
            \begin{gather*}
                E \in {\vR}^n, n \geq 2\\
                 E - \text{ открытое }, F: E \rightarrow {\vR}^n \end{gather*}
            \[d_F(X) \ne 0 \forall x \in E \implies F - \text{ открытое отображение} \]
        \end{corollary}
        \begin{proof}
         \begin{gather*}
             w \in E, w - \text{ открытое }, F(w) = G, Y_0 \in G \\
             \exists x_o \in w \text{ т.ч. } F(X_0) = Y_0 \exists r > 0 \text{ т.ч. }
            B_r(X_0) \subset w, d_F(x_0) \ne 0 \text{ т.е. } \\
            \exists (DF(X_0))^{-1} \\
             \lambda > 0, ||(DF(X_0)^{-1})|| = \frac{1}{4\lambda} \\
            \implies B_{\lambda r}(Y_0) \subset F(B_r(X_0)) \tag{шаг 3} \\
             F(B_r(X_0)) \subset F(w)  \end{gather*}

        \end{proof}
      \begin{theorem}[О неявном отображении]
         \begin{align*}
          n \geq 1, m \geq 1 \\
           X = \begin{bmatrix*}
            X_1 \\
            \vdots \\
            X_n
         \end{bmatrix*} &&  Y = \begin{bmatrix*}
            y_1 \\
            \vdots \\
            y_m
         \end{bmatrix*} &&
         Z = \begin{bmatrix*}
            X \\
            Y \\
         \end{bmatrix*} 
      \end{align*}
      \begin{align*}
         % а - обратима
         A = \begin{bmatrix*}
            a_{11} & \ldots & a_{1n} \\
            \ldots & \ddots & \ldots \\
            a_{n1} & \ldots & a_{nn}
         \end{bmatrix*} &&
         B = \begin{bmatrix*}
            b_{11} & \ldots & b_{1m} \\
            \ldots & \ddots & \ldots \\
            b_{m1} & \ldots & b_{mm}
         \end{bmatrix*} 
      \end{align*}
      $A$ - обратима
      \begin{gather*}
          C= [AB]  = \begin{bmatrix*}
            a_{11} & \ldots & a_{1n} & b_{11} & \ldots & b_{1m} \\
            \ldots & \ddots & \ldots & \ddots & \ldots \\
            a_{n1} & \ldots & a_{nn} & b_{n1} & \ldots & b_{nm}
         \end{bmatrix*} \\
       CZ_0 = \mathbb{0}_n \\
       CZ = \mathbb{0}_n \\
       [AB] \cdot \begin{bmatrix*}
         X \\
         Y
      \end{bmatrix*} = \mathbb{0}_n \tag{1} \\
      (1) \Leftrightarrow AX + BY = \mathbb{0}_n \Leftrightarrow AX = -BY \Leftrightarrow X
      = -(A^{-1}B)Y \tag{2}  \\
       X_0 = -(A^{-1}B)Y_0 \tag{2'} 
      \end{gather*}
      \end{theorem}

      \begin{theorem}[О неявном отображении в общем случае]
         \begin{gather*}
         E \subset {\vR}^{n+m}, Z \in E, Z = \begin{bmatrix*}
            X \\
            Y
         \end{bmatrix*},
         X \in {\vR}^n, Y \in {\vR}^m, Z_0 = \begin{bmatrix*}
            X_0 \\
            Y_0
         \end{bmatrix*} \\
          F: E \rightarrow {\vR}^n, F \in C^1(E) \\
         F(Z) = \begin{bmatrix*}
            f_1(Z) \\
            \vdots \\
            f_n(Z)
         \end{bmatrix*} \\
         DF(Z_0) = \begin{bmatrix*}
            f^\prime_{1x_1}(Z_0) & \ldots & f^\prime_{1x_n}(Z_0) & f^\prime_{1y_1}(Z_0) & \ldots & f^\prime_{1y_m}(Z_0) \\
            \ldots & \ddots & \ldots & \ldots \ddots & \ldots \\
            f^\prime_{nx_1}(Z_0) & \ldots & f^\prime_{nx_n}(Z_0) & f^\prime_{ny_1}(Z_0) & \ldots & f^\prime_{ny_m}(Z_0) 
         \end{bmatrix*} 
      \end{gather*}
         \begin{align*}
            A = \begin{bmatrix*}
               f^\prime_{1x_1}(Z_0) & \ldots & f^\prime_{1x_n}(Z_0) \\
               \ldots & \ddots & \ldots \\
               f^\prime_{nx_1}(Z_0) & \ldots & f^\prime_{nx_n}(Z_0)
            \end{bmatrix*} && B =
            \begin{bmatrix*}
               f^\prime_{1y_1}(Z_0) & \ldots & f^\prime_{1y_m}(Z_0) \\
               \ldots & \ddots & \ldots \\
               f^\prime_{my_1}(Z_0) & \ldots & f^\prime_{my_m}(Z_0) \\
            \end{bmatrix*}
         \end{align*}

         \begin{gather*}
          DF(Z_0) = [AB] \\
         A \text{ обратима } \tag{1} \\
          F(Z_0) = \mathbb{0}_n \tag{2} \\
         \exists Y_0 \in W \\
         g: W \rightarrow {\vR}^n, g \in C^1(W) \\
          g(Y_0) = X_0 \text{  } \forall Y \in W \tag{3} \\
         F\left(\begin{bmatrix*}
            g(y) \\
            Y
         \end{bmatrix*}\right)
          = \mathbb{0}_n  \tag{4} \\ 
          F\left(\begin{bmatrix*}
            X \\
            Y
          \end{bmatrix*}\right) = \mathbb{0}_n \tag{2'}
          \end{gather*}
         \end{theorem}

         \begin{longProof}
            \begin{gather*}
          \Phi: E \rightarrow {\vR}^{n+m} \\ 
           \Phi\left(\begin{bmatrix*}
            X \\
            Y \end{bmatrix*}\right) = \begin{bmatrix*}
               F\left(\begin{bmatrix*}
                  X\\
                  Y
               \end{bmatrix*}\right) \\
               Y
            \end{bmatrix*} \tag{5} \\
             \Phi(Z_0) = \begin{bmatrix*}
               F(Z_0) \\
               Y_0
            \end{bmatrix*} = \begin{bmatrix*}
               \mathbb{0}_n \\
               Y_0
            \end{bmatrix*} \tag{6} \\
            D \Phi(Z_0) = \begin{bmatrix*}
               f^\prime_{1x_1}(\ldots) & \ldots & f^\prime_{1x_n} &f^\prime_{1y_1} & \ldots & f^\prime_{1y_m} \\
               \ldots & \ddots & \ldots & \ldots & \ddots & \ldots \\
               f^\prime_{nx_1} & \ldots & f^\prime_{nx_n} & f^\prime_{ny_1} & \ldots & f^\prime_{ny_m} \\
                & & &  \begin{array}{c|c}
                    & \\
                  \hline
                  0 & \begin{matrix}
                     1 & \ldots & 0 \\
                     \vdots & \ddots & \vdots \\
                     0 & \ldots & 1
                  \end{matrix} \\
               \end{array} & &
            \end{bmatrix*}  \\
             def D\Phi(Z_0) = \det A \ne 0 \\
            \text{По теореме об обратном отображении} \\
             \exists Z_0 \in U, \begin{bmatrix*}
               \mathbb{0}_n \\
               Y_0
            \end{bmatrix*} \in V \\
             S \in {\vR}^n, T \in {\vR}^m \\
             \Phi \text{ гомеоморфизм } U \text{ на } V \\
             \text{для} \begin{bmatrix*}
               S\\
               T
            \end{bmatrix*}
             \in V \exists \Psi \left(\begin{bmatrix*}
               S \\
               T
             \end{bmatrix*}\right) - \text{ обратный к} \Phi \tag{7} \\
             \Psi\left(\begin{bmatrix*}
               \mathbb{0}^n \\
               Y_0
             \end{bmatrix*}\right) = Z_0 \tag{8} \\
             \Psi \in C^1(V) \\
              (5) \implies \Psi\left(\begin{bmatrix*}
                 S \\
                 T
             \end{bmatrix*}\right) = \begin{bmatrix*}
               \psi\left(\begin{bmatrix*}
                  S \\
                  T
               \end{bmatrix*}\right) \\
               T
             \end{bmatrix*} \tag{9} \\
             \text{хотим выбрать такой вектор} \\
             P_0 =  \begin{bmatrix*}
               \mathbb{0}_n \\
               Y_0
             \end{bmatrix*} \in V, \exists r > 0 : B_r^{m+n}(P_0) \subset V 
            \end{gather*}
             \begin{align*}
               \forall Y \in B_r^m(Y_0)  && \begin{bmatrix*}
                  \mathbb{0}_n \\
                  Y 
               \end{bmatrix*} \in B_r^{m+n}(P_0)
             \end{align*}

            \begin{gather*}
              W = B_r^m(Y_0) \\
             Хотим определить функцию 
             T \in W, g(T) \stackrel{def}{=} \psi\left(\begin{bmatrix*}
               \mathbb{0}_n \\
               T               
             \end{bmatrix*}\right) \tag{10} \\
             (10) \implies g(Y_0) = \psi\left(\begin{bmatrix*}
               \mathbb{0}_n \\
               Y_0   
             \end{bmatrix*}\right) \\
             \Phi\left(\begin{bmatrix*}
                  g(T) \\
                  T
             \end{bmatrix*}\right) \stackrel{(10)}{=} \Phi\left(\begin{bmatrix*}
               \psi\left(\begin{bmatrix*}
                  \mathbb{0}_n \\
                  T
               \end{bmatrix*}\right)
               T
             \end{bmatrix*}\right) \stackrel{(9)}{=} \Phi \left( \Psi\left(\begin{bmatrix*}
               \mathbb{0}_n \\
               T
             \end{bmatrix*}\right) \right)  = \begin{bmatrix*}
               \mathbb{0}_n \\
               T
             \end{bmatrix*} \tag{11} \\
             (11) \implies F\left(\begin{bmatrix*}
               g(T) \\
               T
             \end{bmatrix*}\right) = \mathbb{0}_n \tag{12} \\
             \Psi\left(\begin{bmatrix*}
                  \mathbb{0}_n \\
                  Y_0
             \end{bmatrix*}\right) = \begin{bmatrix*}
               X_0 \\
               Y_0
             \end{bmatrix*} \stackrel{(9)}{\implies} \psi\left(\begin{bmatrix*}
               \mathbb{0}_n \\
               Y_0
             \end{bmatrix*}\right) = X_0 \Leftrightarrow g(Y_0) = X_0 \tag{13} \\
             \intertext{Предположим, что такая функция не единственная.}
               W \\
               Y_0 \in W_1 \\
               W_2 = W \cap W_1 \\
              \Phi\left(\begin{bmatrix*}
               g(y) \\
               y
             \end{bmatrix*}\right) = \begin{bmatrix*}
               F\left(\begin{bmatrix*}
                  g(y) \\
                  y
               \end{bmatrix*}\right) \\
               Y
             \end{bmatrix*} \stackrel{(12)}{=} \begin{bmatrix*}
               \mathbb{0}_n \\
               Y\\
             \end{bmatrix*} ; 
             \Phi\left(\begin{bmatrix*}
               g_1(Y) \\
               Y
             \end{bmatrix*} \right) \stackrel{(12}{=} \begin{bmatrix*}
               \mathbb{0}_n \\
               X
             \end{bmatrix*} \tag{14}  \\
             (14) \implies g(Y) = g_1(Y) \end{gather*}
      \end{longProof} 
\end{document}