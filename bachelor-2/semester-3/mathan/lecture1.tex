% !TeX root = ./main.tex
\documentclass[main]{subfiles}
\begin{document}
\chapter*{Норма линейного отображения}
\begin{definition}[Линейный оператор]
    $A: \mathbb{R}^m \rightarrow \mathbb{R}^n$ -- линейный оператор, если
    $ \forall x_1, x_2 \in \mathbb{R}^m, \forall p, q \in \mathbb{R}^n \Rightarrow$
    $A(px_1 qx_2) = pA(x_1) + qA(x_2)$
\end{definition}
\begin{remark}
    $A \leftrightarrow \tilde{A}_{m \times n}$
    \newline
    $X =
        \begin{pmatrix}
            x_1 \\
            \vdots \\
            x_m  \\
        \end{pmatrix}$
    \newline 
    $A(x) = \tilde{A}_{m \times n} X$ 
    \newline
    $\tilde{A} = 
    \begin{pmatrix}
        a_{11} & \ldots & a_{1m} \\
        a_{21} & \ldots & a_{2m} \\
        \vdots & \ldots & \ldots \\
        a_{n1} & \ldots & a_{nm} \\
    \end{pmatrix}$
\end{remark}

\begin{definition}[Норма линейного отображения]
    $||A|| \stackrel{def}{=} \underset{x \in \mathbb{R}^m, ||X|_{\mathbb{R}^m \leq 1}}{\sup}
     {||AX||_{\mathbb{R}^n}}$ 
\end{definition}

\begin{displaymath}
    AX = \begin{pmatrix}
        a_{11}x_1 + \ldots a_{1n}x_m \\
        a_{21}x_1 + \ldots a_{2n}x_m \\
        \vdots \\
        a_{n1}x_1 + \ldots a_{nm}x_m
    \end{pmatrix}
\end{displaymath}

$$
    \begin{gathered}
    ||AX||^2_{\mathbb{R}^n} = \sum_{k=1}^n(a_{k1}x_1 + \ldots a_{km}x_n)^2 \leq
    \sum_{k=1}^n (a^2_{k1} + \ldots + a^2_{km})\underbrace{(x_1^2 + \ldots x_m^2)}_{=||X||^2_{\mathbb{R}^n}} \leq \\
    \sum_{k=1}^n (a^2_{k1} + \ldots + a^2_{km}) = \sum_{k=1}^n \sum_{l=1}^m a_{kl}^2 = ||A||_2
\end{gathered} \eqno\mathrm{(2)}
$$ 

$(2) \Rightarrow ||A|| \leq ||A||_2 \geq 0$

\subsection*{Свойства нормы линейного отображения}
\begin{enumerate}
    \item . \begin{theorem}$||A|| \geq 0, ||A|| = 0 \Leftrightarrow A = \mathbb{0}$ \end{theorem}
    \begin{proof}
        Пусть $||A|| = 0$ 
        $A = \begin{pmatrix}
            a_{11} & \ldots & a_{1m} \\
            \vdots & \vdots & \vdots \\
            a_{1n} & \ldots & a_{nm} \\
        \end{pmatrix}$
        
        $1 \leq i \leq n, 1 \leq j \leq m$ не зависят друг от друга.
        \newline
        $e_i = (0, \ldots, \underbrace{1}_i, \ldots 0)$
        $f_j = \begin{pmatrix}
            0 \\
            \vdots \\
            \underbrace{1}_j \\
            \vdots \\
            0 \\ \end{pmatrix}$
        \newline
        $||A|| = 0 \Rightarrow Af_j = \mathbb{0}_{\mathbb{R}^n} $.
        $Af_j =
        \begin{pmatrix}
            a_{1j} \\
            a_{2j}\\
            \vdots \\
            a_{nj}
        \end{pmatrix} $
        Теперь рассмотрим $e_i \underbrace{(Af_j)}_{\mathbb{0}_{\mathbb{R}^n}} =$
        $ (0 \ldots 1 \ldots) \begin{pmatrix}
            0 \\
            \vdots \\
            1 \\
            \vdots \\
            0 \\ \end{pmatrix}$ = $a_{ij} = 0$
    \end{proof}
    \item $A, c \in \mathbb{R}, (cA(x)) \stackrel{def}{=} c(A(x))$ 
    \begin{theorem}$||cA|| = |c| \cdot ||A||$ \end{theorem}
    \begin{proof}
        $$
        \begin{gathered}
        ||cA|| = \underset{X \in \mathbb{R}^m, ||X||_{\mathbb{R}^n}\leq 1}{\sup}||cA(X)||_{\mathbb{R}^n} = \\
        \underset{X \in \mathbb{R}^m, ||X||_{\mathbb{R}^n}\leq 1}{\sup} ||c(AX)||_{\mathbb{R^n}}
        =\underset{X \in \mathbb{R}^m, ||X||_{\mathbb{R}^n}\leq 1}{\sup} |c| \cdot ||AX||_{\mathbb{R}^n}
        \end{gathered} $$
    \end{proof}
    \item $A, B : \mathbb{R}^m \rightarrow \mathbb{R}^n$
        \begin{theorem}
           $ ||A+B|| \leq ||A|| + ||B|| $
        \end{theorem}
        \begin{proof}
            $$
            ||A + B|| = \underset{x \in \mathbb{R}^m, ||X||_{\mathbb{R}^m} \leq 1}{\sup} ||(A+B)X||_{\mathbb{R}^n}
            = \underset{\ldots}{\sup} ||AX + BX||_{\mathbb{R}^n} \leq $$
            $$
            \leq \underbrace{\sup(\underbrace{||AX||}_{\leq ||A||_2} + \underbrace{||BX||}_{\leq ||B||_2})}_{M}
            \leq \underset{\ldots}{\sup} ||AX||_{\mathbb{R}^n} + \underset{\ldots}{\sup} ||BX||_{\mathbb{R}^n}
            $$
            $$
            \begin{gathered}
                \exists x_1 \in \mathbb{R}^m, ||x_1||_{\mathbb{R}^m} \leq 1 \text{и}
                ||Ax_1|| + ||Bx_1|| > M - \varepsilon
            \end{gathered} \eqno\mathrm{(3)}
            $$
            $(3) \Rightarrow M - \varepsilon < ||A|| + ||B| \Rightarrow 
            \underset{x \in \mathbb{R}^m, ||X||_{\mathbb{R}^m}\leq1}{\sup}
            (||AX||_{\mathbb{R}^n} + ||BX||_{\mathbb{R}^n}) \leq ||A|| + ||B||$
        \end{proof}
        \item . \begin{theorem}
           $ ||AX||_{\mathbb{R}^n} \leq ||A|| \cdot ||X||_{\mathbb{R}^m}$
        \end{theorem}
        \begin{proof}
            $$
            \begin{gathered}
            X \neq \mathbb{0}^m \Leftrightarrow ||X||_{\mathbb{R}^m} \stackrel{def}{=} t > 0 \\
            x_0 = \frac{1}{t}x \\
            ||x_0||_{\mathbb{R}^m} =||\frac{1}{t}x||_{\mathbb{R}^m}= \frac{1}{t}||X||_{\mathbb{R}^m}
            = \frac{1}{t} \cdot t = 1 \Rightarrow ||Ax_0||_{\mathbb{R}^n} \leq ||A||
        \end{gathered} \eqno\mathrm{(5)} $$
        $ (5) \Rightarrow ||Ax_0||_{\mathbb{R}^n}=||A \left(\frac{1}{t}x \right)||_{\mathbb{R}^n} $
        $\leq ||A|| \Rightarrow 4.$
        \end{proof}
        \item . \begin{theorem}
            $$c > 0, \forall x \in \mathbb{R}^m$$ $$||Ax||_{\mathbb{R}^n} \leq c||X||_{\mathbb{R}^m} 
            \forall x \in \mathbb{R}^m \eqno\mathrm{(6)} $$ 
            $$\Rightarrow ||A|| \leq c \eqno\mathrm{(6^\prime)}$$
        \end{theorem}
        \begin{proof}
            $$
            \begin{gathered}
            (6) \Rightarrow \text{при } ||X||_{\mathbb{R}^m} < 1 \text{ имеем} \\
            ||AX||_{\mathbb{R}^n} \leq c \cdot ||X||_{\mathbb{R}^m} \leq c \Rightarrow
            \underset{||X|| \leq 1}{\sup} ||AX|| \leq c \Rightarrow (6^\prime)
            \end{gathered}
            $$
        \end{proof}
    \item . \begin{theorem}
        $$E = \{ c \in \mathbb{R}, с > 0 : \forall x \in \mathbb{R}^m \text{ имеем }
        ||Ax||_{\mathbb{R}^n} \leq C ||X||_{\mathbb{R}^m} \} \eqno\mathrm{(7)} $$
        В случае $A \neq 0$
        \newline
        $||A|| = \inf E$, $||A||_2 \in E$, $\inf E \stackrel{def}{=} m$
        $$ ||A|| = \inf E \eqno\mathrm{(8)}$$
    \end{theorem}
    \begin{proof}
        \begin{enumerate}
            \item $ m = 0$
            $$
            \begin{gathered}
                \forall \varepsilon > 0 \exists c_1 \in E : с_1 < \varepsilon \\
                (7) \Rightarrow ||Ax||_{\mathbb{R}^n} \leq c_1||x||_{\mathbb{R}^m}
                \forall x \in \mathbb{R}^m \Rightarrow ||A|| \leq c_1 < \varepsilon \\
                \Rightarrow ||A|| = 0
            \end{gathered}
            $$
            \item $m > 0$
            $$
            \begin{gathered}
                ||A|| \in E, m \leq ||A|| \\
                \text{пусть } m < ||A|| \Rightarrow \exists c_2 : m < c_2 < ||A|| \\
                (7) \Rightarrow ||Ax||_{\mathbb{R}^n} \leq c_2||x||_{\mathbb{R}^m} \forall x \in \mathbb{R}^m
            \end{gathered} \eqno\mathrm{(9)} 
            $$
            $$(9) \Rightarrow ||A|| \leq c_2 \text{ противоречие}$$
        \end{enumerate}
    \end{proof}
    \item . \begin{theorem}
    $$
    \begin{gathered}
        A : \mathbb{R}^m \rightarrow \mathbb{R}^n, B: \mathbb{R}^n \rightarrow \mathbb{R}^k \\
        L: \mathbb{R}^m \rightarrow \mathbb{R}^k \\
        Lx = B(A(x)) \\ 
        \text{каждая норма соответствует своей паре пространств!} \\
        ||L|| \leq ||A|| \cdot ||B||
    \end{gathered} \eqno\mathrm{(10)}
    $$
    \end{theorem}
    \begin{proof}
        $$
        \begin{gathered}
            \forall x \in \mathbb{R}^m \text{  }||LX||_{\mathbb{R}^k} = ||B(AX)||_{\mathbb{R}^k}
        \leq ||B|| \cdot ||AX||_{\mathbb{R}^n} \\
        \leq \underbrace{||B|| \cdot ||A||}_{c}
         \cdot ||X||_{\mathbb{R}^m}
        \end{gathered}
        \eqno\mathrm{(11)}
        $$
        по свойству 5 $(11) \Rightarrow (10)$
    \end{proof}
\end{enumerate}

\subsection*{Дифференцируемость суперпозиции линейных отображений}
    $$
    \begin{gathered}
        \Omega \subset \mathbb{R}^m, m \geq 1 \\
        X_o \in \Omega - \text{ внутрення точка}\\
        G \in \mathbb{R}^n, Y_0 \in G, Y_0 - \text{ внутренняя точка в } G \\
        F = \begin{pmatrix}
            f_1 \\
            \vdots \\
            f_n
        \end{pmatrix} : \Omega \rightarrow \mathbb{R}^n \text{ }\forall x \in \Omega 
        \text{ }F(x) \in G, F(X_0) = Y_0 \\
        \Phi = \begin{pmatrix}
            \varphi_1(y) \\
            \vdots \\
            \varphi_k(y)
        \end{pmatrix} : G \rightarrow \mathbb{R^k} 
    \end{gathered} $$
    $$\exists P: \Omega \rightarrow \mathbb{R}^k, P(x) \stackrel{def}{=} \Phi(F(x))
     eqno\mathrm{(15)} $$


    \begin{theorem}
        $$
        \begin{gathered}
        F \text{ дифференцируема в } X_0, \Phi \text{ дифференцируема в } X_0 \\
        \Rightarrow P \text{ дифференцируема в } X_0 \text{ и } \\
        DP(X_0) = D\Phi(X_0) \cdot DF(X_0) \text{( это матрицы Якоби)}
        \end{gathered} \eqno\mathrm{(16)}
        $$ 
    \end{theorem}

    \begin{proof}
        $$
        \begin{gathered}
            \Phi(Y_0 + \lambda) - \Phi(Y_0) = B \lambda + \rho(\lambda)
        \end{gathered} \eqno\mathrm{(17)}
        $$
        $$
        B = D\Phi(Y_0), \rho(\lambda) \in \mathbb{R}^k \text{ и }
         \frac{||\rho(\lambda)||_{\mathbb{R}^k}}{||\lambda||_{\mathbb{R}^n}} 
         \underset{\lambda \to \mathbb{0}_n}{\longrightarrow} 0 \eqno{\mathrm{(18)}}
        $$
        $$ 
        \lambda = \mathbb{0}_n, \Phi(Y_0) - \Phi(Y_0) = \mathbb{0}_k + \rho(\mathbb{0}_n)
        \Rightarrow \rho(\mathbb{0}_n) = \mathbb{0}_k$$
        $$ \forall \eta > 0 \exists \delta_1 > 0 :$$
        $$ (18) \Rightarrow \forall \lambda \in \mathbb{R}^n, \lambda \neq \mathbb{0}_n
        \text{ и } ||\lambda||_{\mathbb{R}_n} <\delta_1 \text{ будет }
        \frac{||\rho(\lambda)||_{\mathbb{R}_k}}{||\lambda||_{\mathbb{R}^n}} < \eta
        \eqno\mathrm{(19)}$$ и
        $$ ||\rho(\lambda)||_{\mathbb{R}^k} \leq \eta \cdot ||\lambda||_{\mathbb{R}^n}  \eqno\mathrm{(20)} $$
        $$ (19) \Rightarrow ||\rho(\lambda)||_{\mathbb{R}^k} \leq \eta \cdot ||\lambda||_{\mathbb{R}^n} 
        \eqno\mathrm{(19^\prime)}$$
        $$ \forall \varepsilon > 0 \exists \delta_2 > 0 : \forall H \in \mathbb{R}^m,
        ||H||_{\mathbb{R}^m} \delta_2  \text{ имеем }$$
        $$ ||r(H)||_{\mathbb{R}^n} \leq \varepsilon||H||_{\mathbb{R}^m}
        \eqno\mathrm{(21)} $$
        $$ H \in \mathbb{R}^m, x_0 + H \in \Omega \text{ возможно т.к. внутрення точка} $$
        $$F(X_0) = Y_0 $$
        $$\text{определим } F(x_0 + H) - F(x_0) = \lambda \eqno\mathrm{(22)} $$
        $$F(x_0+H) - Y_0 = \lambda \eqno\mathrm{(22^\prime)} $$
        $$
        \begin{gathered}
            P(x_0+H) - P(x_0) = \Phi(F(x_0 + H)) - \Phi(F(x_0)) \stackrel{(22^\prime)}{=} \\
            \Phi(Y_0 + \lambda) - \Phi(Y_0) \\
             \stackrel{(17)}{=} B\lambda + \rho(\lambda) \stackrel{(12),(22)}{=} \\
             B(AH+r(H)) + \rho(AH + r(H)) = \\
        \end{gathered} \eqno\mathrm{(23^\prime)}
        $$
        $$ = \overbrace{(BA)}^{P(x_0+H)-P(x_0)}H + \overbrace{Br(H) + \rho(AH+r(H))}^{r_1(H)} \eqno\mathrm{(23^\prime)}$$
        $$ ||H||_{\mathbb{R}^m} < \delta_2 \Rightarrow ||r(H)||_{\mathbb{R}^m} \leq \varepsilon ||H||_{\mathbb{R}^m}
        \eqno\mathrm{{23}}$$
        $$ \delta = \varepsilon \text{ } \exists \delta_1 : \text{ выполнено (20) при }
        ||\lambda||_{\mathbb{R}^n} \delta_1 $$
        $$
        \begin{gathered} ||\lambda||_{\mathbb{R}^n} = ||AH + r(H)||_{\mathbb{R}^n} \leq ||AH||_{\mathbb{R}^n}
        + ||r(H)||_{\mathbb{R}^n} \\
        \stackrel{(22)}{\leq} ||A|| \cdot ||H||_{\mathbb{R}^m} + \varepsilon \cdot ||H||_{\mathbb{R}^m}
        =(||A||+\varepsilon)||H||_{\mathbb{R}^m}$$ <\delta_1$$
        то есть $$||H||_{\mathbb{R}^m} \leq \frac{\delta_1}{||A||+\varepsilon}
        \end{gathered} \eqno\mathrm{(24)}
        $$

        $$\delta_0 = \min(\delta_2, \frac{\delta_1}{||A|| + \varepsilon}) \eqno\mathrm{(25)}$$
        и полагаем $||H||_{\mathbb{R}^n} < \delta_0$

        $$
        \begin{gathered}
            \text{При } ||H||_{\mathbb{R}^m} < \delta_0 (26) \Rightarrow ||r_1(H)||_{\mathbb{R}^k} \\ 
            \leq ||Br(H)||_{\mathbb{R}^k} + ||\rho(AH+r(H))||_{\mathbb{R}^k} \\
            \leq ||B|| \cdot ||r(H)||_{\mathbb{R}^n} + \varepsilon||AH+r(H)||_{\mathbb{R}^n} \\
            \stackrel{(20), (23)}{\leq} ||B|| \cdot \varepsilon||H||_{\mathbb{R}^m} +
            \varepsilon(||AH||_{\mathbb{R}^n}+||r(H)||_{\mathbb{R}^n}) \leq \\
        \end{gathered}
        $$
        $$ \leq ||B||\varepsilon||H||_{\mathbb{R}^m} + \varepsilon(||A|| \cdot ||H||_{\mathbb{R}^m}
        +\varepsilon||H||_{\mathbb{R}^m}) = \varepsilon(||B|| + ||A|| + 
        \varepsilon)||H||_{\mathbb{R}^m} \eqno\mathrm{(22)}
        $$
        $$P(x_0)+H - P(X_0) = (BA)H + r_1(H) \eqno\mathrm{(27^\prime)} $$
        $$ \text{При } ||H||_{\mathbb{R}^n} < \delta_0 \text{ имеем}
        ||r_1(H)||_{\mathbb{R}^k} \leq \varepsilon(||B|| + ||A|| + \varepsilon)||H||_{\mathbb{R}^m} 
        \eqno\mathrm{(27^{\prime\prime})}$$ 
        $$ (27^{\prime\prime}) \Rightarrow \frac{||r_1(H)||_{\mathbb{R}^k}}{||H||_{\mathbb{R}^m}}
        \underset{H \to \mathbb{0}_m}{\longrightarrow} 0 \eqno\mathrm{(28)} $$
        $(23^\prime), (27^\prime), (28) \Rightarrow (16)$
    \end{proof}
\end{document}