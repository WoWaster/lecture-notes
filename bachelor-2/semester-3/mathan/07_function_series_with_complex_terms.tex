% !TeX root = ./main.tex
\documentclass[main]{subfiles}
\begin{document}
\chapter{Числовые и функциональные ряды с комплексными слагаемыми}
\section*{Числовые ряды}
\begin{gather*}
    i^2 = -1 \quad c = a + bi \quad a,b \in \vR \\
    \mathbb{C}\\
    \overline{c} = a -bi \\
    |c| = \sqrt{a^2 + b^2}
\end{gather*}
\begin{definition}[Ряд комплексных слагаемых]
     Cимвол \begin{gather*}
        \sum_{n=1}^\infty c_n , c_n \in \mathbb{C} \quad c_n = a_n + ib_n \tag{1}
    \end{gather*}
\end{definition}

\begin{definition}[Сходимость]
    \begin{gather*}
    \sum^\infty_{n=1} a_n \tag{2}\\
    \sum^\infty_{n=1} b_n \tag{3}\\
    \text{Будем говорить, что ряд (1) сходится, если сходятся (1) и (2)} \\
    \sum^\infty_{n=1} c_n \stackrel{def}{=} \sum^\infty_{n=1} a_n + i \sum^\infty_{n=1} b_n \tag{4}\\
    \end{gather*}
\end{definition}
Если сходится ряд из $c_n$, то сходится и ряд из комплексно сопряжённых. 

\begin{gather*}
    p \in \mathbb{C} \quad p \ne 0 \quad p = s + it \\
    \sum^\infty_{n=1} pc_n = p \sum_{n=1}^\infty c_n \tag{5}\\
    pc_n = (s+it)(a_n + ib_n) = (sa_n-tb_n) + i(sb_n+ ta_n) 
\end{gather*}
\begin{definition}[Абсолютная сходимость]
    Ряд (1) сходится абсолютно, если сходится 
    \[\sum_{n=1}^\infty |c_n| < \infty \tag{6} \]
\end{definition}
\begin{theorem}
    Ряд (1) сходится тогда и только когда, когда сходятся ряды (2) и (3)
\end{theorem}
% \begin{proof}
%     $\implies$ \\
%     По признаку сравнений рядов
%     \begin{gather*}
%         |a_n| \leq |c_n| \quad |b_n| \leq |c_n| \\
%         |c_n| \leq |a_n| + |b_n| \\
%         \Leftarrow \text{ очевидно }
%     \end{gather*}
% \end{proof}
\begin{corollary}
    Ряд (1) сходится, если он сходится абсолютно.
\end{corollary}

\section*{Функциональные ряды}
\begin{definition}[Комплексно-значная функция]
    Будем говорить, что на множестве $E \ne \varnothing$ задана комплексно-значная функция, если
    \begin{gather*}
        u(x) \quad v(x) :  E \rightarrow \vR \\
        w(x) = u(x) + i v(x)
    \end{gather*}
\end{definition}

\begin{gather*}
    x \in E \quad \{ w_n(x) \}^\infty_{n=1}  \quad w_n(x) \in \mathbb{C}\\
    w_n(x) = u_n(x) + i v_n(x) \\
\end{gather*}
\begin{definition}[Равномерное стремление]
    Будем говорить, что $w_n$ равномерно стремится к $\phi(x)$
    \[ w_n(x) \darrow{\underset{x \in E}{n \to \infty}} \phi(x) \] если
    \begin{gather*}
        u_n(x) \darrow{\underset{x \in E}{n \to \infty}} \alpha(x) \text{ и }
        v_n(x) \darrow{\underset{x \in E}{n \to \infty}} \beta(x) \tag{7}\\
    \end{gather*}
\end{definition}
\begin{theorem}
    \begin{gather*}
        (7) \Leftrightarrow \quad \forall \varepsilon : \forall n > N \text{ и  } \forall x \in E \\
        |w_n(x) - \phi(x) | < \varepsilon \tag{8}
    \end{gather*}
\end{theorem}
\begin{longProof}
    $\Leftarrow$ проводится точно так же, как в предыдущей теореме\\
    $ \Rightarrow$ Если выполнено (7), то для  определения равномерной сходимости:
    \begin{gather*}
        \exists N_1 \text{ для } \{u_n(x) \}^\infty_{n=1} \\
        \exists N_2 \text{ для } \{v_n(x) \}^\infty_{n=1} \\
        N = \max{N_1, N_2} \\
        |w_n(x) - \phi(x)| \leq \underbrace{| u_n(x) - \alpha(x) |}_{< \frac{\varepsilon}{2}} + \underbrace{|v_n(x) - \beta(x)|}_{< \frac{\varepsilon}{2}}
    \end{gather*}
\end{longProof}

\begin{gather*}
    x \in E \quad \{c_n(x) \}^\infty_{n=1} \quad c_n(x) \in \mathbb{C} \\
    \sum^\infty_{n=1} c_n(x) \tag{9}\\
    S_n(x) = \sum^n_{k=1} c_k(x) 
\end{gather*}
\begin{definition}[Равномерная сходимость]
    Будем говорить, что ряд (9) сходится равномерно на множестве $E$, если
    \begin{gather*}
        \exists \phi(x) \quad S_n(x) \darrow{\underset{x \to E}{n \to \infty}} \phi(x) \tag{10}\\
    \end{gather*}
\end{definition}
\begin{theorem}[Признак Вейерштрасса равномерной сходимости функционального ряда с комплексными слагаемыми]
    Имеется функциональный ряд (9) с комплексными слагаемыми. 
    \begin{gather*}
        \text{ Имеется } A_N \geq 0 \\
        \sum^\infty_{n=1} A_n < \infty \tag{11}\\
        \forall n \text{ и } \forall x \in E |c_n(x)| \leq A_n \tag{12} \\
        \text{тогда ряд (9) сходится равномерно}
    \end{gather*}
\end{theorem}

\begin{longProof}
    \begin{gather*}
        \forall n \text{ и } \forall x \in E \quad |c_n(x)| \leq A_N \tag{12} \\
        c_n(x) = a_n(x) = ib_n(x) \\
        (12) \implies |a_n(x) | \leq A_N \quad \forall n \text{ и } \forall x \in E \tag{13}\\
        (12) \implies |b_n(x) | \leq A_N \quad \forall n \text{ и } \forall x \in E \tag{14}\\
    \end{gather*}
    (13), (14) с вещественными слагаемыми, (11) и (13) $\implies$
    \begin{gather*}
        \sum^\infty_{n=1} a_n(x) \text{ сходится равномерно на } E \tag{15}\\
        \sum^\infty_{n=1} b_n(x) \text{ сходится равномерно на } E \tag{16}\\
        s=S_n(x) = \sum^n_{k=1} a_n(x) + i \sum^n_{k=1} b_k(x)
    \end{gather*}
    Поскольку частичные суммы выглядят таким образом, то по определению равномерной сходимости ряда, $a_n$
    равномерно стремятся к некоторой функции. Суммы мнимых частей $b_n$ тоже равномерно стремятся к какой-то функции.
    По определению равномерной сходимости получаем, что требовалось доказать.
\end{longProof} 
\begin{gather*}
    \alpha \in \mathbb{C} \quad \{ c_n \} ^\infty_{n=0} \\
    c_n \in \mathbb{C} \\
    z = x + iy \\
    z - \alpha
\end{gather*}
\begin{definition}[Степенной ряд]
    \[ c_0 + \sum^\infty_{n=1} c_n(z-\alpha)^n \tag{17} \]
\end{definition}
\begin{lemma}[Абеля]
    Предположим, что ряд (17) сходится. При фиксированном $z_0 \ne \alpha$ это ряд с комплексными 
    слагаемыми.
    \begin{gather*}
        R = |z_0 - \alpha| > 0 \\
        \forall z : |z-\alpha| < R \tag{18}\\
        \text{ фиксируем } 0 < r < R  \\
        \text{ при } |z-\alpha| \leq r \text{ ряд сходится равномерно }
    \end{gather*} 
\end{lemma}
Пока отвлечемся от доказательства, рассмотрим ряд.
    \begin{gather*}
        w_n = p_n + iq_n \\
        \sum^\infty_{n=1} w_n \text{ сходится } и
        \sum^\infty_{n=1} q_n \text{ сходится } \\
        p_n \underset{n \to \infty}{\longrightarrow} 0 \implies \exists M_1 : |p_n| \leq M_1 \forall n \\
        q_n \underset{n \to \infty}{\longrightarrow} 0 \implies \exists M_2 : |q_n| \leq M_2 \forall n
    \end{gather*}
    Провели такое рассуждение, показав, что сумма ограниченна
\begin{longProof}
    Напомню свойство комплексных чисел  
    \[|z_1| \cdot |z_2| = |z_1 \cdot z_2| \]
    В условии сказано, что 
    \begin{gather*}
        c_0 + \sum^\infty_{n=1} c_n(z-z_0)^n \text{ сходится } \implies\\
         \text{по предыдущему рассуждению} \exists M : \forall n \\
        |c_n(z_0 - \alpha)^n | \leq M \tag{20} \\
        (20) : |c_n| \cdot |z_0 - \alpha|^n \leq M \Leftrightarrow |c_n| \leq \frac{M}{R^n} \tag{21}\\
        0 < \frac{|z-\alpha|}{R} = q < 1 \\
        (21) \implies |c_n(z-alpha)^n | = |c_n| \cdot |z-\alpha|^n \leq \frac{M}{R^n} \cdot |z-\alpha|^n = Mq^n \tag{22}\\
        \sum^\infty_{n=1} Mq^n = \frac{Mq}{1-q} \\
    \end{gather*}
    Получаем, что ряд (17) сходится абсолютно, значит, он просто сходится
    \begin{gather*}
        \frac{r}{R} = q_0 < 1
        \text{Предположим, что выполнено(19)} \\
        (19),(21) \implies |c_n(z-\alpha)^n| < |c_n| \cdot |z-\alpha|^n \leq \frac{M}{R^n} \cdot r^n = Mq_0^n \tag{23}\\
    \end{gather*}
    Ряд из $Mq_0^n$ сходится так как $q_0< 1$.  $q_0$ не зависит от $z$, 
    если выполнено соотношение (19). Значит мы получаем, что сходимость ряда равномерная.
\end{longProof}

\section*{Радиус сходимости и круг сходимости степенного ряда}
   \[ c_0 + \sum^\infty_{n=1} c_n(z-\alpha)^n  \quad R \quad B\tag{1} \]
\begin{enumerate}
    \item \begin{gather*}
        (1) \text{ сходится лишь при } z = \alpha \\
        R \stackrel{def}{=} 0 \quad B \stackrel{def}{=} \varnothing
    \end{gather*}
    \item \begin{gather*}
        (1) \text{ сходится при } \forall z \in \mathbb{C} \\
        R \stackrel{def}{=} +\infty \quad B \stackrel{def}{=} \mathbb{C}
    \end{gather*}
    \item \begin{gather*}
        \exists z_1 \ne \alpha : (1) \text{ сходится  в } z_1 \text{ и } \exists z_2 : (1) \text{ не сходится в } z_2 \\
        R_2 = |z_2 - \alpha| \implies \forall z : |z-\alpha| > R_2 \quad (1) \text{ расходится } \\
        \text{ т. ч. } E = \{ |z-\alpha|: (1) \text{ сходится в } z \} \\
        |z_1 - \alpha| \in E \quad \text{ и } E \text{ ограниченно сверху} \quad \forall \rho \in E \quad \rho \leq |z_2 - \alpha|
    \end {gather*}
\end{enumerate}
Понятно, что (3) дополняет (1) и (2)
\begin{definition}
    \[ R \stackrel{def}{=} \sup E \quad B \stackrel{def}{=} \{ z: |z-\alpha| < R \} \]
\end{definition}
\subsection*{Свойства сходимости} 
\begin{theorem}
    \begin{gather*}
        \forall z \in B \quad (1) \text{ сходится в } z \tag{2} \\
        \forall z \in \overline{B} \text{ замыкание } \quad (1) \text{ расходится в } z \tag{3} 
    \end{gather*}
\end{theorem}

\begin{longProof}
    \begin{gather*}
        \rho = |z-\alpha| \quad \rho < R \implies \exists \rho_1, \rho < \rho_1 < R \tag{4\prime}\\
        \rho_1 \in E \tag{4} \\
        \text{ т.е. } \exists z_1 : |z_1-\alpha| = \rho+1 \tag{5} \\
        \text{ и } (1) \text{ сходится в } z_1 \tag{6} 
    \end{gather*} 
    В точке $z_1$ применима лемма Абеля.
    \begin{gather*}
        (4\prime), (4), (5),(6) \implies (2) \\
        z_0 \notin \overline{B} \Leftrightarrow |z_0 - \alpha| > R \\
        \text{ если бы (1) сходится в } z_0
        \forall z : |z-\alpha| < |z_0 - \alpha| \quad (1) \text{ сходится в } z \\
        \tilde{z} : |\tilde{z} - \alpha| = \frac{1}{2} (R + |z_0 - \alpha|) \quad |\tilde{z} - \alpha| < |z_0 - \alpha| \quad (1) \text{ сходится в } \tilde{z} \\
        |\tilde{z} - \alpha| > R \quad |\tilde{z} - \alpha | \in E
    \end{gather*}
    Получили противоречие с определением супремума
\end{longProof}

\subsection*{Непрерывность}
\begin{theorem}
    \begin{gather*}
        0 < r < R \\
        \{ z: |z-\alpha| \leq r \} \quad  \quad r < \rho < R \quad |z_0 - \alpha| = \rho \\
        z = x + iy \Leftrightarrow (x,y)  \quad \quad z - \alpha = x - \beta + i(y - \gamma)\\
        |z| = ||(x,y)|| \quad \quad \alpha = \beta + i\gamma \\
        c_0 + \sum^\infty_{n=1} u_n(z) + i \sum^\infty_{n=1} v_n(z)
    \end{gather*}
    Это ряды из непрерывных функций, поэтому их сумма тоже непрерывная функция
\end{theorem}

\begin{definition}[Непрерывность]
    \[w(z) = A(Z) + iB(z) \] будем называть непрерывной, если непрерывны мнимая и вещественная часть
\end{definition}

(1) непрерывна на $B$ \\
Получено следующее утверждение: сумма степенного ряда по степеням $z-\alpha$ непрерывна в круге сходимости.

\subsection*{Вычисление радиуса сходимости степенного ряда}
Имеется ряд (1).
    \[t = \overline{\underset{n \to \infty}{\lim}} \sqrt[n]{|a_n|} \tag{7} \]
Он может быть равен $+\infty$ \\
Теперь хотим определить \[R_0 = \begin{cases}
    0 \quad \text{ если } t = +\infty \\
    +\infty \quad \text{ если } t = 0  \\
    \frac{1}{t} \quad \text{ если } 0 < t < \infty
\end{cases} \tag{8} \]
\[ R_0 = R \tag{9} \]
\begin{theorem}
    $R_0$, опредленное в соотношении (8), является радиусом сходимости
\end{theorem}
\begin{longProof}
    Надо рассмотреть 3 случая. Случаи $0$ и $+\infty$ мы рассматривать не будем, потому что это они простые. Рассмотрим
    более интересный случай
    \begin{gather*}
        |z_0 - \alpha| > R_0 \tag{10} \\
        \varepsilon = |z_0 - \alpha| - R_0 > 0 \\
        \delta_0 = \frac{t^2 \varepsilon}{1-t\varepsilon} \tag{11}
    \end{gather*}
    В силу свойств верхнего предела (это было еще в далёком первом семестре)
    \begin{gather*}
        \exists \{ n_k \}^\infty_{k=1} : \sqrt[n_k]{|c_{n_k}|} > t - \delta_0 \tag{12} \\
        (12) \Leftrightarrow |c_{n_k}| > (t-\delta_0)^{n_k} \tag{12\prime} \\
        (12^\prime) \implies |c_{n_k} (z_0-\alpha)^{n_k} | > (t-\delta_0)^{n_k} \cdot (R_0 + \varepsilon)^{n_k} = \\
       = ((t-\delta_0)(R_0 + \varepsilon))^{n_k}
    \end{gather*}
    Посчитаем теперь отдельно выражение в скобке
    \begin{multline*}
        (t-\delta_0)(R_0+ \varepsilon) =  \left (t- \frac{t^2\varepsilon}{1+t\varepsilon}\right ) \left (\frac{1}{t} + \varepsilon  \right ) = \\
        = \frac{t+t^2\varepsilon-t^2\varepsilon}{1+t\varepsilon} \cdot \frac{1 + t \varepsilon}{t} = 1\\
        (13) \implies c_{n_k} (z_0 - \alpha)^{n_k} \text{ не стремится к 0 при } k \to \infty
    \end{multline*}
    Ряд в точке $z_0$ разошёлся

\begin{gather*}
    |z_r - \alpha| < R_0 \quad \varepsilon = R_0 - |z_1 - \alpha| > 0 \\
    \delta_1 = \frac{1}{z} \cdot \frac{t^2 \varepsilon}{1 - t \varepsilon}
\end{gather*}
Опять же по свойствам верхнего предела из первого семестра. 
\begin{gather*}
    \exists N_1 : \forall n > N_1 \quad \sqrt[n]{|c_n|}  < t + \delta_1 \tag{14} \\
    (14) \Leftrightarrow |c_n| < (t+ \delta_1)^n \tag{14\prime} \\
    (14\prime) \implies |c_n| |z_1 - \alpha|^n < (t+\delta_1)^n (R_0 - \varepsilon) =
    ((t+\delta_1)(R_0 - \varepsilon))^n \tag{15} \\
    (t+2\delta_1)(R_0 - \varepsilon) = \left ( t + \frac{t^2\varepsilon}{1-t\varepsilon} \right ) \left ( \frac{1}{t} - \varepsilon \right ) =
    \frac{t - t^2\varepsilon + t^2\varepsilon}{1-t\varepsilon} \cdot \frac{1 -t \varepsilon}{t} = 1 \implies \\
    \implies (t+\delta_1)(R_0 - \varepsilon) = 1 - \delta_1(R_0 - \varepsilon) \stackrel{def}{=} q < 1 \tag{16}
\end{gather*} 
$q$ не зависит от $n$, поэтому
\begin{gather*}
    (15), (16) \implies |c_n(z-\alpha)^n| < q^n \\
    \sum_1^\infty q^n = \frac{1}{1-q} \\
    \intertext{получим}
    \forall z : |z-\alpha| > R_0 \quad  (1) \text{ расходится } \\
    \forall z : |z-\alpha| < R_0 (1) \quad \text{ сходится } \\
    (18) \implies R = R_0
\end{gather*}
Если бы $R \ne R_0$, то один из них больше другого. Возьмём $z$ между ними. 
Тогда посмотрим на радиус сходимости $R$ в точке $z$, там
будет расходимость, хотя должна быть сходимость.
\end{longProof}

\section*{Вещественные степенные ряды} 

\[ \vR \leftrightarrow x + i0 \]
\begin{definition}[Вещественный степенной ряд]
    \begin{gather*}
        c_n \in \vR, n \geq 0 \\
        с_0 + \sum^\infty_{n=1} c_n(x-\alpha)^n \quad x \in \vR \tag{1} \\
    \end{gather*}
\end{definition}

Рассмотрим
\begin{gather*}
    (2) c_0 + \sum^\infty_{n=1}  c_n (z-\alpha)^n \quad z \in \mathbb{C} \\
    R, B \text{ -- круг сходимости } (2) \\
    B = \{ z: |z-\alpha| < R  \} \quad R > 0 \\
    I = B \cap \vR = (\alpha - R, \alpha + R) \tag{3}
\end{gather*}

\begin{definition}[Интервал сходимости]
    $I$, которое может быть пустым, называется интервалом сходимости вещественного ряда (1).
\end{definition}

\begin{theorem*}
    Он обладает следующими свойствами: \\
  если $x_0 \in I \implies (1) $ сходится в $x_0$
    если $x_1 \notin \overline{I}  \implies (1) $ расходится в $x_1$
\end{theorem*}


\begin{proof}
    \[ x_0 \in I \implies x_0 \in B \implies \text{ ряд (2) сходится в }  x_0 \Leftrightarrow (1) \text{ сходится в } x_0 \] 
    \[ x_1 \notin \overline{I} \Leftrightarrow |x_1 - \alpha| > R \Leftrightarrow x_1 \notin \overline{B} \implies (3) \text{ расходится в } x_1  \] 
\end{proof}

\begin{gather*}
    R = \frac{1}{t} \\
    t = \overline{\underset{n \to \infty}{\lim}} \sqrt[n]{|c_n|}
\end{gather*}
Есть еще одна формула для радиуса сходимости.
\begin{theorem*}
    \begin{gather*}
        c_1 \in \mathbb{C} \quad c_n \ne \forall n \\
        \exists \underset{n \to \infty}{\lim} \frac{|c_n|}{|c_{n+1}} = R
    \end{gather*}
\end{theorem*}
\begin{proof}
        Доказательство аналогично. Тут случай проще потому что предполагается,
         что предел существует.
\end{proof}

\begin{theorem}
    \begin{gather*}
        \forall r \quad 0 < r < R \quad (1) \text{ сходится равномерно при } |x-\alpha| \leq r \implies \\
        \implies c_0 + \sum^\infty_{n=1} c_n(x-\alpha)^n \in C(I) \quad I = [\alpha - r, \alpha + r] 
    \end{gather*}
\end{theorem}

\begin{theorem}[Абеля]
    \begin{gather*}
        (1) \text{ сходится в } \alpha - R \implies (1) \text{ равномерно сходится на } [\alpha - R,\alpha]  \quad \\
        \text{ сумма (1) } \in C([\alpha - R,R]) \\
        (1) \text{ сходится в } \alpha + R \implies (1) \text{ равномерно сходится на } [\alpha,\alpha+R]  \quad \\
        \text{ сумма (1) } \in C([\alpha , \alpha +R])
    \end{gather*}
\end{theorem}

\begin{longProof}
    Оба случая доказываются аналогично, рассмотрим первый. $c_0$ и так добавляется, его писать не будем
    \begin{gather*}
        \sum^\infty_{n=1} c_n((\alpha-R)-\alpha)^n \text{ сходится } \\
        \sum^\infty_{n=1} (-1)^n c_n R^n \text{ сходится } \tag{5} \\
        \alpha - x > 0 \quad \quad \alpha - R < x < \alpha \\
        c_n(x-\alpha)^n = (-1)^n c_n (\alpha - x)^n = (-1)^n c_n R^n \cdot \left ( \frac{\alpha-x}{R} \right ) \tag{6}
    \end{gather*}
    Хотим применить признак Абеля равномерной сходимости функциональных рядов \\
    \begin{gather*}
        u_n(x) = (-1)^n c_n R^n \\
        \sum^\infty_{n=1} u_n(x) v_n(x) \\
        \sum^\infty_1 u_n(x) \text{ сходится равномерно } \\
         v_n(x) \text{ монотонна } \forall x \\
         |v_n(x)| \leq M \\
         0 \leq v_n(x) = \left ( \frac{\alpha-x}{R} \right ) \leq 1 \\
         (6) \implies (1) \text{ равномерно сходится на } [\alpha - R, R] 
    \end{gather*}
\end{longProof}

\end{document}