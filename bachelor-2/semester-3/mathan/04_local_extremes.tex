% !TeX root = ./main.tex
\documentclass[main]{subfiles}
\begin{document}
\chapter{Условные экстремумы}
\begin{definition}
        \[E \subset {\vR}^n, n \geq 2 \quad M \subset E \quad
        f: E \rightarrow \mathbb{R} \quad
        X_0 \in M \]
        Тогда говорят, что функция в $X_0$ имеет локальный экстремум при условии $M$, 
        если $\exists w \subset {\vR}^n, X_0 \in w : f|_{w \cap M}$ имеет в точке $X_0$
        локальный экстремум. 
\end{definition}
 
\begin{theorem}[О множителях Лагранжа]
    \begin{gather*}
        E \subset {\vR}^{n+m} \quad E - \text{ открытое} \quad
        F: E \rightarrow {\vR}^n \quad F \in C^1(E) \\
        \begin{bmatrix*}
            X \\
            Y
        \end{bmatrix*} : X \in {\vR}^n, Y \in {\vR}^m \\
        \forall \begin{bmatrix*}
            X \\
            Y
        \end{bmatrix*} \in E \quad rank{DF\left( \begin{bmatrix*}
            X \\
            Y
        \end{bmatrix*} \right)} = n \tag{9} \\
        \begin{bmatrix*}
            X_0 \\
            Y_0
        \end{bmatrix*} \in E \quad F \left( \begin{bmatrix*}
            X_0 \\
            Y_0
        \end{bmatrix*} \right) = \mathbb{0}_n \\ Z = \begin{bmatrix}
            X \\
            Y
        \end{bmatrix} \quad
        M = \left\{ \begin{bmatrix*}
            X \\
            Y
        \end{bmatrix*} \in E: F\left( \begin{bmatrix*}
            X \\
            Y
        \end{bmatrix*}\right) = \mathbb{0}_n  \right\} \\
        f: E \rightarrow {\vR}  \quad  f \in C^1(E)\\
        \begin{bmatrix*}
            X_0 \\
            Y_0
        \end{bmatrix*} - \text{ локальный экстремум } f \text{ при условии } M. 
        \text{ Тогда} \\ 
        \exists! \Lambda = (\lambda_1, \ldots, \lambda_n) : \phi_\Lambda(Z) = f(Z) + \Lambda F(Z) \tag{11} \\
        \nabla  \phi_\Lambda(Z_0) = \mathbb{0}^{T}_{n+m} (\nabla - \text{ градиент}) \tag{12} \\
        (12) : \phi^\prime_{\Lambda x_1}(Z_0) = 0, \ldots, \phi^\prime_{\Lambda x_n}(Z_0) = 0 \\
        \phi^\prime_{\Lambda y_1}(Z_0), \ldots, \phi^\prime_{\Lambda y_m}(Z_0) = 0
    \end{gather*}
\end{theorem} 
\begin{longProof}
    Мы не случайно обозначили так координаты, по условию сказано, что ранг матрицы Якоби в любой точке равен n. Мы можем выбрать n столбцов
        так, чтобы соответсвующий определитель матрицы Якоби не был равен 0.
    \begin{gather*}
            \left |\begin{matrix*}
                F^\prime_{1x_1}(Z_0) & \ldots & F^\prime_{1x_n}(Z_0) \\
                \ldots & \ddots & \ldots \\
                F^\prime_{nx_1}(Z_0) & \ldots & F^\prime_{nx_n}(Z_0)
            \end{matrix*}  \right | \ne 0 \tag{13} \\
            \exists W \in {\vR}^m \quad Y_0 \in W \quad g: W \rightarrow {\vR}^n \quad g(Y_0) = X_0 \\
            \text{ при } y \in W \quad F\left(\begin{matrix*}
                X \\
                Y
            \end{matrix*}\right) = \mathbb{0}_n \Leftrightarrow X =  g(Y) \tag{14} \\
            (14) : \begin{bmatrix*}
                X \\
                Y
            \end{bmatrix*} \in M \text{ при } Y \in W \Leftrightarrow X = g(Y) \tag{14\prime} \\
            h(Y) = f \left( \begin{bmatrix*}
                g(Y) \\
                y
            \end{bmatrix*} \right) \quad w \in {\vR}^{n+m} \\
            \begin{bmatrix*}
                X \\
                Y
            \end{bmatrix*} \in w \implies Y \in W \tag{15} \\
            (14\prime), (15) \implies \begin{bmatrix*}
                X \\
                Y
            \end{bmatrix*} \in w \cap M \Leftrightarrow X = g(Y) \tag{16} \\
            (16) \text{ и определение локального условного экстремума } \implies\\
             Y_0 - \text { локальный экстремум для } h(Y) \tag{17}\\
             h \in C^1(W) \quad g \in C^1(W) \\
            (17) \Rightarrow \nabla h(Y_0) = \mathbb{0}^T_m  \implies
            Dh(Y_0) = \mathbb{0}^T_m \tag{18} \\
            % Когда мы заменили градиент на матрицу Якоби ... (вставить текст)
            \text{Хотим ввести отображение } P(Y) = \begin{bmatrix*}
                g(Y) \\
                y
            \end{bmatrix*} \\
             h(Y) = f(P(Y)) \tag{19} \\
             h(Y) = f \left( \begin{bmatrix*}
                g(Y) \\
                Y
             \end{bmatrix*} \right) \\
             h \in C^1(W) \\
             (18), (19) \implies Df(P(Y_0))DP(Y_0) = \mathbb{0}^T_m \tag{20} \\
             P(Y_0) = \begin{bmatrix*}
                g(Y_0) \\
                Y_0
             \end{bmatrix*} = \begin{bmatrix*}
                X_0 \\
                Y_0 
             \end{bmatrix*} = Z_0 \\
             DF(Z_0) = \nabla f(Z_0) = (\underbrace{f^\prime_{x_1}(Z_0) \ldots f^\prime_{x_n}(Z_0)}_{\nabla_1 f(Z_0)} \underbrace{(f^\prime_{y_1}(Z_0) \ldots f^\prime_{y_m}(Z_0))}_{\nabla 2 f(Z_0)} \tag{21} \\
             (4) \implies DP(Y_0)  = \begin{bmatrix*}
                Dg(Y_0) \\
                I_m
             \end{bmatrix*} \tag{4\prime} \\
             (4), (20), (21) \implies (\nabla_1 f(Z_0) \nabla_2 f(Z_0)) \begin{bmatrix*}
                Dg(Y_0) \\
                I_m
             \end{bmatrix*} = \mathbb{0}^T_m \Leftrightarrow \\
             \nabla_1 f(Z_0) Dg(Y_0) + \nabla_2 f(Z_0) = \mathbb{0}^T_m \tag{22} \\
             (1), (2), (8), (22) \implies \nabla_1 f(Z_0)(-A_0^{-1}B_0) + \nabla_2 f(Z_0) = \mathbb{0}^T_m \\
             \Leftrightarrow \nabla_2 f(Z_0) = \nabla_1 f(Z_0)(A_0^{-1}B_0) \tag{23} \\
            \end{gather*}
             Соотношение (23) получилось на основании того, что есть теорема о неявном отображении вместе с вычислением матрицы Якоби этого отображения (1).
             И  необходимого условия локального экстремума функции класса $C^1$ (2) 
             \begin{gather*}
             \nabla \phi_\Lambda(Z_0) = \mathbb{0}^T_{m+n} \tag{?} \\
             \text{Сейчсас будем проверять это.} \\
             \text{Выпишем часть требуемых равенств:} 
             \begin{cases}
                \phi^\prime_{\Lambda x_1} (Z_0) = 0 \\
                \vdots \\
                \phi^\prime_{\Lambda x_n} (Z_0) = 0
              \end{cases} \tag{24} \\
              \begin{cases}
                f^\prime_{x_1}(Z_0) + \lambda_1 F^\prime_{1x_1}(Z_0) + \ldots + \lambda_n F^\prime_{nx_1}(Z_0) = 0 \\
                \vdots \\
                f^\prime_{x_n}(Z_0) + \lambda_1 F^\prime_{1x_n}(Z_0) + \ldots + \lambda_n F^\prime_{nx_n}(Z_0) = 0 \\
              \end{cases} \\
              (24): \nabla_1 f(Z_0) + \Lambda A_0 = \mathbb{0}^T_m \tag{24\prime} \\
              (24\prime) \implies \Lambda = -\nabla_1 f(Z_0) A_0^{-1} \tag{25} \\
            \end{gather*}
              Соотношение 25 означает, что если есть вектор-строка хотя бы для первых n равенств, то она одна. То, что она есть, означает
              что выполнены еще m равенств, которые мы пока не проверили.
              \begin{gather*}
              \begin{cases}
                \phi^\prime_{\Lambda y_1} (Z_0) = 0 \\
                \vdots \\
                \phi^\prime_{\Lambda y_m} = 0
              \end{cases}
              \begin{cases}
                f^\prime_{y_1}(Z_0) + \lambda_1 F^\prime_{1y_1}(Z_0) + \ldots + \lambda_n F^\prime_{ny_1}(Z_0) = 0 \\
                \vdots \\
                f^\prime_{y_m}(Z_0) + \lambda_1 F^\prime_{1y_m}(Z_0) + \ldots + \lambda_n F^\prime_{ny_m}(Z_0) = 0 \\
              \end{cases} \tag{26} \\
              (26): \nabla_2 f(Z_0) + \Lambda B_0 = \mathbb{0}^T_m \tag{?} \\
              (23), (25) \implies \nabla_1 f(Z_0)(A_0^{-1}B_0) - \nabla_1 f(Z_0)(A_0^{-1}B_0) = \mathbb{0}^T_m
        \end{gather*}
        \end{longProof}
        \begin{corollary}[Приложение теоремы Лагранжа]
        \begin{gather*}
            X = \begin{bmatrix*}
                X_1 \\
                \vdots \\
                x_n
            \end{bmatrix*} \\
            A(X) = \sum^n_{i=1} \sum^n_{j=1} a_{ij} x_i x_j \quad a_{ij} = a_{ji} \quad i \ne j \\
            max A(X), min A(X) \text{ при } ||X||^2_{{\vR}^n} = 1 (\text{то есть при условии } x_1^2 + \ldots + x_n^2 = 1)\\
            F_1(X) = x_1^2 + \ldots + x_n^2 - 1  = 0 \quad
            X_+ = \begin{bmatrix*}
                X_1^+ \\
                \vdots \\
                X_n^+ \\
            \end{bmatrix*} \\
            \phi_\lambda(X) = A(X) + \lambda(x_1^2 + \ldots + x_n^2 - 1) \\
            \phi^\prime_{\lambda x_1} (X_+) = 0, \ldots, \phi^\prime_{\lambda x_n}(X_+) = 0 \\
            \text{напишем сумму для одного i} \ne k 
        \end{gather*}
            \begin{multline*}
            (a_{ik}x_i x_k + a_{kk}x_k^2 + a_{ki} x_k x_i)\prime_{x_k} = a_{ik} x_i + 2a_{kk} x_k + a_{ki} x_i = \\
            = 2a_{kk} x_k + 2a_{ki} x_i 
            \end{multline*}
            \begin{gather*}
            A\prime_{x_k}(X) = 2 \sum_{i=1}^n a_{ki} x_i \\ 
            2 \sum^n_{i=1} a_{ki} x_i^+  + 2 \lambda x_k^+ = 0 \\
            \sum^n_{i=1} a_{ki}x^+_i = -\lambda x^+_k, 1 \leq k \leq n  \implies ax_+ = -\lambda x_+ \\
            \implies\lambda - \text { составное число} \\
            \sum_{i=1}^n a_{ki}x_i^+x_k^+ = -\lambda x_k^{+2} \implies 
            A(x_+) = \sum^n_{k=1}\sum^n{i=1} a_{ki}x_i^+x_k^+ = \\ = 
            -\lambda \sum^n_{k=1}x_k^{+2} = -\lambda
        \end{gather*}
    \end{corollary}
    \begin{proposition}
        $A$ - симметричная матрица $n \times n$. Тогда $||A|| = \max | \lambda_j |$.
    \end{proposition}
        \begin{theorem}
            $A$ - симметричная матрица $n \times n$. $B$ - любая матрица, не обязательно квадатная.
            $U = B^TB$ квадратная и симметричная, все собственные числа неотрицательные. Тогда
            $||B|| = \sqrt{||U||}$
        \end{theorem}
\end{document}