% !TeX root = ./main.tex
\documentclass[main]{subfiles}
\begin{document}
\chapter{Криволинейные интегралы второго рода}
\section{Криволинейные интегралы второго рода от комплекснозначных функций} \marginpar{16.02.23}
Для ориентированных кривых Николай Алексеевич использует гнутую стрелочку, здесь будет использоваться обычная.
Так же вместе $\stackrel{\text{def}}{=}$ будет использоваться $\coloneq$.
\begin{definition}[Криволинейный интеграл второго рода от комплекснозначной функции]
    Имеется $\AnClOrCurve \subset \R^2$ -- ориентированная кривая.
    И имеется функция \[f(x_1, x_2) = u(x_1, x_2) + i v(x_1, x_2),\] такая что $f: \AnClOrCurve \to \C$, $f \in C(\Gamma)$.

    Тогда криволинейным интегралом второго рода от функции $f$ по ориентированной кривой $\AnClOrCurve$ по аргументу $x_j$ будем называть следующее выражение, в котором $j=1,2$:
    \[\AnClOrInt f(x_1, x_2) dx_j \coloneq \AnClOrInt u(x_1, x_2) dx_j + i \AnClOrInt v(x_1, x_2) dx_j\]
\end{definition}

\subsection{Свойства криволинейного интеграла второго рода от комплекснозначных функций}
Далее везде предполагается, что все функции непрерывны на кривой, а сама кривая кусочно-гладкая.

\begin{property}
    \[\AnClOrInt (f+g) dx_j = \AnClOrInt f dx_j + \AnClOrInt g dx_j\]
\end{property}

\begin{property}
    Пусть $c \in \C$, тогда
    \[\AnClOrInt c f dx_j = c \AnClOrInt f dx_j\]
\end{property}
\begin{proof}
    Допустим, что $c = a + ib$, $f = u + iv$, тогда
    \[cf = au - bv + i(av + bu)\]
    Подставим это в левый интеграл:
    \begin{multline*}
        \AnClOrInt c f dx_j = \AnClOrInt (au - bv) dx_j + i \AnClOrInt (av + bu) dx_j =\\
        = a \AnClOrInt u dx_j - b \AnClOrInt v dx_j + i\left(a \AnClOrInt v dx_j + b \AnClOrInt u dx_j\right) = \\
        = c \AnClOrInt u dx_j + i c \AnClOrInt v dx_j = c \AnClOrInt f dx_j
    \end{multline*}
\end{proof}

\begin{property}
    \[\ClOrInt f dx_j  = - \AnClOrInt f dx_j\]
\end{property}

\begin{property}
    Имеется $\AnClOrCurve([a,b])$, разбиение $P = \{t_k\}_{k=0}^m$, где $t_0 = a, t_m = b$, а так же его оснащение $T = \{\tau_k\}_{k=1}^m$, где $\tau_k \in [t_{k-1}, t_k]$.
    Кроме того имеется комплекснозначная функция $f: \Gamma \to \C$, $f \in C(\Gamma)$.

    Введем обозначения: $\AnClOrCurve(\tau_k) = M_k$, $\AnClOrCurve(t_k) =
        \begin{bmatrix}
            x_{1k} & x_{2k}
        \end{bmatrix}^T$.
    Суммой Римана для функции $f$, разбиения $P$, оснащения $T$ и для аргумента $x_j$  будем называть такое выражение:
    \[S(f, P, T, j) = \sum_{k=1}^m f(M_k)(x_{j,k} - x_{j, k-1}),\]
    где  $j = 1, 2$.

    Тогда свойство состоит в следующем:
    \begin{multline*}
        \forall \epsilon > 0\ \exists \delta >0 \text{ т.ч. } \forall P \text{ т.ч. } t_k - t_{k-1} < \delta, k \le m \text{ и } \forall T \\
        \left|\AnClOrInt[\AnClOrCurve([a,b])] f dx_j - S(f, P, T, j)\right| < \epsilon \tag{1}
    \end{multline*}
\end{property}
\begin{proof}
    Если $f = u + iv$, то выполнено
    \begin{gather*}
        S(f, P, T, j) = S(u, P, T, j) + iS(v, P, T, j)\\
        \intertext{Применив аналогичную теорему для вещественных функций, можем найти такое $\delta$, что выполняются}
        \left| \AnClOrInt u dx_j - S (u, P, T, j)\right| < \epsilon/2 \tag{2}\\
        \left| \AnClOrInt v dx_j - S (v, P, T, j)\right| < \epsilon/2 \tag{3}
    \end{gather*}
    $(2), (3) \implies (1)$
\end{proof}

\section{Направление на кривой}
\begin{center}
    \import{figures}{orientation_on_curve.pdf_tex}
\end{center}
Сначала рассмотрим окружность, как самый простой пример.
Можем обходить ее <<по часовой стрелке>> и <<против часовой стрелки>>.
Положительным обходом с этого момента и далее будем называть обход против часовой стрелки.
В отрицательном направлении по часовой стрелке, соответственно.

Теперь рассмотрим некую замкнутую кривую на плоскости.
На ней так же определены положительное и отрицательное направления.
В некотором смысле они аналогичны окружности.
Но как это понять?

\begin{theorem}[Жордана]
    Имеется некая кривая на плоскости $\AnClOrCurve: [a, b] \to \R^2$, такая что $\AnClOrCurve(a) = \AnClOrCurve(b)$, и если $t_1 \neq t_2$, то $\AnClOrCurve(t_1) \neq \AnClOrCurve(t_2)$.
    Тогда эта кривая делит плоскость на 2 области.

    Одна из этих областей содержит точки, которые произвольно далеки от начала координат, и называется внешностью кривой, вторая называется внутренностью.
    При этом, если мы возьмем любую точку $M$ внутри области и любую точку $N$ вне области, и соединим их любой кривой $L$, то эта кривая обязательно пересечет кривую $\Gamma$, то есть $L \cap \Gamma \neq \varnothing$.
\end{theorem}
\begin{proof}
    Примем без доказательства.
\end{proof}

Пусть $\Gamma(t) =
    \begin{bmatrix}
        x_1(t) & x_2(t)
    \end{bmatrix}^T$~--- замкнутая кусочно-гладкая кривая.
Функции $x_1$ и $x_2$ непрерывны на $[a,b]$ и дифференцируемы за исключением конечного числа точек.
Так же выполнено $(x_1'(t))^2 + (x_2'(t))^2 \ge \delta_0 > 0$, если $t \neq c_1, ..., c_l$.

Положим, что $\nu(t) =
    \begin{bmatrix}
        -x_2'(t) & x_1'(t)
    \end{bmatrix}^T$.
Ориентация называется положительной, если для достаточно малого $\epsilon >0$ $\Gamma(t) + \epsilon \nu (t)$ лежит во внутренности $\Gamma$.

Пояснение: для достаточно малого $\epsilon$
\[\forall t \neq c_j\ \exists \epsilon_0(t): 0 < \epsilon \le \epsilon_0(t)\]
и $\Gamma(t) + \epsilon \nu (t)$ лежит во внутренности $\Gamma$.

\begin{example}
    С окружностью понятно как действует теорема Жордана
    \begin{center}
        \import{figures}{circle_on_x1_x_2.pdf_tex}
    \end{center}
    \begin{gather*}
        \begin{cases}
            x_1 = R \cos t \\
            x_2 = R \sin t
        \end{cases} \quad  0 \le t \le 2 \pi\\
        \begin{cases}
            x_1' = -R \sin t \\
            x_2' = R \cos t
        \end{cases} \implies
        \nu =
        \begin{bmatrix}
            -R \cos t \\ -R \sin t
        \end{bmatrix}\\
        \begin{bmatrix}
            x_1 \\
            x_2
        \end{bmatrix} + \epsilon \nu =
        \begin{bmatrix}
            R (1 - \epsilon) \cos t \\
            R (1 - \epsilon) \sin t
        \end{bmatrix}
    \end{gather*}
    Если $0 < \epsilon < R$, то эта точка лежит внутри круга.
\end{example}

\section{Криволинейные интегралы второго рода от комплекснозначных функций на комплексной плоскости}
Изменим обозначения координат: вместо $x_1, x_2$ будем рассматривать $x,y$.
Тогда можем рассматривать однозначное соответствие с комплексной плоскостью.

Есть кривая $\AnClOrCurve(t): [a,b] \to \R^2 \leftrightarrow \AnClOrCurve_C(t): [a,b] \to \C$
\[\Gamma(t) = \begin{bmatrix}
        x(t) & y(t)
    \end{bmatrix}^T \leftrightarrow z(t) = x(t) + i y(t)\]
Будем также определять ориентацию на кривых в комплексной плоскости, полагая, что она задается на обычной плоскости.
Если мы посмотрим на замкнутые кривые, то теорема Жордана  выполняется.
Будем называть ориентацию замкнутой кривой на комплексной плоскости положительной, если положительной является ориентация этой кривой на $\R^2$.

Вспомним, что $\C \supset E \leftrightarrow E^* \subset \R^2$.
Тогда  $f: \Gamma_C \to \C \leftrightarrow f^*: \Gamma \to \C$, а так же $f^*(M(t)) \coloneq f(z(t)), t \in [a,b]$.
Далее $\ ^*$ и $\ _C$ будем опускать.

Пусть у нас имеется ориентированная кривая $f: \AnClOrCurve_C \to \C \leftrightarrow f^*: \AnClOrCurve \to \C$, тогда определению полагаем
\begin{gather*}
    \AnClOrInt[\AnClOrCurve_C] f(z) dx \coloneq \AnClOrInt f^*(M) dx \tag{4}\\
    \AnClOrInt[\AnClOrCurve_C] f(z) dy \coloneq \AnClOrInt f^*(M) dy \tag{5}\\
\end{gather*}
Тогда можем определить
\[\AnClOrInt[\AnClOrCurve_C] f(z) dz \coloneq \AnClOrInt[\AnClOrCurve_C] f(z) dx + i \AnClOrInt[\AnClOrCurve_C] f(z) dy \tag{6}\]

Если $\AnClOrCurve_C(t)$~--- гладкая и $\AnClOrCurve_C(t) = z(t) = x(t) + iy(t), t \in [a,b]$, то (4)--(6) влекут, что
\[\AnClOrInt[\AnClOrCurve_C([a,b])] f(z) dz = \int_a^b f(z(t))(x'(t) + i y'(t))dt \tag{7}\]

\subsection{Свойства криволинейного интеграла второго рода $f(z)dz$}
Далее все функции предполагаются непрерывными.
\begin{property}
    \[\AnClOrInt (f+g)dz = \AnClOrInt fdz + \AnClOrInt gdz\]
\end{property}

\begin{property} Пусть $c \in \C$, тогда
    \[\AnClOrInt c f dz = c \AnClOrInt f dz\]
\end{property}

\begin{property}
    \[\AnClOrInt[\ClOrCurve] f dz = - \AnClOrInt f dz\]
\end{property}

\begin{property}\label{1:RiemannSumC}
    Введем суммы Римана.
    Имеем $\AnClOrCurve([a,b]) \subset \C$, разбиение $P = \{t_k\}_{k=0}^m$ и оснащение $T = \{\tau_k\}_{k=1}^m$.
    \[S(f, P, T, z) = \sum_{k=1}^m f(z(\tau_k)) (z(t_k) -z(t_{k-1})) \tag{8}\]
    Пусть $\Gamma$ есть кусочно-гладкая кривая, $f \in C(\Gamma)$, тогда
    \begin{multline*}
        \forall \epsilon > 0\ \exists \delta >0 \text{ т.ч. } \forall P \text{ т.ч. } t_k - t_{k-1} < \delta \text{ и } \forall T \\
        \left|\AnClOrInt f(z) dz - S(f, P, T, z)\right| < \epsilon \tag{9}
    \end{multline*}
\end{property}

\begin{property}
    Пусть $\AnClOrCurve \subset \C$ -- кусочно-гладкая кривая с параметризацией $z(t), t \in [a,b]$.
    $c \in \C$, тогда
    \begin{gather}
        \AnClOrInt c dz = c(z(b) - z(a)) \tag{10}\\
        \intertext{в частности, если кривая $\Gamma$ замкнутая, то}
        \AnClOrInt c dz = 0 \tag{10'}
    \end{gather}
\end{property}
\begin{proof}
    Для доказательства используем \ref{1:RiemannSumC}.
    $\forall \epsilon > 0$ выберем подходящие $\delta, P, T$
    \begin{gather*}
        \begin{multlined}
            S(c, P, T, z) = \sum_{k=1}^m c(z(t_k) - z(t_{k-1})) =\\
            = c (z(t_m) - z(t_0)) = c (z(b) - z(a))
        \end{multlined} \tag{11}\\
        \left| \AnClOrInt c dz - S(c, P, T, z)\right| < \epsilon \tag{12}\\
        (11), (12) \implies (10)
    \end{gather*}
\end{proof}

\begin{property}\label{1:PartsSum}
    Пусть у нас есть кусочно-гладкая кривая $\AnClOrCurve \subset \C$, где $f: \AnClOrCurve \to \C$.
    Допустим у нас есть точки $a < t_1 < ... < t_m < b$ и имеются кривые $\AnClOrCurve_1:[a, t_1] \to \C, ..., \AnClOrCurve_m: [t_{m-1}, t_m] \to \C, \AnClOrCurve_{m+1}: [t_m, b] \to \C$.
    \[\AnClOrInt fdz = \sum_{k=1}^{m+1} \AnClOrInt[\AnClOrCurve_k] f dz\]
\end{property}

\begin{property} \label{1:LessThanLength}\marginpar{23.02.23}
    Пусть $\AnClOrCurve_C([a,b])$~--- ориентированная кусочно-гладкая кривая, $f \in C(\Gamma([a,b]))$.
    Тогда
    \[\left| \AnClOrInt[\AnClOrCurve_C ([a,b])] f dz \right| \le \int_{\Gamma([a,b])} \left| f^* \right| dl, \tag{1}\]
    где в правой части стоит криволинейный интеграл первого рода.
\end{property}
\begin{proof}
    Возьмем $\forall \epsilon >0$, найдем такое разбиение $P = \{t_j\}_{j=0}^n$, $a=t_0 < \dotsc < t_n = b$, чтобы для любого оснащения $T = \{\tau_j\}_{j=1}^n$ выполнялись соотношения
    \[\left| \sum_{j=1}^{n} f(z_j') (z_j - z_{j-1}) - \AnClOrInt[\AnClOrCurve_C ([a,b])] f dz \right| < \epsilon, \tag{2}\]
    где $z_j = \AnClOrCurve_C (t_j)$, $z_j' = \AnClOrCurve_C(\tau_j)$, и
    \[\left| \sum_{j=1}^{n} \left| f^*(\Gamma(\tau_j)) \right| l(\gamma(t_{j-1}, t_j)) - \int_{\Gamma([a,b])} \left| f^* \right| dl \right| < \epsilon, \tag{3}\]
    где $l(\gamma(t_{j-1}, t_j))$~--- длина дуги $\gamma(t_{j-1}, t_j) \subset \Gamma([a,b])$ между точками $\Gamma(t_{j-1})$ и $\Gamma(t_j)$.

    Тогда (2) и (3) $\implies$
    \begin{multline*}
        \left| \AnClOrInt[\AnClOrCurve_C ([a,b])] f dz \right| = \\
        = \left| \AnClOrInt[\AnClOrCurve_C ([a,b])] f dz - \sum_{j=1}^{n} f(z_j') (z_j - z_{j-1}) + \sum_{j=1}^{n} f(z_j') (z_j - z_{j-1}) \right| \le \\
        \le \left| \AnClOrInt[\AnClOrCurve_C ([a,b])] f dz - \sum_{j=1}^{n} f(z_j') (z_j - z_{j-1}) \right| + \left| \sum_{j=1}^{n} f(z_j') (z_j - z_{j-1}) \right| < \\
        < \epsilon + \sum_{j=1}^{n} |f(z_j')| |z_j - z_{j-1}| \le \epsilon + \sum_{j=1}^{n} \left| f^*(\Gamma(\tau_j)) \right| l(\gamma(t_{j-1}, t_j)) = \\
        = \epsilon + \left( \sum_{j=1}^{n} \left| f^*(\Gamma(\tau_j)) \right| l(\gamma(t_{j-1}, t_j)) - \int_{\mathrlap{\Gamma([a,b])}} \left| f^* \right| dl \right) + \int_{\mathrlap{\Gamma([a,b])}} \left| f^* \right| dl < \\
        < 2\epsilon + \int_{\Gamma([a,b])} \left| f^* \right| dl \tag{4}
    \end{multline*}
    В силу произвольности $\epsilon >0$ $(4) \implies (1)$.
\end{proof}

\begin{property}
    Пусть $\AnClOrCurve_C([a,b])$~--- ориентированная кусочно-гладкая кривая, $f \in C(\Gamma([a,b]))$, $\theta \in (0, 2\pi)$,
    \begin{gather*}
        \AnClOrCurve_\theta([a,b]) \coloneq e^{i \theta} \AnClOrCurve([a,b]),
        \intertext{то есть}
        \Gamma_\theta([a,b]) = \left\{ z \in \C: z = e^{i \theta} \xi, \xi \in \Gamma([a,b])\right\}
    \end{gather*}
    и ориентация на $\AnClOrCurve_\theta([a,b])$ задается с $\AnClOrCurve([a,b])$.
    Для $z \in \AnClOrCurve([a,b])$ положим $f_\theta(z) \coloneq f(e^{-i\theta}z)$.
    И справедливо равенство
    \[\AnClOrInt[\AnClOrCurve ([a,b])] f(z) dz = e^{-i\theta} \AnClOrInt[\AnClOrCurve_\theta ([a,b])] f(\xi) d\xi. \tag{5}\]
\end{property}
\begin{proof}
    Пусть $\{z_j\}_{j=0}^n$~--- точки на $\AnClOrCurve([a,b])$, полученные из разбиения $\{t_j\}_{j=0}^n$, $\{\tau_j\}_{j=1}^n$~--- оснащение, $z_j' = \Gamma(\tau_j)$.
    Тогда точки $\{e^{i\theta} z_j\}_{j=0}^n$~--- точки на $\AnClOrCurve_\theta([a,b])$, полученные из того же разбиения и $\{e^{i\theta} z_j'\}_{j=1}^n$~--- точки на $\AnClOrCurve_\theta([a,b])$, полученные из оснащения.
    Теперь имеем:
    \[\sum_{j=1}^{n} f(z_j') (z_j - z_{j-1}) = e^{-i \theta}\sum_{j=1}^{n} f_\theta(e^{i \theta}z_j') (e^{i \theta}z_j - e^{i \theta}z_{j-1}) \tag{6}\]
    Теперь (5) следует из (6) и свойства \ref{1:RiemannSumC}.
\end{proof}
\end{document}
