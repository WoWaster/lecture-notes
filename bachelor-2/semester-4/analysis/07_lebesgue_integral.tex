% !TeX root = ./main.tex
\documentclass[main]{subfiles}
\begin{document}
\chapter{Интеграл Лебега}
\section{Построение интеграла Лебега}
\begin{definition}
    Пусть имеется $E \subset \R^n$ и функция $f_0:E \to \R$.
    Функция $f_0$ называется простой, если она принимает конечное число значений.
    Пусть $c_1, \dotsc, c_m$~--- различные значения $f_0$, тогда рассмотрим множества
    \[E_k = \{ x \in E: f(x) = c_k\},\]
    т.к. все $c_k$ различны, то $E_k \cap E_l = \varnothing$,  при $k \neq l$.
    Вспомним характеристическую функцию $K_{E_l}(x)$, то
    \[f_0(x) = \sum_{l=1}^{m} c_l K_{E_l}(x)\]
\end{definition}
Рассмотрим множества $F_1, \dotsc, F_q \subset E$, где $d_1, \dotsc, d_q$ не обязательно различные, то
\[\sum_{l=1}^{q} d_l K_{F_l}(x)\]
тоже простая функция.

\begin{theorem}[об аппроксимации простыми функциями]
    Пусть имеется произвольное непустое множество $E \subset \R^n$ и функция $f: E \to \overline{\R}$, тогда существует последовательность простых функций $\{f_m(x)\}_{n=1}^\infty$, т.ч.
    \[\forall x \in E\ f_m(x) \xrightarrow[m \to \infty]{} f(x) \tag{4}\]
    Если $f(x) \ge 0$, то можно выбрать $f_m$ так, чтобы
    \[f_m(x) \le f_{m+1}(x)\ \forall m\ \forall x \in E \tag{5}\]
    Если $E$ измеримо, $f$ измеримо, то можно выбрать $f_m$ измеримо $\forall m$~(6).
\end{theorem}
\begin{proof}
    Определим множества при $-2^{2m} \le i \le 2^{2m}$:
    \begin{align*}
        E_{i,m} & = \{x \in E: (i-1) 2^{-m} \le f(x) < i \cdot 2^{-m}\} \\
        E_m     & = \{x \in E: f(x) < -2^m \lor f(x) \ge 2^m\}
    \end{align*}
    Теперь определим $f_m$:
    \[f_m (x) = \begin{cases}
            (i-1) \cdot 2^{-m} & x \in E_{i,m}            \\
            -2^m               & x \in E_m \land f(x) < 0 \\
            2^m                & x \in E_m \land f(x) > 0
        \end{cases}\]
    Получили, что все $f_m$ измеримы, кроме того, в терминах множеств Лебега
    \[E_{i,m} = E_{\ge (i-1) 2^{-m}} (f) \cap E_{< i 2^{-m}}(f),\]
    т.е. в случае измеримой функции все эти множества измеримы.

    Предел есть, т.к. если $x \in E_{i,m}$, то $0 \le f(x) - f_m(x) \le 2^{-m}$, монотонность тоже проверяется.
\end{proof}

Далее считаем, что все множества и функции измеримы.
\begin{remark}
    Измеримые множества могут иметь и бесконечную меру.
\end{remark}
\begin{definition}
    Пусть имеется множество $E$ и простая функция
    \[f_0 = \sum_{k=1}^{\nu} c_k K_{E_k}(x), \tag{7}\]
    все $c_k$ различны, поэтому $E_k \cap E_l = \varnothing$, если $k \neq l$~(8).
    Положим
    \[I_E(f_0) \coloneq \sum_{k=1}^{\nu} c_k mE_k \tag{9}\]
    Если $m E_{k_0} = + \infty$, то $c_{k_0} = 0$ и $0 \cdot \infty = 0$.
\end{definition}

\begin{proposition}
    Если $f_0(x) = \sum_{k=1}^\mu d_k K_{F_k}(x)$, то
    \[I(f_0) = \sum_{k=1}^{\mu} d_k m(F_k) \tag{10}\]
\end{proposition}
\begin{proof}
    Проверяется по определению.
\end{proof}

\begin{proposition}
    Если $m E < +\infty$ и $a \le f(x) \le b$, когда $x \in E$, тогда $a m E \le I(f_0) \le b m E$.
\end{proposition}
\begin{proof}
    Посмотрим на (9), для всех $c_k$ выполнено $a \le c_k \le b$, все меры неотрицательные, поэтому замена всех $c_k$ на $b$ увеличит сумму, аналогично для $a$.
\end{proof}

\begin{proposition}
    $I(c f_0) = c I(f_0)$, где $c \in \R$.
\end{proposition}
\begin{proof}
    Если домножим (9) на $c$, то получим соответствующее равенство.
\end{proof}

\begin{proposition}
    Пусть имеются две простые функции $f_0$ и $g_0$, тогда $I(f_0 + g_0) = I(f_0) + I(g_0)$.
\end{proposition}
\begin{proof}
    Запишем простые функции как суммы:
    \begin{gather*}
        f_0 = \sum_{k=1}^{\nu} c_k K_{E_k}(x) \\
        g_0 = \sum_{l=1}^{\mu} d_l K_{F_l}(x)
    \end{gather*}
    Рассмотрим $G_{kl} = E_k \cap F_l$, тогда
    \begin{gather*}
        f_0(x) = \sum_{k=1}^{\nu} \sum_{l=1}^{\mu} c_k K_{G_{kl}}(x) \\
        g_0(x) = \sum_{k=1}^{\nu} \sum_{l=1}^{\mu} d_l K_{G_{kl}}(x)
    \end{gather*}
    Тогда верна сумма
    \[f_0(x) + g_0(x) = \sum_{k=1}^{\nu} \sum_{l=1}^{\mu} (c_k + d_l) K_{G_{kl}}(x)\]
\end{proof}

\begin{proposition}
    Если $f_0 (x) \le g_0(x)\ \forall x \in E$, то $I(f_0) \le I(g_0)$.
\end{proposition}
\begin{proof}
    Воспользуемся равенствами из доказательства выше.
    Если $x \in G_{kl}$, тогда $c_k = f_0(x) \le g_0(x) = d_l$.
    \begin{gather*}
        I(f_0) = \sum_{k=1}^{\nu} \sum_{l=1}^{\mu} c_k mG_{kl}(x) \\
        I(g_0) = \sum_{k=1}^{\nu} \sum_{l=1}^{\mu} d_l mG_{kl}(x)
    \end{gather*}
    Для каждой пары $k, l$, получаем, что $c_k \le d_l$, что и доказывает теорему.
\end{proof}

\marginpar{11.05.23}
\begin{definition}[интеграл Лебега]
    Имеем измеримое множество $E \subset \R^n$, где $n \ge 1$.
    Так же имеем $f: E \to \overline{\R}$, $f(x) \ge 0$.
    Обозначим
    \begin{align*}
        B_E(f) = \{ & s: s\text{ --- простая}, s: E \to \R,          \\
                    & 0 \le s(x) \le f(x)\ \forall x \in E\} \tag{1}
    \end{align*}
    $B_E(f)$ не пусто.
    Рассмотрим $s_0(x) \equiv 0$ для любого $x \in E$, тогда $s_0(x) \in B_E(f)$.
    Тогда интегралом Лебега от функции $f$ по множеству $E$ называется
    \[\int_E f dm \coloneq \sup_{s \in B_E(f)} I_E(s) \tag{2}\]
    Супремум может быть равен $+\infty$.
\end{definition}
\subsection{Первые свойства интеграла Лебега}
\begin{property}
    Если $c > 0$, то
    \[\int_E cf dm = c \int_E f dm. \tag{3}\]
\end{property}

\begin{property}
    Если $0 \le f(x) \le g(x)$ $\forall x \in E$, тогда
    \[\int_E fdm \le \int_E gdm \tag{4}\]
\end{property}
\begin{proof}
    Это следует из того, что $B_E(f) \subset B_E(g)$.
\end{proof}

\begin{property}
    Если $mE = 0$, то для любой функции $f$
    \[\int_E fdm = 0 \tag{5}\]
\end{property}
\begin{proof}
    Для любой простой $s$
    \[\int_E sdm = 0,\]
    откуда и следует (5).
    Если $F \subset E$ и $mE = 0$, тогда $0 \le mF \le mE = 0$, значит $mF = 0$.
\end{proof}

\section{Определенный интеграл Лебега для неотрицательных функций}
\begin{definition}
    Говорят, что неотрицательная функция $f$ суммируема на $E$, если
    \[\int_E f dm < +\infty \tag{6}\]
    Множество всех суммируемых положительных функций на множестве $E$ будем обозначать $\calL_+(E)$.
\end{definition}

Пусть имеется функция $f(x)$ произвольного знака.
Определим функции:
\begin{align*}
    f_+(x) & = \begin{cases}
                   f(x), & f(x) \ge 0 \\
                   0,    & f(x) < 0
               \end{cases}   \\
    f_-(x) & = \begin{cases}
                   |f(x)|, & f(x) < 0   \\
                   0,      & f(x) \ge 0
               \end{cases} \tag{7}
\end{align*}
Для них справедливо тождество $f = f_+ - f_-$ (следует из (7)).
К тому же, $f_+$, $f_-$ измеримы.
\begin{definition}
    Будем говорить, что функция $f$ суммируема на $E$, если $f_+$ и $f_- \in \calL_+(E)$.
    Множество таких функций обозначим $\calL(E)$.
    Если $f$ суммируема на $E$, то
    \[\int_E f dm \coloneq \int_E f_+ dm - \int_E f_- dm \tag{8}\]
\end{definition}
\subsection{Первые свойства}
\begin{property}
    Если $c > 0$, то
    \[\int_E cf dm = c \int_E f dm \tag{9}\]
\end{property}

\begin{property}
    \[\int_E (-f) dm = - \int_E f dm \tag{9'}\]
\end{property}
\begin{proof}
    Если умножить $f$ на $-1$, то $f_+$ и $f_-$ поменяются местами.
\end{proof}

\begin{property}
    $\forall c \in \R$ выполнено
    \[\int_E cf dm = c \int_E f dm \tag{9''}\]
\end{property}

\begin{property}
    Если $mE = 0$, а функция $f$ измеримая произвольного знака, то
    \[\int_E f dm = 0\]
\end{property}

\subsection{Важнейшее свойство}

\begin{theorem}
    Имеем измеримое множество $E$ и $f \in \calL(E)$.
    Рассмотрим любое измеримое $E_k \subset E$, $f$ на $E_k$ тоже будет измеримой, так же
    \[E = \bigcup_{k=1}^{\infty} E_k\]
    Допустим, что $E_k \cap E_l = \varnothing$, если $k \neq l$ и $1 \le k, l \le +\infty$.
    Рассмотрим функцию
    \[\phi(E_k) = \int_{E_k} fdm, \]
    тогда $f$ суммируема на каждом $E_k$ и справедливо соотношение
    \[\phi(E) = \sum_{k=1}^{\infty}\phi(E_k) \tag{10}\]
    Это свойство называется счетной аддитивностью функции $\phi$, таким образом интеграл Лебега является счетно аддитивной функцией на множестве всех измеримых по Лебегу множеств.
\end{theorem}
\begin{remark}
    Если $s_0(x) \ge 0$ и $s_0$~--- простая, то
    \[\int_E s_0 dm = I_E(s_0)\]
\end{remark}
\begin{longProof}
    Рассмотрим несколько случаев:
    \begin{enumerate}
        \item Пусть $f(x) = K_F(x)$, где $F \subset E$ и $mF < +\infty$.
              Такая $f(x)$~--- простая, тогда
              \begin{gather*}
                  \int_E K_F dm = I_E(K_F) = 1 \cdot m(F \cap E) = m(E \cap F) \tag{11} \\
                  \int_{E_l} K_F dm = I_{E_l}(K_F) = m(E_l \cap F) \tag{12}
              \end{gather*}
              Вспомним важнейшее свойства меры Лебега и $\sigma$-кольца измеримых множеств~--- мера Лебега является счетно аддитивной функцией на множестве всех измеримых множеств.
              Если $k \neq l$, то
              \begin{gather*}
                  (E_k \cap F) \cap (E_l \cap F) = (E_k \cap E_l) \cap F = \varnothing \tag{13}\\
                  \bigcup_{l=1}^\infty (E_l \cap F) = \left(\bigcup_{l=1}^\infty E_l\right) \cap F = E \cap F \tag{14}
              \end{gather*}
              В силу счетно аддитивности (13) и (14) влекут, что
              \[m(E \cap F) = \sum_{l=1}^{\infty} m(E_l \cap F) \tag{15}\]
              При этом (11), (12) и (15) влекут, что
              \[\int_E K_F dm = \sum_{l=1}^{\infty} \int_{E_l} K_F dm \tag{16}\]
        \item Пусть
              \[f(x) = \sum_{\nu=1}^{m} c_\nu K_{F_\nu}, \tag{17}\]
              и $c_\nu > 0$.
              Это простая функция, тогда
              \begin{multline*}
                  \int_E f dm = I_E(f) = \sum_{\nu=1}^m c_\nu I_E(K_{F_\nu}) = \sum_{\nu=1}^{m} c_\nu \int_E K_{F_\nu} dm =\\
                  = \sum_{\nu=1}^{m} c_\nu \sum_{l=1}^{\infty} I_{E_l}(K_{F_\nu}) = \sum_{l=1}^{\infty} \sum_{\nu=1}^{m} c_\nu I_{E_l}(K_{F_\nu}) = \\
                  = \sum_{l=1}^{\infty} I_{E_l} \left(\sum_{\nu=1}^{m} c_\nu K_{F_\nu}\right) = \sum_{l=1}^{\infty} \int_{E_l} \left(\sum_{\nu=1}^{m} c_\nu K_{F_\nu}\right)dm =\\
                  = \sum_{l=1}^{\infty} \int_{E_l} f dm \tag{18}
              \end{multline*}
        \item Пусть $f(x) \ge 0$ и $f \in \calL_+(E)$.
              Рассмотрим любую функцию $s \in B_E(f)$, тогда по шагу 2
              \[\int_E sdm = \sum_{l=1}^{\infty} \int_{E_l} sdm = \sum_{l=1}^{\infty} I_{E_l}(s) \le \sum_{l=1}^{\infty} \int_{E_l} fdm, \tag{19}\]
              т.к. $s \in B_{E_l}(f)$ $\forall l$

              Теперь зафиксируем натуральное $N$, и выберем функцию $s_l$, где $1 \le l \le N$ и $s_l \in B_{E_l}(f)$ и
              \[I_{E_l} (s_l) > \int_{E_l} f dm - \frac{\epsilon}{N} \tag{20}\]
              Рассмотрим функцию $s^*$:
              \[s^*(x) = \begin{cases}
                      s_l(x), & x \in E_l, 1 \le l \le N              \\
                      0,      & x \in E \setminus \bigcup_{l=1}^N E_l
                  \end{cases}\]
              Понятно, что $s^* \in B_E(f)$.
              В таком случае,
              \begin{multline*}
                  \int_E s^* dm = \sum_{l=1}^{\infty} \int_{E_l} s^* dm = \sum_{l=1}^{N} \int_{E_l} s^* dm + \sum_{l = N+1}^{\infty} \int_{E_l} s^* dm =\\
                  = \sum_{l=1}^{N} \int_{E_l} s^* dm = \sum_{l=1}^{N} \int_{E_l} s_l dm >\\
                  > \sum_{l=1}^{N} \left(\int_{E_l} f dm - \frac{\epsilon}{N}\right) = \sum_{l=1}^{N} \int_{E_l} f dm - \epsilon \tag{21}
              \end{multline*}
              Тогда (21) влечет
              \[\sum_{l=1}^{N} \int_{E_l} fdm < \int_E s^* dm + \epsilon \le \int_{E} fdm + \epsilon \tag{22}\]
              Поскольку $\epsilon$ произволен, то (22) влечет, что
              \[\sum_{l=1}^{N} \int_{E_l} fdm \le \int_{E} fdm \tag{23}\]
              В свою очередь, (23) влечет
              \[\sum_{l=1}^{\infty} \int_{E_l} fdm \le \int_{E} fdm \tag{24}\]
              (19) и (24) влекут
              \[\sum_{l=1}^{\infty} \int_{E_l} fdm = \int_{E} fdm \tag{25}\]
    \end{enumerate}

    Для функций любого знака это тоже проверяется с помощью (8)
\end{longProof}

\begin{corollary}
    Функция $f \in \calL(E)$ тогда и только тогда, когда $|f| \in \calL_+(E)$.
\end{corollary}
\begin{proof}
    Заведем множества
    \begin{gather*}
        E_+ = \{x: f(x) \ge 0\} \\
        E_- = \{x: f(x) < 0\},
    \end{gather*}
    тогда $E_+ \cap E_- = \varnothing$, $E = E_+ \cup E_-$.
    В таком случае
    \[\int_E f dm = \int_{E_+} fdm + \int_{E_-} fdm = \int_{E_+} f_+ dm - \int_{E_-} f_- dm\]
    и
    \[\int_E |f| dm = \int_{E_+} |f| dm + \int_{E_-} |f| dm = \int_{E_+} f_+ dm + \int_{E_-} f_- dm,\]
    Кроме того, получили, что
    \[\left| \int_E fdm \right| \le \int_E |f| dm\]
\end{proof}

\begin{definition}
    Пусть имеется множество $E$, не обязательно измеримое, а также функции $f$ и $g: E \to \overline{\R}$.
    Говорят, что функции $f$ и $g$ эквивалентны и пишут $f \sim g$, если
    \[T = \{x \in E: f(x) \neq g(x)\}\]
    и $mT = 0$.
\end{definition}

\begin{corollary}
    Пусть имеется суммируемые функции $f, g \in \calL(E)$ и $f \sim g$, тогда
    \[\int_E fdm = \int_E gdm\]
\end{corollary}
\begin{proof}
    Пусть
    \[T = \{x \in E: f(x) \neq g(x)\}\]
    и $S = E \setminus T$, тогда
    \begin{multline*}
        \int_E fdm = \int_S fdm + \int_T fdm = \\
        = \int_S gdm + 0 = \int_S gdm + \int_T gdm = \int_E gdm
    \end{multline*}
\end{proof}

Далее будет много результатов без доказательств.
\begin{theorem}[Беппо Леви]
    Пусть имеется последовательность функций $\{v_n(x)\}_{n=1}^\infty$, где $v_n(x) \ge 0$, когда $x \in E$, и $v_n(x) \le v_{n+1}(x)\ \forall n\ \forall x$.
    Пусть
    \[V(x) = \lim_{n \to \infty} v_n(x),\]
    считаем, что данный предел всегда существует и может быть равен $+\infty$.
    По свойствам измеримых функций $V(x)$ измерима.
    Тогда теорема Беппо Леви состоит в том, что
    \[\lim_{n \to \infty} \int_E v_n dm = \int_E V dm\]
\end{theorem}
\begin{proof}
    Примем без доказательства.
\end{proof}

\begin{corollary}
    Пусть $f, g \ge 0$ и $f,g \in \calL_+(E)$, тогда
    \[\int_E (f+g) dm = \int_E fdm + \int_E gdm\]
\end{corollary}
\begin{proof}
    Воспользуемся теоремой об аппроксимации, построим простые функции $0 \le u_n(x) \le f(x)$, где $u_n(x) \le u_{n+1}(x)$ и $u_n(x) \xrightarrow[n \to \infty]{} f(x)$.
    И простые функции $0 \le v_n(x) \le g(x)$, где $v_n(x) \le v_{n+1}(x)$ и $v_n(x) \xrightarrow[n \to \infty]{} g(x)$.
    Тогда
    \begin{multline*}
        \int_E (u_n + v_n) dm = I_E(u_n + v_n) = I_E(u_n) + I_E(v_n) =\\
        = \int_E u_n dm + \int_E v_n dm \tag{26}
    \end{multline*}
    по теореме Беппо Леви
    \begin{gather*}
        \int_E (u_n + v_n) dm \to \int_E (f + g) dm \\
        \int_E u_n dm \to \int_E f dm \\
        \int_E v_n dm \to \int_E g dm
    \end{gather*}
    и (26) влечет соответствующее свойство.
\end{proof}
\begin{remark}
    Свойство сформулировано только для положительных функций, однако выполнено для любых.
\end{remark}

\marginpar{18.05.23}
\begin{theorem}
    Пусть имеются функции $v_n: E \to \R$, при этом $v_n(x) \ge 0$ $\forall x \in E$, $\forall n$.
    И пусть
    \[S(x) = \sum_{n=1}^{\infty} v_n(x),\]
    если ряд расходится, то сумма полагается равной $+\infty$, иначе
    \[\int_E S dm  = \sum_{n=1}^{\infty} \int_E v_n dm\]
\end{theorem}
\begin{proof}
    Положим, что
    \[S_n(x) = \sum_{k=1}^{n} v_k(x),\]
    понятно, что $0 \le S_n(x) \le S_{n+1}(x)$.
    По теореме об аддитивности интеграла Лебега имеем
    \[\int_E S_n dm = \sum_{k=1}^{n} \int_E v_k dm,\]
    тогда по теореме Беппо Леви
    \[\int_E S_n dm \xrightarrow[n \to \infty]{} \int_E S dm, \]
    т.к.
    \[\sum_{k=1}^{n} \int_E v_k dm \xrightarrow[n \to \infty]{} \sum_{k=1}^{\infty} \int_E v_k dm\]
\end{proof}

\begin{theorem}[Лебега о мажорируемой сходимости]
    Пусть имеется множество $E \subset \R^n$, а так же функция $g(x) \ge 0$ и $g \in \calL(E)$.
    Допустим, что имеется последовательность функций $\{f_n(x)\}_{n=1}^\infty$, которые удовлетворяют условию $|f_n(x)| \le g(x)$ $\forall n$ $\forall x \in E$.
    Предположим, что
    \[\forall x\ f_n(x) \xrightarrow[n \to \infty]{} f(x),\]
    тогда $f \in \calL(E)$ и
    \[\int_E f_n dm \xrightarrow[n \to \infty]{} \int_E f dm\]
\end{theorem}

\begin{theorem}[Связь интеграла Римана и интеграла Лебега]
    Пусть функция $f \in \mathcal{R}((a,b))$\footnote{интегрируема по Риману}, тогда $f$ измерима на $(a, b)$, $f \in \calL((a,b))$ и
    \[\int_{a}^{b} f(x) dx = \int_{(a, b)} f dm\]
\end{theorem}
Все интегрируемые по Риману функции интегрируемы по Лебегу.
Обратное неверно.
\begin{example}
    Функция Дирихле:
    \[\chi(x) = \begin{cases}
            1, & x \in (0, 1), x \text{ рациональное}   \\
            0, & x \in (0, 1), x \text{ иррациональное}
        \end{cases}\]
    Интеграл Римана от нее не существует.
    С другой стороны, функция принимает всего два значения; множество рациональных чисел счетно, где каждая точка имеет меру 0, поэтому мера рациональных чисел тоже 0, и тогда получаем, что
    \[\int_{(0, 1)} \chi dm = \int_{(0, 1) \cap \Q} \chi dm + \int_{(0, 1) \setminus \Q} \chi dm = 0 + \int_{(0, 1) \setminus \Q} 0 dm = 0,\]
    поэтому интеграл Лебега является расширением интеграла Римана.
\end{example}

\section{Теорема Фубини}

Имеется множество $E \subset \R^{n+k}$, где $n \ge 1$, $k \ge 1$.

Введем обозначение п.в.$(m)$, которое означает, что какое-то свойство выполнено почти всюду относительно меры Лебега в пространстве $\R^m$.
А также $\M_m$, которое означает множество всех множеств, измеримых по Лебегу в $\R^m$.

Пусть имеется точка $M \in \R^{n+k}$, будем записывать $M(x, y)$, где $x \in \R^n$ и $y \in \R^k$.

Пусть $y \in \R^k$, обозначим
\[E(\cdot, y) = \{x \in \R^n: (x, y) \in E\},\]
пусть $x \in \R^n$, обозначим
\[E(x, \cdot) = \{y \in \R^k: (x, y) \in E\},\]
любое из этих множеств может быть пустым.

Пусть имеется функция $f: E \to \R$ и $y \in \R^k$, тогда
\[\phi_y(x) = \begin{cases}
        \text{не определена}, & E(\cdot, y) = \varnothing                                   \\
        f(x, y),              & E(\cdot, y) \neq \varnothing \text{ при } x \in E(\cdot, y)
    \end{cases},\]
аналогично, если $x \in \R^n$, тогда
\[\psi_x(y) = \begin{cases}
        \text{не определена}, & E(x, \cdot) = \varnothing                                   \\
        f(x, y),              & E(x, \cdot) \neq \varnothing \text{ при } y \in E(x, \cdot)
    \end{cases}\]

\begin{proposition}
    Пусть $E \in \M_{n+k}$, тогда при п.в.$(k)$ $y$ $E(\cdot, y) \in \M_n$, а так же при п.в.$(n)$ $x$ $E(x, \cdot) \in \M_k$.
\end{proposition}

\begin{proposition}
    Пусть функция $f$ $(n+k)$ измерима на $E$, тогда при п.в.$(k)$ $y$ $\phi_y$ $(n)$ измерима на $E(\cdot, y)$, а так же при п.в.$(n)$ $x$ $\psi_x$ $(k)$ измерима на $E(x, \cdot)$.
\end{proposition}

\begin{proposition}
    Пусть функция $f \in \calL_{n+k}(E)$, тогда при п.в.$(k)$ $y$ $\phi_y \in \calL_n(E(\cdot, y))$, а так же при п.в.$(n)$ $x$ $\psi_x \in \calL_k(E(x, \cdot))$.
\end{proposition}

Обозначим $\Omega \subset \R^k$ как множество таких $y$, что $E(\cdot, y) \neq \varnothing$ и $\phi_y \in \calL(E(\cdot, y))$, аналогично обозначим множество $G$ как множество таких $x$, что $E(x, \cdot) \neq \varnothing$ и $\psi_x \in \calL(E(x, \cdot))$.

Определим функцию
\[\Psi (y) = \int_{E(\cdot, y)} \phi_y dm_n\]
при $y \in \Omega$.
Аналогично определим
\[\Phi(x) = \int_{E(x, \cdot)} \psi_x dm_k\]
при $x \in G$.

Тогда $\Omega \in \M_k$ и $G \in \M_n$ и справедлива следующая формула:
\[
    \int_E f dm_{n+k} = \int_\Omega \Psi dm_k = \int_G \Phi dm_n
\]
Это и есть основное утверждение теоремы Фубини.

Из этой теоремы следует теорема о несобственном интеграле, зависящем от параметра, за прошлый семестр.
\end{document}
