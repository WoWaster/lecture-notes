% !TeX root = ./main.tex
\documentclass[main]{subfiles}
\begin{document}
\chapter{Формула Коши}

\begin{theorem}
    Имеется область $\Omega \subset \C$, $a \in \Omega$, $\overline{B_r}(a) = \{z: |z-a| \le r\} \subset \Omega$~--- замкнутый круг.
    Допустим, что $f \in A(\Omega)$.
    Пусть $z_0 \in B_r(a)$, $\gamma_r(a) = \{z: |z-a| = r\}$~--- окружность.
    Тогда справедлива формула Коши
    \[f(z_0) = \frac{1}{2 \pi i} \int_{\overrightarrow{\gamma_r(a)}} \frac{f(z)}{z - z_0} dz \tag{1}\]
\end{theorem}

\begin{lemma}
    Пусть $\rho > 0$, $\gamma_\rho(z_0)$~--- окружность, тогда
    \[\int_{\overrightarrow{\gamma_\rho(z_0)}} \frac{dz}{z - z_0} = 2 \pi i \tag{2} \]
\end{lemma}

\begin{proof}
    Окружность параметризуется как
    \[\overrightarrow{\gamma_\rho(z_0) } = \{z: z = z_0 +\rho(\cos \theta + i \sin \theta)\}, 0 \le \theta \le 2 \pi \tag{3}\]
    $(3) \implies$ при $z \in \gamma_\rho(z_0)$:
    \[z - z_0 = \rho(\cos \theta + i \sin \theta) \tag{4}\]
    Окружность является гладкой кривой, поэтому (2), (4) и формула (7) от 16.02 влекут
    \begin{multline*}
        \int_{\overrightarrow{\gamma_r(z_0)}} \frac{dz}{z - z_0} = \int_{0}^{2\pi} \frac{\rho(-\sin\theta + i \cos \theta)}{\rho(\cos \theta + i \sin \theta)} d\theta =\\
        = i \int_{0}^{2 \pi} \frac{cos \theta + i \sin \theta}{cos \theta + i \sin \theta} d\theta = 2 \pi i
    \end{multline*}
\end{proof}

\begin{longProof}[теоремы]
    Выберем $0< \rho < r - |z_0 - a|$.
    Рассмотрим область $D_\rho = B_r(a) \setminus \overline{B_\rho}(z_0)$: $\partial D_\rho = \gamma_r(a) \cup \gamma_\rho(z_0)$
    \begin{center}
        \import{figures}{cauchy_formula_D_rho.pdf_tex}
    \end{center}
    и функцию
    \[\phi (z) = \frac{f(z)}{z - z_0} \tag{5}\]
    Функция $f(z) \in A(\Omega)$, поэтому  $\phi \in A(\Omega \setminus \{z_0\})$.
    Понятно, что $\overline{D_\rho} \subset \Omega$, применим теорему Коши
    \[\int_{\overrightarrow{\partial D_\rho}} \phi(z) dz = 0 \tag{6}\]
    Будем считать, что на $\partial D_\rho$ задана положительная ориентация.
    \begin{multline*}
        (6) \implies \int_{\overrightarrow{\partial D_\rho}} \phi(z) dz = \int_{\overrightarrow{\gamma_r(a)}} \phi(z) dz + \int_{\overleftarrow{\gamma_\rho(z_0)}} \phi(z) dz = \\
        = \int_{\overrightarrow{\gamma_r(a)}} \phi(z) dz  - \int_{\overrightarrow{\gamma_\rho(z_0)}} \phi(z) dz = 0 \tag{7}
    \end{multline*}
    Перенесем последний интеграл в другую сторону
    \[(7) \Leftrightarrow  \int_{\overrightarrow{\gamma_r(a)}} \phi(z) dz  = \int_{\overrightarrow{\gamma_\rho(z_0)}} \phi(z) dz \tag{8}\]
    теперь займемся интегралом в правой части
    \begin{multline*}
        (6) \implies \int_{\overrightarrow{\gamma_\rho(z_0)}} \phi(z) dz = \int_{\overrightarrow{\gamma_\rho(z_0)}}  \frac{f(z)}{z - z_0} dz = \\
        = \int_{\overrightarrow{\gamma_\rho(z_0)}} \frac{f(z) - f(z_0)}{z - z_0} dz + \int_{\overrightarrow{\gamma_\rho(z_0)}} \frac{f(z_0)}{z - z_0} dz \underset{(2)}{=} \\
        \underset{(2)}{=} \int_{\overrightarrow{\gamma_\rho(z_0)}} \frac{f(z) - f(z_0)}{z - z_0} dz + 2 \pi i f(z_0) \tag{9}
    \end{multline*}
    поскольку $f(z)$~--- аналитична, то можем применить формулу для $f(z) - f(z_0)$
    \begin{gather*}
        f(z) - f(z_0) = f'(z_0) (z- z_0) + S(z) \tag{10} \\
        \frac{|S(z)|}{|z-z_0|} \xrightarrow[z \to z_0]{} 0 \tag{11} \\
        (11) \implies \exists \delta >0 \text{ т.ч. } \forall z: |z-z_0| \le \delta \text{ имеем } \frac{|S(z)|}{|z-z_0|} < 1 \tag{12}
    \end{gather*}
    тогда (10), (11), (12) влекут, что при $\rho < \delta$, $|z - z_0| = \rho$ имеем
    \begin{multline*}
        |f(z) - f(z_0)| \le |f'(z_0)| \cdot |z - z_0| + |S(z)| < \\
        < |f'(z_0)| \cdot  |z - z_0| + |z - z_0| = (|f'(z_0)| + 1)|z - z_0| \tag{13}
    \end{multline*}
    По свойству \ref{1:LessThanLength}, (13) влечет, при $0 < \rho < \delta$ имеем
    \begin{multline*}
        \left| \int_{\overrightarrow{\gamma_\rho(z_0)}} \frac{f(z) - f(z_0)}{z - z_0} dz \right| \le \int_{\gamma_\rho(z_0)} \left| \frac{f(z) - f(z_0)}{z - z_0} \right| dl(z) \underset{(13)}{\le} \\
        \underset{(13)}{\le} \int_{\gamma_\rho(z_0)} \frac{(|f'(z_0)| + 1) |z - z_0|}{|z - z_0|} dl(z)  = (|f'(z_0)| + 1) dl(\gamma_\rho(z_0)) =\\
        = \rho \cdot 2 \pi (|f'(z_0)| + 1) \tag{14}
    \end{multline*}
    в выкладке (9) введем обозначение
    \[C (\rho) = \int_{\overrightarrow{\gamma_\rho(z_0)}} \frac{f(z) - f(z_0)}{z - z_0} dz\]
    а в выкладке (14) $A = 2 \pi (|f'(z_0)| + 1)$, тогда
    \begin{gather*}
        (14) \implies \exists \delta > 0 \text{ т.ч. при } 0 < \rho < \delta \text{ имеем } |C(\rho)| < A \rho \tag{15}  \\
        (6), (8), (9) \implies \int_{\gamma_r(a)} \frac{f(z)}{z - z_0} dz - 2 \pi i f(z_0) = C(\rho) \tag{16}
    \end{gather*}
    теперь возьмем $\forall \epsilon > 0$, $\rho < \epsilon / A$ и $\rho < \delta$, и при таких $\rho$
    \[(16) \implies \left| \int_{\gamma_r(a)} \frac{f(z)}{z - z_0} dz - 2 \pi i f(z_0) \right| = |C(\rho)| \le A \frac{\epsilon}{A} = \epsilon \tag{17}\]
    $(17) \implies (1)$
\end{longProof}

\section{Теорема о частных производных криволинейного интеграла второго рода, зависящего от параметра}
\marginpar{16.03.23}
\begin{theorem}
    Имеется гладкая ориентированная кривая $\overrightarrow{\Gamma}: [a, b] \to \C$.
    А так же, имеется функция $f(\zeta, x,y)$, где $\zeta \in \Gamma$, $(x,y) \in D \subset \R^2$.
    Кроме того имеется прямоугольник $Q$:
    \[\overline{Q} = \{\alpha \le x \le \beta,  \gamma\le y \le \delta\}\]
    При этом $\overline{Q} \subset D$.
    Предположим, что $f \in C(\Gamma \times D)$ и
    \begin{align*}
         & \forall \zeta \in \Gamma\ \forall (x,y) \in \overline{Q} &  & \exists f'_x (\zeta, x,y) \in C(\Gamma \times \overline{Q}) \\
         &                                                          &  & \exists f'_y (\zeta, x,y) \in C(\Gamma \times \overline{Q})
    \end{align*}

    Рассмотрим $F (x,y) = \int_{\overrightarrow{\Gamma}} f(\zeta, x, y)d \zeta$, тогда
    \begin{align*}
         & \forall (x,y) \in \overline{Q} &  & \exists F'_x(x,y) = \int_{\overrightarrow{\Gamma}} f'_x (\zeta, x,y) d \zeta \tag{1} \\
         &                                &  & \exists F'_y(x,y) = \int_{\overrightarrow{\Gamma}} f'_y (\zeta, x,y) d \zeta \tag{2}
    \end{align*}
\end{theorem}
\begin{proof}
    Поскольку $\Gamma$~--- гладкая, то можем записать криволинейный интеграл второго рода, как обычный интеграл.
    Предположим, что $\overrightarrow{\Gamma} = \{u(t) + i v(t): t\in [a,b]\}$, тогда
    \[F(x,y) = \int_{a}^{b} f(u(t) + i v(t),x,y)(u'(t) + i v'(t))dt \tag{3} \]
    Введем комплексно-значную функцию $\Phi$:
    \[\Phi (t, x, y) = f(u(t) + i v(t),x,y)(u'(t) + i v'(t))\]
    Она непрерывна и имеет непрерывные частные производные, когда $(x,y) \in Q$.
    Если мы зафиксируем $y$ и будем рассматривать $\Phi$ как функцию от $x$ на соответствующем промежутке, то ее вещественная и мнимая часть удовлетворяют условиям теоремы о существовании производных.
    Частные производные у $\Phi$ существуют, если они существуют, соответственно у $f$.
    Аналогично, если фиксировать $x$.
    Тогда (3) и теорема из 3-его семестра влекут
    \begin{multline*}
        F'_x (x,y) = \\
        = \int_{a}^{b} f'_x(u(t) + i v(t),x,y)(u'(t) + i v'(t)) dt = \\
        = \int_{\overrightarrow{\Gamma}} f'_x(\zeta, x, y) d\zeta \implies (1)
    \end{multline*}
    Аналогично следует (2)
\end{proof}
\end{document}
