% !TeX root = ./main.tex
\documentclass[main]{subfiles}
\begin{document}
\chapter[Дальнейшие свойства аналитических функций][Свойства аналитических функций]{Дальнейшие свойства аналитических функций}
\section{Теорема о бесконечной гладкости аналитической функции}
\begin{theorem}\label{4:theorem_with_defs}
    Пусть имеется $B = \{ z: |z - a| < R\}$, и функция $f \in A(B)$, тогда $\forall r \ge 1\ f\in C^r (B)$.
\end{theorem}
\begin{remark}
    Напоминание: если $g \in A(\Omega)$, где $\Omega$~--- область, тогда
    \[\begin{rcases}
            g'_z(z) = g'(z) \\
            g'_{\overline{z}}(z) = 0
        \end{rcases} \implies g'_x = g', g'_y = ig'\]
\end{remark}
\begin{proof}
    Возьмем любое $0< r < R$.
    Пусть $\overrightarrow{\gamma_r} = \{z: |z - a| = r\}$~--- окружность, а
    $B_r = \{z: |z-a|< r\}$~--- открытый круг, где $a = p + iq$ и $B_r^* = \{(x,y) : (x-p)^2 + (y-q)^2 < r^2\}$.
    Для таких $z$ справедливо соотношение
    \[f(z) = \frac{1}{2 \pi i} \int_{\overrightarrow{\gamma_r}} \frac{f(\zeta)}{\zeta - z} d\zeta \tag{4}\]
    Пусть $z = x + iy$, тогда рассмотрим выражение
    \[\frac{f(\zeta)}{\zeta - x - iy}\]
    Тогда может взять любой $z$ и любой $Q$, т.ч. $\overline{Q} \subset B_r^*$ и $(x,y) \in \overline{Q}$.
    При фиксированном $\zeta \neq z$ выполнено:
    \begin{gather*}
        \left( \frac{1}{\zeta - z} \right)' = \frac{1}{(\zeta - z)^2} \\
        \left( \frac{1}{\zeta - z} \right)^{(n)} \coloneq \left(\left(\frac{1}{\zeta - z}\right)^{(n-1)} \right)' = \frac{n!}{(\zeta - z)^{(n+1)}} \tag{5}\\
        (5) \implies \left(\frac{1}{\zeta - x -iy}\right)^{(m+n)}_{\underbrace{x \dotsc x}_m \underbrace{y \dotsc y}_n} = i^n \frac{(m+n)!}{(\zeta -x -iy)^{m+n+1}} \tag{6}
    \end{gather*}
    Тогда (1), (2), (4), (6) $\implies$
    \[f^{(m+n)}_{\underbrace{x \dotsc x}_m \underbrace{y \dotsc y}_n}(z) = i^n \frac{(m+n)!}{2 \pi i} \int_{\overrightarrow{\gamma_r}} \frac{f(\zeta)}{(\zeta - x -iy)^{m+n+1}} \tag{7}\]
    (7) выполнена $\forall z \in B_r$ (8), т.к. $r$ произвольно, (8) влечет, что (7) выполнено $\forall z \in B$.
    Строго говоря, доказательство нужно вести через индукцию по $(m+n)$.
\end{proof}
\begin{corollary}
    Пусть $f \in A(\Omega)$, тогда $\forall r \ge 1\ f\in C^r(\Omega)$.
\end{corollary}
\begin{proof}
    Возьмем любое $z_0 \in \Omega$, $\exists B = \{z: |z - z_0| < R\}$, $B\subset \Omega$ $\implies f \in A(B)$.
\end{proof}

\section{Теорема об аналитичности производной аналитической функции}
\begin{theorem}
    В обозначениях прошлой теоремы
    \[f \in A(B) \implies f' \in A(B) \tag{9}\]
\end{theorem}
\begin{proof}
    Рассмотрим $f'(z) = f'_x(z)$, то есть $m = 1, n = 0$, тогда
    \[(7) \implies f'(z) = \frac{1}{2 \pi i}  \int_{\overrightarrow{\gamma_r}} \frac{f(\zeta)}{(\zeta - z)^{2}} d \zeta \tag{10}\]
    \begin{multline*}
        (10) \implies (f'(z))'_x + i (f'(z))'_y = \\
        = \frac{1}{2 \pi i} \int_{\overrightarrow{\gamma_r}} f(\zeta) \left(\left(\frac{1}{(\zeta - z)^{2}}\right)'_x + i \left(\frac{1}{(\zeta - z)^{2}}\right)'_y\right) d\zeta = \\
        = \frac{1}{2 \pi i} \int_{\overrightarrow{\gamma_r}} f(\zeta) \cdot 2 \cdot \left(\frac{1}{(\zeta - z)^{2}}\right)'_{\overline{z}} d\zeta = 0
    \end{multline*}
    Так как в силу аналитичность производная по $\overline{z}$ равна нулю.
    Вообще говоря, эта формула верна для $B_r$, но поскольку мы можем брать $r$ сколько угодно близкое к $R$, то если мы возьмем $z$ из $B$, то мы можем взять такое $r$, чтобы $z$ лежала в $B_r$, и тогда формула верна в окрестности $z$ вследствие выбора $r$.
\end{proof}
\begin{corollary}
    Пусть $f \in A(\Omega)$, тогда $f' \in A(\Omega)$.
\end{corollary}
\begin{proof}
    Возьмем любое $z_0 \in \Omega$, тогда $\exists B = \{z: |z - z_0| < R\}, B \subset \Omega$, и по теореме $f'\in A(B)$.
    Поскольку свойство локальное, получили, что $f$ аналитична в окрестности любой точки из $\Omega$.
\end{proof}

\section{Формула для \texorpdfstring{$n$}{n}-ной производной аналитической функции}

\begin{definition}[Любая производная аналитической функции]
    Есть открытое множество $\Omega$ и $f \in A(\Omega)$.
    По предыдущему следствию $f' \in A(\Omega)$, это влечет, что
    \begin{gather*}
        \exists (f')'(z), z \in \Omega
        \intertext{по определению полагаем}
        f''(z) \coloneqq (f')'(z) \tag{11}
        \intertext{вторая производная~--- производная некоторой аналитической функции, поэтому}
        (11) \implies f'' \in A(\Omega) \tag{12}
        \intertext{раз она аналитична в $\Omega$, значит у нее есть производная}
        (12) \implies \forall z \in \Omega \exists (f'')' (z) \coloneqq f'''(z) \tag{13}
        \intertext{и так далее, по индукции}
        n \ge 3 \ f^{(n)} (z) \in A(\Omega) \tag{14}\\
        (14) \implies \forall z \in \Omega \exists (f^{(n)})' = f^{(n+1)}(z) \tag{15}\\
        (15) \implies f^{(n+1)} \in A(\Omega)
    \end{gather*}
\end{definition}
Опять вернемся к обозначениям из \ref{4:theorem_with_defs}.
Запишем формулу Коши для $z \in B_r$:
\[f(z) = \frac{1}{2 \pi i} \int_{\overrightarrow{\gamma_r}} \frac{f(\zeta)}{\zeta - z} d\zeta\]
При $m=1, n=0$ формула (7) влечет
\[f'(z) = f'_x(z) = \frac{1}{2 \pi i} \int_{\overrightarrow{\gamma_r}} \frac{f(\zeta)}{(\zeta - z)^2} d\zeta \tag{10}\]
Мы уже знаем, что $f'$ аналитична, поэтому это равенство можно использовать и дальше.
Теперь (10) влечет
\begin{multline*}
    f''(z) = (f')'(z) = (f')'_x(z) = \\
    = \left(\frac{1}{2 \pi i} \int_{\overrightarrow{\gamma_r}} \frac{f(\zeta)}{(\zeta - x - iy)^2} d\zeta \right)'_x = \frac{1}{2 \pi i} \int_{\overrightarrow{\gamma_r}} f(\zeta) \left(\frac{1}{(\zeta - x - iy)^2}\right)'_x d\zeta = \\
    = \frac{2}{2 \pi i} \int_{\overrightarrow{\gamma_r}} \frac{f(\zeta)}{(\zeta - x - iy)^3} d\zeta = \frac{2}{2 \pi i} \int_{\overrightarrow{\gamma_r}} \frac{f(\zeta)}{(\zeta - z)^3} d\zeta \tag{16}
\end{multline*}
Далее по индукции при $n \ge 2$, $z \in B_r$ предполагаем, что
\[f^{(n)}(z) = \frac{n!}{2 \pi i} \int_{\overrightarrow{\gamma_r}} \frac{f(\zeta)}{(\zeta - z)^{n+1}} d\zeta \tag{17} \]
Тогда (17) влечет
\begin{multline*}
    f^{(n+1)}(z) = (f^{(n)})'(z) = (f^{(n)})'_x(z) = \\
    = \frac{n!}{2 \pi i} \int_{\overrightarrow{\gamma_r}} f(\zeta) \left(\frac{1}{(\zeta - x - iy)^{n+1}}\right)'_x d\zeta = \\
    = \frac{(n+1)!}{2 \pi i} \int_{\overrightarrow{\gamma_r}} \frac{f(\zeta)}{(\zeta - z)^{n+2}} d\zeta
\end{multline*}

\section{Теорема о разложении аналитической функции в степенной ряд}

Возьмем $z = a$, тогда при $n \ge 1$
\begin{gather*}
    (17) \implies f^{(n)} (a) = \frac{n!}{2 \pi i} \int_{\overrightarrow{\gamma_r}} \frac{f(\zeta)}{(\zeta - a)^{(n+1)}} \tag{18}
\end{gather*}
\begin{theorem}
    Все еще пользуемся обозначениями из \ref{4:theorem_with_defs}.
    Пусть $f \in A(B)$, тогда $\forall z \in B$ справедливо равенство
    \[f(z) = f(a) + \sum_{n=1}^{\infty} \frac{f^{(n)}  (a)}{n!}(z-a)^n \tag{19}\]
\end{theorem}
\begin{longProof}
    Зафиксируем $z$ и выберем $r: |z - a| < r < R$.
    Функция $f \in C(\gamma_r)$, $\gamma_r$~--- компакт, поэтому по первой теореме Вейерштрасса, которую мы отдельно применим к вещественной и мнимой частям, функция на компакте ограниченна.
    Поэтому
    \[\exists M_r: |f(z)| \le  M_r,\ z \in \gamma_r \tag{20}\]
    Теперь применим формулу Коши к $z$ и $\gamma_r$:
    \[f(z) = \frac{1}{2 \pi i} \int_{\overrightarrow{\gamma_r}} \frac{f(\zeta)}{\zeta - z} d\zeta \tag{4}\]
    Теперь запишем следующее выражение:
    \begin{multline*}
        \frac{1}{\zeta - z} = \frac{1}{(\zeta - a) - (z - a)} = \\
        = \frac{1}{\zeta - a} \cdot \frac{1}{1 - \frac{z - a}{\zeta - a}} = \frac{1}{\zeta - a} \sum_{n = 0 }^{\infty} \left(\frac{z - a}{\zeta -a}\right)^n = \\
        = \frac{1}{\zeta - a} + \frac{1}{\zeta - a} \sum_{n = 1 }^{\infty} \left(\frac{z - a}{\zeta -a}\right)^n \tag{21}
    \end{multline*}
    Введем обозначение $q = \frac{|z - a|}{r} < 1$.
    При $\zeta \in \gamma_r$ справедливы соотношения:
    \begin{gather*}
        \left| \frac{z - a}{\zeta - a}\right| = q\\
        \left| \frac{f(\zeta)}{\zeta - a} \cdot \left( \frac{z - a}{\zeta - a} \right)^n \right| \le \frac{M_r}{r} \cdot q^n \tag{22}
    \end{gather*}
    Тогда (4), (22) влекут
    \begin{multline*}
        f(z) = \frac{1}{2 \pi i} \int_{\overrightarrow{\gamma_r}} \frac{f(\zeta)}{\zeta - a} d\zeta + \frac{1}{2 \pi i} \int_{\overrightarrow{\gamma_r}} f(\zeta) \left(\frac{1}{\zeta - a}  \sum_{n = 1 }^{\infty} \left(\frac{z - a}{\zeta -a}\right)^n\right)d\zeta = \\
        = f(a) + \sum_{n=1}^{\infty} (z-a)^n \cdot \frac{1}{2 \pi i} \int_{\overrightarrow{\gamma_r}} \frac{f(\zeta)}{(\zeta - a)^{n+1}} d\zeta
    \end{multline*}
    Здесь мы интегрируем целый ряд.
    Заметим, что
    \[f(\zeta) \left(\frac{1}{\zeta - a}  \sum_{n = 1 }^{\infty} \left(\frac{z - a}{\zeta -a}\right)^n\right)\]
    сходится равномерно по $\zeta$ по признаку Вейерштрасса (следует из (22)).
    В итоге, полученный ряд состоит из слагаемых, как в правой части (18), что влечет (19).
\end{longProof}
\begin{remark}
    В прошлом семестре были рассуждения о сходимости степенных рядов.
    Здесь получили, что данный ряд, который называется рядом Тейлора для аналитической функции $f$, сходится для любого $z$ из $B$.
    Тогда в любом замкнутом круге меньшем по радиусу, чем $B$, он сходится равномерно.

    Принципиальный момент: если функция $f$ аналитична в круге, то она раскладывается в нем в степенной ряд.
    В конце прошлого семестра был получен результат, что если степенной ряд сходится в круге, то он является аналитической функцией.

    Таким образом получается, что если есть функция, заданная в круге, то условие того, что она аналитична в круге и того, что она равна сумме степенного ряда, это эквивалентные условия.
    Это еще одно важное свойство аналитической функции.
\end{remark}

\subsection[Разложение элементарных функций в ряды в комплексной плоскости]{Разложение элементарных функций в ряды в комплексной плоскости\footnote{За основу конспекта за эту дату, взят конспект от 2021 года авторства Антона Чижова}}\marginpar{23.03.23}

\begin{enumerate}
    \item $f_1(z) = e^z \in A(\C)$, поэтому степенной ряд для $f_1$ сходится в $\C$.
          Далее, при $x\in\R$
          \[ (f_1(x))^{(n)} = {f_1}_{\underbrace{x \dotsc x}_n}^{(n)}(x) = e^x,\  (f_1(z))^{(n)}_{|_{z=0}}=1, \]
          отсюда следует
          \[ e^z = 1 + \sum_{n=1}^\infty \frac{z^n}{n!},\ z\in\C \]
    \item $f_2(z) = \cos{z}$, аналогично $(\cos{z})^{(n)} = (\cos{z})^{(n)}_{\underbrace{x \dotsc x}_n}$, поэтому
          \[ \cos{z} = 1 + \sum_{n=1}^\infty \frac{(-1)^n}{(2n)!} z^{2n},\ z \in \C \]
    \item $f_3(z) = \sin{z}$, аналогично 1. и 2.
          \[ \sin{z} = \sum_{n=1}^\infty \frac{(-1)^{n-1}}{(2n-1)!} z^{2n-1},\ z \in \C \]
    \item $f_4(z) = \ln{(1 + z)},\ |z|<1$, полагаем
          \[\ln{(1 + z)} = \ln{|1 + z|} + i \arg(1 + z),\]
          $|\arg(1 + z)| < \pi$ при $|z| < 1$.
          Опять пользуемся равенством $(\ln{(1 + z)})^{(n)}=(\ln{(1 + z)})^{(n)}_{\underbrace{x \dotsc x}_n}$, при $z=x$, $-1 < x < 1$ имеем
          \[(\ln{(1 + x)})^{(n)}_{x \dotsc x}=(-1)^{n-1}\cdot(n-1)!(1+x)^{-n},\]
          поэтому
          \[\ln{(1 + z)} = \sum_{n=1}^\infty \frac{(-1)^n}{n} z^n,\]
          $|z|<1$, $\ln{(1+x)}\in\R$ при $-1<x<1$.
    \item $f_5(z) = (1 + z)^r$, $r \in \R \setminus \N$, $r \neq 0$.
          Аналогично 4. получаем
          \[(1 + z)^r = 1 + rz + \frac{r(r - 1)}{2!} z^2 + \dotsb + \frac{r(r - 1) \dotsm (r - n + 1)}{n!} z^n + \dotsb \]
          при $|z| < 1$, $(1 + x)^r > 0$ при $(-1 < x < 1)$
    \item Пусть $\alpha \in \C \setminus \R$, положим при $|z| < 1$
          \[(1 + z)^\alpha \coloneq e^{\alpha \ln{(1 + z)}},\]
          где $\ln(1 + z)$ определена в 4.
          Тогда
          \begin{multline*}
              \left( (1 + z)^\alpha \right)' = e^{\alpha \ln{(1 + z)}}  \cdot (\alpha \ln{(1 + z)})' = \alpha e^{\alpha \ln{(1 + z)}} \frac{1}{1 + z}         \\
              = \alpha e^{\alpha \ln{(1 + z)}} e^{-\ln{(1 + z)}} = \alpha e^{(\alpha - 1) \ln{(1 + z)}} = \alpha (1 + z)^{\alpha - 1}
          \end{multline*}
          и аналогично получаем
          \[ \left( (1 + z)^\alpha \right)^{(n)} = \alpha (\alpha - 1) \dotsm (\alpha - n + 1) (1 + z)^{\alpha - n}, \]
          поэтому при $|z|<1$ имеем
          \[ (1 + z)^\alpha = 1 + \alpha z + \frac{\alpha(\alpha - 1)}{2!} z^2 + \dotsb + \frac{\alpha(\alpha - 1) \dotsm (\alpha - n + 1)}{n!} z^n+ \dotsb \]
\end{enumerate}

\section{Теоремы единственности для аналитических функций.}
\begin{theorem}[с производными функциями]
    Пусть $G\subset\C$~--- область, $f\in A(G)$, $a \in G$, $f(a) = 0$, $f^{(n)}(a) = 0$, $\forall n \in \N$.
    Тогда $f(z)\equiv 0$ в $G$.
\end{theorem}
\begin{longProof}
    Пусть
    \[E = \{z \in G: f(z) = 0 \text{ и } f^{(n)}(z) = 0\ \forall n \in \N\}.\]
    Тогда $a \in E \implies E \neq \varnothing$.

    \begin{description}
        \item[Множество $E$ относительно замкнуто в $G$:]
            Пусть $z_m\in E$,
            \[z_m \xrightarrow[m\to\infty]{} z_*,\]
            $z_*\in G$.
            Тогда в силу $f \in C^\infty(G)$, $f^{(n)} \in C^\infty(G)$ имеем
            \begin{gather*}
                f(z_m) \to f(z_*),\\
                f^{(n)}(z_m) \to f^{(n)}(z_*)\ \forall n\in\N.
            \end{gather*}
            Но $z_m \in E \implies f(z_m)=0$, $f^{(n)}(z_m) = 0 \implies\ 0 \xrightarrow[m\to\infty]{} f(z_*)$, $0 \xrightarrow[m\to\infty]{} f^{(n)}(z_*) \implies f(z_*) = f^{(n)}(z_*) = 0$, т.е. $z_* \in E$.
        \item[Множество $E$ относительно открыто в $G$:]
            Пусть $z_0 \in E$, тогда $\exists r > 0: \{ z : |z - z_0| < r\} \subset G$, поэтому при $|z - z_0| < r$ имеем разложение в ряд:
            \[ f(z) = f(z_0) + \sum_{n=1}^\infty \frac{f^{(n)}(z_0)}{n!} (z - z_0)^n \equiv 0, \]
            поскольку $z_0 \in E$, поэтому $\{ z : |z - z_0| < r\} \subset E$.
    \end{description}
    Поскольку $G$ связно, $E \neq \varnothing$, $E$ относительно открыто и замкнуто в $G$, то $E = G$, т.е. $f(z) = 0\ \forall z \in G$.
\end{longProof}

\begin{theorem}[со значениями функции]
    Пусть $G$~--- область, $E \subset G$, $z_*$~--- точка сгущения $E$, $z_* \in G$, $f\in A(G)$, $f(z_0) = 0\ \forall z_0 \in E$.
    Тогда $f(z) \equiv 0$ в $G$.
\end{theorem}

\begin{longProof}
    Поскольку $z_*$~--- точка сгущения $E$, то $\exists \{z_m\}_{m = 1}^\infty$, $z_m \in E$ и $z_m \xrightarrow[m\to\infty]{} z_*$, поэтому $f(z_m) \to f(z_*)$, $0\to f(z_*) \implies f(z_*) = 0$.

    Если $f^{(n)}(z_*) = 0\ \forall n \in \N$, то по предыдущей теореме $f(z) \equiv 0$.
    Предположим, что $\exists n_0 \in \N$ т.ч. $f^{(n_0)}(z_*) \neq 0$.
    Пусть $n_0$~--- наименьшее такое число, т.ч. $f^{(k)}(z_0) = 0$ при $1 \le  k \le n_0 - 1$; если $n_0 = 1$, то $f'(z_0) \neq 0$.

    Выберем $r > 0$ так, чтобы $\{z:|z - z_*| \le r\} \subset G$, пусть
    \[M_f(r)=\max_{z:|z-z_*|=r} |f(z)|.\]
    Тогда при $|z - z_*| \le r$ имеем равенство
    \begin{multline*}
        f(z) = f(z_*) + \sum_{n = 1}^\infty \frac{f^{(n)}(z_*)}{n!} (z-z_*)^n = \sum \limits_{n = n_0}^\infty \frac{f^{(n)} (z_*)}{n!} (z-z_*)^n = \\
        = \frac{f^{(n_0)}(z_*)}{n_0!} (z-z_*)^n + \sum_{n = n_0 + 1}^\infty \frac{f^{(n)}(z_*)}{n!} (z - z_*)^n\tag{1}
    \end{multline*}
    Неравенство (22) в лекции от $16.03.23$ влечёт %???
    \[ \left| \frac{f^{(n)}(z_*)}{n!} \right| \le \frac{M_f(r)}{r^n},\ n \in \N,\tag{2} \]
    поэтому при $|z - z_*| =  \delta r$, $0 < \delta < 1$, из (2) получаем
    \begin{multline*}
        \left| \sum_{n = n_0 + 1}^\infty \frac{f^{(n)}(z_*)}{n!} (z-z_*)^n \right| \le \sum_{n = n_0 + 1}^\infty \left| \frac{f^{(n)}(z_*)}{n!} \right| \cdot |z-z_*|^n \le\\
        \le \sum_{n = n_0 + 1}^\infty \frac{M_f(r)}{r^n} (\delta r)^n = M_f(r) \delta^{n_0+1} \cdot \frac{1}{1 - \delta}\tag{3}
    \end{multline*}
    Выберем $\delta_0$ из равенства
    \[ \frac{|f^{(n_0)}(z_*)|}{n_0!} (\delta_0 r)^{n_0} = 2 M_f \delta_0^{n_0 + 1} \cdot \frac{1}{1 - \delta_0}, \tag{4} \]
    тогда при $0 < |z - z_*| < \delta_0 r$, $|z - z_*| = \delta r$
    \begin{multline*}
        |f(z)| \ge \left| \frac{f^{(n_0)}(z_*)}{n_0!} (z - z_*)^{n_0} \right| - \left|\sum_{n = n_0 + 1}^\infty \frac{f^{(n)} (z_*)}{n!} (z-z_*)^n \right| \ge                                    \\
        \ge \left| \frac{f^{(n_0)}(z_*)}{n_0!} \right| (\delta r)^{n_0} - M_f(r) \delta^{n_0 + 1} \cdot \frac{1}{1 - \delta} \ge \frac{1}{2}\frac{|f^{(n_0)}(z_*)|}{n_0!} (br)^{n_0} > 0 \tag{5}
    \end{multline*}

    Из (5) следует, что $E\cap\{z:|z-z_*|<\delta_0\}=\{z_*\}$, что противоречит тому, что $z_*$ -- точка сгущения $E$. Следовательно, $f^{(n)}(z_*)=0\ \forall n$ и $f(z)\equiv 0,\ z\in G$.
\end{longProof}

\begin{corollary}
    Пусть $G$~--- область, $E\subset G$, как в теореме 2, $g, h \in A(G)$ и $g|_{E} = h|_E$.
    Тогда $g(z) \equiv h(z)$, $z \in G$.
\end{corollary}
\begin{proof}
    Пусть $f(z) = g(z) - h(z)$, тогда $f|_E = 0 \implies f(z) \equiv 0$, $z \in G \implies h(z) \equiv g(z)$.
\end{proof}

\begin{corollary}[о структуре аналитической функции в окрестности нуля]
    Пусть $G$~--- область, $z_0 \in G$, $f \in A(G)$, $f \not \equiv 0$ в $G$, $f(z_0) = 0$.
    Тогда $\exists n_0 \in \N$ и $\phi \in A(G)$ т.ч. $\phi(z_0)\neq 0$ и $f(z) = (z - z_0)^{n_0} \phi(z)$ и $\exists \delta > 0$ т.ч. при $z \neq z_0$, $|z - z_0| < \delta$ $f(z) \neq 0$.
\end{corollary}
\begin{longProof}
    Пусть $n_0 \in \N$~--- минимальное число, для которого $f^{(n_0)}(z_0) \neq 0$, такой $n_0$ существует в силу $f \not\equiv 0$.

    Выберем $r$ так, чтобы $\{ z:|z - z_0| \le r\} \subset G$.
    Проводя рассуждения из доказательства теоремы 2 (со значениями функции), полагая
    \[M_f(r) = \max_{|z-z_0|=r} |f(z)|,\]
    аналогично получаем при $|z-z_0|\le r$
    \[ f(z) = (z - z_0)^{n_0} \left( \frac{f^{(n_0)}(z_0)}{n_0!} + \sum_{n = n_0 + 1}^{\infty} \frac{f^{(n)}(z_0)}{n!} (z - z_0)^{n - n_0} \right) \tag{6} \]
    Положим
    \[\phi(z) =
        \begin{cases}
            \displaystyle
            \frac{f(z)}{(z - z_0)^{n_0}},\ z \in G \setminus \{z_0\} \\
            \displaystyle
            \frac{f^{(n_0)}(z_0)}{n_0!} + \sum_{n = n_0 + 1}^\infty \frac{f^{(n)}(z_0)}{n!} (z - z_0)^{n - n_0},\ |z - z_0|\le r \tag{7}
        \end{cases}
    \]
    Соотношение (6) показывает, что (7) определено корректно при $|z - z_0|<r$, $z \neq z_0$, поэтому (7) определяет функцию $\phi(z)$ при $z\in G$.
    Первая строка в (7) показывает, что $\phi \in A(G \setminus \{z_0\})$, вторая строка показывает, что $\phi \in A(\{|z - z_0| < r\})$, поэтому (7) $\implies \phi \in A(G)$.
    Далее,
    \[\phi(z_0) = \frac{f^{(n_0)}(z_0)}{n_0!} \neq 0,\]
    поэтому $\exists \delta > 0$ т.ч. $\phi(z) \neq 0$ при $|z - z_0| < \delta$, тогда при $z \neq z_0$, $|z - z_0| < r$, $f(z) = (z - z_0)^n \phi(z) \neq 0$.
\end{longProof}

\section{Аналитическое продолжение} \marginpar{30.03.23}
\begin{definition}
    Пусть имеются области $\Omega$ и $D$, при этом $\Omega \cap D \neq \varnothing$.
    Кроме того имеются функции $f \in A(\Omega)$, $g \in A(D)$.
    Предположим, что
    \[f(z) = g(z)\ \forall z \in \Omega \cap D, \]
    в таком случае говорят, что функция $g$ аналитически продолжает функцию $f$ в область $D$.
\end{definition}
\begin{theorem}
    Аналитическое продолжение функции $f$ в область $D$ единственно.
    То есть, если имеется $g_1 \in A(D)$, т.ч.
    \[f(z) = g_1(z)\ \forall z \in \Omega \cap D, \]
    то $g_1(z) = g(z)\ \forall z \in D$.
\end{theorem}
\begin{proof}
    Поскольку $\Omega$ и $D$~--- области и пересечение их не пусто, то их пересечение является открытым множеством.
    В открытом множестве каждая точка является точкой сгущения для этого множества.
    В области $D$ имеются две аналитические функции: $g$ и $g_1$.
    Они совпадают на множестве, которое имеет предельную точку в области $D$, поэтому по теореме единственности они совпадают.
\end{proof}

\subsection{Продолжение аналитической функции по пути}
\begin{definition}
    Путем называется непрерывное отображение $\phi: [a,b] \to \C$, $\phi \in C([a,b])$\footnote{Инъективность не требуется!}.
    Говорят, что $\phi(a)$~--- начало пути, $\phi(b)$~--- конец пути.
\end{definition}
\begin{definition}
    Пусть имеется область $D \subset \C$, а так же круг $B = B_r(z_0)$, где $z_0 \in D$ и $B \subset D$.
    Предполагаем, что у нас есть некий путь $\phi: [a,b] \to \C$, при этом $\phi(t) \in D\ \forall t \in [a,b]$, и $\phi(a) = z_0$, а $\phi(b) = z_1$.

    Кроме того, имеется вектор-функция $f \in A(B)$ и круг $B_1 = B_{r_1}(z_1)$~(при этом $z_1$ может совпадать с $z_0$).
    Так же имеется разбиение $\{t_k\}_{k=0}^n$, где $t_0 = a$, $t_k < t_{k+1}$, $t_n =b$, и круги $B_{\rho_k}(\zeta_k) \subset D$, при этом $\zeta_k = \phi(t_k), 0 \le k \le n$, т.о. $\zeta_0 = z_0$, $\zeta_n = z_1$, а $\rho_0 = r_0$, $\rho_n = r_1$.
    В этих обозначениях $B = B_{\rho_0}(\zeta_0)$.
    Дальше предполагаем, что $\forall k = 0, \dotsc, n-1$
    \[B_{\rho_k}(\zeta_k) \cap B_{\rho_{k+1}}(\zeta_{k+1}) \neq \varnothing. \]
    Предположим, что $f$ аналитически продолжена в $B_{\rho_1}(\zeta_1)$, т.е. существует $f_1 \in A(B_{\rho_1}(\zeta_1))$, т.ч.
    \[f|_{B_{\rho_0}(\zeta_0) \cap B_{\rho_{1}}(\zeta_{1})} = f_1|_{B_{\rho_0}(\zeta_0) \cap B_{\rho_{1}}(\zeta_{1})}.\]
    Далее, $f_1$ аналитически продолжена в $B_{\rho_2}(\zeta_2)$, т.е. существует $f_2 \in A(B_{\rho_2}(\zeta_2))$, т.ч.
    \[f_2|_{B_{\rho_1}(\zeta_1) \cap B_{\rho_{2}}(\zeta_{2})} = f_1|_{B_{\rho_1}(\zeta_1) \cap B_{\rho_{2}}(\zeta_{2})}.\]
    При $k < n$ существует $f_k \in A(B_{\rho_k}(\zeta_k))$, $f_k$ аналитически продолжена в $B_{\rho_{k+1}}(\zeta_{k+1})$.
    И так далее до $k=n$, где $f_n \in A(B_{\rho_n}(\zeta_n))$ или $f_n \in A(B_1)$.
    \begin{center}
        \import{figures}{analytic_continuation.pdf_tex}
    \end{center}
    В таком случаем, будем говорить, что функция $f$ аналитически аналитически продолжена вдоль пути, лежащего в области $D$ из круга $B$ в круг $B_1$.
\end{definition}

В процессе построения используется довольно много промежуточных кругов, которые можно выбирать бесконечным числом способов, т.к. можно менять их количество.
Если мы для одного набора кругов реализовали продолжение, получим ли мы в конце ту же самую функцию при выборе другого набора кругов, удовлетворяющего условиям?

\begin{theorem}
    Аналитическое продолжение не зависит от промежуточных кругов $B_{\rho_1}(\zeta_1), \dotsc, B_{\rho_{n-1}}(\zeta_{n-1})$, то есть, если имеется другой набор кругов, но с теми же условиями и начальным и конечными кругами $B$ и $B_1$ соответственно, то функция, которая будет получена в круге $B_1$ будет совпадать с той, что была получена до этого.
\end{theorem}
\begin{proof}
    Без доказательства.
\end{proof}

\begin{theorem}
    Пусть имеется область $D \subset \C$, круг $B = B_r(z_0)$, где $z_0 \in D$ и некоторая функция $f \in A(B)$.
    Рассмотрим любой путь $\gamma: [a,b] \to D$, т.е $\gamma(t) \in D$.
    $\gamma(a) = z_0$, $\gamma(b) = z_1$, где $z_1 \in D$.
    И имеется ещё один круг $B_1 = B_{r_1}(z_1) \subset D$.
    Тогда говорят, что функция $f$ продолжима в область $D$ из круга $B$ по любому пути.
\end{theorem}
\begin{proof}
    Без доказательства.
\end{proof}

\begin{definition}
    Множество называется односвязной областью, если любой замкнутый путь можно непрерывно стянуть в точку.
\end{definition}
\begin{theorem}[о монодромии]
    Пусть $D$~--- односвязная область, и имеется круг $B \subset D$, функция $f \in A(B)$ и $f$ продолжима в $D$ из $B$ по любому пути.
    Тогда $f \in A(D)$, т.е. $\exists F \in A(D)$, т.ч. $F|_B = f|_B$.
\end{theorem}
\begin{proof}
    Без доказательства.
\end{proof}

\begin{example}[функции, продолжимой по любому пути]
    Функция $\log z$ была определена в односвязной области $D = \C \setminus (- \infty,0]$ формулой
    \[\log z = \log |z| + i \arg z\]
    при $- \pi < \arg z < \pi$ и выполнено $e^{\log z} = z$.
    Теперь будем писать
    \[\log_k z = \log |z| + i \arg z + 2\pi i k, k \in \Z\]
    Это годится в качестве логарифма, т.к.
    \[e^{\log_k z} = e^{\log z + 2 \pi i k} = z\]
    Пусть теперь $\zeta \in B_r(z)$, $z \neq 0$, $ r \le |z|$.
    Выберем круг:
    \begin{center}
        \import{figures}{log_cont_1.pdf_tex}
    \end{center}
    Определим в этом круге $\arg \zeta$, тогда можем полагать
    \[\log \zeta = \log |\zeta| + i \arg \zeta\]
    Тогда по результатам прошлого семестра, если мы смотрим на любой круг в $\C$, такой что 0 не лежит в нем, то для любого такого круга мы можем определить корректный $\log \zeta$ бесконечным числом способов.

    Теперь рассмотрим круг $B = B_1(1)$, зададим каким-либо образом функцию $\log z$, при этом в данном круге $ - \pi/2 < \arg z < \pi /2$.
    Возьмем любой путь, который проходит в $\C \setminus \{0\}$, то мы можем продолжить функцию $\log z$ вдоль этого пути.
    \begin{center}
        \def\svgwidth{0.6\linewidth}
        \import{figures}{log_cont_2.pdf_tex}
    \end{center}
    Если в $k$-ом круге мы полагаем, что
    \[\log z = \log |z| + i \arg_{(k)} z,\]
    где $(k)$ есть номер круга.
    Если у нас есть два круга, которые пересекаются и проходят через точку $0$, то мы можем выбрать круги с номерами $k$ и $k+1$, так что в круге $k+1$ можно выбрать аргумент с условием, что
    \[\arg_{(k+1)} z = \arg_{(k)} z\]
    в пересечении кругов.
    Чтобы это сделать, посмотрим на аргумент из круга $k$ в пересечении, тогда знаем как доопределить в круге $k+1$, т.к. все аргументы отличаются на $2 \pi i k$, поэтому если у нас есть в какой-то точке круга заданный аргумент, то во всем круге может его задать.
    Получается, что $\log z$ является продолжимой по любому пути в $\C \setminus \{0\}$.
    Но она не является аналитической!

    Посмотрим, что произойдет, если мы обойдем начало координат по единичной окружности.
    \begin{center}
        \import{figures}{log_cont_3.pdf_tex}
    \end{center}
    В первом круге:
    \[- \frac{\pi}{2} < \arg z < \frac{\pi}{2}\]
    Переходим во второй круг, на биссектрисе первой четверти аргумент равен $\pi /4$.
    Хотим, чтобы аргумент во втором круге был непрерывной функцией, поэтому
    \[0 < \arg z < \pi\]
    На биссектрисе второй четверти аргумент равен $3 \pi/4$, поэтому в третьем круге
    \[ \frac{\pi}{2} < \arg z < \frac{3}{2} \pi\]
    На биссектрисе третьей четверти аргумент равен $5 \pi/4$, поэтому в четвертом круге
    \[\pi < \arg z < 2 \pi\]
    На биссектрисе  четвертой четверти аргумент равен $7 \pi/4$, поэтому в пятом, совпадающем с первым, круге
    \[ \frac{3}{2}\pi < \arg z < \frac{5}{2} \pi\]
    До обхода $\log z$ в первом круге определялся как
    \[\log z = \log |z| + i \arg z,\]
    а после стал определяться как
    \[\log z = \log |z| + i \arg z + 2 \pi i.\]

    Таким образом получается, что $\C \setminus \{0\}$ не односвязное множество, функция $\log z$ продолжима там по любому пути, но не является аналитической.
\end{example}

\end{document}
