% !TeX root = ./main.tex
\documentclass[main]{subfiles}
\begin{document}
\chapter{Ряды Фурье}
\section{Определение рядов Фурье}
\begin{definition}
    Пусть $f \in \calL((-\pi, \pi))$, коэффициентами Фурье функции $f$ называются числа
    \[\begin{gathered}
            a_0 = \frac{1}{2 \pi} \int_{-\pi}^{\pi} fdm \\
            a_n = \frac{1}{\pi} \int_{-\pi}^{\pi} f(x) \cos nx dm \\
            b_n = \frac{1}{\pi} \int_{-\pi}^{\pi} f(x) \sin nx dm
        \end{gathered} \tag{1}\]
\end{definition}
Запись $\int_{-\pi}^{\pi} f(x) \cos nx dm$ или $\int_{-\pi}^{\pi} f(x) \sin nx dm$ означает интеграл Лебега от функций $f(x) \cos nx$, $f(x) \sin nx$ по множеству $(-\pi, \pi)$ по мере Лебега в $\R^1$.
\begin{definition}
    Рядом Фурье для функции $f$ называется функциональный ряд
    \[a_0 + \sum_{n=1}^{\infty} (a_n \cos nx + b_n \sin nx)\]
\end{definition}

Нам понадобятся еще два утверждения из теории интеграла Лебега, которые мы примем без доказательств.
\begin{proposition}
    Пусть $f \in \calL((a, b))$, тогда
    \[\int_{(a, b-h)} |f(x+h) - f(x)| dm \xrightarrow[h \to +0]{} 0 \tag{2}\]
\end{proposition}

\begin{proposition}
    Пусть $f \in \calL((a, b))$, тогда
    \[\int_{(b-h, b)} |f(x)| dm \xrightarrow[h \to +0]{} 0 \tag{3}\]
\end{proposition}

Для интеграла Лебега справедливо также следующее свойство инвариантности относительно сдвига: пусть $E \subset \R$, $E \in \M_1$, $f \in \calL(E)$,
\[E_t = \{x \in \R: x = y+t, y \in E\},\]
$t \neq 0$, $f_t(x) = f(x-t)$, $x \in E_t$.
Тогда
\[\int_{E_t} f_t dm = \int_E f dm \tag{4}\]

\begin{theorem}[Римана-Лебега]
    Справедливы соотношения
    \[\begin{gathered}
            \int_{-\pi}^{\pi} f(x) \cos dx dm \xrightarrow[|d| \to \infty]{} 0\\
            \int_{-\pi}^{\pi} f(x) \sin dx dm \xrightarrow[|d| \to \infty]{} 0
        \end{gathered}\tag{5}\]
\end{theorem}
\begin{longProof}
    Докажем первое соотношение в (5) при $d \to \infty$, остальные доказываются аналогично.
    Пусть $t = \pi/d$, $E = (-\pi, \pi)$, $E_t = (-\pi + t, \pi +t)$, тогда (4) влечет
    \begin{multline*}
        \int_E f(x) \cos dx dm = \int_{E_t}(f(x) \cos dx)_t dm = \\
        = \int_{E_t} f(x-t) \cos d(x-t) dm = - \int_{E_t} f(x-t) \cos dx dm, \tag{6}
    \end{multline*}
    поскольку $\cos d(x - t) = \cos (dx - \pi) = - \cos dx$.

    Прибавим слева и справа в (6) $\int_E f(x) \cos dx dm$, получим
    \begin{multline*}
        2 \int_{-\pi}^{\pi} f(x) \cos dx dm = \int_{-\pi}^{\pi} f(x) \cos dx dm - \int_{-\pi + t}^{\pi + t} f(x - t) \cos dx dm = \\
        = \int_{-\pi}^{-\pi + t} f(x)\cos dx dm + \int_{-\pi+t}^{\pi} (f(x) - f(x - t)) \cos dx dm - \\
        - \int_{\pi}^{\pi + t} f(x - t) \cos dx dm \coloneq I_1(d) + I_2(d) - I_3(d) \tag{7}
    \end{multline*}
    Теперь получаем соотношения
    \begin{gather*}
        \left| I_1(d) \right| \le \int_{-\pi}^{-\pi + t} |f(x)\cos dx| dm \le \int_{-\pi}^{-\pi + t} |f(x)| dm \xrightarrow[d \to \infty]{} 0 \tag{8} \\
        \begin{multlined}
            \left| I_2(d) \right| \le \int_{-\pi+t}^{\pi} |f(x) - f(x - t)||\cos dx| dm \le \\
            \le \int_{-\pi+t}^{\pi} |f(x) - f(x - t)| dm \xrightarrow[d \to \infty]{} 0
        \end{multlined} \tag{9} \\
        \begin{multlined}
            \left| I_3(d) \right| \le \int_{\pi}^{\pi + t} |f(x - t) \cos dx| dm \le \int_{\pi}^{\pi + t} |f(x - t)| dm = \\
            = \int_{\pi-t}^{\pi} |f(x - t)| dm \xrightarrow[d \to \infty]{} 0
        \end{multlined} \tag{10}
    \end{gather*}
    Соотношения (7)-(10) влекут (5).
\end{longProof}

\section{Частичные суммы ряда Фурье}

В дальнейшем распространим функцию $f \in \calL((-\pi, \pi))$ на все вещественные аргументы с периодом $2 \pi$, т.е. полагаем, что $f: \R \to \R$, $f(x + 2\pi) = f(x)$ $\forall x \in \R$.
Учтем, что тогда в силу (4) будет выполняться равенство
\[\int_{(-\pi + 2 k \pi, \pi + 2 k \pi)} f dm = \int_{(-\pi, \pi)} f dm \tag{11}\]

Пусть $a \in (-\pi, \pi)$.
Тогда опять по (4)
\begin{multline*}
    \int_{a, a + 2\pi} f dm = \int_{(a, \pi)} f dm + \int_{(\pi, a + 2\pi)} f dm = \\
    = \int_{(a, \pi)} f dm + \int_{(-\pi, a)} f dm = \int_{(-\pi, \pi)} f dm \tag{12}
\end{multline*}
С учетом (11) и (12), получаем, что равенство (12) справедливо для $\forall a \in \R$.

Пусть
\[a_0 + \sum_{n=1}^{\infty}(a_n \cos nx + b_n \sin nx) \tag{13}\]
ряд Фурье функции $f$.
Частичной суммой ряда (13) называется выражение
\[S_n(x) = a_0 + \sum_{k=1}^{n} (a_k \cos kx + b_k \sin kx) \tag{14}\]

Применяя формулы (1), найдем, что
\begin{multline*}
    S_n(x) = \frac{1}{2 \pi} \int_{-\pi}^{\pi} f(t) dm + \sum_{k=1}^{n} \left(\cos kx \frac{1}{\pi} \int_{-\pi}^{\pi} f(t) \cos kt dm + \right. \\
    \left. + \sin kx \frac{1}{\pi} \int_{-\pi}^{\pi} f(t) \sin kt dm\right) = \frac{1}{\pi} \int_{-\pi}^{\pi} f(t) \left( \frac{1}{2} + \sum_{k=1}^{n} \cos kx \cos kt + \right. \\
    \left. + \sin kx \sin kt\right) dm = \frac{1}{\pi} \int_{-\pi}^{\pi} f(t) \left( \frac{1}{2} + \sum_{k=1}^{n} \cos k(t-x) \right) dm \tag{15}
\end{multline*}
Положим
\[D_n(y) \coloneq \frac{1}{2} + \sum_{k=1}^{n} \cos ky \tag{16}\]
Тогда $D_n(2 m \pi) = n + 1/2$.
Если $y \neq 2 m \pi$, то
\begin{multline*}
    \sin \frac{y}{2} D_n(y) = \frac{1}{2} \sin \frac{y}{2} + \sum_{k=1}^{n} \sin \frac{y}{2} \cos ky = \\
    = \frac{1}{2} \sin \frac{y}{2} + \sum_{k=1}^{n} \frac{1}{2}\left( \sin \left(k + \frac{1}{2}\right)y - \sin \left(k - \frac{1}{2}\right)y\right) = \\
    = \frac{1}{2} \sin \left(n + \frac{1}{2}\right)y,
\end{multline*}
поэтому
\[D_n(y) = \frac{\sin \left(n + \frac{1}{2}\right)y}{2 \sin \frac{y}{2}}, \tag{17}\]
(15) и (17) влекут, что
\[S_n(x) = \frac{1}{2\pi} \int_{-\pi}^{\pi} f(t) \frac{\sin \left(n + \frac{1}{2}\right)(t-x)}{\sin \frac{t-x}{2}} dm \tag{18}\]
Применяя формулу (12), преобразуя (18):
\begin{multline*}
    S_n(x) = \frac{1}{2\pi} \int_{-\pi - x}^{\pi - x} f(v + x) \frac{\sin \left(n + \frac{1}{2}\right)v}{\sin \frac{v}{2}} dm = \\
    = \frac{1}{2\pi} \int_{-\pi}^{\pi} f(v + x) \frac{\sin \left(n + \frac{1}{2}\right)v}{\sin \frac{v}{2}} dm \tag{19}
\end{multline*}

\section{Признак \texorpdfstring{Д\'ини}{Дини} сходимости ряда Фурье}
\begin{theorem}
    Предположим, что справедливо соотношение
    \[\frac{f(v + x) - f(x)}{v} \in \calL((-\pi, \pi))\]
    Тогда
    \[S_n(x) \xrightarrow[n \to \infty]{} f(x) \tag{20}\]
\end{theorem}
\begin{longProof}
    Применяя (16), находим, что
    \[\int_{-\pi, \pi} D_n(v) dm = \int_{-\pi}^{\pi} D_n(v) dv = \pi, \tag{21}\]
    тогда (19) и (21) влекут
    \begin{multline*}
        S_n(x) - f(x) = \frac{1}{2 \pi} \int_{-\pi}^{\pi} (f(x + v) - f(x))  \frac{\sin \left(n + \frac{1}{2}\right)v}{\sin \frac{v}{2}} dm = \\
        = \frac{1}{2 \pi} \int_{-\pi}^{\pi} \frac{f(x + v) - f(x)}{v} \cdot \frac{v}{\sin \frac{v}{2}} \cdot \sin\left(n + \frac{1}{2}\right) v dm \tag{22}
    \end{multline*}
    При $|v| < \pi$ имеем
    \[\left|\sin \frac{v}{2}\right| \ge \frac{2}{\pi} \cdot \frac{|v|}{2} = \frac{|v|}{\pi},\]
    поэтому $|v / \sin(v / 2)| \le \pi$,
    \[\left|\frac{f(x + v) - f(x)}{v} \cdot \frac{v}{\sin \frac{v}{2}}\right| \le \pi \left|\frac{f(x + v) - f(x)}{v}\right|,\]
    т.е.
    \[g(v) \coloneq \frac{f(x + v) - f(x)}{v} \cdot \frac{v}{\sin \frac{v}{2}} \in \calL((-\pi, \pi))\]
    Тогда по теореме Римана-Лебега
    \[S_n(x) - f(x) = \frac{1}{2 \pi} \int_{(-\pi, \pi)} g(v) \sin\left(n + \frac{1}{2}\right) v dm \xrightarrow[n \to \infty]{} 0\]
\end{longProof}

\section{Равенство Парсеваля}
\begin{theorem}
    Пусть $f^2 \in \calL(-\pi, \pi)$, $a_n$, $n \ge 0$, $b_n$, $n \ge 1$~--- коэффициенты Фурье $f$.
    Тогда
    \[\frac{1}{2 \pi} \int_{(-\pi, \pi)} f^2 dm = a^2_0 + \frac{1}{2} \sum_{n=1}^{\infty} (a^2_n + b^2_n)\]
\end{theorem}
Примем без доказательства.

\section{Единственность ряда Фурье}
\begin{theorem}
    Пусть $f, g \in \calL((-\pi, \pi))$, $a_n(f)$, $b_n(f)$, $a_n(g)$, $b_n(g)$~--- коэффициенты Фурье $f, g$.

    Пусть $a_n(f) = a_n(g)$, $n \ge 0$, $b_n(f) = b_n(g)$, $n \ge 1$, тогда $f \sim g$.
\end{theorem}
Примем без доказательства.

Если $f \sim g$, то $a_n(f) = a_n(g)$, $b_n(f) = b_n(g)$, теорема дает обратное утверждение.
В частности, если $a_n(f_0) = 0$, $n \ge 0$, $b_n(f_0) = 0$, $n \ge 1$, то $f_0 \sim 0$, $0$~--- функция, тождественно равная нулю.

Конец курса.
\end{document}
