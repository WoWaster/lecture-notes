% !TeX root = ./main.tex
\documentclass[main]{subfiles}
\begin{document}
\chapter{Секвенциальное исчисление высказываний (СИВ)}
\begin{definition}[Секвенция]
    Формулы СИВ~--- секвенция. То есть выражения вида $\Gamma \vdash \Delta$, где $\Gamma, \Delta$~--- наборы пропозициональных формул.
\end{definition}
\begin{definition}[Формульный образ секвенции]
    \[\Phi(\overbrace{A_1, \dots, A_n}^{\text{антецедент}}
        \vdash \overbrace{B_1, \dots , B_m}^{\text{сукцедент}}) \eqcirc A_1 \and A_2\and\dots \and A_n \to (B_1 \lor B_2 \lor \dots \lor B_m),\] где $\eqcirc$~--- графическое равенство, т.е. то, что слева это <<точь в точь>> то, что справа.
\end{definition}
\begin{definition}[Аксиомы СИВ]
    Аксиомы СИВ~--- это секвенции вида $\Gamma_1 A\Gamma_2 \vdash \Delta_1 A \Delta_2$.
\end{definition}
\begin{lemma}
    Формульный образ аксиомы является тавтологией.
\end{lemma}
\begin{proof}
    Если применить правило де~Моргана и раскрыть импликацию, то аргумент $\Phi$ в общем виде можно записать так:
    \begin{gather*}
        A_1 \and \ldots \and A_n \to B_1 \lor \dots \lor B_m \Leftrightarrow \lnot A_1 \lor \dots \lor \lnot A_n \lor B_1 \lor \dots \lor B_m\\
        \intertext{из этого следует, что }
        \Phi(A_1, \dots , A_k, A, A_{k+1}, \dots, A_n \vdash B_1, \dots, B_l, A, B_{l+1}, \dots, B_m) \Leftrightarrow \\
        \Leftrightarrow \lnot A_1 \lor \dots \lor \lnot A_k \lor \textcolor{red}{\lnot A} \lor \lnot A_{k+1}\lor\dots \lor\lnot A_n \lor\\
        \lor B_1\lor\dots \lor B_l \lor \textcolor{red}{A} \lor B_{l+1}\lor \dots \lor B_m
    \end{gather*}
    \textcolor{red}{красное} даёт нам тавтологию.
\end{proof}
\section{Правила вывода в СИВ}
$(*\ \vdash)$~--- правило вывода в антецеденте для бинарной связки $*$

$(\vdash \ *)$~--- правило вывода в сукцеденте для бинарной связки $*$
\begin{align*}
     & \begin{prooftree}
           \hypo{\Gamma_1 A\Gamma_2 B\Gamma_3 &\vdash \Delta}
           \infer[left label=$(\and \ \vdash)$]1{\Gamma_1 A \and B \Gamma_2 \Gamma_3 &\vdash \Delta}
       \end{prooftree} &
     & \begin{prooftree}
           \hypo{ \Gamma &\vdash \Delta_1 B\Delta_2}
           \infer[no rule]1{ \Gamma &\vdash \Delta_1A\Delta_2 }
           \infer[left label=$(\vdash \ \and)$]1{ \Gamma &\vdash \Delta_1 A \and B \Delta_2}
       \end{prooftree}
    \\
    \\
     & \begin{prooftree}
           \hypo{\Gamma_1 B\Gamma_2 &\vdash \Delta}
           \infer[left label=$(\lor \ \vdash)$]1{\Gamma_1 A\lor B\Gamma_2 &\vdash \Delta}
       \end{prooftree}            &
     & \begin{prooftree}
           \hypo{\Gamma &\vdash \Delta_1A\Delta_2B\Delta_3}
           \infer[left label=$(\vdash \ \lor)$]1{\Gamma &\vdash \Delta_1 A\lor B\Delta_2\Delta_3}
       \end{prooftree}
    \\
    \\
     & \begin{prooftree}
           \hypo{\Gamma_1B\Gamma_2 &\vdash\Delta_1\Delta_2}
           \infer[no rule]1{\Gamma_1\Gamma_2 &\vdash \Delta_1 A \Delta_2}
           \infer[left label=$(\to \ \vdash)$]1{\Gamma_1 A\to B\Gamma_2 &\vdash \Delta_1 \Delta_2}
       \end{prooftree}   &
     & \begin{prooftree}
           \hypo{\Gamma_1A\Gamma_2 &\vdash \Delta_1B\Delta_2}
           \infer[left label=$(\vdash \ \to)$]1{\Gamma_1\Gamma_2 &\vdash \Delta_1A\to B \Delta_2}
       \end{prooftree}
    \\
    \\
     & \begin{prooftree}
           \hypo{\Gamma_1 \Gamma_2 &\vdash \Delta_1A\Delta_2}
           \infer[left label=$(\lnot\ \vdash)$]1{\Gamma_1 \lnot A\Gamma_2 &\vdash \Delta_1\Delta_2}
       \end{prooftree}  &
     & \begin{prooftree}
           \hypo{\Gamma_1 A\Gamma_2 &\vdash \Delta_1\Delta_2}
           \infer[left label=$(\vdash \ \lnot)$]1{\Gamma_1 \Gamma_2 &\vdash \Delta_1\lnot A\Delta_2}
       \end{prooftree}
\end{align*}
\end{document}