% !TeX root = ./main.tex
\documentclass[main]{subfiles}
\begin{document}
\chapter{Пропозициональные формулы}
\begin{definition}[Высказывание]
    Высказывание~--- это утверждение, про которое можно сказать, истинное оно или ложное.
\end{definition}
\begin{definition}[Пропозициональные переменные]
    Переменные для высказывания ~--- пропозициональные переменные.
\end{definition}
Всего есть 2 константы: Const=\{истина, ложь\}.
Для них есть много разных обозначений:

$true;\ t;\ T;\ \top;\ 1;\ 0$~--- истина,

$false;\ f;\ F;\ \bot;\ 0;\ 1$~--- ложь.

Здесь и далее запись $*\ |\ \dots$ будет означать, что $*$~--- обозначение, которое будет использоваться в курсе, а всё, что после $|$~--- альтернативные обозначения.

Дальше напишем логические связки:
\begin{enumerate}
    \item Конъюнкция $\and\ |\ \land;\ \cdot;\ and $~--- логическое <<и>>
    \item Дизъюнкция $\lor\ |\ +;\ or$~---  логическое <<или>>
    \item Импликация $\to\ |\ \supset;\ \Rightarrow$~--- <<если ... , то>>
    \item Эквивалентность $\leftrightarrow\ |\ \equiv;\ \Leftrightarrow;\ \sim$
    \item Операции неэквивалентность ($\not\leftrightarrow$), исключающее <<или>> ($\stackrel{\cdot}{\lor}$), Сложение по модулю 2 ($+_2$) имеют одинаковые таблицы истинности. Мы будем обозначать эту операцию символом $\oplus$
    \item Символ Шеффера $ \mid$.   $x\mid y = \lnot(x\and y)$
    \item Стрелка Пирса $\downarrow$.   $x\downarrow y = \lnot(x\lor y)$
    \item Отрицание $\lnot\ |\ \sim;\ \overline{x}$
\end{enumerate}
\begin{definition}[Пропозициональные формулы]
    \begin{enumerate}
        \item Пропозициональные переменные и $Const$~---
              пропозициональные формулы.
        \item Если $A$~--- пропозициональная
              формула, то и $\lnot A$~--- пропозициональная формула.
        \item Если $A, B$~--- пропозициональные формулы , $*$~--- любая бинарная
              связка, то $(A * B)$~--- пропозициональная формула.
    \end{enumerate}
\end{definition}
Приоритеты идут в таком порядке:
\begin{enumerate}
    \item $\lnot$
    \item $\and$
    \item Всё остальное
\end{enumerate}
\begin{remark}
    Важно заметить, что в данном курсе формула $x\and y \to y\lor z$ равносильна формуле $((x\and y \to y) \lor z)$.
\end{remark}
\begin{definition}[Равносильность формул]
    2 формулы равносильны, если их значения совпадают на любом наборе значений,
    входящих в них пропозициональных переменных.
\end{definition}
Основные равносильности:
\begin{enumerate}

    \item $A\and B \Leftrightarrow B\and A$ коммутативность конъюнкции
    \item $A\lor B \Leftrightarrow B\lor A$ коммутативность дизъюнкции
    \item $A\and A \Leftrightarrow A$ идемпотентность конъюнкции
    \item $A\lor A \Leftrightarrow  A$ идемпотентность дизъюнкции
    \item $A\and(B\lor C) \Leftrightarrow  A\and B \lor A\and C$ дистрибутивность конъюнкции относительно дизъюнкции
    \item $A\lor B\and C  \Leftrightarrow (A\lor B)\and (A\lor C)$ дистрибутивность дизъюнкции относительно конъюнкции
    \item $A\and (B\and C) \Leftrightarrow (A\and B)\and C$ ассоциативность конъюнкции
    \item $A\lor (B\lor C) \Leftrightarrow  (A\lor B)\lor C$ ассоциативность дизъюнкции
    \item $\lnot\lnot x \Leftrightarrow x$
    \item $\lnot(A\and B) \Leftrightarrow \lnot A \lor \lnot B$ правило де~Моргана
    \item $\lnot(A\lor B) \Leftrightarrow  \lnot A \and \lnot B$
    \item $A\lor A \and B \Leftrightarrow  A$ правило поглощения
    \item $A\and(A\lor B) \Leftrightarrow  A$ правило поглощения
    \item $A\and B \lor \lnot A\and C \Leftrightarrow  A\and B \lor \lnot A \and C \lor B\and C $ правило склеивания
    \item $(A\lor B) \and (\lnot A \lor C) \Leftrightarrow  (A\lor B) \and (\lnot A \lor C) \and (B\lor C)$ правило склеивания
    \item $A\to B \Leftrightarrow \lnot A \lor B$
    \item
          \begin{align*}
              A\leftrightarrow B & \Leftrightarrow (A\to B) \& (B\to A)                  \\
                                 & \Leftrightarrow (\lnot A\lor B) \and (\lnot B \lor A) \\
                                 & \Leftrightarrow A\and B \lor \lnot A \and \lnot B
          \end{align*}
    \item
          \begin{align*}
              A\oplus B & \Leftrightarrow \lnot(A \leftrightarrow B)            \\
                        & \Leftrightarrow \lnot A\and B \lor A \and \lnot B     \\
                        & \Leftrightarrow (A\lor B) \and (\lnot A \lor \lnot B)
          \end{align*}
    \item $A \mid B \Leftrightarrow  \lnot(A \and B)$
    \item $A \downarrow B \Leftrightarrow \lnot (A\lor B)$
\end{enumerate}
\begin{definition}[Тавтология]
    Формула называется тавтологией, если она истинна на любых наборах
    пропозициональных переменных.
\end{definition}
\begin{definition}[Противоречие]
    Формула называется противоречием, если она ложна на любых наборах
    пропозициональных переменных.
\end{definition}
Основные тавтологии:
\begin{enumerate}
    \item $x \lor \lnot x$
    \item $x\to x$ или $x\leftrightarrow x$
\end{enumerate}
Основные противоречия:
\begin{enumerate}
    \item $x\and \lnot x$
    \item $x\oplus x$
\end{enumerate}
\chapter{Секвенциальное исчисление высказываний (СИВ)}
\begin{definition}[Секвенция]
    Формулы СИВ~--- секвенция. То есть выражения вида $\Gamma \vdash \Delta$, где
    $\Gamma, \Delta$~--- наборы пропозициональных формул.
\end{definition}
\begin{definition}[Формульный образ секвенции]
    Формульный образ секвенции \[\Phi(\overbrace{A_1, \dots, A_n}^{\text{антецедент}}
        \vdash \overbrace{B_1, \dots , B_m}^{\text{сукцедент}}) \eqcirc
        A_1 \and A_2\and\dots \and A_n \to (B_1 \lor B_2 \lor \dots \lor B_m)\].
\end{definition}
\begin{definition}[Аксиомы СИВ]
    Аксиомы СИВ~--- это секвенции вида $\Gamma_1 A\Gamma_2 \vdash \Delta_1 A \Delta_2$.
\end{definition}
\begin{lemma}
    Формульный образ аксиомы является тавтологией.
\end{lemma}
\begin{proof}
    Если применить правило де~Моргана и раскрыть импликацию, то аргумент $\Phi$ в общем виде можно записать так:
    \begin{gather*}
        A_1 \and \ldots \and A_n \to B_1 \lor \dots \lor B_m \Leftrightarrow \lnot A_1 \lor \dots \lor \lnot A_n \lor B_1 \lor \dots \lor B_m\\
        \intertext{из этого следует, что }
        \Phi(A_1, \dots , A_k, A, A_{k+1}, \dots, A_n \vdash B_1, \dots, B_l, A, B_{l+1}, \dots, B_m) \Leftrightarrow \\
        \Leftrightarrow \lnot A_1 \lor \dots \lor \lnot A_k \lor \textcolor{red}{\lnot A} \lor \lnot A_{k+1}\lor\dots \lor\lnot A_n \lor\\
        \lor B_1\lor\dots \lor B_l \lor \textcolor{red}{A} \lor B_{l+1}\lor \dots \lor B_m
    \end{gather*}
    \textcolor{red}{красное} даёт нам тавтологию.
\end{proof}
\section{Правила вывода в СИВ}
$(*\ \vdash)$~--- правило вывода в антецеденте для бинарной связки $*$

$(\vdash \ *)$~--- правило вывода в сукцеденте для бинарной связки $*$

\begin{align*}
     & \begin{prooftree}
           \hypo{\Gamma_1 A\Gamma_2 B\Gamma_3 &\vdash \Delta}
           \infer[left label=$(\and \ \vdash)$]1{\Gamma_1 A \and B \Gamma_2 \Gamma_3 &\vdash \Delta}
       \end{prooftree} &
    \qquad
     & \begin{prooftree}
           \hypo{ \Gamma &\vdash \Delta_1A\Delta_2 }
           \hypo{ \Gamma &\vdash \Delta_1 B\Delta_2}
           \infer[left label=$(\vdash \ \and)$]2{ \Gamma \vdash \Delta_1 A \and B \Delta_2}
       \end{prooftree}          &
    \\
    \\
     & \begin{prooftree}
           \hypo{\Gamma_1 B\Gamma_2 &\vdash \Delta}
           \infer[left label=$(\lor \ \vdash)$]1{\Gamma_1 A\lor B\Gamma_2 &\vdash \Delta}
       \end{prooftree}            &
    \qquad
     & \begin{prooftree}
           \hypo{\Gamma &\vdash \Delta_1A\Delta_2B\Delta_3}
           \infer[left label=$(\vdash \ \lor)$]1{\Gamma &\vdash \Delta_1 A\lor B\Delta_2\Delta_3}
       \end{prooftree}    &
    \\
    \\
     & \begin{prooftree}
           \hypo{\Gamma_1\Gamma_2 &\vdash \Delta_1 A \Delta_2}
           \hypo{\Gamma_1B\Gamma_2 &\vdash\Delta_1\Delta_2}
           \infer[left label=$(\to \ \vdash)$]2{\Gamma_1 A\to B\Gamma_2 \vdash \Delta_1 \Delta_2}
       \end{prooftree}    &
    \qquad
     & \begin{prooftree}
           \hypo{\Gamma_1A\Gamma_2 &\vdash \Delta_1B\Delta_2}
           \infer[left label=$(\vdash \ \to)$]1{\Gamma_1\Gamma_2 &\vdash \Delta_1A\to B \Delta_2}
       \end{prooftree}    &
    \\
    \\
     & \begin{prooftree}
           \hypo{\Gamma_1 \Gamma_2 &\vdash \Delta_1A\Delta_2}
           \infer[left label=$(\lnot\ \vdash)$]1{\Gamma_1 \lnot A\Gamma_2 &\vdash \Delta_1\Delta_2}
       \end{prooftree}  &
    \qquad
     & \begin{prooftree}
           \hypo{\Gamma_1 A\Gamma_2 &\vdash \Delta_1\Delta_2}
           \infer[left label=$(\vdash \ \lnot)$]1{\Gamma_1 \Gamma_2 &\vdash \Delta_1\lnot A\Delta_2}
       \end{prooftree} &
\end{align*}
\end{document}