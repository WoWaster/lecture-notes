% !TeX root = ./main.tex
\documentclass[main]{subfiles}
\begin{document}
\chapter{Пропозициональные формулы}\marginpar{14.02.23}
\begin{definition}[Высказывание]
    Высказывание~--- это утверждение, про которое можно сказать, истинное оно или ложное.
\end{definition}
\begin{definition}[Пропозициональные переменные]
    Переменные для высказывания~--- пропозициональные переменные.
\end{definition}
Всего есть 2 константы: $Const$=\{истина, ложь\}.
Для них есть много разных обозначений:

$true;\ t;\ T;\ \top;\ 1;\ 0$~--- истина,

$false;\ f;\ F;\ \bot;\ 0;\ 1$~--- ложь.

Здесь и далее запись $*\ |\ \dots$ будет означать, что $*$~--- обозначение, которое будет использоваться в курсе, а всё, что после $|$~--- альтернативные обозначения.

Дальше напишем логические связки:
\begin{enumerate}
    \item Конъюнкция $\and\ |\ \land;\ \cdot;\ and $~--- логическое <<и>>
    \item Дизъюнкция $\lor\ |\ +;\ or$~---  логическое <<или>>
    \item Импликация $\to\ |\ \supset;\ \Rightarrow$~--- <<если ... , то>>
    \item Эквивалентность $\leftrightarrow\ |\ \equiv;\ \Leftrightarrow;\ \sim$
    \item Операции неэквивалентность ($\not\leftrightarrow$), исключающее <<или>> ($\stackrel{\cdot}{\lor}$), сложение по модулю 2 ($+_2$) имеют одинаковые таблицы истинности. Мы будем обозначать эту операцию символом $\oplus$
    \item Символ Шеффера $ \mid$.   $x\mid y = \lnot(x\and y)$
    \item Стрелка Пирса $\downarrow$.   $x\downarrow y = \lnot(x\lor y)$
    \item Отрицание $\lnot\ |\ \sim;\ \overline{x}$
\end{enumerate}
\begin{definition}[Пропозициональные формулы]
    \

    \begin{enumerate}
        \item Пропозициональные переменные и $Const$~--- пропозициональные формулы.
        \item Если $A$~--- пропозициональная формула, то и $\lnot A$~--- пропозициональная формула.
        \item Если $A, B$~--- пропозициональные формулы , $*$~--- любая бинарная связка, то $(A * B)$~--- пропозициональная формула.
    \end{enumerate}
\end{definition}
Приоритеты идут в таком порядке:
\begin{enumerate}
    \item $\lnot$
    \item $\and$
    \item Всё остальное
\end{enumerate}
\begin{remark}
    Важно заметить, что в данном курсе формула $x\and y \to y\lor z$ равносильна формуле $((x\and y \to y) \lor z)$.
\end{remark}
\begin{definition}[Равносильность формул]
    Две формулы равносильны, если их значения совпадают на любом наборе значений,
    входящих в них пропозициональных переменных.
\end{definition}
Основные равносильности:
\begin{enumerate}
    \item $A\and B \Leftrightarrow B\and A$ коммутативность конъюнкции
    \item $A\lor B \Leftrightarrow B\lor A$ коммутативность дизъюнкции
    \item $A\and A \Leftrightarrow A$ идемпотентность конъюнкции
    \item $A\lor A \Leftrightarrow  A$ идемпотентность дизъюнкции
    \item $A\and(B\lor C) \Leftrightarrow  A\and B \lor A\and C$ дистрибутивность конъюнкции относительно дизъюнкции
    \item $A\lor B\and C  \Leftrightarrow (A\lor B)\and (A\lor C)$ дистрибутивность дизъюнкции относительно конъюнкции
    \item $A\and (B\and C) \Leftrightarrow (A\and B)\and C$ ассоциативность конъюнкции
    \item $A\lor (B\lor C) \Leftrightarrow  (A\lor B)\lor C$ ассоциативность дизъюнкции
    \item $\lnot\lnot x \Leftrightarrow x$ правило двойного отрицания
    \item $\lnot(A\and B) \Leftrightarrow \lnot A \lor \lnot B$ правило де~Моргана
    \item $\lnot(A\lor B) \Leftrightarrow  \lnot A \and \lnot B$ правило де~Моргана
    \item $A\lor A \and B \Leftrightarrow  A$ правило поглощения
    \item $A\and(A\lor B) \Leftrightarrow  A$ правило поглощения
    \item $A\and B \lor \lnot A\and C \Leftrightarrow  A\and B \lor \lnot A \and C \lor B\and C $ правило склеивания
    \item $(A\lor B) \and (\lnot A \lor C) \Leftrightarrow  (A\lor B) \and (\lnot A \lor C) \and (B\lor C)$ правило склеивания
    \item $A\to B \Leftrightarrow \lnot A \lor B$
    \item
          \begin{align*}
              A\leftrightarrow B & \Leftrightarrow (A\to B) \& (B\to A)                  \\
                                 & \Leftrightarrow (\lnot A\lor B) \and (\lnot B \lor A) \\
                                 & \Leftrightarrow A\and B \lor \lnot A \and \lnot B
          \end{align*}
    \item
          \begin{align*}
              A\oplus B & \Leftrightarrow \lnot(A \leftrightarrow B)            \\
                        & \Leftrightarrow \lnot A\and B \lor A \and \lnot B     \\
                        & \Leftrightarrow (A\lor B) \and (\lnot A \lor \lnot B)
          \end{align*}
    \item $A \mid B \Leftrightarrow  \lnot(A \and B)$
    \item $A \downarrow B \Leftrightarrow \lnot (A\lor B)$
\end{enumerate}
\begin{definition}[Тавтология]
    Формула называется тавтологией, если она истинна на любых наборах
    пропозициональных переменных.
\end{definition}
\begin{definition}[Противоречие]
    Формула называется противоречием, если она ложна на любых наборах
    пропозициональных переменных.
\end{definition}
Основные тавтологии:
\begin{enumerate}
    \item $x \lor \lnot x$
    \item $x\to x$ или $x\leftrightarrow x$
\end{enumerate}
Основные противоречия:
\begin{enumerate}
    \item $x\and \lnot x$
    \item $x\oplus x$
\end{enumerate}
\end{document}