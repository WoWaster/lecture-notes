% !TeX root = ./main.tex
\documentclass[main]{subfiles}
\begin{document}
\chapter{Пропозициональные формулы}
\begin{definition}
    Высказывание -- это утверждение, про которое можно сказать, истинное оно или ложное
\end{definition}
\begin{definition}
    Переменные для высказывания  -- пропозициональные переменные
\end{definition}
Всего есть 2 константы: $\text{Const} = \{\text{true}, \text{false}\}$.
Для них есть много разных обозначений:\\
$true; t; T; \top; 1; 0$ -- истина, \\
$false; f; F; \bot; 0; 1$ -- ложь.\\
Дальше напишем логические связки:
\begin{enumerate}
    \item Конъюнкция $\& | \land ; \cdot ; and $ -- логическое "и"
    \item Дизъюнкция $\lor | +; or$ --  логическое "или"
    \item Импликация $\to | \supset ; \Rightarrow$ -- "если ... , то"
    \item Эквивалентность $\leftrightarrow | \equiv ; \Leftrightarrow ; \sim$
    \item Неэквивалентность $\not\leftrightarrow$
    \item Исключающее "или" $\stackrel{\cdot}{\lor}$
    \item Сложение по модулю 2 $+_2$
    \item Заметим, что 5, 6, 7 имеют одинаковые таблицы истинности. Они имеют 
    альтернативное обозначение -- $\oplus$
    \item Символ Шеффера $ |$.   $x|y = \lnot(x\&y)$
    \item Стрелка Пирса $\downarrow$.   $x\downarrow y = \lnot(x\lor y)$
    \item Отрицание $\lnot | \sim; \overline{x}$
\end{enumerate}
\begin{definition}[Пропозициональные формулы]
    \begin{enumerate}
        \item Пропозициональные переменные и $Const$ -- 
        пропозициональные формулы
        \item Если $A$ -- пропозициональная
        формула, то и $\lnot A$ -- пропозициональная формула
        \item Если $A, B$ -- пропозициональные формулы , $*$ -- любая бинарная
         связка, то $(A * B)$ -- пропозициональная формула
    \end{enumerate}
\end{definition}
Приоритеты идут в таком порядке
\begin{enumerate}
    \item $\lor$
    \item $\&$
    \item Всё остальное
\end{enumerate}
\begin{remark}
    Важно заметить, что в данном курсе формула $x\& y \to y\lor z$ 
    равносильна формуле $((x\& y \to y) \lor z)$
\end{remark}
\begin{definition}
    2 формулы равносильны, если их значения совпадают на любом наборе значений, 
    входящих в них пропозициональных переменных
\end{definition}
Основные равносильности:
\begin{enumerate}
    
    \item $A\& B \Leftrightarrow B\& A$ коммутативность
    \item $A\lor B \Leftrightarrow B\lor A$
    \item $A\& A \Leftrightarrow A$ идемпотентность
    \item $A\lor A \Leftrightarrow  A$
    \item $A\&(B\lor C) \Leftrightarrow  A\& B \lor A\& C$ дистрибутивность
    \item $A\lor B\& C  \Leftrightarrow (A\lor B)\& (A\lor C)$
    \item $A\& (B\&C) \Leftrightarrow (A\& B)\&C$ ассоциативность 
    \item $A\lor (B\lor C) \Leftrightarrow  (A\lor B)\lor C$
    \item $\lnot\lnot x \Leftrightarrow x$
    \item $\lnot(A\&B) \Leftrightarrow \lnot A \lor \lnot B$ правило Де-Моргана
    \item $\lnot(A\lor B) \Leftrightarrow  \lnot A \& \lnot B$
    \item $A\lor A \& B \Leftrightarrow  A$ правило поглощения 
    \item $A\&(A\lor B) \Leftrightarrow  A$
    \item $A\& B \lor \lnot A\& C \Leftrightarrow  A\& B \lor \lnot A \& C \lor B\& C $ правило склеивания
    \item $(A\lor B) \& (\lnot A \lor C) \Leftrightarrow  (A\lor B) \& (\lnot A \lor C) \& (B\lor C)$
    \item $A\to B \Leftrightarrow \lnot A \lor B$
    \item 
        \begin{align*}
        A\leftrightarrow B \Leftrightarrow (A\to B) \& (B\to A)\\
        \Leftrightarrow (\lnot A\lor B) \& (\lnot B \lor A)\\
        \Leftrightarrow A\& B \lor \lnot A \& \lnot B
        \end{align*}
    \item 
        \begin{align*}
            A\oplus B \Leftrightarrow  \lnot(A\leftrightarrow B) \\
            \Leftrightarrow \lnot A\& B \lor A \& \lnot B \\
            \Leftrightarrow (A\lor B) \& (\lnot A \lor \lnot B)
        \end{align*} 
    \item $A|B \Leftrightarrow  \lnot(A\& B)$
    \item $A\downarrow B \Leftrightarrow \lnot (A\lor B)$
\end{enumerate}
\begin{definition}[Тавтология]
    Формула называется тавтологией, если она истина на любых наборах 
    пропозициональных переменных
\end{definition}
\begin{definition}[Противоречие]
    Формула называется противоречием, если она ложна на любых наборах 
    пропозициональных переменных
\end{definition}
Основные тавтологии:
\begin{enumerate}
    \item $x \lor \lnot x$
    \item $x\to x$ или $x\leftrightarrow x$
\end{enumerate}
Основные противоречия:
\begin{enumerate}
    \item $x\& \lnot x$
    \item $x\oplus x$
\end{enumerate}
\chapter{Секвенциальное исчисление высказываний (СИВ)}
\begin{definition}
    Формулы СИВ -- секвенция. То есть выражения вида $\Gamma \vdash \Delta$, где
    $\Gamma, \Delta$ -- наборы пропозициональных формул
\end{definition}
\begin{definition}
    Формульный образ секвенции \[\Phi(\overbrace{A_1, \dots, A_n}^{\text{антецедент}} 
    \vdash \overbrace{B_1, \dots , B_m}^{\text{сукцедент}}) =
     A_1 \& A_2\&\dots \&A_n \to (B_1 \lor B_2 \lor \dots \lor B_m)\]
\end{definition}
\begin{definition}[Аксиомы СИВ]
    Аксиомы СИВ -- это секвенции вида $\Gamma_1 A\Gamma_2 \vdash \Delta_1 A \Delta_2$
\end{definition}
\begin{lemma}
    Формульный образ аксиомы является тавтологией
\end{lemma}
\begin{proof}
    \begin{gather*}
        \text{Если применить правило Де-Моргана и раскрыть импликацию, то}\\ 
        \text{аргумент } \Phi \text{ в общем виде можно записать так:}\\
        A_1\&\dots\&A_n \to B_1\lor\dots B_m \Leftrightarrow \\
        \lnot A_1 \lor \dots \lor \lnot A_n \lor B_1 \dots \lor B_m\\
        \text{из этого следует, что } \\
        \Phi(A_1, \dots , A_k, A, A_{k+1}, \dots, A_n \vdash B_1, \dots, B_l, A, B_{l+1}, \dots, B_m)\\
        \Leftrightarrow \lnot A_1 \lor \dots \lor \lnot A_k \lor \textcolor{red}{\lnot A} \lor 
        \lnot A_{k+1}\lor\dots \lor\lnot A_n \\ 
        \lor B_1\lor\dots \lor B_l \lor \textcolor{red}{A} 
        \lor B_{l+1}\lor \dots \lor B_m
    \end{gather*}
    \textcolor{red}{красное} даёт нам тавтологию
\end{proof}
\section{Правила вывода в СИВ}
\begin{gather*}
    (\& \quad \vdash): 
    \cfrac{\Gamma_1A\Gamma_2 B\Gamma_3 \vdash \Delta}
    {\Gamma_1 A \& B \Gamma_2 \Gamma_3 \vdash \Delta}\quad
    (\vdash \quad \&): \cfrac{(\Gamma\vdash \Delta_1A\Delta_2), 
    (\Gamma \vdash \Delta_1 B\Delta_2)}{\Gamma \vdash \Delta_1 A \& B \Delta_2}
    \\
    \\
    (\lor \quad \vdash):
    \cfrac{\Gamma_1 B\Gamma_2}{\Gamma_1 A\lor B\Gamma_2 \vdash \Delta}\quad
    (\vdash \quad \lor):
    \cfrac{\Gamma \vdash \Delta_1A\Delta_2B\Delta_3}
    {\Gamma \vdash \Delta_1 A\lor B\Delta_2\Delta_3}
    \\
    \\
    (\to \quad \vdash):
    \cfrac{(\Gamma_1\Gamma_2\vdash\Delta_1A\Delta_2), 
    (\Gamma_1B\Gamma_2\vdash\Delta_1\Delta_2)}
    {\Gamma_1 A\to B\Gamma_2 \vdash \Delta_1 \Delta_2}\quad
    (\vdash \quad \to):
    \cfrac{\Gamma_1A\Gamma_2 \vdash \Delta_1B\Delta_2}
    {\Gamma_1\Gamma_2 \vdash \Delta_1A\to B \Delta_2}
    \\
    \\
    (\lnot\quad \vdash):
    \cfrac{\Gamma_1 \Gamma_2 \vdash \Delta_1A\Delta_2}
    {\Gamma_1 \lnot A\Gamma_2 \vdash \Delta_1\Delta_2}\quad
    (\vdash \quad \lnot):
    \cfrac{\Gamma_1 A\Gamma_2 \vdash \Delta_1\Delta_2}
    {\Gamma_1 \Gamma_2 \vdash \Delta_1\lnot A\Delta_2}\quad
\end{gather*}
\end{document}