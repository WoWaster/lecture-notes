% !TeX root = ./main.tex
\documentclass[main]{subfiles}
\begin{document}
\chapter[Предикатные формулы]{Формулы исчисления предикатов}
\begin{definition}[Предметная переменная]
    Переменная для констант называется предметной переменной.
\end{definition}
\begin{definition}[Терм]
    Определение индуктивное:
    \begin{itemize}
        \item Предметная переменная и предметная константа являются термом
        \item Если $t_1, \dots, t_n$~--- термы, $f$~--- n-местный функциональный символ, то $f(t_1, \dots, t_n)$~--- терм
    \end{itemize}
\end{definition}
\begin{definition}[Атомарная формула]
    Если $t_1, \dots, t_n$~--- термы, $p$~--- n-местный предикатный символ, то $p(t_1, \dots, t_n)$~--- атомарная формула.
\end{definition}
\begin{definition}[Предикатная формула]
    \

    \begin{itemize}
        \item Атомарная формула является предикатной
        \item Если $F_1, F_2$~--- предикатные формулы, $*$~--- бинарная логическая связка, то $\lnot F_1, \  (F_1 * F_2)$~--- предикатные формулы
        \item Если $F$~--- предикатная формула, $x$~--- предметная переменная, то $\forall x F, \ \exists x F$~--- предикатные формулы
    \end{itemize}
\end{definition}
\begin{definition}[Кванторный комплекс]
    Выражения $\forall x,\, \exists x$ называются кванторными комплексами.
\end{definition}
\begin{definition}[Область действия квантора]
    Область действия квантора~--- формула, стоящая непосредственно за кванторным комплексом.
\end{definition}
\begin{example}
    $\textcolor{red}{\forall} x  \textcolor{red}{(}P \to \textcolor{green}{\exists} y \textcolor{green}{(}Q\lor R\textcolor{green}{)} \and \textcolor{blue}{\exists}y \textcolor{blue}{P}\textcolor{red}{)}$. Цветом выделены области действия соответствующих кванторов.
\end{example}
\begin{definition}[Связанное вхождение переменной]
    Вхождение переменной, которая находится в области действия квантора по этой переменной называется связанным.
\end{definition}
\begin{definition}[Свободное вхождение переменной]
    В противном случае вхождение называется свободным.
\end{definition}
\begin{example}
    $\forall x (P(\textcolor{red}{x},y,z) \to \exists y (Q(\textcolor{red}{y}, z) \lor \forall z R(\textcolor{red}{x}, \textcolor{red}{z})))$. Красным помечены все связанные переменные.
\end{example}
\begin{definition}[Интерпретация]
    Для её задания достаточно:
    \begin{enumerate}
        \item Область интерпретации $D$, то есть множество констант
        \item Каждому n-местному предикатному символу поставить в соответствие конкретное отношение из $D^n$
        \item Каждому n-местному функциональному символу поставить в соответствие конкретную функцию $f: D^n \to D$
    \end{enumerate}
\end{definition}
% Тут должен быть блок с описанием квантора существования и всеобщности на псевдокоде, я не уверен, как это обосновать и не понимаю, какую роль он играет, так что пока опущу его
\begin{definition}[Замкнутая формула]
    Формула, не содержащая свободных вхождений переменных, называется замкнутой.
\end{definition}
Заметим тот факт, что замкнутая формула в каждой конкретной интерпретации задаёт высказывание, которое либо истинное, либо ложное.
\begin{definition}[Свойство]
    Формула с одной свободной переменной задаёт свойство объекта в этой интерпретации
\end{definition}
\begin{definition}[Общезначимость]
    Предикатная формула называется общезначимой, если она истинна в любой интерпретации на любом наборе значений свободных переменных
\end{definition}
\begin{definition}[Равносильность]
    Две формулы называются равносильными, если в любой интерпретации на любом наборе значений свободных переменных значение формул совпадают. (Обозначение: $P \lra Q$)
\end{definition}
Основные равносильности:
\begin{enumerate}
    \item $\lnot \forall x P \lra \exists x \lnot P$
    \item $\lnot \exists x P \lra \forall x \lnot P$
    \item $\forall x (P\and Q) \lra \forall x P \and Q$, если $x$ не входит свободно в $Q$
    \item $\exists x (P\and Q) \lra \exists x P \and Q$, если $x$ не входит свободно в $Q$
    \item $\forall x (P\lor Q) \lra \forall x P \lor Q$, если $x$ не входит свободно в $Q$
    \item $\exists x (P\lor Q) \lra \exists x P \lor Q$, если $x$ не входит свободно в $Q$
    \item $P \to \forall x Q \lra \forall x(P\to Q)$, если $x$ не входит свободно в $P$
    \item $P \to \exists x Q \lra \exists x(P\to Q)$, если $x$ не входит свободно в $P$
    \item $\forall x P \to Q \lra \exists x(P\to Q)$, если $x$ не входит свободно в $Q$
    \item $\exists x P \to Q \lra \forall x(P\to Q)$, если $x$ не входит свободно в $Q$
\end{enumerate}
\begin{definition}[Свобода для подстановки]
    Терм $t$ свободен для подстановки в формулу $F$ вместо свободно входящих предметных переменных $x$, если он не содержит переменных, в области действия которых имеются его вхождения.
\end{definition}
\begin{example}
    $S(x) \lor \forall x(P(x,y,z) \to \forall x\exists y(Q(x,z) \lor R(y,x,z)))$

    Напишем, табличку, какой терм (обозначенный за $t$) свободен для подстановки в написанную выше формулу вместо соответствующих переменных:
    % \begin{gather*}
    %     \begin{tabular}{|c|c|c|c|c|}
    %         \hline
    %         t       & x & y & z & u \\
    %         \hline
    %         f(x)    & + & - & - & + \\
    %         \hline
    %         f(y)    & + & + & - & + \\
    %         \hline
    %         f(z)    & + & + & + & + \\
    %         \hline
    %         f(u)    & + & + & + & + \\
    %         \hline
    %         g(x, y) & + & - & - & + \\
    %         \hline
    %         g(x, z) & + & - & - & + \\
    %         \hline
    %         g(x, u) & + & - & - & + \\
    %         \hline
    %         g(y, z) & + & + & - & + \\
    %         \hline
    %         \dots                   \\
    %         \hline
    %     \end{tabular}
    % \end{gather*}
    \[\begin{array}{*5{c}}
            \toprule
            t       & x & y & z & u \\
            \midrule
            f(x)    & + & - & - & + \\
            \midrule
            f(y)    & + & + & - & + \\
            \midrule
            f(z)    & + & + & + & + \\
            \midrule
            f(u)    & + & + & + & + \\
            \midrule
            g(x, y) & + & - & - & + \\
            \midrule
            g(x, z) & + & - & - & + \\
            \midrule
            g(x, u) & + & - & - & + \\
            \midrule
            g(y, z) & + & + & - & + \\
            \midrule
            \dots   &   &   &   &   \\
            \bottomrule
        \end{array}\]

    Логика здесь такая: смотрим на то, какие переменные есть в терме, смотрим, сломается ли что-то, если мы везде вместо переменной напишем терм (то есть не станет ли то, что было свободным вдруг связным). Если ничего не ломается, то можем подставить, а если ломается, то, соответственно, не можем.
\end{example}
\end{document}
