% !TeX root = ./main.tex
\documentclass[main]{subfiles}
\begin{document}
\section{Знаки препинания в предложениях с обособленными членами.}
\textbf{Согласованные определения:}

\begin{longtabu}{X[l] X[l]}
      \toprule
      \multicolumn{1}{c}{\textbf{Обособляются}} &
      \multicolumn{1}{c}{\textbf{Не обособляются}} \\
      \midrule
      \endfirsthead
      \midrule
      \endhead
      \endfoot
      \bottomrule
      \endlastfoot
      Если выражены причастием или прилагательным с зависимыми словами и стоящие после определяемого слова. \newline
      \case{На березах, \textbf{обращенных к солнцу}, […]} &
      Если стоят после определяемого слова и сливаются с ним по смыслу. \newline
      \case{Аркадий встал с утра с настроением \textbf{игриво радостным}.} \\
      \midrule
      Определения, стоящие перед определяем существительным и с добавочным обстоятельственным значением. (близки к~оборотам с \textit{будучи}) \newline
      \case{\textbf{Еще прозрачные}, леса как будто пухом зеленеют.} &
      Если стоят перед определяемым словом и не имеют добавочных обстоятельственных значений. \newline
      \case{Кому-нибудь […] \textbf{на крытом черепицей старинном} чердаке.} \\
      \midrule
      Определения, независимо от их положения к определяемому слову, отделенные от него, другими словами. \newline
      \case{На улицу меня пускали редко, каждый раз я возвращался домой, \textbf{избитый мальчишками}.} &
      Если связаны по смыслу и с подлежащим, и со сказуемым при этом определение входит в состав составного именного сказуемого. \newline
      \case{Ночи были \textbf{короткие, светлые}.} \\
      \midrule
      Одиночные определения, стоящие после определяемого слова, если они имеют добавочное значение. \newline
      \case{Люди же, \textbf{изумленные}, стояли как камни.} &
      Если выражены сложной формой степеней сравнения прилагательных. \newline
      \case{Гент никогда не видел человека \textbf{более бесхитростного}, чем он.} \\
      \midrule
      \multicolumn{2}{p{\dimexpr\textwidth-2\tabcolsep\relax}}{\noindent Два или более одиночных определения, стоящих после определяемого слова. \newline
            \case{Впереди были горы, \textbf{высокие, неприступные}.} \newline
            Если перед определяемым словом нет определения, то последующие определения могут \newline не~обособляться в зависимости от смысла и их интонационно-смысловой близости. \newline
            \case{Кухарка \textbf{жирная} у скаред на сковороде мясо жарит.}} \\
\end{longtabu}

\textbf{Согласованные определения-местоимения:}
\begin{longtabu}{X[l] X[l]}
      \toprule
      \multicolumn{1}{c}{\textbf{Обособляются}} &
      \multicolumn{1}{c}{\textbf{Не обособляются}} \\
      \midrule
      \endfirsthead
      \midrule
      \endhead
      \endfoot
      \bottomrule
      \endlastfoot
      Определения, относящееся к личному местоимению, независимо от его места в~предложении. \newline
      \case{Я читал, \textbf{возбужденный и взволнованный}.} &
      Определения, относящееся к личному местоимению, если оно связано и с~подлежащим, и со сказуемым. \newline
      \case{Они шли домой \textbf{веселые и довольные}.} \\
      \midrule\relax
      (следует из правого столбца.) Если связь менее тесная, а после местоимения делается пауза. \newline
      \case{Что-то, \textbf{лиловое, игривое}, плескалось и пенилось сзади его головы.} &
      Определения, стоящие после неопределенного местоимения, если они образуют с ним единое смысловое целое. \newline
      \case{Сзади его головы плескалось и пенилось что-то \textbf{лиловое, игривое}.} \\
      \midrule\relax
      (следует из правого столбца). Если есть добавочное значение и интонационное выделение. \newline
      \case{Этот, \textbf{блестящий на солнце}, снег мне запомнился.} &
      Причастные обороты, стоящие после определительных, указательных и притяжательных местоимений, если они связаны по смыслу и интонационно. \newline
      \case{Мне запомнился этот \textbf{блестящий на солнце} снег.} \\
      \midrule
      & Определения, относящееся к личному местоимению, в восклицательных предложениях. \newline
      \case{Ой вы мои \textbf{хорошие}!} \\
\end{longtabu}

\textbf{Несогласованные определения:}

Несогласованные определения, выраженные косвенными падежами существительных, обособляются, если подчеркивается выражаемое ими значение. \newline
\case{Дед, \textbf{в бабушкиной кацавейке, в старом картузе без козырька}, щурится, чему-то улыбается.} \newline
Несогласованные определения \textbf{обособляются} если:
\begin{enumerate}
      \item Относятся к личному местоимению. \newline
            \case{\textbf{В широком плаще, с лицом под густой вуалью}, она долгие часы простаивала у подъезда.}
      \item Относятся к имени собственному. \newline
            \case{\textbf{Русый, с кудрявой головой, без шапки и с расстегнутой на груди рубахой}, Дымов казался красивым и необыкновенно сильным.}
      \item Отделены от определяемого слова другими членами предложения. \newline
            \case{Лицо её, \textbf{с симпатично вздёрнутым веснушчатым носиком}, светилось радостью.}
      \item Образуют ряд однородных членов с согласованными определениями. \newline
            \case{\textbf{Квадратный, широкогрудый, с огромной кудрявой головой}, он явился под вечер.}
      \item Употреблены при существительном со значением родства, профессии, должности и т.~д. и выражают добавочное сообщение, содержат конкретизацию. \newline
            \case{Мать, \textbf{полуголая, в красной юбке}, стоит на коленях.}
      \item Выражено оборотом с прилагательным в сравнительной степени. \newline
            \case{Маленькие усы, \textbf{немного темнее волос}, делали его лицо привлекательным.}
\end{enumerate}
Несогласованные определения могут \textbf{выделяться тире}:
\begin{enumerate}
      \item Если они выражены инфинитивом (можно подставить \textit{а именно}).\newline
            \case{Таким образом, волей-неволей мы должны были отказаться от первоначального намерения — \textbf{пройти во Владивосток вдоль берегов Сибири}.}
      \item В середине предложения, выделяются с обеих сторон.
      \item Если по условиям контекста после определения, выраженного инфинитивом, должна ставиться запятая, то второе тире опускается. \newline
            \case{Это была моя обязанность — \textbf{нести отцу обед}, и я ею очень гордился.}
\end{enumerate}

Тире при согласованных и несогласованных определениях, имеющих различные способы выражения, можно поставить тогда, когда необходимо особо подчеркнуть их смысловую значимость; в данных случаях они выполняют также функцию пояснения.

\case{На пороге избы встретил меня старик — \textbf{лысый, низкого роста, плечистый и плотный}. Правда, что птица не знает человеческих — \textbf{моральных, психологических государственных} — границ.}
\end{document}