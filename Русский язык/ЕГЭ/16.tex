\documentclass[12pt]{article}
%%% PDF settings
\pdfvariable minorversion 7 % Set PDF version to 1.7.

%%% Fonts and language setup.
% Setup fonts.
\usepackage{fontspec}
\setmainfont{CMU Serif}
% TODO: Find a sans-serif font.
\setmonofont[Contextuals={Alternate}]{Fira Code Regular}

\usepackage[english,russian]{babel} % Enable russian language.

% Enable ligatures to work in verbatim environment.
\usepackage{verbatim}
\makeatletter
\def\verbatim@nolig@list{}
\makeatother

\usepackage{hologo} % Add fancy logos like \LuaLaTeX.

\usepackage{csquotes} % Quotes based on babel settings

\usepackage{indentfirst} % Indent first paragraph after heading.


%%% Page settings.
% Fancy page geometry.
\usepackage{geometry}
\geometry{
    margin=2cm,
    % headheight=15pt,
    % includefoot=true,
}

\usepackage{multicol} % Multicols environment.

\usepackage{lastpage} % Shows last page number.

\usepackage[usenames,dvipsnames,svgnames,table,rgb]{xcolor} % Enable color support.


%%% Particular subjects helper tools.
%% Math
\usepackage{amsmath, amsfonts, amssymb, amsthm, mathtools} % Advanced math tools.
\usepackage{unicode-math} % Allow TTF and OTF fonts in math and allow direct typing unicode math characters
\unimathsetup{
    warnings-off={
            mathtools-colon,
            mathtools-overbracket
        }
}

%% Physics
\usepackage{siunitx} % Fancy method to write physic formulas.

%% Chemistry
% Fix babel and chemformula confusion
\let\Ch\ch % save the command for the hyperbolic cosine
\let\ch\relax % undefine \ch
\usepackage{chemmacros} % Chemistry signs and other.
\usepackage{chemformula} % Write down chemical formulas.
\usepackage{chemfig} % Draw structural formulas.


%%% Tables
\usepackage{array} % Improved column definition.
\usepackage{tabularx} % Adds X columns that are stretched to have equal width.
\usepackage{tabulary} % Makes rows equal height by adjusting column width.
\usepackage{booktabs} % Adds fancy rules to use with tables.
\usepackage{longtable} % Table that spans across multiple pages.
\usepackage{multirow} % Combine rows in table.


%%% Images
% Support for images.
\usepackage{graphicx}
\graphicspath{{images/}}
\usepackage{wrapfig} % Floating images.



% Please add everything above.

%%% HyperRef
% Add hyperlinks to PDF and make them invisible.
\usepackage{hyperref}
\hypersetup{
    hidelinks
}

\usepackage{microtype}
\newcommand{\exam}[1]{{\small \textit{#1}}} % Make font one step smaller and adds italicizes content.
\begin{document}
\begin{center}
    {\large \textit{\textbf{Задание №16}}}

    \textit{\textbf{Пунктуация в сложносочиненном предложении и в предложении с~однородными членами.}}
\end{center}
\begin{longtabu}[c]{X[l]X[l]}
    \toprule
    \centering {\large Запятая ставится} &
    \centering {\large Запятая НЕ ставится} \\
    \midrule
    \endfirsthead
    \midrule
    \centering {\large Запятая ставится} &
    \centering {\large Запятая НЕ ставится} \\
    \midrule
    \endhead
    \endfoot
    \bottomrule
    \endlastfoot

    \multicolumn{2}{c}{\large \textbf{Определения}} \\
    \midrule
    \multicolumn{1}{c}{\textbf{Однородные}} &
    \multicolumn{1}{c}{\textbf{Неоднородные}} \\
    \midrule
    Обозначают отличительные признаки разных предметов. \newline
    \exam{Пионы белые, розоватые, розовые, темно-вишневые.} &
    Характеризуют предмет с~разных сторон. \newline
    \exam{Большая железная печь.} \\
    \midrule
    Обозначат признаки одного предмета, с~одной стороны. \newline
    \exam{Крупный, короткий, благодатный дождь.} &
    Выражены сочетанием качественного и относительного прилагательных. \newline
    \exam{Кожаные большие сапоги.} \\
    \midrule
    Являются контекстными синонимами или эпитетами. \newline
    \exam{Бледно-голубые, стеклянные глаза.} &
    \multirow[t]{7}{=}{Имеют предшествующее определение, которое относится не непосредственно к~определяемому существительному, а к~сочетанию последующего определения и определяемого существительного. \newline
        \exam{Профессор открыл свой новенький кожаный саквояж.}}\\
    \cmidrule{1-1}
    Образуют смысловую градацию. \newline
    \exam{Темная, мрачная, душная комната.} & \\
    \cmidrule{1-1}
    Выражены причастным оборотом, стоящим за одиночным определением. \newline
    \exam{Пожилая, гладко причесанная на прямой пробор женщина.} & \\
    \cmidrule{1-1}
    Стоят после определяемого имени существительного. \newline
    \exam{Лицо Егора, бледное, одутловатое, […]} & \\
    \cmidrule{1-1}
    Противопоставлены другим определениям при одном существительном. \newline
    \exam{Цветы яркие, красочные, но неестественно крупные и благоухающие.} & \\
    \midrule


    \multicolumn{1}{c}{\textbf{Однородные}} &
    \multicolumn{1}{c}{\textbf{Неоднородные}} \\
    \midrule
    Характеризуют предмет с~одной стороны, обозначают его близкие признаки. \newline
    \exam{Доктор педагогических наук, профессор Ступин.} &
    Характеризуют предмет с~разных сторон, обозначают его разные признаки. \newline
    \exam{Командир дивизии генерал-майор Панфилов.} \\
    \midrule

    \multicolumn{2}{c}{\large \textbf{Однородные члены с~неповторяющимися союзами}} \\
    \midrule
    С~союзами \textib{а, но, да}~(в значении \textib{но})\textib{, однако, зато, тем не менее, хотя}~(с~уступительными значением). \newline
    \exam{Резв, но мил.} &
    \multirow[t]{3}{=}{С~союзами \textib{и, да}~(в значении \textib{и})\textib{, или, либо}. \newline
        \exam{Читать да писать.}} \\
    \cmidrule{1-1}
    Между двумя однородными сказуемыми, соединенными одиночным \textib{и}, СТАВИТЬСЯ ТИРЕ для указания следствия во втором сказуемом или резкой смены. \newline
    \exam{Хотел объехать целый свет – и не объехал сотой доли.} & \\
    \midrule

    \multicolumn{2}{c}{\large \textbf{Однородные члены с~повторяющимися союзами}} \\
    \midrule
    С~союзами \textib{и…~и, да…~да, ни…~ни, или…~или, ли…~ли, то…~то, не то…~не то, либо…~либо} и др. \newline
    \exam{Разлюбил и брань, и саблю, и свинец.} &
    Если образуется смысловое единство между двумя однородными членами. \newline
    \exam{И день и ночь, и день и ночь!} \\
    \midrule
    >3 слов и союз не стоит перед первым. \newline
    \exam{Заставляли бледнеть мужчин, и женщин, и детей.} &
    Члены соединяются попарно, запятая ставится между парными группами. \newline
    \exam{Пьяные и трезвые, робкие и отчаянные.} \\
    \midrule
    \multirow[t]{5}{=}{С~союзами \textib{как…~так и, не только…~но и, не столько…~сколько}, запятая ставится только перед второй частью. \newline
        \exam{Не только для писателей, но и для людей.}} &
    В фразеологических выражениях из двух слов с~повторением и или ни. \newline
    \exam{Ни то ни се.} \\
    \cmidrule{2-2}
    & В союзах \textib{не то что…~а, не то чтобы…~а~(но)}, запятая перед \textib{что} и \textib{чтобы} не ставится.\\
    \midrule

    \multicolumn{2}{c}{\large \textbf{Сложносочиненные предложения}} \\
    \midrule
    Если части соединены соединительными, противительными или разделительными союзами. \newline
    \exam{Зацвела черемуха, и весь город тащил себе ветки с~белыми цветами.} &
    Если есть общий для обеих частей второстепенный член. \newline
    \exam{\textbf{Во время частых зимних штормов} в порту ревели басами океанские пароходы и скрипело от ветра окно.} \\
    \midrule
    Если безличные предложения в составе ССП неоднородны по своему составу. \newline
    \exam{В комнате было душно, и мне захотелось выйти на свежий воздух.} &
    \multirow[t]{4}{=}{Если обе части восклицательные или вопросительные предложения, объединенные общей интонацией. \newline
        \exam{Зачем ты послан был и кто тебя послал?}} \\
    \cmidrule{1-1}
    >2 номинативных предложений. \newline
    \exam{Ночь, тишина, и звезды, звезды.} & \\
    \midrule
    \multicolumn{2}{p{\linewidth}}{\textib{Точка с~запятой} ставится между частями ССП, которые значительно распространены и имеют знаки препинания внутри.\newline
        \exam{Обыкновенно Вернер исподтишка насмехался над своими больными; но раз я видел, как он плакал над умирающим солдатом.}} \\
    \midrule
    \multicolumn{2}{p{\linewidth}}{\textib{Тире} в ССП ставится между его частями, которые содержат неожиданное присоединение или резкое противопоставление.\newline
        \exam{Я сажусь на трамвай – и через 20 минут я опять в поле.}}
\end{longtabu}
\end{document}