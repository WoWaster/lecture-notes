% !TeX root = ./main.tex
\documentclass[main]{subfile}
\begin{document}
\section{Знаки препинания в сложных предложениях с разными видами связи.}


\begin{longtabu}{X[l] X[l]}
    \toprule
    \multicolumn{1}{c}{Запятая ставится} &
    \multicolumn{1}{c}{Запятая НЕ ставится} \\
    \midrule
    \endfirsthead
    \midrule
    \endhead
    \endfoot
    \bottomrule
    \endlastfoot

    Между однородными придаточными, не соединенными сочинительными союзами. \newline
    \case{Он был из тех, кого судьба вела \newline
        Кремнистыми путями испытанья, \newline
        Кого везде опасность стерегла, \newline
        Насмешливо грозя тоской изгнанья. \newline} &
    Если однородные придаточные соединяются неповторяющимися соединительными или разделительными союзами. \newline
    \case{Каждый раз, приезжая сюда, я видел, как разрастается сад и как старятся дом и его хозяйка.} \\
    \midrule
    Между однородными придаточными, соединенными повторяющимися сочинительными союзами. \newline
    \case{Матвей подробно рассказал и какие блюда ел, и из чего эти блюда приготовлены.} &
    На стыке сочинительного и подчинительного союзов или двух подчинительных союзов, если за придаточным следует вторая часть сложного союза – \textib{то, так, но} (1). \\
    \cmidrule{1-1}
    Между придаточными с последовательным подчинением.\newline
    \case{Немножко далее речка, вероятно, сливалась с другой такою же речонкой, потому что шагах в ста от холма по ее течению зеленела густая, пышная осока, из которой, когда подъезжала бричка, с криком вылетело три бекаса.} &
    Однако \textbf{запятая ставится}, если вторая часть сложного подчинительного союза отсутствует (2).
    \begin{enumerate}
        \item[(1)] \case{Вся моя мысль в том, \textib{что ежели} люди порочные связаны между собой и составляют силу, \textib{то} людям честным надо сделать только то же самое.}
        \item[(2)] \case{Весь день стояла прекрасная погода, \textib{но, когда} мы подплывали к Одессе, пошел сильных дождь.}
    \end{enumerate} \\
    \midrule
    \multicolumn{2}{p{\dimexpr\textwidth-2\tabcolsep\relax}}{\textbf{Точка с запятой} ставится между однородными придаточными, если они сильно распространены или имеет внутри запятые. \newline
        \case{О чем же он думал? О том, что был он беден; что трудом он должен был себе доставить и независимость и честь; что мог бы Бог ему прибавить ума и денег; что ведь есть такие праздные счастливцы, ума недальнего, ленивцы, которым жизнь куда легка.}}\\
\end{longtabu}

\end{document}