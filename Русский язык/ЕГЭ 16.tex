\documentclass[12pt]{article}
%%% PDF settings
\pdfvariable minorversion 7 % Set PDF version to 1.7.

%%% Fonts and language setup.
% Setup fonts.
\usepackage{fontspec}
\setmainfont{CMU Serif}
% TODO: Find a sans-serif font.
\setmonofont[Contextuals={Alternate}]{Fira Code Regular}

\usepackage[english,russian]{babel} % Enable russian language.

% Enable ligatures to work in verbatim environment.
\usepackage{verbatim}
\makeatletter
\def\verbatim@nolig@list{}
\makeatother

\usepackage{hologo} % Add fancy logos like \LuaLaTeX.

\usepackage{csquotes} % Quotes based on babel settings

\usepackage{indentfirst} % Indent first paragraph after heading.


%%% Page settings.
% Fancy page geometry.
\usepackage{geometry}
\geometry{
    margin=2cm,
    % headheight=15pt,
    % includefoot=true,
}

\usepackage{multicol} % Multicols environment.

\usepackage{lastpage} % Shows last page number.

\usepackage[usenames,dvipsnames,svgnames,table,rgb]{xcolor} % Enable color support.


%%% Particular subjects helper tools.
%% Math
\usepackage{amsmath, amsfonts, amssymb, amsthm, mathtools} % Advanced math tools.
\usepackage{unicode-math} % Allow TTF and OTF fonts in math and allow direct typing unicode math characters
\unimathsetup{
    warnings-off={
            mathtools-colon,
            mathtools-overbracket
        }
}

%% Physics
\usepackage{siunitx} % Fancy method to write physic formulas.

%% Chemistry
% Fix babel and chemformula confusion
\let\Ch\ch % save the command for the hyperbolic cosine
\let\ch\relax % undefine \ch
\usepackage{chemmacros} % Chemistry signs and other.
\usepackage{chemformula} % Write down chemical formulas.
\usepackage{chemfig} % Draw structural formulas.


%%% Tables
\usepackage{array} % Improved column definition.
\usepackage{tabularx} % Adds X columns that are stretched to have equal width.
\usepackage{tabulary} % Makes rows equal height by adjusting column width.
\usepackage{booktabs} % Adds fancy rules to use with tables.
\usepackage{longtable} % Table that spans across multiple pages.
\usepackage{multirow} % Combine rows in table.


%%% Images
% Support for images.
\usepackage{graphicx}
\graphicspath{{images/}}
\usepackage{wrapfig} % Floating images.



% Please add everything above.

%%% HyperRef
% Add hyperlinks to PDF and make them invisible.
\usepackage{hyperref}
\hypersetup{
    hidelinks
}

\usepackage{wasysym}
\renewcommand{\labelenumii}{\theenumii}
\renewcommand{\theenumii}{\theenumi.\arabic{enumii}.}
\usepackage[normalem]{ulem}
\begin{document}

\section{16 задание --- ССП и однородные члены}

\subsection{Алгоритм}
\begin{enumerate}
    \item Выделить грамматические основы.
          \begin{itemize}
              \item[!] если части ССП --- односоставные предложения
          \end{itemize}
    \item Если предложение простое, смотри на однородные члены
          \begin{enumerate}
              \item  повторяющиеся союзы
              \item  двойные союзы (запятая перед второй частью)
              \item  союз <<да>> (и, но)
              \item  союз <<да и>>
              \item  фразеологизмы
              \item  неоднородные определение
          \end{enumerate}
          ОБЯЗАТЕЛЬНО СХЕМА
    \item Если предложение сложное, ищи общий второстепенный член
\end{enumerate}

\subsection{Примеры}
\begin{enumerate}
    \item Уже \uuline{смеркалось} и совсем \uuline{не хотелось спать}.
    \item \begin{enumerate}
              \item \fullmoon, \fullmoon{} и \fullmoon; \fullmoon{}, и \fullmoon{}, и \fullmoon; и \fullmoon, и \fullmoon, и \fullmoon; \fullmoon{} и \fullmoon{},\fullmoon{} и \fullmoon{}
              \item не только \fullmoon , но и \fullmoon;как \fullmoon, так \fullmoon; хотя и \fullmoon, но и \fullmoon; если не \fullmoon, то \fullmoon; не столько \fullmoon, сколько \fullmoon; не то что бы \fullmoon, но \fullmoon
              \item да = <<и>> или <<, но>>
              \item не ставится, если союз указывает на неожиданный переход от одного действия к другому (чаще глагольные сочетания)\\
                    \textit{Он взял да и сказал.}\\
                    ставится, если союз вводит замечания и разъяснения\\
                    \textit{Лежали только фантики, да и те никому не нужные.}
              \item ни дать ни взять; ни слуху ни духу
              \item зеленые, радужные пятна; масляные радужные пятна
          \end{enumerate}
    \item Ищи общее:
          \begin{itemize}
              \item второстепенный член --- этим утром ярко светило солнце и щебетали птицы
              \item вводная конструкция (слово) --- во-первых, льется дождь и дует ветер, во-вторых...
              \item общее придаточное предложение
              \item запятая не ставится если части ССП - это два:
                    \begin{itemize}
                        \item назывных предложения --- смех и слезы
                        \item вопросительных --- кто ты и как ты живешь?
                        \item восклицательных --- откройся мир и засияй любовь!
                        \item побудительных --- откройся мир и засияй любовь!
                        \item безличных --- темно и душно
                        \item неопределенно личных (местоимение <<они>>) --- грустили вместе и мечтали вместе
                    \end{itemize}
          \end{itemize}
\end{enumerate}

\end{document}