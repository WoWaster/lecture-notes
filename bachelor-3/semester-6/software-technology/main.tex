\documentclass[12pt, a4paper, oneside]{memoir}
%%% PDF settings
\pdfvariable minorversion 7 % Set PDF version to 1.7.

%%% Fonts and language setup.
\usepackage{polyglossia}
% Setup fonts.
\usepackage{fontspec}
\setmainfont{CMU Serif}
\setsansfont{CMU Sans Serif}
\setmonofont{CMU Typewriter Text}

\usepackage{microtype} % Add fancy-schmancy font tricks.

\usepackage{xcolor} % Add colors support.

%% Math
\usepackage{amsmath, amsfonts, amssymb, amsthm, mathtools} % Advanced math tools.
\usepackage{unicode-math} % Allow TTF and OTF fonts in math and allow direct typing unicode math characters.
\unimathsetup{
    warnings-off={
            mathtools-colon,
            mathtools-overbracket
        }
}
\setmathfont{Latin Modern Math} % default
\setmathfont[range={\setminus,\varnothing,\smashtimes}]{Asana Math}

%%% Images
\usepackage{graphicx}
\graphicspath{{figures/}}
\usepackage{import}

%%% Polyglossia setup after (nearly) everything as described in documentation.
\setdefaultlanguage{russian}
\setotherlanguage{english}

\usepackage{csquotes}

%%% Custom commands
\AtBeginDocument{\renewcommand{\leq}{\leqslant}}
\AtBeginDocument{\renewcommand{\geq}{\geqslant}}

%%% HyperRef
\usepackage{hyperref}

\title{Денежная сторона IT проекта}
\author{Кириленко Я.~А.\thanks{Конспект Пономарев Н.~А.}}
\date{28 мая 2024 г.}

\begin{document}

\maketitle

\section*{Введение}
ЮВЛ просил рассказать по деньгам в проекте, но ЯА знает только как их тратить, так что будет рассказ о том как пролететь.

Проекты~--- это некоторая осмысленная деятельность, ведущая к какой-то цели, в рамках ограниченных ресурсов.

\enquote{Ваш труд нужен, только если он кому-то нужен!}

Нам надо выполнить что-то и получить денег.
Как получить денег?
Надо понять наши потребности.
Наша задача выполнить требования, а точнее вопрос управления ожиданиями ($\neq$ то, что задокументировано).
Платят, потому что пользователь рад!
Так что будем смотреть на все эти вещи сугубо с экономической стороны.
Ценность вашей работы~--- ценность для потребителя.

Сколько труда уйдёт на удовлетворения требований?
Для этого служат методологии разработки, ибо цена разработки огромна, а требования не всегда отражают ожидания.

К вам пришли и заказали простую и понятную вещь.
Пришел Кириленко и говорит: \enquote{сделай мне калькулятор, но надо завтра!}.
Сказал \enquote{нет}~--- Кириленко к врагам.
Надо вселять уверенность, что ты на стороне заказчика.
Вот это все будем сегодня говорить

\section*{Оценка проектов}

Справимся или нет~--- грубая оценка.
Сказать готовы или не готовы.
Не предполагает обширного диалога.
Чувствуются высокие риски.

ТКП (Технико-коммерческое предложение)~--- уже сильно более подробная штука.
Это пойдёт в основу договора.

Круто, но вообще надо уметь оценивать и по ходу проекта, ибо всё может пойти не так.

...

Бывает переоценка проекта.

Надо различать оценку, бизнес-цель и обязательство.

Оценка вообще имеет вероятностную природу.
Простое правило: если можем измерять~--- измеряем, если не может~--- придумываем как.

Программное управление~--- как запрограмировали так и идёт, т.е. косяков не бывает и всё работает как предсказано.
Но это детская мечта, в жизни так не бывает!
ВСЕГДА нужна обратная связь.
(Это вообще про кибернетику)
Поэтому всегда надо измерять, чтобы управлять.

Зачем нужны всякие диаграммы Гантта?
Хорошо планируем оптимальную траекторию, но вообще происходит всё что угодно.
А проект надо завершать по срокам.

Самый простой способ измерения объема проекта~--- SLOC.
Но очевидно, что у него слишком много подковырок, хотя метрика достаточно приемлемая.

Есть ли связь количества строк и количества фич?
Удивительно, но есть!
Особенно хорошо работает для крупных областей, например в сайтостроении.
С большими фреймворками вообще бывает линейное соответствие.

UCP~--- как-то распределяем попугаев по элементам программы.
FP и всякое остальное: ссылки будут потом.
Эти штуки с одной стороны ресерч, с другой стороны хороши, когда одинаковые проекты клепаются пачками.

(Определения всех трёх смотри на слайде)

Когда надо выполнить оценку, надо сделать \textit{объективную} оценку трудозатрат.
Но обычно есть какая-то цепочка, которая говорит \enquote{надо сделать}.

Оценка  в попугаях $\implies$ SLOC $\implies$ ещё более крутые метрики, но в конце проекта~--- печаль.

Бизнес-цель, что от нас ждут (субъективна).
Обязательства, на что договорились.

Обязательства можно обсуждать, но потом должны стараться их выполнить как можно круче.

В оценке не должно быть \enquote{кажется}.
Правильно \enquote{Займет от ... до ..., вероятно столько-то}.

Информации в начале сильно меньше, и она регулярно увеличивается в количеств.
Есть ресерч старый всё ещё неплохо подходит, потому что совпадает с современным жизненным опытом (пока что верим религиозно).
При первичной оценке может быть расхождения вплоть до четырёх раз в любую сторону.
На моменте подписания договора уже до двух раз.
Около 20\% будут где-то на моменте готового дизайна приложения.
Расхождения будут нулевые, только когда (почти) закончим проект.
Поэтому нужна регулярная переоценка.
(Есть ещё числа от NASA, смотри слайд)

\section*{Калибровка}

Чтобы делать более точную оценку, нужны прошлые данные, лучше всего личные.
В НИРах такого не бывает!

Данные индустрии~--- хорошо, что есть, но слабо натягивается на собственные проекты.

\section*{Ограничения}

А что вообще влияет на оценку?
Почему 3 попугая из скрама~--- это неделя одного разработчика?
Житейский опыт определенной команды.

Влияющие факторы, смотри на слайде.
Какое-то исследование, выводы:
если с заказчиком беседовал придурок, то есть шанс сделать проект в 1.5 раза быстрее или в 2 раза медленнее.
Нужно очень хорошо знать команду.
Если вы не компетентны, то прежде чем ввязываться в бизнес, надо сильно прокачаться, иначе будет больно.
Для этого есть модель с супер формулой COCOMOII (с нелинейной зависимостью).

Но обычно работает экспертный метод~(\enquote{Я считаю...}).
Формально это не так, поэтому надо разобраться.

\section*{Этапы процесса оценки}

Желательно не одному, при этом люди должны быть с разными ролями, так как про чужие задачи никто не помнит.
Больше инфы~--- меньше риски.
Всегда можно позвать даже внешнего человека.

Любая методика и инструментарий~--- результат взвешенного выбора командой, который нужно документировать.

А что нужно будет сделать надо декомпозировать, это позволит найти похожие задачи в прошлом для лучше оценки.

Дальше понятно, но что мы оцениваем???
Это обсудим дальше.

\section*{Сбор и анализ информации}
Смотри жирное на слайде, остальное уже было.

Заключить договор~--- взять обязательства.
Наша задача быстро сужать конус, тем самым повышая точность.

На каждом этапе есть понимание в виде допущений и открытых вопросов, НО документировано!
Профессионализм аналитика/архитектора~--- предугадать граничные случаи.
Всё, что не было записано, сыграет против вас.
Условно \enquote{Мы имеем право выбрать фреймворк}.
В идеале открытые вопросы обсуждаются и переходят в допущения, который понятны заказчику.

Декомпозиция нужна~--- WBS основа (узнай сам).

Модель оценки желательно формально описанный (с формулами) подход к проекту.
(COCOMOII~--- монстр, а вот Scrum норм).

\section*{Выполнение оценки}
См. слайд

Не бывает просто оценки.

Всегда есть какой-то размер системы.

Оценка трудозатрат.
Сколько, что и как, но не размер системы.

Длительность тоже понятно.

Стоимость~--- как бы следует, но куча нюансов.
Сложно загрузить на 100\%, поэтому платить как бы надо.
Поэтому стоимость включает операционные расходы и внезапные риски.

\section*{Риски}
Про управления рисками и лирическое отступление можно почитать на слайде.
(вообще было же)
Риски все должны быть на заказчике, то есть он о них знает, а значит обговорены и (в идеале) вложены в стоимость.
На слайде чеклист контролируете вы или нет, но это уже скорее для взрослого менеджера, а не третьекурсника.

Что делать с рисками?

Их можно заложить в стоимость, но если загнать все риски, то цена будет запредельной, что на текущем рынке равносильно смерти.
Поэтому вместо этого планируем буфер. (см слайд)

Риск общения с заказчиком: я ему сказал, он не понял $\implies$ коэффициент цены 1.1.

Недооценить~--- ужасно.
Переоценить можно, но делать будете всё равно всё доступное время и за все возможные деньги.
Найти идеальную оценку сложно, но делает вас крутыми.

Пока предупреждение: разработчики склонны забывать по 30\% задач.
А по опыту ЯА, а ещё про 50\% задач не знают.
Поэтому читай чеклист типичных пропущенные требования.

... тут я выпал

Теперь про стоимость ещё.
Написать код~--- только часть расходов.
В смету нужно записать риски, операционные расходы, налоги, зарплаты сотрудников, прибыль собственника.

Всё хочется в договор, ещё и на год сразу.
Выглядит сложно.
Как решаем?
Не работает Fixed Price, Time \& Material чуть лучше, поэтому договор по этапам.
Сделали этап, договорились заново, примерно раз в 3-6 месяцев

\end{document}
