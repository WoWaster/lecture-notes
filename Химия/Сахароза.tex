\documentclass[12pt]{article}
%%% PDF settings
\pdfvariable minorversion 7 % Set PDF version to 1.7.

%%% Fonts and language setup.
% Setup fonts.
\usepackage{fontspec}
\setmainfont{CMU Serif}
% TODO: Find a sans-serif font.
\setmonofont[Contextuals={Alternate}]{Fira Code Regular}

\usepackage[english,russian]{babel} % Enable russian language.

% Enable ligatures to work in verbatim environment.
\usepackage{verbatim}
\makeatletter
\def\verbatim@nolig@list{}
\makeatother

\usepackage{hologo} % Add fancy logos like \LuaLaTeX.

\usepackage{csquotes} % Quotes based on babel settings

\usepackage{indentfirst} % Indent first paragraph after heading.


%%% Page settings.
% Fancy page geometry.
\usepackage{geometry}
\geometry{
    margin=2cm,
    % headheight=15pt,
    % includefoot=true,
}

\usepackage{multicol} % Multicols environment.

\usepackage{lastpage} % Shows last page number.

\usepackage[usenames,dvipsnames,svgnames,table,rgb]{xcolor} % Enable color support.


%%% Particular subjects helper tools.
%% Math
\usepackage{amsmath, amsfonts, amssymb, amsthm, mathtools} % Advanced math tools.
\usepackage{unicode-math} % Allow TTF and OTF fonts in math and allow direct typing unicode math characters
\unimathsetup{
    warnings-off={
            mathtools-colon,
            mathtools-overbracket
        }
}

%% Physics
\usepackage{siunitx} % Fancy method to write physic formulas.

%% Chemistry
% Fix babel and chemformula confusion
\let\Ch\ch % save the command for the hyperbolic cosine
\let\ch\relax % undefine \ch
\usepackage{chemmacros} % Chemistry signs and other.
\usepackage{chemformula} % Write down chemical formulas.
\usepackage{chemfig} % Draw structural formulas.


%%% Tables
\usepackage{array} % Improved column definition.
\usepackage{tabularx} % Adds X columns that are stretched to have equal width.
\usepackage{tabulary} % Makes rows equal height by adjusting column width.
\usepackage{booktabs} % Adds fancy rules to use with tables.
\usepackage{longtable} % Table that spans across multiple pages.
\usepackage{multirow} % Combine rows in table.


%%% Images
% Support for images.
\usepackage{graphicx}
\graphicspath{{images/}}
\usepackage{wrapfig} % Floating images.



% Please add everything above.

%%% HyperRef
% Add hyperlinks to PDF and make them invisible.
\usepackage{hyperref}
\hypersetup{
    hidelinks
}

\usepackage{parskip}
\usepackage{gensymb}
\begin{document}
\pagestyle{empty}
\subsection*{Фруктоза}
\ch{C6H12O6}

Белое кристаллическое вещество, хорошо растворимое в воде.
В 2 раза слаще глюкозы и в 4-5 раз слаще лактозы.

\subsubsection*{Строение}
\begin{center}
    \chemfig{CH_2OH-C(-[6]HO)(-[2]H)-C(-[6]HO)(-[2]H)-C(-[2]HO)(-[6]H)-C(=[6]O)-CH_2OH}
\end{center}

\subsubsection*{Свойства}
\begin{enumerate}
    \item Реакция серебряного зеркала
    \item Восстановление водородом
    \item Качественная реакция --- проба Селиванова
          \begin{figure}[!h]
              \centering
              \def\svgwidth{\columnwidth}
              \input{images/Seliwanow.pdf_tex}
              \raggedleft
              появляется вишнево-красное окрашивание
          \end{figure}
    \item Брожение со спиртом \\
          \ch{C6H12O6 ->[+дрожжи][t$^\circ$] 2CO2 ^  + 2C2H5OH}
    \item Молочнокислое брожение \\
          \ch{C6H12O6 ->[t$^\circ$] CH3CH(OH)COOH}

\end{enumerate}

\subsubsection*{Применение}
\begin{enumerate}
    \item Благодаря некоторым особенным свойствам фруктоза широко используется как подсластитель.
          Её повышенная сладость и синергетическое действие с другими подсластителями позволяет добавлять в продукты меньше сахара, поэтому её часто используют в низкокалорийной пище.
    \item Способна усиливать фруктовые вкусы.
    \item Фруктоза обладает высокой растворимостью при низких температурах и сильно понижает температуру плавления своих растворов, поэтому её использование представляет интерес в производстве мороженого, где эти свойства важны для текстуры продукта.
    \item Фруктоза широко применяется в напитках (газированных, спортивных, низкокалорийных и т. д.), замороженных десертах, выпечке, консервированных фруктах, шоколаде, конфетах и молочных продуктах.
    \item Благодаря хорошей растворимости в этаноле она применяется в сладких ликёрах.
\end{enumerate}
\end{document}