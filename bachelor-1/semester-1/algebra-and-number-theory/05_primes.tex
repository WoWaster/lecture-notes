% !TeX root = ./main.tex
\documentclass[main]{subfiles}
\begin{document}
\chapter{Простые числа}
\begin{definition}
    $p \in \Z$ называется простым, если $p \neq 0, \pm 1$ и $\{a: a\mid p\} =
        \{\pm 1, \pm p\}$. Простые числа могут быть отрицательными.
\end{definition}
\[\Z = \{0, \pm 1\} \cup \{\text{простые}\} \cup \{\text{составные}\}\]

\begin{assertion}
    Пусть $a > 1$, тогда наименьший натуральный делитель $a$, отличный от $1$
    -- простое число.
\end{assertion}
\begin{proof}
    $p$ -- наименьший натуральный делитель $n$. Если $p$ составное, то
    $\exists q: 1< q<p, q \mid p$
    \begin{equation*}
        \begin{rcases}
            q \mid p \\
            p \mid n
        \end{rcases}
        \implies q \mid n, q < p \qquad \contradiction
    \end{equation*}
\end{proof}

\begin{corollary}
    Любое целое число, кроме $\pm 1$ делится на простое
\end{corollary}
\begin{corollary}
    Наименьший натуральный делитель, $\neq 1$, составного числа $n$ не больше $\sqrt{n}$.
\end{corollary}
\begin{proof}\footnote{Здесь и далее $\contradiction$ означает противоречие}
    $p$ -- наименьший натуральный делитель $n$, $p \neq 1$, тогда
    \[n = pb\\\]
    Предположим, что $p > \sqrt{n}$, $n$ -- составное $\implies b \neq 1 \implies b
        \ge p > \sqrt{n}$
    \[n=pb>\sqrt{n} \sqrt{n} = n \qquad\contradiction\]
\end{proof}

\begin{theorem}[Евклида]
    Простых бесконечно много.
\end{theorem}
\begin{proof}
    Пусть это не так, $p_1, p_2, ..., p_k$ -- все положительные простые.
    \begin{gather*}
        n = p_1 p_2 ... p_k +1 \\
        n > 1 \implies  \text{ составное} \implies \exists\text{ простое } p \mid n, p > 0\\
        \implies p \in \{p_1, ..., p_k\} \implies p \mid  (n-1)\\
        \begin{rcases}
            p\mid n \\
            p\mid (n-1)
        \end{rcases}
        \implies p \mid 1 \implies p = \pm 1 \qquad \contradiction
    \end{gather*}
\end{proof}
\end{document}