% !TeX root = ./main.tex
\documentclass[main]{subfiles}
\begin{document}
\part{Алгебраические структуры}
\chapter{Множества}
\section{Нотация}

Стандартная запись
\[A' = \{1,3,5,7\}\]
\[A=\{1,3,5, ..., 99\}\]

Общий вид
\[B = \{2,4,6,...0\} = \{2n: n\in \N\}\]

Стандартные числовые множества
\[\N =\{1,2,3,...\} \]
\[\Z =\{...,-1,0,1,2,...\} \]
\[\Q = \left\{ \frac{p}{q}: p,q \in \Z, q \neq 0 \right\} \]
\[\R, \C\]

Подмножества
\[A' \subset A \subset \N, A' \not\subset B\]

\begin{align*}
    C & = \{1,2,3\} & \emptyset, & \{1\},  \{2\}, \{3\}     \\
      &             &            & \{1,2\},\{1,3\}, \{2,3\} \\
      &             &            & \{1,2,3\} = C
\end{align*}

Предикат для подмножеств: $\{n \in \N : n < 5\} = \{1,2,3,4\}$

\section{Операции на множествах}
Пусть A, B --- множества \marginpar{$\oplus \Leftrightarrow \bigtriangleup$}

\[A \cap B = \{a \in A \land  a \in B\}\]
\[A \cup B = \{a: a \in A \lor a \in B\}\]
\[A \setminus B = \{a\in A \land a\not\in B\}\]
\[A \bigtriangleup B = (A \setminus B) \cup (B \setminus A)\]
\[A \times B = \{(a,b):a\in A, b \in B\}\]
\begin{multline*}
    A = \{1,2,3\} \quad
    B = \{-1, 1\} \\
    A \times B = \{ (1,-1), (1,1), (2, -1), (2,1), (3, -1), (3,1)\}
\end{multline*}
\[\bigcap_{i=1}^n A_i \quad \bigcup_{i=1}^n A_i\]

\[A \cup (B \cap C) = (A \cup B) \cap (A \cup C)\]
\[A \cap (B \cup C) = (A \cap B) \cup (A \cap C)\]

\section{Отображение}
$A, B$ --- множества

\begin{definition}
    Задать отображение $A$ в $B$, значит для каждого $a \in A$ задать
    некоторый элемент $B$ (т.н. образ элемента $A$)
\end{definition}

\begin{center}
    \[A = \{1,2,3,4\}\]
    \[B = \R\]

    \begin{tabular}{c|c}
        $a$ & $f(a)$     \\
        \hline
        1   & $\sqrt{2}$ \\
        2   & 0          \\
        3   & $7^5$      \\
        4   & 0          \\
    \end{tabular}
\end{center}


\begin{center}
    \begin{align*}
         & f: \R \to \R &  & f: \R \to \R  \\
         & f(a)=a-3     &  & a \mapsto a-3
    \end{align*}

    \begin{tabular}{c|c}
        $a$ & $f(a)$ \\
        \hline
        1   & -2     \\
        2   & -1     \\
        3   & 0      \\
        4   & 1
    \end{tabular}
\end{center}


\begin{gather*}
    f: \R \to \Z \\
    a \mapsto \begin{cases}
        1,  & a > 0 \\
        0,  & a =0  \\
        -1, & a < 0
    \end{cases}
\end{gather*}

\begin{gather*}
    \varphi: \N \to \N \\
    n \mapsto |\{m \in \N: m \le n \& (m,n) = 1\}|
\end{gather*}

$|M| = \#M = \text{Card } M$ --- мощность множества

$2^M$ --- множество всех подмножеств M, его мощность $|2^M| = 2^{|M|}$

\subsection{Свойства}
$f: A \to B$ называется инъекцией, если $\forall a_1, a_2 \in A: a_1 \neq a_2 \Rightarrow f(a_1)\neq f(a_2)$

отображение называется сюръекцией, если $\forall b \in B, \exists a \in A: f(a) = b$

отображение называется биекцией, если оно одновременно инъекция и сюръекция

\begin{gather*}
    f: A \to B\\
    b\in B \text{ полный прообраз } b \text{ относительно } f: \\
    f^{-1}(b) = \{a \in A : f(a) = b\}
\end{gather*}

\begin{align*}
       & \R \to \R     &            &             \\
    f: & x \mapsto x^2 & f^{-1}(4)  & = \{-2,2\}  \\
       &               & f^{-1}(0)  & = \{0\}     \\
       &               & f^{-1}(-3) & = \emptyset \\
\end{align*}

$f$ -- инъекция $\Leftrightarrow \forall b \in B: |f^{-1}(b)| \le 1$

$f$ -- сюръекция $\Leftrightarrow \forall b \in B:|f^{-1}(b)| \ge 1$

$f$ -- биекция $\Leftrightarrow \forall b \in B:|f^{-1}(b)| = 1$

\subsection{Сужение отображения}

\begin{align*}
    f:        & A \to B       &  & A' \subset A & f: \R \to \R                        \\
    f|_{A'} : & A' \to B      &  &              & x \mapsto x^2                       \\
              & a\mapsto f(a) &  &              & f|_{\R_{\ge 0}} : \R_{\ge 0} \to \R
\end{align*}

\subsection{Образ подмножества}
\begin{align*}
    f: & A \to B     &   &                          \\
       & M \subset A & f & (M) = \{ f(m): m \in M\} \\
       &             & f & (A) = \text{Im}A
\end{align*}

\section{Композиция}


\begin{gather*}
    f: A \to B \quad g: B \to C\\
    g \circ  f:A \to C\\
    a \mapsto g(f(a))
\end{gather*} --- композиция $f$ и $g$

\begin{gather*}
    f, g: \R \to \R\\
    f(x) = x+1\\
    g(x) = 2x\\
    \begin{aligned}
        g \circ f : & \R \to \R      & f \circ g : & \R \to \R       \\
                    & x \mapsto 2x+2 &             & x \mapsto 2x +1
    \end{aligned}
\end{gather*}

\section{Тождественное отображение}
Пусть $M$ -- множество
\begin{gather*}
    \id_M : M \to M \\
    m \mapsto m
\end{gather*}

Пусть $f: X \to Y$, тогда отображение $g: Y \to X$ называется обратным, если
$g\circ f = \id_X, f \circ g = \id_Y$

\begin{theorem}
    У $f: X \to Y$ есть обратное $\Leftrightarrow f $ -- биекция
\end{theorem}

\begin{proof}
    \begin{align*}
        \Longrightarrow & g: Y \to X & g\circ f  & = \id_X \\
                        &            & f\circ g  & = \id_Y \\
                        & y \in Y    & f^{-1}(y) & = \{x\} \\
                        & g(y) := x  &           &
    \end{align*}
    \begin{gather*}
        (g \circ f) (x) = g(f(x)) = x \qquad f^{-1}(f(x)) = \{x\}\\
        (f \circ g) (y) = f(g(y)) = y
    \end{gather*}
    \begin{align*}
        \Longleftarrow g\circ f & = \id_X             & f \circ g       & = \id_Y              \\
        \Downarrow              & \Uparrow            & \Downarrow      & \Uparrow             \\
        f                       & \text{ -- инъекция} & f               & \text{ -- сюръекция} \\
        f(x_1)                  & = f(x_2)            & y               & \in Y \Rightarrow    \\
        \Rightarrow g(f(x_1))   & = g(f(x_2))         & \exists x \in X & : f(x)=y             \\
        \Rightarrow x_1         & = x_2               & f(g(y))         & =y                   \\
    \end{align*}
\end{proof}
\end{document}