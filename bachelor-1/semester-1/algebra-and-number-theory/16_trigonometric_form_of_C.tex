% !TeX root = ./main.tex
\documentclass[main]{subfiles}
\begin{document}
\chapter{Тригонометрическая форма комплексного числа}

Дано $z = a + bi \in \C^* (\C^* = \C \setminus \{0\})$, тогда:
\begin{gather*}
    r = |z|\\
    z = r \left(\frac{a}{r}+\frac{b}{r}i\right)\\
    \left(\frac{a}{r}\right)^2 +  \left(\frac{b}{r}\right)^2 = \frac{a^2+b^2}{r^2}=1\\
    \implies \exists \phi \in \R: \frac{a}{r} = \cos \phi, \frac{b}{r} = \sin \phi\\
    \left[\frac{a}{r} = \cos \phi \implies \left(\frac{b}{r}\right)^2 = 1 - \cos^2 \phi
        = \sin^2 \phi; \qquad \phi = \pm \phi \right]\\
    z = |z| (\cos \phi + i \sin \phi)
\end{gather*}
-- тригонометрическая форма $z$
\begin{gather*}
    \Re z = |z|\cos\phi\\
    \Im z = |z|\sin\phi
\end{gather*}

$\phi$ называется аргументом $z$, $\phi$ определено с точностью до кратных $2\pi$,
т.е. $\phi$ -- аргумент $z$, то и $\phi+2\pi k$ -- аргумент $z \forall k \in \Z$.
Если $\phi\in \left[0, 2\pi\right)$, то такой $\phi$ называется главное значение аргумента $z$.
\begin{remark}
    Верно и обратное:
    \begin{gather*}
        \cos \phi' = \cos \phi \qquad \sin \phi' = \sin \phi\\
        \implies \phi' = \phi + 2 \pi k, k \in \Z
    \end{gather*}
\end{remark}

\begin{theorem}
    \begin{enumerate}
        \item Пусть $z,w \in \C^*$, тогда
              \[\arg(zw) = \arg z + \arg w\]
        \item Пусть $z,w \in \C^*$, тогда
              \[\arg \left(\frac{z}{w}\right) = \arg z - \arg w\]
        \item Пусть $z \in \C^*$, тогда
              \[\arg \overline{z} = - \arg z\]
    \end{enumerate}
\end{theorem}
\begin{proof}
    \begin{gather*}
        \phi = \arg z \qquad \psi = \arg w
        \intertext{Докажем 1:}
        \begin{multlined}
            zw = |z| |w| (\cos \phi + i \sin \phi) (\cos \psi + i \sin \psi) =\\
            = |zw| \left((\cos \phi \cos \psi - \sin \phi \sin \psi) + (\cos \phi \sin \psi + \sin \phi \cos \psi)i\right)=\\
            = |zw| (\cos(\phi + \psi) + i \sin (\phi + \psi)) \\
            \implies \phi + \psi = \arg |zw|
        \end{multlined}
        \intertext{Докажем 2:}
        z = \frac{z}{w}w\\
        \implies \arg z = \arg \frac{z}{w} + \arg w \implies \arg \frac{z}{w} = \arg z - \arg w
        \intertext{Докажем 3:}
        \arg z = \phi\\
        z = |z| (\cos \phi + i \sin \phi)\\
        \overline{z} = |z| (\cos \phi - i \sin \phi) = |z| (\cos (-\phi) + i \sin (-\phi))\\
        \implies - \phi = \arg \overline{z}
    \end{gather*}
\end{proof}

\begin{corollary}[Формула Муавра]
    Пусть $z = r(\cos \phi + i \sin \phi), r = |z|$, тогда
    \[\forall n \in \Z: z^n = r^n (\cos(n \phi) + i \sin (n\phi))\]
\end{corollary}
\begin{proof}
    \begin{gather*}
        \arg z = \phi\\
        n > 0 \qquad z^n = r^n (\cos(n \phi) + i \sin (n\phi))\\
        n = 0 \qquad 1=1\\
        n < 0 \qquad n = -m, m \in \N\\
        z^n = \frac{1}{z^m} =r^{-m} (\cos(0 - m \phi) + i \sin (0 - m \phi)) = r^n (\cos(n \phi) + i \sin (n\phi))
    \end{gather*}
\end{proof}
\end{document}