% !TeX root = ./main.tex
\documentclass[main]{subfiles}
\begin{document}
\chapter{Понятие векторного пространства}
\begin{definition}
    Множество $V$ с двумя операциями:
    $+:V\times V \to V; \quad (a,b) \mapsto a+b$ и
    $\cdot : \R \times V \to V$ называется векторным пространством (над $\R$),
    если при условии $\forall \va, \vb, \vc \in V; \forall \alpha, \beta \in \R$, выполнены следующие свойства:

    \begin{enumerate}
        \item $\va+(\vb+\vc) = (\va+\vb)+\vc$ --- ассоциативность
        \item $\va+\vb=\vb+\va$ --- коммутативность
        \item $\exists \zv:\forall \va \quad \zv+\va=\va+\zv=\va$
        \item $\forall \va \exists (-\va): \va+(-\va)=\zv$%
              \footnote{Если выполнены свойства 1--4, то $V$ называется коммутативной (абелевой) группой.}
        \item $\alpha(\va+\vb)=\alpha \va + \alpha \vb$ --- дистрибутивность \begin{proof}
                  \begin{gather*}
                      \alpha (\va+\vb) = \alpha \va + \alpha \vb \\
                      \va = (x_1, y_1) \quad \vb = (x_2, y_2)
                  \end{gather*}
                  \begin{multline*}
                      \alpha(\va+\vb)=\alpha((x_1, y_1)+(x_2, y_2)) = \alpha (x_1+x_2, y_1+y_2)=\\
                      (\alpha(x_1+x_2), \alpha(y_1+y_2))=(\alpha x_1 + \alpha x_2, \alpha y_1 + \alpha y_2)=\\
                      (\alpha x_1, \alpha y_1)+(\alpha x_2, \alpha y_2) = \alpha (x_1, y_1) + \alpha (x_2, y_2)= \\
                      \alpha \va + \alpha \vb
                  \end{multline*}
              \end{proof}
        \item $(\alpha+\beta)\va = \alpha \va + \beta \va$ --- дистрибутивность
        \item $(\alpha \cdot \beta)\va = \alpha(\beta \va)$ --- ассоциативность
        \item $1\cdot \va = \va$
    \end{enumerate}
\end{definition}

\subsection{Свойства векторного пространства}

\begin{enumerate}
    \item $\zv$ --- единственный \begin{proof}
              $\zv_1 = \zv_1 + \zv_2 = \zv_2$
          \end{proof}
    \item $-\va$ --- единственный \begin{proof}
              Пусть $\vb_1, \vb_2$ -- противоположные к $\va$

              $\vb_1 + \va = \zv \quad \vb_2 + \va = \zv$

              $\vb_1 = \vb_1 + \zv = \vb_1 + (\va + \vb_2) = (\vb_1+\va)+\vb_2 = \zv +\vb_2 = \vb_2$
          \end{proof}
    \item $\zv\cdot \va = \zv$ \begin{proof}
              !!!
          \end{proof}
    \item $-1\cdot \va = -\va$ \begin{proof}
              !!!
          \end{proof}
\end{enumerate}

\subsection{Примеры векторных пространств}
\begin{enumerate}
    \item Координатная плоскость $\{(x,y): x,y\in \R\}$
    \item Координатное трехмерное пространство $\{(x,y,z): x,y,z\in \R\}$
    \item Строки длины $n$ из вещественных чисел \\
          $V = \{(x_1, x_2, ..., x_n): x_i \in \R\}$ или матрицы (2d массивы)
\end{enumerate}

\section{Операции над векторами}
\begin{equation*}
    \va = (x_1, y_1) \quad \vb = (x_2, y_2)
\end{equation*}
\subsection{Сложение}
\begin{equation*}
    \va + \vb = (x_1+x_2, y_1+y_2)
\end{equation*}

\subsection{Умножение вектора на число}
\begin{equation*}
    \alpha \va = (\alpha\cdot x_1, \alpha\cdot y_1)
\end{equation*}

\section[ЛК, ЛЗ и ЛНЗ]{Линейные комбинация, зависимость и независимость}
\begin{definition}
    $V$ - векторное пространство и векторы \\ $\vv_1,\vv_2,\vv_3,..., \vv_n \in V$.
    Система $\vv_1,...,\vv_n$ называется линейно независимой (ЛНЗ), если из
    $\alpha_1 \vv_1 + \alpha_2 \vv_2 + ... + \alpha_n \vv_n = 0 \implies \alpha_1=\alpha_2=...=\alpha_n =0$.
\end{definition}

\begin{definition}
    Если $\alpha_1,..., \alpha_n \in \R$, $\vv_1,...,\vv_n \in V$.
    То $\alpha_1 \vv_1 + \alpha_2 \vv_2 + ... + \alpha_n \vv_n$ -- линейная комбинация (ЛК)
    векторов $\vv_1,...,\vv_n$.
\end{definition}

\begin{definition}
    Если $\exists \alpha_1,..., \alpha_n$, не все $=0$, но $\alpha_1 \vv_1 + \alpha_2 \vv_2 + ... + \alpha_n \vv_n = 0$,
    то система $\vv_1,...,\vv_n$ называется линейно зависимой (ЛЗ).
\end{definition}

\begin{assertion}
    $\vv_1,...,\vv_n$ -- ЛЗ $\Leftrightarrow$ один из этих векторов можно представить как ЛК остальных.
    $\exists i: \vv_i= \alpha_1 \vv_1 +\alpha_2 \vv_2 + ... + \alpha_{i-1} \vv_{i-1} + \alpha_{i+1} \vv_{i+1} + ... +\alpha_n \vv_n$
\end{assertion}
\begin{proof}
    $\Rightarrow : \exists \alpha_1,...,\alpha_n (\exists i: \alpha_i \neq 0)$
    \begin{gather*}
        \alpha_1 \vv_1 + \alpha_2 \vv_2 + ... + \alpha_n \vv_n = 0\\
        \alpha_i \vv_i = - \alpha_1 \vv_1 - \alpha_2 \vv_2 - ... - \alpha_{i-1} \vv_{i-1} - \alpha_{i+1} \vv_{i+1} - ... -\alpha_n \vv_n\\
        \alpha_i \neq 0 \quad \vv_i = -\frac{\alpha_1}{\alpha_i}\vv_1 -... - \frac{\alpha_n}{\alpha_i}\vv_n\\
        \Leftarrow: \vv_i = \alpha_1 \vv_1 + ... + \alpha_n \vv_n \text{ без } i\text{-ого слагаемого}\\
        \alpha_1 \vv_1 + \alpha_2 \vv_2 + ... + \mathbf{(-1)}\vv_i + ... + \alpha_n \vv_n =0\\
        \text{ЛК } = 0 \text{ не все коэффициенты} = 0
    \end{gather*}
\end{proof}

\begin{prop}
    $\vv_1,...,\vv_n $ -- ЛНЗ, то любой его поднабор тоже ЛНЗ.\\
    $\vv_1,...,\vv_n $ -- ЛЗ, то при добавлении векторов, набор останется ЛЗ.
\end{prop}


\begin{assertion}
    $\vv_1,..., \vv_n$ -- ЛНЗ $\Leftrightarrow$ если
    \begin{gather*}
        \alpha_1 \vv_1 + ... + \alpha_n \vv_n = \beta_1 \vv_1 + ... + \beta_n \vv_n\\
        \Rightarrow \alpha_1 = \beta_1; \alpha_2 = \beta_2; ... ; \alpha_n = \beta_n
    \end{gather*}
\end{assertion}
\begin{proof}
    \begin{gather*}
        (\alpha_1 - \beta_1)\vv_1 + (\alpha_2 - \beta_2)\vv_2 + ...
        + (\alpha_n - \beta_n)\vv_n = \zv\\
        \alpha_i - \beta_i = 0 \Leftrightarrow \vv_1, ..., \vv_n\text{-- ЛНЗ}
    \end{gather*}
\end{proof}
\end{document}