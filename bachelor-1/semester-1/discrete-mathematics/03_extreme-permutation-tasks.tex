% !TeX root = ./main.tex
\documentclass[main]{subfiles}
\begin{document}

\section{Задача о минимуме скалярного произведения.}
\subsection{Условие}
Пусть даны два вектора $x = (x_1, \ldots, x_n)$ и $y = (y_1, \ldots, x_n)$.

Рассмотрим скалярное произведение $(x, y) = \sum_{i=1}^n x_iy_i$.

Как переставить элементы этих векторов, чтобы минимизировать $(x,y)$?

\subsection{Решение}
\begin{theorem}
    Скалярное произведение будет минимальным, если $x_1 \geq \ldots \geq x_n$,
    а $y_1 \leq \ldots \leq y_n$.
\end{theorem}

\begin{proof}
    Пусть нашлась пара $i$ и $j$, т.ч. $x_i \geq x_j$ и $y_i \geq y_j$.
    Переставим их местами. Значение скалярного произведения не увеличилось:
    \begin{align*}
        x_iy_i + x_jy_j \geq x_iy_j + x_jy_i \\
        (x_i - x_j)(y_i - y_j) \geq 0
    \end{align*}
\end{proof}

\section{Задача о максимальной возрастающей подпоследовательности}
\subsection{Условие}
Дана перестановка. Требуется найти самую длинную возрастающую 
подпоследовательность.

\subsection{Решение}
\paragraph{Алгоритм}
\begin{enumerate}
    \item Просматриваем перестановку с начала и формируем первую 
подпоследовательность, записывая элементы перестановки, пока они идут в
убывающем порядке
    \item Если очередной элемент больше предыдущего, то начинаем с него
новую подпоследовательность
    \item Каждый следующий элемент перестановки пытаемся поставить в 
подпоследовательность с минимальным номером, а если это не удается, начинаем
с него новую подпоследовательность
    \item Из каждой подпоследовательности, начиная с последней, выбираем по 
одному элементу и строим возрастающую подпоследовательность. Каждый раз
выбираем элемент, меньший текущего и стоящий левее.
\end{enumerate} 
\begin{case}
    Если дана подпоследовательность
    \[18, 2, 4, 15, 17, 16, 3, 20, 5, 1, 7, 19, 6, 14, 11, 9, 8, 13, 12, 10\]
    В результате работы алгоритма сформируются подпоследовательности:
    \begin{align*}
        &18, 2, 1 \\
        &4, 3 \\
        &15, 5 \\
        &17, 16, 7, 6 \\
        &20, 19, 14, 11, 9, 8 \\
        &13, 12, 10
    \end{align*}
    Результат:
    \[2, 3, 5, 6, 8, 10\]
\end{case}

\end{document}
