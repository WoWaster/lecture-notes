% !TeX root = ./main.tex
\documentclass[main]{subfiles}
\begin{document}
\chapter{Векторы}

\section{Двоичные векторы}

\begin{definition}
    Двоичный вектор --- упорядоченный набор фиксированного размера из 0 и 1.
\end{definition}

\subsection{Примеры}
\begin{enumerate}
    \item вершины единичного куба в n-мерном пространстве 
    \item характеристический вектор подмножества. (1, 0, 1, 0, 0) - подмножество 
входят первый и третий элементы множества из 5 элементов
    \item двоичное число
    \item кодирование изображения
\end{enumerate}

\section{Перебор векторов}
\subsection{Последовательно}
Рассматривая вектор как двоичное число. Начиная с нулевого, последовательно 
прибавляем единицу. $0000 -> 0001 -> 0010$

\subsection{Код Грея}
Алгоритм: Определим два набора: $x = \left(0, 0, \ldots, 0\right)$ и 
$y = \left(0, 0, \ldots, 0\right)$
\begin{enumerate}
    \item Прибавляем 1 к числу $y$
    \item Фиксируем позицию $k$, в которой 0 сменился на 1
    \item $x[k] = \lnot x[k]$
\end{enumerate}

\section{Прямое произведение множеств}
\begin{definition}
    Пусть даны $k$ множеств $M_i$. Прямым произведением этих множеств 
    $M = M_1 \times M_2 \times \ldots \times M_i$ называется множество всех 
    упорядоченных наборов $(a_1, \ldots, a_k)$, где $a_i \in M_i$.
\end{definition}
\subsection{Количество элементов в прямом произведении:}
\[
    |M_i| = |M_1 \times \ldots \times M_k| = m_1 \cdot \ldots \cdot m_k 
    \text{, где $m_i=|M_i|$}
\]

\subsection{Нумерация элементов прямого произведения}
Занумеруем элементы каждого множества от 0 до $m_i - 1$. Элемент произведения ---
вектор чисел $(r_1, \ldots, r_k)$.

Сопоставим каждому набору номер:
\begin{multline}
    num(r_1, \ldots r_k) = \sum_{i = 1}^{k}\left(r_i \cdot \prod\limits_{j=1}^{i-1}m_j\right) = \\
    = r_1 + r_2 \cdot m_1 + r_3 \cdot m_1 \cdot m_2 + \ldots + r_k m_1 m_2 \ldots m_{k-1}
\end{multline}

\section{Перебор перестановок v1}
\subsection{Идея}
Введем вспомогательный набор множеств $M_i$, т.ч. $|M_i| = i$.
$T_k = M_k \times M_{k-1} \times \ldots \times M_1, |T_k| = k!$ - 
ровно столько, сколько перестановок в $P_k$. $T_k$ перебирать умеем. 
Чтобы научиться перебирать и перенумеровывать $P_k$, построим
взаимно-однозначное соответсвие между $T_k$ и $P_k$

\begin{center}
    \begin{tabular}[c]{*{8}{c|}c}
        $i$   & 1 & 2 & 3 & 4 & 5 & 6 & 7 & 8 \\ \cline{1-9}
        $r_i$ & 4 & 8 & 1 & 5 & 7 & 2 & 3 & 6 \\ \cline{1-9}
        $t_i$ & 3 & 6 & 0 & 2 & 3 & 0 & 0 & 0 
    \end{tabular}
    \marginpar{$r_i \in P_k$}
    \marginpar{$t_i \in T_k$}
\end{center}

\subsection{Алгоритм}
Используем вспомогательный вектор t, представляющий номре перестановки в
факториальной системе счисления, начиная с $t=(0,0,0,0)$
\begin{enumerate}
    \item Прибавить 1 к текущему значению вектора $t$
    \item По $t$ построить перестановку $p$
    \item Если $t=(k-1, \ldots, 1, 0)$, завершить работу
    \item Перейти к шагу 1
\end{enumerate}

\begin{center}
    \begin{tabular}[c]{c|c|c}
        $№$ & 0\;0\;0\;0 & 1\;2\;3\;4 \\ \cline{1-3}
        1   & 0\;0\;1\;0 & 1\;2\;4\;3 \\ \cline{1-3}
        2   & 0\;1\;0\;0 & 1\;3\;2\;4 \\ \cline{1-3}
        3   & 0\;1\;1\;0 & 1\;3\;4\;2 \\ \cline{1-3}
        4   & 0\;2\;0\;0 & 1\;4\;2\;3 \\ \cline{1-3}
        5   & 0\;2\;1\;0 & 1\;4\;3\;2 \\ \cline{1-3}
        6   & 1\;0\;0\;0 & 2\;1\;3\;4
    \end{tabular}
\end{center}

\section{Перебор перестановок v2}
\subsection{Алгоритм}
Так же перебирает перестановки в лексикоргафическом порядке, но гораздо проще
и быстрее. Начинаем с перестановки $(1, 2, \ldots, k)$.
\begin{enumerate}
    \item В данной перестановке $(r_1, \ldots, r_k)$ найти такое $q$, что \\
$r_q > r_{q + 1} > \ldots > r_k$ и $r_{q-1} < r_q$
    \item Если была перестановка $(k, k - 1, \ldots, 1)$, то алгоритм завершает
работу
    \item Выбрать в суффиксе $(r_q, \ldots, r_k)$ элемент, следующий по значению
после $r_{q-1}$ и поменять его и $r_{q-1}$ местами
    \item Упорядочить суффикс по возрастанию
\end{enumerate}

\subsection{Пример}
\begin{enumerate}
    \item (3,4,1,2,6,5,\textcolor{red}{8,7})
    \item (3,4,1,2,6,7,5,\textcolor{red}{8})
    \item (3,4,1,2,6,7,\textcolor{red}{8,5})\marginpar{Красным цветом выделен суффикс}
    \item (3,4,1,2,6,8,5,\textcolor{red}{7})
    \item (3,4,1,2,6,\textcolor{red}{8,7,5})
    \item (3,4,1,2,7,5,6,\textcolor{red}{8})
    \item (3,4,1,2,7,5,\textcolor{red}{8,6})
    \item (3,4,1,2,7,6,5,\textcolor{red}{8})
\end{enumerate}

\section{Перебор перестановок v3: "аналог кода Грея"}
\subsection{Алгоритм}
Дополнительно условие: хотим, чтобы только два соседних элемента менялись 
местами. \\
Перестановка $p = (1, 2, \ldots, n)$ \\
Вектор (младший разряд последний) $t = (0, 0, \ldots, 0)$ \\
Вектор, ответственный за смену навправлений $d = (-1, -1, \ldots, -1)$ \\
Номер разряда $j$, в котором значение увеличилось на 1
\begin{enumerate}
    \item Прибавить 1 к вектору $t$
    \item Определить номер разряда $j$, в котором значение увеличилось на 1
    \item Изменить направление движения у всех элементов, больших $j$, т.е.
для всех $i > g$ положим $d_i = -d_i$
    \item Поменять местами элемент перестановки $j$ с элементом справа
(если $d_j > 0$) или слева (если $d_j < 0$)
\end{enumerate}

\end{document}
