% !TeX root = ./main.tex
\documentclass[main]{subfiles}
\begin{document}
\chapter{Пределы}

\section{Общее определение предела последовательности}
\begin{definition}
    Пусть $E \neq \emptyset, \exists$ по крайней мере 2 точки $x_1, x_2 \in E$.
    Множество $E$ называется метрическим пространством, если $\forall x, y 
    \in E$ определена функция $\rho(x, y)$, удовлетворяющая следующим 
    свойствам:
    \begin{enumerate}
        \item $\rho(x, y) \geq 0; \rho(x, y) = 0 \Leftrightarrow x = y$
        \item $\rho(x, y) = \rho(y, x)$
        \item $\forall x, y, z \in E \rho(x, z) \leq \rho(x, y) + \rho(y, z)$
    \end{enumerate}
    Функцию $\rho$ называют метрикой, заданной на $E$, а $\rho(x, y)$ называют 
    расстоянием в $E$ между $x, y$. Соотношение 3 называется неравенством
    треугольника в $E$. Точка $a \in E$ называется точкой сгущения множества
    $E$, если $\forall \epsilon > 0 \exists x_\epsilon \in E : x_\epsilon 
    \neq a \wedge \rho(x_\epsilon, a) < \epsilon$. Точка $b \in E$ называется
    изолированной точкой множества $E$, если $\exists \epsilon_0 > 0 : 
    \forall x \in E, x \neq b$, выполнено $\rho(x, b) \geq \epsilon_0$.
    Последовательностью $\left\{v_n\right\}_{n = 1}^{\infty}$ в $E$ 
    называется отображение $F: \N \rightarrow E, F(n) = v_n$.
\end{definition}

\begin{definition}
    Пусть $E$ - метрическое пространство с метрикой $\rho$, $a \in E$ - 
    точка сгущения, $\left\{v_n\right\}_{n = 1}^{\infty}, v_n \in E$, - 
    последовательность в $E$. Говорят, что $v_n$ стремится к $a$ при $n$, 
    стремящимся к бесконечности, пишут $v_n 
    \underset{n \rightarrow \infty}{\to} a$, или, что равносильно, что предел 
    $v_n$ при $n$, стремящемся к бесконечности, равен $a$, пишут 
    $\lim_{n\to\infty}v_n = a$, если 
    \begin{equation}\label{limdef}
        \forall \epsilon > 0 \exists N \in \N :
        \forall n > N \rho(v_n, a) < \epsilon
    \end{equation}
\end{definition}

\begin{theorem}
    О единственности предела. Пусть $E$ - метрическое пространство с метрикой
    $\rho$, $\left\{v_n\right\}_{n = 1}^{\infty}$ - последовательность 
    элементов $E$ и предположим, что $v_n \underset{n\to\infty}{\to} a_1$ и 
    $v_n \underset{n\to\infty}{\to} a_2$, $a_1, a_2 \in E$. Тогда $a_1 = a_2$
\end{theorem}
\begin{proof}
    Предположим, что $a_1 \neq a_2$, тогда $\rho(a_1, a_2) \overset{def}{=} 
    \delta > 0$. Положим $\epsilon = \frac{\delta}{4}$. Тогда
    $\exists N_1 : \forall n > N_1 \rho(v_n, a_1) < \epsilon$ и
    $\exists N_2 : \forall n > N_2 \rho(v_n, a_2) < \epsilon$.
    Пусть $n_0 = N_1 + N_2 + 1$, тогда $n_0 > N_1, n_0 > N_2$, поэтому 
    $\rho(v_{n_0}, a_1) < \epsilon \wedge \rho(v_{n_0}, a_2) < \epsilon$.
    Тогда получаем
    \begin{equation*}
        \rho(a_1, a_2) \leq \rho(a_1, v_{n_0}) + \rho(v_{n_0}, a_2) < 
        \epsilon + \epsilon = 2\epsilon = \delta / 2 < \delta
    \end{equation*}
    что противоречит выбору $\delta$.
\end{proof}

\begin{theorem}
    Об ограниченности последовательности, имеющей предел.
    Пусть $E$ - метрическое пространство с метрикой
    $\rho$, $\left\{v_n\right\}_{n = 1}^{\infty}$ - последовательность 
    элементов $E$ и $v_n \underset{n\to\infty}{\to} a$. Тогда 
    $\exists M > 0 : \forall n \in N$ имеем соотношение
    \begin{equation}\label{rvnaleqm}
        \rho(v_n, a) \leq M
    \end{equation}
\end{theorem}
\begin{proof}
    Выберем $\epsilon = 1$, тогда $\exists N_0 : \forall n > N_0   
    \rho(v_n, a) < 1$. Пусть $M_1 = max(\rho(v_1, a), \ldots, \rho(v_{N_0}, a))$.
    Тогда для $M = max(M_1, 1)$ соотношение \ref{rvnaleqm} выполнено 
    $\forall n \in \N$. 
\end{proof}

\section{Предел числовой последовательности}
Множество вещественных чисел является метрическим пространством: для 
$a, b \in \R$ положим $\rho(a, b) \overset{def}{=} |a - b|$, нужные свойства
следуют из свойства модуля. Поэтому, если $\left\{x_n\right\}_{n = 1}^{\infty}$ -
последовательность в $\R, a \in \R$, то общее определение предела переносится
так:
\begin{equation*}
    x_n \underset{n\to\infty}{\to} a \text{, если} \forall \epsilon > 0
    \exists N \in \N : \forall n > N \left|x_n - a\right| < \epsilon
\end{equation*}
Если $x_n \underset{n\to\infty}{\to} a, a \in \R$, то по теореме об 
ограниченности последовательности, имеющей предел, $\exists M : |x_n - a| 
\leq M \forall n$. Тогда
\begin{equation}\label{xn}
    |x_n| = |x_n - a + a| \leq |x_n - a| + |a| \leq M + |a| \forall n
\end{equation}

\section{Расширение множества вещественных чисел}
Добавим к множеству $\R$ два элемента, $\{+\infty\}, \{-\infty\}$.
Полагаем, что по определению, $a < +\infty \forall a \in \R, -\infty < a 
\forall a \in \R, -\infty < +\infty$.

\section{Определение бесконечных пределов}
Пусть $\left\{y_n\right\}_{n = 1}^{\infty}$ - последовательность вещественных 
чисел. Говорят, что $y_n$ стремится к $+\infty$, пишут 
$y_n \underset{n\to\infty}{\to} +\infty$, или, что предел $y_n$ равен $+\infty$
при $n$, стремящемся к бесконечности. Пишут $\lim_{n\to\infty}y_n = +\infty$, 
если $\forall K > 0 \exists N : \forall n > N y_n > K$.

Аналогично, для последовательности $\left\{t_n\right\}_{n = 1}^{\infty}$
$t_n \underset{n\to\infty}{\to} -\infty$ или $\lim_{n\to\infty}t_n = -\infty$,
если $\forall L < 0 \exists N_1 : \forall n > N_1 t_n < L$.

Если предел некоторой последовательности вещественное число, то говорят, что
её конечен, а если равен $\pm\infty$, то говорят, что предел бесконечен.

\begin{theorem}
    Критерий Коши существования конечного предела последовательности.
    Пусть $\left\{x_n\right\}_{n = 1}^{\infty}$ - последовательность 
    вещественных чисел. Для того, чтобы эта последовательность имела 
    конечный предел, необходимо и достаточно, чтобы $\forall \epsilon > 0
    \exists N \in \N : \forall n_1 > N, n_2 > N$ выполнялось соотношение
    \begin{equation}\label{koshi}
        |x_{n_2} - x_{n_1}| < \epsilon
    \end{equation}
\end{theorem}
\begin{proof}
    Достаточность. Построим сечение $\R$ с нижним классом $A$ и верхним классом
    $A'$ следующим образом: $\alpha \in A \Leftrightarrow \exists N_\alpha$, 
    зависящее от $\alpha$, т.ч. $\forall n > N_\alpha$ выполнено 
    \begin{equation}\label{kek5}
        x_n > \alpha
    \end{equation}
    $A' \overset{def}{=} R \setminus A$.

    Проверим, что $A \neq \emptyset$: возьмем $\epsilon = 1$, тогда \ref{koshi}
    $\Rightarrow \exists N_1 : \forall n_1, n_2 > N_1$ выполнено
    $|x_{n_2} - x_{n_1}| < 1$, что эквивалентно соотношению 
    \begin{equation}\label{kek6}
        x_{n_1} - 1 < x_{n_2} < x_{n_1} + 1
    \end{equation}
    Положим $n_1 = N_1 + 1$, тогда \ref{kek6} выполнено при $n_2 \geq n_1$,
    т.е. $x_{n_1} - 1 \in A$. Поскольку правое неравенство в \ref{kek6} 
    тоже выполнено при $n_2 \geq N_1 + 1$, то определение \ref{kek5} 
    $\Rightarrow x_{n_1} + 1 \notin A$, т.е. $x_{n_1} + 1 \in A'$, т.е. 
    $A \neq \R$.

    Определение $A, A' \Rightarrow A \cup A' = \R, A \cap A' = \emptyset$.

    Возьмем $\forall \alpha \in A, \forall \beta \in A'$. Тогда \ref{kek5}
    $\Rightarrow \exists N_\alpha : \forall n > N_\alpha$ выполнено \ref{kek5}.
    Поскольку $\beta \notin A$, то $\exists n_0 > N_\alpha : x_{n_0} \leq \beta$,
    поскольку в противоположном случае при отсутсвии такого $n_0$ из \ref{kek5}
    $\Rightarrow \beta \in A$, что неверно. Таким образом 
    \[\alpha < x_{n_0} \leq \beta \Rightarrow \alpha < \beta\]
    Итак, $A, A'$ - сечения $\R$. По теореме Дедекинда $\exists \gamma:
    \forall \alpha \in A \alpha \leq \gamma \wedge \forall \beta \in A' 
    \gamma \leq \beta$. Возьмем $\forall \epsilon > 0$, тогда $\gamma - 
    \frac{\epsilon}{2} \in A, \gamma + \frac{\epsilon}{2} \in A'$. 
    Выберем $N_2$ так, чтобы при $\forall n_1, n_2 > N_2$ выполнялось
    \begin{equation}\label{kek7}
        |x_{n_1} - x_{n_2}| \leq \frac{\epsilon}{2}
    \end{equation}
    Поскольку $\gamma -  \frac{\epsilon}{2} \in A$, то $\exists N_3 : 
    \forall n > N_3$ выполняется 
    \begin{equation}\label{kek8}
        x_n > \gamma - \frac{\epsilon}{2}
    \end{equation}
    Не уменьшая общности, считаем, что $N_3 \geq N_2$.

    Поскольку $\gamma + \frac{\epsilon}{2} \in A'$, то 
    \begin{equation}\label{kek9}
        \exists n' > N_3 : x_{n'} \leq \gamma + \frac{\epsilon}{2}
    \end{equation}
    Положим теперь $N = n'$, тогда \ref{kek8}, \ref{kek9} $\Rightarrow$
    \begin{equation}\label{kek10}
        \left( 
            \gamma - \frac{\epsilon}{2} < x_{n'} \leq \gamma + \frac{\epsilon}{2}
        \right) \Rightarrow
        |x_{n'} - \gamma| \leq \frac{\epsilon}{2}
    \end{equation}
    Теперь \ref{kek7} и \ref{kek10} при $n > N = n'$ влекут:
    \begin{equation}
        |x_{n} - \gamma| = |(x_{n} - x_{n'}) + (x_{n'} - \gamma)| \leq 
        |x_{n} - x_{n'}| + |x_{n'} - \gamma| < \frac{\epsilon}{2} + 
        \frac{\epsilon}{2} = \epsilon
    \end{equation}
    Т.е. из определения предела $x_n \underset{n\to\infty}{\to} \gamma$.
    Достаточность доказана.

    Доказательство необходимости. Пусть $x_n \underset{n\to\infty}{\to} a,
    a \in \R$, тогда $\forall \epsilon > 0 \exists N : \forall n > N$
    выполнено 
    \begin{equation}\label{kek11}
        |x_n - a| < \frac{\epsilon}{2}
    \end{equation}
    Возьмем $\forall n_1, n_2 > N$, тогда \ref{kek11} $\Rightarrow$
    \begin{equation*}
        |x_{n_2} - x_{n_1}| = |(x_{n_2} - a) - (x_{n_2} - a)| \leq 
        |x_{n_2} - a| + |x_{n_2} - a| < \frac{\epsilon}{2} + \frac{\epsilon}{2} 
        = \epsilon
    \end{equation*}
    Необходимость доказана.
\end{proof}

\section{Предельные переходы в арифметических действиях}
Далее для сокращения вместо $x_n \underset{n\to\infty}{\to} a$ пишем
$x_n \to a$. Далее $a, b, ... \in \R$.

\begin{theorem}
    \begin{enumerate}
        \item Пусть $x_n = a, n \geq 1 \Rightarrow x_n \to a$
        \item Пусть $x_n \to a, c \in \R \Rightarrow cx_n \to ca$
        \item Пусть $x_n \to a, y_n \to b \Rightarrow x_n + y_n \to a + b$
        \item Пусть $x_n \to a, y_n \to b \Rightarrow x_ny_n \to ab$
        \item Пусть $x_n \to a, a \neq 0, x_n \neq 0 \forall n \Rightarrow
\frac{1}{x_n} \to \frac{1}{a}$
        \item Пусть $x_n \to a, a \neq 0, x_n \neq 0 \forall n, y_n \to b
\Rightarrow \frac{y_n}{x_n} \to \frac{b}{a}$
    \end{enumerate}
\end{theorem}
\begin{proof}
    1) следует из определения. 
    
    Для 2): возьмем $\forall \epsilon > 0, \exists N
    : \forall n > N, |x_n - a| < \epsilon$, что влечет  $|c||x_n - a| < 
    (|c| + 1)\epsilon, |cx_n - ca| < (|c| + 1)\epsilon$.
    Поскольку $\epsilon > 0$ произвольно, то и $(|c| + 1)\epsilon$ произвольно.

    Для 3): возьмем $\forall \epsilon > 0, \exists N_1 : \forall n > N_1, 
    |x_n - a| < \frac{\epsilon}{2}, \exists N_2 : \forall n > N_2 |y_n - b| < 
    \frac{\epsilon}{2}$, пусть $N = max(N_1, N_2), \forall n > N$ имеем:
    \begin{equation*}
        |(x_n + y_n) - (a + b)| \leq |x_n - a| + |y_n - b| < 
        \frac{\epsilon}{2} + \frac{\epsilon}{2} = \epsilon
    \end{equation*}

    Для 4): $x_ny_n - ab = (xn - a)yn + a(y_n - b)$. Поскольку $y_n \to b$, то
    $\exists M > 0: |y_n| \leq M \forall n$. Возьмем $\forall \epsilon > 0,
    \exists N_1 : \forall n > N_1 |x_n - a| < \epsilon \wedge \exists N_2 :
    \forall n > N_2 |y_n - b| < \epsilon; N = \overset{def}{=} max(N_1, N_2)
    \Rightarrow \forall n > N \Rightarrow$
    \begin{equation*}
        |x_ny_n - ab| \leq |x_n - a|y_n + |a||y_n - b| < \epsilon * M + 
        |a| * \epsilon = (M + |a|)\epsilon
    \end{equation*}
    Выражение $(M + |a|)\epsilon$ может быть выбрано произвольным $> 0$ 
    вместе с $\epsilon$.

    Для 5): $\exists N_1 : \forall n > N_1 \; |x_n - a| < \frac{|a|}{2}$, 
    тогда $\forall n > N_1$ имеем:
    \begin{equation*}
        |x_n| = |(x_n - a) + a| \geq |a| - |x_n - a| > |a| - \frac{|a|}{2} = 
        \frac{|a|}{2}
    \end{equation*}
    т.е. при $n > N_1$ $\frac{1}{x_n} < \frac{2}{|a|}$, $\frac{1}{x_n} - 
    \frac{1}{a} = \frac{a - x_n}{x_na}$. Возьмем $\forall \epsilon > 0, 
    \exists N_2 : \forall n > N_2 \; |x_n - a| < \epsilon$, пусть 
    $N = max(N_1, N_2)$. При $n > N$ имеем:
    \begin{equation*}
        |\frac{1}{x_n} - \frac{1}{a}| = \frac{|a - x_n|}{|x_n||a|} < 
        \frac{2}{|a|} \cdot \frac{1}{|a|}\epsilon = \frac{2}{a^2}\epsilon
    \end{equation*}
    $\frac{2}{a^2}\epsilon$ может быть любым положительным числом.

    Для 6) $\frac{y_n}{x_n} = \frac{1}{x_n} \cdot y_n$, тогда 4) и 5)
    $\Rightarrow$ 6).
\end{proof}

\section{Переход к пределу в неравенствах}
\begin{theorem} О двух миллиционерах??
    \begin{enumerate}
        \item Пусть $x_n \leq y_n \forall n, x_n \to a, y_n \to b; a, b \in \R
    \Rightarrow a \leq b$
        \item Пусть $x_n \leq y_n \leq z_n \forall n, x_n \to a, z_n \to a,
    a \in \R \Rightarrow y_n \to a$
    \end{enumerate} 
\end{theorem}
\begin{proof}
    1. Пусть $a > b, a - b = \delta > 0 \Rightarrow \exists N_1 : \forall n > N_1
    \; |x_n - a| \leq \frac{\delta}{4}, \exists N_2 : \forall n > N_2 \; 
    |y_n - b| < \frac{\delta}{4}$, возьмем $n_0 = N_1 + N_2 + 1 \Rightarrow$
    \begin{equation*}
        x_{n_0} > a - \frac{\delta}{4} = b + \delta - \frac{\delta}{4} = 
        b + \frac{3}{4}\delta = (b + \frac{\delta}{4}) + \frac{\delta}{2} >
        y_{n_0} + \frac{\delta}{2} > x_{n_0}
    \end{equation*}
    что противоречит условию.

    2. Возьмем $\forall \epsilon > 0, \exists N_1 : \forall n < N_1 \; 
    |x_n - a| < \epsilon, \exists N_2 : \forall n > N_2 \; |z_n - a| < \epsilon$,
    тогда для $N = max(N_1, N_2)$ имеем при $n > N$
    \begin{equation*}
        a - \epsilon < x_n \leq y_n \leq z_n < a + e \Rightarrow 
        |y_n - a| < \epsilon 
    \end{equation*}
\end{proof}

\begin{term}
    Если $x_n \underset{n\to\infty}{\to} 0$, то говорят, что 
    $\{x_n\}_{n=1}^{+\infty}$ бесконечно малая; если $|y_n| \underset{n\to\infty}{\to}
    +\infty$, то говорят, что последовательность $\{y_n\}_{n=1}^{+\infty}$ 
    бесконечно большая.
\end{term}

\begin{assertion}
    Пусть последовательность $\{y_n\}_{n=1}^{+\infty}$ бесконечно большая, 
    $y_n \neq 0 \forall n, x_n = \frac{1}{y_n}$. Тогда 
    $\{x_n\}_{n=1}^{+\infty}$ бесконечно малая последовательность.
    Пусть $\{x_n\}_{n=1}^{+\infty}$ бесконечно малая последовательность, 
    $x_n \neq 0 \forall n, y_n = \frac{1}{x_n}$. Тогда $\{y_n\}_{n=1}^{+\infty}$
    бесконечно большая последовательность.
\end{assertion}
\begin{proof}
    Докажем первое, второе доказывается аналогично. Возьмем $\forall \epsilon > 0$,
    пусть $L = 1 / \epsilon$. Тогда $\exists N : \forall n > N \; |y_n| > L$, но
    $|y_n| > L \Leftrightarrow \frac{1}{|y_n|} < \frac{1}{L} = \epsilon$, т.е.
    $|x_n - 0| = |x_n| = \frac{1}{|y_n|} < \epsilon$
\end{proof}

\section{Предельные переходы и бесконечные пределы}
\subsection{Дополнение к предельным переходам в арифметических действиях}
\begin{theorem}
    Справедливы следующие утвреждения:
    \begin{enumerate}
        \item $c > 0, x_n \to +\infty, y_n \to -\infty \Rightarrow
    cx_n \to +\infty, cy_n \to -\infty; d < 0, dx_n \to -\infty 
    dy_n \to +\infty$
        \item $x_n \to +\infty, y_n \to a \in \R, z_n \to +\infty \Rightarrow
    x_n + y_n \to +\infty, x_n + z_n \to +\infty$ 
    $u_n \to -\infty, v_n \to b \in \R, w_n \to -\infty \Rightarrow
    u_n + v_n \to -\infty, u_n + w_n \to -\infty$ 
        \item $x_n \to +\infty, y_n \to a > 0, z_n \to b < 0 \Rightarrow 
    x_ny_n \to +\infty, x_nz_n \to -\infty$
    $u_n \to -\infty, v_n \to c > 0, w_n \to d < 0 \Rightarrow 
    u_nv_n \to -\infty, u_nw_n \to +\infty$
    $x_n \to +\infty, t_n \to +\infty, s_n \to -\infty \Rightarrow
    x_nt_n \to +\infty, x_ns_n \to -\infty$
    $u_n \to -\infty, v_n \to -\infty \Rightarrow u_nv_n \to +\infty$;
    \end{enumerate}
\end{theorem}

\subsection{Дополнение к предельным переходам в неравенствах}
\begin{theorem}
    Справедливы следующие утверждения:
    \begin{enumerate}
        \item Пусть $x_n \leq y_n \forall n, x_n \to a \in \overline{\R},
        y_n \to b \in \overline{\R} \Rightarrow a \leq b$
        \item Пусть $x_n \leq y_n \leq z_n \forall n, x_n \to a \in \overline{\R},
        z_n \to a \in \overline{\R} \Rightarrow y_n \to a$ 
    \end{enumerate}
\end{theorem}
\begin{proof}
    Доказательство приведенных выше теорем проще доказательств соответсвующих
    теорем для конечных пределов и в дальнейшем курсе не используются, 
    поэтому будут приняты без доказательства.
\end{proof}

\begin{definition}
    Последовательность $\{x_n\}_{n=1}^{+\infty}$ называется возрастающей, если
    $x_n \leq x_{n+1} \forall n$; последовательность $\{y_n\}_{n=1}^{+\infty}$
    называется строго возрастающей, если $y_n < y_{n+1} \forall n$.
    Последовательность $\{u_n\}_{n=1}^{+\infty}$ называется убывающей, если
    $u_n \geq u_{n+1} \forall n$; последовательность $\{v_n\}_{n=1}^{+\infty}$
    называется строго убывающей, если $v_n > v_{n+1} \forall n$.
    Последовательность $\{t_n\}_{n=1}^{+\infty}$ называется монотонной, если
    она возрастающая или убывающая; последовательность $\{w_n\}_{n=1}^{+\infty}$
    называется строго монотонной, если она строго возрастающая или строго 
    убывающая.
\end{definition}

\begin{theorem}
    \begin{enumerate}
        \item Пусть $\{x_n\}_{n=1}^{+\infty}$ - монотонная последовательность.
        Тогда $\exists \lim_{n\to\infty}x_n \in \overline{\R}$
        \item Пусть $\{y_n\}_{n=1}^{+\infty}$ - возрастающая.
        Тогда $\lim_{n\to\infty}y_n \in \R \Leftrightarrow \exists M \in \R :
        y_n \leq M \forall n$
        \item Пусть $\{u_n\}_{n=1}^{+\infty}$ - убывающая.
        Тогда $\lim_{n\to\infty}u_n \in \R \Leftrightarrow \exists K \in \R :
        u_n \geq K \forall n$
        \item Пусть $\{v_n\}_{n=1}^{+\infty}$ - строго возрастающая и 
        $\lim_{n\to\infty}v_n \in \R$. Тогда $v_n < \lim_{n\to\infty}v_n
        \forall n$
        \item Пусть $\{w_n\}_{n=1}^{+\infty}$ - строго убывающая и 
        $\lim_{n\to\infty}w_n \in \R$. Тогда $w_n > \lim_{n\to\infty}w_n
        \forall n$
    \end{enumerate}
\end{theorem}
\begin{proof}
    Доказательство соотношений 1), 2), 4). Предположим, что последовательность
    $\{y_n\}_{n=1}^{+\infty}$ неограничена сверху. Возьмем $\forall L > 0$,
    тогда $\exists N : y_N > L \Rightarrow \forall n > N \; 
    y_n \geq y_{n-1} \geq \ldots \geq y_N > L \Rightarrow y_n 
    \underset{n\to\infty}{\to} +\infty$.
    Предположим, что $\{y_n\}_{n=1}^{+\infty}$ ограничена сверху, т.е. 
    $\exists M : y_n \leq M \forall n$. Пусть $E = \{y \in \R : \exists n :
    y = y_n\}$. Тогда множество $E$ непусто и ограничено сверху, $M$ - его
    верхняя граница. Пусть $a = supE$. Тогда $y_n \leq a \forall n$. Возьмем
    $\epsilon > 0$, тогда $a - \epsilon$ - не верхняя граница $E$, тогда
    $\exists N : y_N > a - \epsilon$. Тогда при $\forall n > N$ имеем
    $y_n \geq y_{n-1} \geq \ldots \geq y_N > a - \epsilon, y_n \leq a$, т.е. 
    $|y_n - a| < \epsilon$, т.е. $y_n \underset{n\to\infty}{\to} a$. 
    Утверждение 1) для возрастающей последовательности и достаточность в
    утверждении 2) доказаны. Необходимость в утверждении 2) следует из 
    ограниченности последовательности, имеющей конечный предел.
    Для доказательства 4) пишем $(y_n < y_{n+1} \leq \lim_{n\to\infty}y_n)$,
    $(v_n < v_{n+1} \leq \lim_{n\to\infty}v_n)$.
    Доказательство утверждений 1), 3), 5) следуют из того, что если $u_n$ - 
    убывающая, то $y_n \overset{def}{=} -u_n$ - возрастающая, если $w_n$ - 
    строго убывающая, то $v_n \overset{def}{=} -w_n$ - строго возрастающая, и 
    далее применим утверждения 1), 2), 4).
\end{proof}

\begin{theorem}
    Теорема о вложенных промежутках. Пусть $[a_{n+1}, b_{n+1}] \subset 
    [a_n, b_n], n = 1, 2, \ldots, n; \; b_n - a_n \underset{n\to\infty}{\to} 0$.
    Тогда $\exists! \; c \in \R : c \in [a_n, b_n] \forall n$. 
\end{theorem}
\begin{remark}
    Условие замкнутости промежутков существенно: имеем 
    $\left(0, \frac{1}{n + 1}\right] \subset \left(0, \frac{1}{n}\right], 
    \frac{1}{n} - 0  = \frac{1}{n} \underset{n\to\infty}{\to} 0$, но 
    $\bigcap_{n=1}^\infty\left(0, \frac{1}{n}\right] \neq \emptyset$.
\end{remark}
\begin{proof}
    Имеем неравенство $a_n < b_n \leq b_{n-1} \leq \ldots \leq b_1, 
    b_n > a_n \geq a_{n-1} \geq \ldots \geq a_1 \forall n$, т.е. 
    $a_n \leq b_1 \forall n, b_n \geq a_1 \forall n$. Тогда в силу 
    возрастания $\{a_n\}_{n=1}^{+\infty}$ и убывания $\{b_n\}_{n=1}^{+\infty}$
    по предыдущей теореме $\exists c_1 = lim_{n\to\infty}a_n; \exists c_2 = 
    lim_{n\to\infty}b_n; c_1, c_2 \in \R$. По свойству перехода к пределу 
    в неравенстве имеем: 
    \begin{equation*}
        a_n < b_n \Rightarrow lim_{n\to\infty}a_n \leq lim_{n\to\infty}b_n
        \Rightarrow c_1 \leq c_2
    \end{equation*}
    По доказательству предыдущей теоремы имеем $a_n \leq c_1 \forall n; 
    b_n \geq c_2 \forall n$, поэтому $0 \leq c_2 - c_1 \leq b_n - a_n$,
    тогда
    \begin{equation*}
        0 \leq c_2 - c_1 \leq lim_{n\to\infty}(b_n - a_n) = 0 \Rightarrow
        c_1 = c_2 \overset{def}{=} c
    \end{equation*}
    Тогда имеем $a_n \leq c \leq b_n$, т.е. $c \in [a_n, b_n]$.
    Если бы $\exists c_0 \in [a_n, b_n] \forall n$, то $|c_0 - c| \leq b_n - a_n,
    |c_0 - c| \leq lim_{n\to\infty}(b_n - a_n) = 0$, т.е. $c$ - единственно.
\end{proof}

\section{Число $e$}
\begin{theorem}
    Пусть $x_n = \left(1 + \frac{1}{n}\right)^n, y_n = 
    \left(1 + \frac{1}{n}\right)^{n+1}, n = 1,2, \ldots$. Тогда 
    $\{x_n\}_{n=1}^{+\infty}$ строго возрастает, $y_n$ строго убывает,
    $\exists lim_{n\to\infty}x_n = lim_{n\to\infty}y_n \overset{def}{=} e 
    \wedge x_n < e < y_n \forall n$, в частности $2 < e < 3$.
\end{theorem}
\begin{remark}
    Вычислено, что $e = 2,718\dots$
\end{remark}
\begin{proof}
    Докажем строгое возрастание $\{x_n\}_{n=1}^{+\infty}$. Считаем $n\geq 3$,
    при $n = 1, n = 2$ - явные вычисления. Применяя бином Ньютона, имеем:
    \begin{multline}\label{1lol}
        x_n = \left(1 + \frac{1}{n}\right)^n = 1^n + 
        \sum_{k=1}^{n-1}C_n^k\cdot 1^{n-k}\cdot\frac{1}{n^k} + \frac{1}{n^n} =\\
        1 + C_n^1\cdot\frac{1}{n} + \sum_{k=2}^{n-1}C_n^k\cdot\frac{1}{n^k} + 
        \frac{1}{n^n} = 2 + \sum_{k=2}^{n-1}\frac{n!}{k!(n-k)!}\cdot\frac{1}{n^k} +
        \frac{1}{n^n}
    \end{multline}
    Преобразуем отдельно:
    \begin{multline}\label{2lol}
        \frac{n!}{k!(n-k)!}\frac{1}{n^k} = 
        \frac{n(n-1)\ldots(n-k+1)(n-k)!}{k!(n-k)!n^k} = \\
        \frac{n(n-1)(n-k+1)}{k!n^k} = 
        \frac{1}{k!}\left(1 - \frac{1}{n}\right)\left(1 - \frac{2}{n}\right)
        \cdot\ldots\cdot\left(1 - \frac{k-1}{n}\right)
    \end{multline}
    \begin{multline}\label{3lol}
        \frac{1}{n^n} = \frac{n!}{n^n}\cdot\frac{1}{n!} = 
        \frac{(n-1)!}{n^{n-1}}\cdot\frac{1}{n!} = 
        \frac{1}{n!}\left(1-\frac{1}{n}\right)\cdot\left(1-\frac{2}{n}\right)
        \cdot\ldots\cdot\left(1-\frac{n-1}{n}\right)
    \end{multline}
    Тогда соотношения \ref{1lol}, \ref{2lol}, \ref{3lol} влекут 
    \begin{equation}\label{4lol}
        x_n = 2 + \sum_{k=2}^n\frac{1}{k!}\left(1 - \frac{1}{n}\right)\ldots
        \left(1 - \frac{k-1}{n}\right)
    \end{equation}
    Тогда \ref{4lol} $\Rightarrow$
    \begin{equation}\label{5lol}
        x_{n+1} = 2 + \sum_{k=2}^{n+1}\frac{1}{k!}\left(1 - \frac{1}{n+1}\right)\ldots
        \left(1 - \frac{k-1}{n+1}\right)
    \end{equation}
    Поскольку $1 - \frac{e}{n} < 1 - \frac{e}{n + 1}, 1 \leq e \leq n - 1$, и в
    \ref{5lol} есть ещё одно слагаемое по сравнению с $x_n$, то 
    $x_n < x_{n+1}$

    Докажем строгое убывание $y_n$. Пусть $n \geq 0$.
    \begin{multline}\label{6lol}
        \frac{y_{n-1}}{y_n} = 
        \frac{\left(1 + \frac{1}{n-1}\right)^n}{\left(1 + \frac{1}{n}\right)^{n+1}} =
        \frac{\left(\frac{n}{n-1}\right)^n}{\left(\frac{n+1}{n}\right)^{n+1}} = 
        \frac{n}{n+1}\cdot\frac{\left(\frac{n}{n-1}\right)^n}{\left(\frac{n+1}{n}\right)^{n}} = \\
        \frac{n}{n+1}\left(\frac{n}{n-1}\right)^n\left(\frac{n+1}{n}\right)^n = 
        \frac{n}{n+1}\frac{n^{2n}}{(n^2 - 1)^n} = \\
        \frac{n}{n+1}\left(\frac{n^{2}}{n^2 - 1}\right)^n = 
        \frac{n}{n+1}\left(1 + \frac{1}{n^2 - 1}\right)^n
    \end{multline}
    Применяя бином Ньютона к выражению $(1 + x)^n, x > 0, n \geq 2$, имеем
    $(1 + x)^n = 1^n + C_n^11^{n-1}x + \ldots = 1 + nx + \ldots$,
    где многоточие означает положительные слагаемые, поэтому получаем
    неравенство Бернулли:
    \begin{equation}\label{6dotlol}
        (1 + x)^n > 1 + nx, n\geq 2, x > 0
    \end{equation}
    Применим \ref{6lol} к \ref{6dotlol} с $x = \frac{1}{n^2 - 1}$. Тогда
    \ref{6lol}, \ref{6dotlol} $\Rightarrow$
    \begin{equation*}
        \frac{y_{n-1}}{y_n} > \frac{n}{n + 1}\left(1 + \frac{n}{n^2 - 1}\right) =
        \frac{n(n^2+n-1)}{(n+1)(n^2-1)} = \frac{n^3 + n^2 - n}{n^3 + n^2 - n - 1} 
        > 1
    \end{equation*}
    Строгое убывание $\{y_n\}_{n=1}^{+\infty}$ доказано.
    
    Далее,
    \begin{equation}\label{7lol}
        y_n = \left(1 + \frac{1}{n}\right)^{n+1} = \left(1 + \frac{1}{n}\right)
        \left(1 + \frac{1}{n}\right)^n = \left(1 + \frac{1}{n}\right)x_n
    \end{equation}
    \ref{7lol} $\Rightarrow y_n > x_n, y_n - x_n = \frac{1}{n}x_n$. 
    Теперь доказано, что
    $x_n < x_{n+1} < y_{n+1} < y_n$, т.е. $[x_{n+1}, y_{n+1}] \subset [x_n, y_n],
    0 < x_n < y_n < y_{n-1} < y_5 < 3, n \geq 5 \Rightarrow y_n - x_n < \frac{3}{n},
    n \geq 5 \Rightarrow y_n - x_n \underset{n\to\infty}{\to} 0$

    К мромежуткам $[x_n, y_n]$ применим теорему о вложенных промежутках, 
    тогда $\exists! e \in [x_n, y_n] \forall n$, из доказательства теоремы 
    о вложенных промежутках следует, что $e = \lim_{n\to\infty}x_n = 
    \lim_{n\to\infty}y_n$. По утверждениям 4 и 5 теоремы о пределах 
    монотонных последовательностей в силу строгого возрастания 
    $\{x_n\}_{n=1}^{+\infty}$ и строгого убывания $\{y_n\}_{n=1}^{+\infty}$
    имеем $x_n < e < y_n \forall n$
\end{proof}

\section{Подпоследовательности}
\begin{definition}
    Пусть $F: \N \to \R$ - последовательность, $\phi: \N \to \N$ - 
    инъективное отображение и $\phi(n) < \phi(m), n < m$.
    Подпоследовательностью последовательности $F$ называется 
    $G \overset{def}{=} F(\phi): \N \to \R$. $G$ - тоже последовательность.

    В приведенном определении фигурируют последовательности вещественных 
    чисел, но определение сохраняется, если рассматривать любую последовательность
    $F: \N \to E$ из элементов множества $E$, подпоследовательность будет 
    определяяться как суперпозиция $F(\phi): \N \to E$.
    
    Исторически сложившееся обозначение подпоследовательности следующее.
    Обозначают $\phi(k) = n_k$, тогдa, если $F(n) = x_n$, то $F(\phi(k)) = x_{n_k}$,
    т.е. рассматриваются элементы с номерами $x_{n_1}, x_{n_2}, \ldots, x_{n_k},
    \ldots$, стандартное обозначение $\{x_{n_k}\}_{k=1}^{+\infty}$ и
    говорят, что подпоследовательность выбрана из последовательности, т.е. 
    выбраны номера $n_1, n_2, \ldots; n_1 < n_2 < \ldots < n_k < \ldots$,
    и рассматриваются элементы $x_{n_k}$ только с этими номерами.
\end{definition}

\begin{assertion}
    Предположим, что $x_n \underset{n\to\infty}{\to} a, a \in \overline{\R},
    \{x_{n_k}\}_{k=1}^{+\infty}$ - подпоследотвальность. Тогда
    $x_{n_k} \underset{k\to\infty}{\to} a$.
\end{assertion}
\begin{proof}
    Для $a \in \R$, для $a = \pm\infty$ аналогично. Поскольку $n_1 < n_2 <
    \ldots < n_k < \ldots$, то $n_k \geq k$; возьмем $\epsilon > 0$ и пусть 
    $N : \forall n > N \; |x_n - a| < \epsilon$. Тогда $\forall k > N$ имеем
    $n_k \geq k > N$, поэтому $\forall k > N |x_{n_k} - a| < \epsilon$.
\end{proof}

\begin{theorem}
    Принцип выбора Больцано — Вейерштрасса. Пусть $\{x_{n}\}_{n=1}^{+\infty}$
    - последовательность $M > 0, |x_n| \leq M \forall n$. Тогда 
    $\exists a, a \in [-M, M]$ и подпоследовательность 
    $\{x_{n_k}\}_{k=1}^{+\infty} : x_{n_k} \underset{k\to\infty}{\to} a$.
\end{theorem}
\begin{proof}
    Положим $a_1 = -M, b_1 = M, c_1 = 0 = \frac{a_1 + b_1}{2}$. Тогда либо для 
    $[a_1, c_1]$, либо для $[c_1, b_1]$, либо для обоих, выполнено следующее 
    утверждение 1: существует бесконечно много номеров $n$ таких, что 
    $x_n$ принадлежит $x_n$ лежит на этом промежутке. Пусть $[a_2, b_2]$ - 
    именно этот промежуток, $n_1 : x_{n_1} \in [a_2, b_2]$. Здесь либо
    $a_2 = a_1$, $b_2 = c_1$, либо $a_2 = c_1$, $b_2 = b_1$. Пусть 
    $c_2 = \frac{a_2 + b_2}{2}$. Тогда либо для $[a_2, c_2]$, либо для 
    $[c_2, b_2]$, либо для обоих выполнено утверждение 2: существует
    бесконечно много номеров $n > n_1 : x_n$ лежит на этом промежутке. 
    Пусть $[a_3, b_3]$ именно этот промежуток, т.е. либо $a_3 = a_2$, 
    $b_3 = c_2$, либо $a_3 = c_2$, $b_3 = b_2$, и пусть $x_{n_2} \in 
    [a_3, b_3], n_2 > n_1$. Далее по индкуции пусть уже выбраны 
    $[a_k, b_k], n_{k-1} > n_{k-2}$. Пусть $c_k = \frac{a_k + b_k}{2}$.
    Для $[a_k, c_k]$ или для $[c_k, b_k]$, или для обоих выполнено утверждение
    $k$: существует бесконечно много номеров $n > n_{k-1} : x_n$ лежит 
    на этом промежутке. Пусть $[a_{k+1}, b_{k+1}]$ этот промежуток, т.е. 
    либо $a_{k+1} = a_k, b_{k+1} = c_k$, либо $a_{k+1} = c_k, b_{k+1} = b_k$,
    и пусть $x_{n_k} \in [a_{k+1}, b_{k+1}], n_k > n_{k - 1}$. 

    Построена последовательность $\{x_{n_k}\}_{k=1}^{+\infty}$ и 
    последовательность вложенных промежутков $\ldots [a_{k+1}, b_{k+1}] 
    \subset [a_k, b_k] \subset \ldots \subset [a_1, b_1]$, при этом 
    \begin{equation*}
        b_k - a_k = \frac{2M}{2^{k-1}} = 2^{-k+1}M \underset{k\to\infty}{\to} 0
    \end{equation*}

    По теореме о вложенных промежутках $\exists a \in [a_k, b_k] \forall k,
    a_k \underset{k\to\infty}{\to} a, b_k \underset{k\to\infty}{\to} a$, 
    при этом $x_{n_k} \in [a_{k+1}, b_{k+1}]$, т.е. $a_{k+1} \leq x_{n_k} \leq 
    b_{k+1}$.

    По утверждению 2 теоремы о переходе к пределу в неравенстве $x_n 
    \underset{k\to\infty}{\to} a$.
\end{proof}

\end{document}
