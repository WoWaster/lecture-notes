% !TeX root = ./main.tex
\documentclass[main]{subfiles}
\begin{document}
\chapter{Вещественные числа}
\section{Обозначения и нотация}

В дальнейшем множество будем понимать как совокупность объектов, называемых
его элементами. Приведенное высказывание не является определением, однако в
дальнейшем при операциях с конкретными множествами, математический контекст
рассматриваемые множества определяет.

Если $a,b$ -- некие элементы, $A$ -- множество, то запись $a\in A$ означает,
что $a$ принадлежит множеству $A$; запись $b \not\in A$ означает,
что элемент $b$ не принадлежит множеству $A$.

Символ $\forall$ означает высказывание <<для всякого>>, далее всегда будет следовать
текст конкретизирующий это высказывание.

Символ $\exists$ означает высказывание <<существует>> и также будет задан
математическим контекстом.

Запись $A \Rightarrow B$ или $B \Leftarrow A$ означает <<из $A$ следует $B$>>;
запись $A \Leftrightarrow B$ означает <<$A$ эквивалентно $B$>>.

Множества $A$ и $B$ называются совпадающими, что записывают формулой $A=B$,
если $(\forall a \in A) \Rightarrow (a \in B)$ и
$(\forall b \in B) \Rightarrow (b \in A)$;
приведенная формальная запись означает, что $A=B$ в том и только в том случае,
когда они состоят из одних и тех же элементов.

Если множества $A$ и $B$ не совпадают, то пишут $A \neq B$.

Определяют также пустое множество, в котором нет элементов,
которое будем обозначать символом $\varnothing$.

Запись $A \subset B$ читается <<$A$ содержится в $B$>> и означает, что
$(\forall a \in A) \Rightarrow (a \in B)$. Полагаем, что $\varnothing \subset A$
для любого множества $A$. Понятно, что
\[A = B \Leftrightarrow (A \subset B) \text{ и } (B \subset A).\]

В дальнейшем при рассмотрении сразу нескольких множеств в качестве синонима
слова <<множество>> будем использовать слова <<семейство>>, <<класс>>,
<<совокупность>>.

\section{Операции над множествами}

Объединением $ A \cup B$ множеств $A$ и $B$ будем называть множество:
\[(a \in A \cup B) \Leftrightarrow (a \in A) \text{ или } (a \in B).\]

Если множество $A$ задается каким-то условием, обозначим его <<условие>>,
то для задания множества $A$ будем использовать обозначение
\[A=\{a:\text{<<условие>> на } a\}\]
\begin{example}
    \[A_1 \cup A_2 = \{a: a \in A_1 \text{ или } a \in A_2\} \]
\end{example}

Если имеется произвольное непустое множество $I$ и $\forall \alpha \in I$
имеется множество $A_\alpha$, то
\[\bigcup_{a \in I} A_\alpha = \{a: \exists \alpha \in I \text{ такое, что } a \in A_\alpha\}\]

Пересечением $A \cap B$ назовем множество
\[A \cap B = \{a: (a \in A) \text{ и } (a \in B)\}.\]

Если элементов $a$, принадлежащих $A$ и $B$, не существует, пишем
\[A \cap B = \varnothing\]
и называем $A$ и $B$ дизъюнктивными. Если есть непустое множество $I$,
то, предполагая, что $\forall\alpha\in I \exists A_\alpha$, Полагаем
\[\bigcap_{\alpha\in I} A_\alpha = \{a: \forall \alpha \in I \quad a \in A_\alpha\}\]

Теоретико-множественной разностью множеств $A$ и $B$, обозначаемой $A \\ B$,
называется множество
\[A \\ B = \{a: a\in A, a \not\in B\}\]

\begin{theorem}
    Предположим, что имеется непустое множество $I$ и для любого $\alpha \in I$
    имеется множество $A_\alpha$. Справедливы следующие формулы:
    \begin{equation}
        \label{thm:1.1}
        B \cap \left( \bigcup_{\alpha \in I} A_\alpha \right) =
        \bigcup_{\alpha \in I} (B \cap A_\alpha)
    \end{equation}
    \begin{equation}
        B \cup \left( \bigcap_{\alpha \in I} A_\alpha \right) =
        \bigcap_{\alpha \in I} (B \cup A_\alpha)
    \end{equation}
    \begin{equation}
        B \setminus \left( \bigcup_{\alpha \in I} A_\alpha \right) =
        \bigcap_{\alpha \in I} (B \setminus A_\alpha)
    \end{equation}
    \begin{equation}
        B \setminus \left( \bigcap_{\alpha \in I} A_\alpha \right) =
        \bigcup_{\alpha \in I} (B \cup A_\alpha)
    \end{equation}
\end{theorem}

\begin{proof}
    Докажем \eqref{thm:1.1}, остальные соотношения доказываются аналогично.
    Обозначим левую часть \eqref{thm:1.1} через $C$, а правую через $D$.
    Если $a \in C$, то $a\in B$ и $a \in \bigcup_{\alpha \in I} A_\alpha$,
    т.е. $\exists \alpha_0 \in I$, такое что $a \in A_\alpha$, тогда
    $a \in B \cap A_{\alpha_0}$, $a \in \bigcup_{\alpha \in I}(B \cap A_\alpha)$,
    $a \in D$, то есть $C \subset D$. Если $b \in D$, то $\exists \alpha_1 \in I$
    такое что $b \in B \cap A_{\alpha_1}$, то есть $b \in B$ и $b \in A_{\alpha_1}$,
    тогда $b \in \bigcup_{\alpha \in I} A_\alpha$,
    $b \in B \cap \bigcup_{\alpha \in I} A_\alpha$, т.е. $b \in C$ и $D \subset C$,
    т.е. $C = D$, что и требовалось доказать.
\end{proof}

\section{Определение вещественных чисел по Р.Дедекинду}
Далее будем считать известными натуральные числа, множество которых всегда
обозначается через $\N$, множество целых чисел $\Z$, множество
рациональных чисел $\Q$. Считаем, что свойства арифметических действий
с числами из $\Q$ и свойства, связанные с упорядочиванием рациональных
чисел по возрастанию, известны.

\begin{definition}
    Пусть $\alpha$ - непустое множество, состоящее из рациональных чисел. Будем
    называть множество $\alpha$ сечением, если выполняются следующие условия:
    \begin{enumerate}
        \item $\alpha \neq \Q$
        \item Если $p \in \alpha, q \in \alpha, q < p$, то $q \in \alpha$
        \item В $\alpha$ нет наибольшего числа, т.е. не существует $p_0 \in \alpha$,
              такого что $\forall p \in \alpha$ выполнено $p \leq p_0$
    \end{enumerate}
\end{definition}

\begin{assertion}
    Пусть $\alpha$ -- сечение. Если $q \in \Q, p \in \alpha, q \notin
        \alpha$, то $p < q$.
\end{assertion}
\begin{proof}
    Из условия следует, что $p \neq q$. Если бы выполнялось $q < p$, то по п.2
    определения сечения $q \in \alpha$, чего нет. Следовательно $p > q$, чтд.
\end{proof}

\begin{term}
    Пусть $\alpha$ -- сечение. Числа из $\Q$, принадлежащие $\alpha$, называются
    нижними числами сечения $\alpha$, а числа из $\Q$, не принадлежащие $\alpha$,
    называются верхними числами сечения $\alpha$.
\end{term}

Сопоставим теперь $\forall z \in \Q$ сечение, которое будем обозначать $z^*$.
Далее запись $A \overset{def}{=} B$ означает, что объект $A$ определяется через
объект $B$. Полагаем:
\begin{equation}
    \label{thm:1.5}
    z^* = \{ p \in \Q : p < z \}
\end{equation}

Запись \eqref{thm:1.5} является сокращением формальной записи \eqref{thm:1.6}
\begin{equation}
    \label{thm:1.6}
    z^* = \{ p: p \in \Q \wedge p < z \}
\end{equation}

Проверим, что $z^*$ -- сечение. $z - 1 < z$, т.е. $z - 1 \in z^*$,
множество $z^*$ непустое. $z + 1 > z, z + 1 \notin z^*, z^* \neq \Q$.
Если $p \in z^* \wedge q \in \Q, q < p$, то $q < p < z \Rightarrow
    q < z, q \in z^*$. Если $p_1 \in z^*$, то ${p_1 < z}$;
пусть $p_2 = \frac{p_1 + z}{2}$, тогда $p_1 < p_2 < z$, $p_2 \in z^*$, т.е.
в $z^*$ нет наибольшего числа.

\begin{definition}
    Множество всех сечений будет называться множеством вещественных чисел, а любое
    конкретное сечение будем называть вещественным числом. Обозначаем множество
    вещественных чисел $\R$.
\end{definition}

Приведенный подход к определению вещественных чисел принадлежит немецкому
математику Р. Дедекинду, поэтому сечения называются сечениями множества
рациональных чисел по Дедекинду.

\section{Упорядочивание по возрастанию и арифметические действия над $\R$ числами}
\begin{definition}
    Пишем $\alpha < \beta$, говорим, что $\alpha$ меньше $\beta$, если
    $\exists p \in \Q$, т.ч. $p \in \beta \wedge p \notin \alpha$.
    Пишем $\alpha \leq \beta$, говорим, что $\alpha$ не превосходит $\beta$,
    если $\alpha < \beta \vee \alpha = \beta$.
\end{definition}

\begin{theorem}
    Пусть $\alpha$, $\beta$ -- сечения. Тогда либо $\alpha < \beta$, либо
    $\alpha = \beta$, либо $\alpha > \beta$.
\end{theorem}
\begin{proof}
    Если $\alpha = \beta$, то определение влечёт, что не может быть при этом
    $\alpha < \beta$ или $\alpha > \beta$. Пусть $\alpha \neq \beta$. Докажем,
    что выполнено только одно соотношение $\alpha < \beta$ или $\alpha > \beta$.
    Предположим, что выполнены оба, т.е. $\alpha < \beta$ и $\beta < \alpha$.
    Тогда $(\alpha < \beta) \Rightarrow (\exists p \in \Q |
        p \in \beta, p \notin \alpha)$; $(\beta < \alpha) \Rightarrow
        (\exists q \in \Q | q \in \alpha, q \notin \beta$).
    По утверждению из предыдущей лекции $(p \in \beta, q \notin \beta) \Rightarrow
        p < q; (q \in \alpha, p \notin \alpha) \Rightarrow q < p$ -- получили
    противоречие.

    Таким образом, $\alpha < \beta$ и $\beta < \alpha$ вместе не могут выполняться.
    Но, если $\alpha \neq \beta$, то в каком-то из этих множеств, например в
    $\beta$ имеется элемент $r \in \Q$, не принадлежащий $\alpha$,
    тогда по определению имеем $\alpha < \beta$. Аналогично для $\beta < \alpha$.
    Следовательно, в случае $\alpha \neq \beta$ обязательно выполнится только
    одно условие $\alpha < \beta$ или $\beta < \alpha$. Теорема доказана.
\end{proof}

\begin{theorem}
    Теорема о трех сечениях. Пусть $\alpha, \beta, \gamma$ -- сечения.
    Если $\alpha < \beta \wedge \beta < \gamma$, то $\alpha < \gamma$.
\end{theorem}
\begin{proof}
    $(\alpha < \beta) \Rightarrow (\exists p \in \Q | p \in \beta, p \notin \alpha)$;
    $(\beta < \gamma) \Rightarrow (\exists q \in \Q | q \in \gamma, q \notin \beta)$.
    Далее, $(p \in \beta, q \notin \beta)$ $\Rightarrow$ по утверждению из прошлой лекции
    $p < q$. Поскольку $p \notin \alpha$, то тогда и $q \notin \alpha$,
    в противоположном случае по свойству 2 в определении сечения было бы и $p \in \alpha$.
    Таким образом, $q \in \gamma, q \notin \alpha$, т.е. $\alpha < \gamma$.
    Теорема доказана.
\end{proof}

\begin{definition}
    Сумма вещественных чисел = сумма сечений.
\end{definition}
\begin{theorem}
    Пусть $\alpha$ и $\beta$ -- сечения, $\gamma$ -- множество рациональных чисел
    $r$, т.ч. $r = p + q$, где $p \in \alpha$ - произвольное число,
    $q \in \beta$ - произвольное число. Тогда $\gamma$ -- сечение.
\end{theorem}
\begin{proof}
    Поскольку $\alpha \neq \varnothing$, $\beta \neq \varnothing$, то
    $\gamma \neq \varnothing$. Поскольку $\alpha \neq \Q, \beta \neq \Q$, то
    $\exists s \in \Q, s \notin \alpha$ и $\exists t \in \Q, t \notin \beta$.
    Пусть $p \in \alpha$, $q \in \beta$.
    По удтверждению из прошлой лекции $(p \in \alpha, s \notin \alpha) \Rightarrow
        (p < s); (q \in \beta, t \notin \beta) \Rightarrow (q < t)$.
    Отсюда следует, что $p + q < s + t$ $\forall p \in \alpha \wedge
        \forall q \in \beta$, т.ч. $\forall r \in \gamma$ выполнено $r < s + t$, т.е.
    $s + t \notin \gamma$, т.е. $\gamma \neq \Q$ -- проверен п.1 в определении
    сечения.

    Пусть $r \in \gamma$, $s < r$. Тогда $r = p + q, p \in \alpha, q \in \beta$.
    Пусть $t = s - q$, тогда $t < r - q = (p + q) - q = p$, из $p \in \alpha$ и
    $t < \alpha$ следует $t \in \alpha$, т.е. $s = t + q, t \in \alpha,
        q \in \beta$, т.е. $s \in \gamma$ -- проверен п.2 в определении сечения.

    Пусть $r \in \gamma, r = p + q, p \in \alpha, q \in \beta$. По п.3
    определения сечения $\exists p_1 \in \alpha, p_1 > p$, тогда
    $r_1 = p_1 + q > p + q = r$, в $\gamma$ нет наибольшего элемента, проверен
    п.3 определения сечения.

    Теорема доказана.
\end{proof}

\begin{definition}
    Сечение $\gamma$, построенное в предыдущей теореме, называется суммой
    сечений $\alpha$ и $\beta$.

    Поскольку вещественные числа определены как сечения, то вещественное число
    $\gamma$ называют суммой вещественных чисед $\alpha$ и $\beta$, пишут
    $\gamma = \alpha + \beta$.
\end{definition}

\paragraph{Свойства сложения}
\begin{theorem}
    Пусть $\alpha, \beta, \gamma$ -- вещественные числа. Тогда:
    \begin{enumerate}
        \item $\alpha + \beta = \beta + \alpha$
        \item $(\alpha + \beta) + \gamma = \alpha + (\beta + \gamma)$
        \item $\alpha + 0^* = \alpha$
    \end{enumerate}
\end{theorem}
\begin{proof}
    Пункты 1 и 2 следуют из определения сложения и свойств сложения
    рациональных чисел. Докажем п.3.

    Пусть $r \in \alpha + 0^*$, тогда $r = p + q$, $p \in \alpha, q \in 0^*$,
    т.е. $q < 0$, поэтому $r = p + q < p$, тогда $r \in \alpha$ по условию 2
    определения сечений, т.ч. $\alpha + 0^* \subset \alpha$, если мы делаем
    акцент на том, что $\alpha + 0^*$ и $\alpha$ -- множества. Пусть теперь
    $t \in \alpha$. Выберем $s > t$, но $s \in \alpha$, что возможно по п.3
    определения сечений. Полагаем $q_0 = t - s$, тогда $t - s < 0 \Rightarrow
        t - s \in 0^*, t = s + (t - s) \in \alpha + 0^*$, т.е. $\alpha \subset
        \alpha + 0^*$, тогда $\alpha = \alpha + 0^*$. Теорема доказана.
\end{proof}

\begin{theorem}
    Теорема о разности верхних и нижних чисел сечения. Пусть $\alpha$ -- сечение,
    и пусть $r \in \Q, r > 0$. Тогда $\exists p \in \Q, \exists q \in \Q$,
    такие что $p \in \alpha, q \notin \alpha$, $q$ не является наименьшим из
    верхних чисел $\alpha$ и $q - p = r$.
\end{theorem}
\begin{proof}
    Возьмем $s \in \alpha$, и пусть $s_n = s + nr, s_0 = s, n = 0, 1, \ldots$.
    Найдется $m_0$, т.ч. $s_{m_0} \notin \alpha$: если бы $s_n \in \alpha
        \forall n \in \N$, то возьмем $\forall t \in \Q, t > s$. По свойствам
    рациональных чисел $\exists n_0$ т.ч. $s = n_0r > t$, и тогда
    $s_{n_0} \in \alpha \Rightarrow t \in \alpha$, т.е. $\alpha = \Q$ в силу
    произвольности $t$, что противоречит условию 1.

    Таким образом, $\exists m_0 \in \N$, т.ч. $s_{m_0} \notin \alpha$.
    Поскольку $s_0 \in \alpha$, то имеется максимальное $m \in \N$, т.ч.
    $s_m \in \alpha, m < m_0$, тогда $s_{m + 1} \notin \alpha$.
    Если $s_{m + 1}$ не является минимальным из верхних чисел сечения,
    то полагаем $p = s_m, q = s_{m+1}$, тогда $q - p = s_{m + 1} - s_m =
        (s + (m + 1)r) - (s + mr) = r$. Если же $s_{m + 1}$ является наименьшим
    из верхних чисел сечения, то пусть $p = s_m + \frac{r}{2}$,
    $q = s_{m + 1} + \frac{r}{2}$, $q - p = r$, $q > s_{m + 1} \Rightarrow
        q \notin \alpha$, $s_{m + 1}$ -- наименьшее из верхних чисел $\alpha$ и
    $p = s_m + \frac{r}{2} = s + mr + \frac{r}{2} < s + (m + 1)r$, поэтому
    $p \in \alpha$. Теорема доказана.
\end{proof}

\paragraph{Существование противоположного числа}
\begin{theorem}
    Пусть $\alpha$ - вещественное число. Тогда существует единственное число
    $\beta$ такое, что $\alpha + \beta = 0^*$
\end{theorem}
\begin{proof}
    Вначале докажем единственность $\beta$. Предположим, что $\exists \beta_0$ т.ч.
    $\alpha + \beta_0 = 0^*$. Тогда, по теореме о свойствах сложения имеем
    \[
        \beta_0 = 0^* + \beta_0 = (\alpha + \beta) + \beta_0 = (\beta + \alpha) +
        \beta_0 = \beta + (\alpha + \beta_0) = \beta + 0^* = \beta
    \]
    т.е. $\beta$ - единственный, если существует.

    Найдем теперь какое-то $\beta$, т.ч. $\alpha + \beta = 0^*$. Пусть $\beta$ --
    множество всех рациональных чисел таких, что $-p$ является верхним числом
    $\alpha$, но не наименьшим из верхним чисел.

    Проверим, что $\beta$ -- сечение (= вещественное число). Взяв любое верхнее не
    наименьшее число $t$ сечения $\alpha$, полагая $p = -t$, имеем $p \in \beta$,
    т.е. $\beta \neq \emptyset$. Взяв любое $s \in \alpha$, получаем, что
    $-s \notin \beta$, т.к. $-(-s) = s \in \alpha$, $s$ - нижнее число $\alpha$,
    т.е. $\beta \neq \Q$ -- проверено условие 1.

    Если $p \in \beta$, $q \in \Q$ и $q < p$, то $-q > -p$, $-p$ -- верхнее число
    $\alpha \Rightarrow -q$ -- верхнее число $\alpha$ и $-q$ -- не наименьшее
    верхнее в $\alpha$, т.е. $q \in \beta$ -- проверено условие 2.

    Если $p \in \beta$, то $-p$ -- врехнее число $\alpha$ и $\exists$ верхнее число
    $\alpha$, обозначим его $-q$, т.ч. $-q < -p$; пусть $-z=^{def}-\frac{q + p}{2}$,
    тогда $-z > -q$, т.е. $-z$ -- верхнее число в $\alpha$ и не наименьшее, поэтому
    $z \in \beta$. Поскольку $-z < -p$, то $z > p$, в $\beta$ нет наибольшего --
    проверено условие 3. Таким образом $\beta$ -- сечение.

    \underline{Проверка свойства $\alpha + \beta = 0^*$}
    Пусть $p \in \alpha + \beta$, тогда $p = q + z, q \in \alpha, z \in \beta$;
    $z \in \beta \Rightarrow -z \notin \alpha$, тогда $q \in \alpha \Rightarrow
        q < -z, q + z < 0, p < 0, p \in 0^*$, т.е. $\alpha + \beta \subset 0^*$,
    если трактовать $\alpha, \beta, 0^*$ как множества.

    Пусть $p \in 0^*$, тогда $p < 0$. По теореме о разности верхних и нижних
    чисел сечения $\exists q \in \alpha, s \notin \alpha, s$ не является наименьшим
    верхним числом $\alpha$, т.ч. $s - q = -p$. Поскольку $-s \in \beta$, то тогда
    $p = q - s = q + (-s) \in \alpha + \beta$, т.е. $0^* \subset \alpha + \beta$;
    в итоге $0^* = \alpha + \beta$, теорема доказана.
\end{proof}

\begin{definition}
    Вещественное число $\beta$, построенное в предыдущей теореме обозначается
    $-\alpha$, и называется числом, противоположным $\alpha$.
\end{definition}

\begin{assertion}
    О сохранении неравенства. Пусть $\beta < \gamma$, тогда $\alpha + \beta <
        \alpha + \gamma$. В частности, если $0^* < \gamma, 0^* < \alpha$, то
    $(\alpha = 0^* + \alpha < \alpha + \gamma, 0^* < \alpha) \Rightarrow
        0^* < \alpha + \gamma$.
\end{assertion}
\begin{proof}
    Из определения сложения вещественных чисел следует, что $\alpha + \beta \leq
        \alpha + \gamma$. Если было бы $\alpha + \beta = \alpha + \gamma$, то тогда
    \[\beta = 0^* + \beta = ((-\alpha) + \alpha) + \gamma = 0^* + \gamma = \gamma\],
    что противоречит условию. Утверждение доказано.
\end{proof}

\paragraph{Определение разности вещественных чисел}
\begin{theorem}
    Пусть $\alpha, \beta$ -- вещественные числа. тогда существует единственное
    вещественное число $\gamma | \alpha + \beta = \gamma$.
\end{theorem}
\begin{proof}
    Полагаем $\gamma = \beta + (-\alpha)$. Тогда $\alpha + \gamma = \alpha +
        (\beta + (-\alpha)) = \alpha + ((-\alpha) + \beta) = (\alpha + (-\alpha)) +
        \beta = 0^* + \beta = \beta$.

    Если бы существовало $\gamma_1 | \alpha + \gamma_1 = \beta$, то если бы
    $\gamma \neq \gamma_1$, то тогда либо $\gamma < \gamma_1$, либо $\gamma_1 <
        \gamma$. Не уменьшая общности, считаем $\gamma < \gamma_1$. Тогда по удтверждению
    о сохранении неравенства мы получаем $\alpha + \gamma < \alpha + \gamma_1$,
    но $\alpha + \gamma = \beta, \alpha + \gamma_1 = \beta$, противоречие.

    Итак, вещественное число $\gamma$ одно. Оно называется разностью $\beta$ и
    $\alpha$, $\gamma = \beta - \alpha$.
\end{proof}

\begin{definition}
    $|\alpha|$. Полагаем
    \begin{equation*}
        |\alpha| =
        \begin{cases}
            \alpha,  & \alpha \geq 0^* \\
            -\alpha, & \alpha < 0^*
        \end{cases}
    \end{equation*}
\end{definition}

\begin{assertion}
    $|\alpha| \geq 0^* \forall \alpha \in \R$
\end{assertion}
\begin{proof}
    Если $\alpha \geq 0^*$, это следует из определения $|\alpha|$. Пусть
    $\alpha < 0^*$, тогда $\alpha \neq 0^*$ и, если неверно, что $\alpha > 0^*$.
    то $-\alpha < 0^*$. По удтверждению о сохранении неравенства тогда бы
    выполнялось $\alpha + (-\alpha) < \alpha + 0^* = \alpha$, но $\alpha < 0^*$,
    тогда $\alpha + (-\alpha) < 0^*, 0^* < 0^*$, что невозможно. Итак
    $|\alpha| \geq 0^*$. Из определения видно, что $|\alpha| = 0^* \Leftrightarrow
        \alpha = 0^*$. Удтверждение доказано.
\end{proof}

\begin{theorem}
    $p^* < \alpha, p \in \Q. p^* < \alpha \Leftrightarrow p \in \alpha, p \in \Q$
\end{theorem}
\begin{proof}
    Пусть $p \in \alpha; p \notin p^* \Rightarrow p^* < \alpha$. Пусть теперь
    $p^* < \alpha$, тогада $\exists q \in \Q | q \notin p^*$, т.е. $q \geq p$,
    и $q \in \alpha$. Тогда $p \in \alpha$. Теорема доказана.
\end{proof}

\section{Произведение вещественных чисел}
\begin{theorem}
    Пусть $\alpha$, $\alpha \geq 0^*$, и $\beta$, $\beta \geq 0^*$ - вещественные 
    числа. Обозначим через $\gamma$ следующее множество рациональных чисел:
    если $p \in \Q, p < 0$, то $p \in \gamma$; если $p = st, s \in \alpha, 
    t \in \beta$ и $s \geq 0, t \geq 0$, то $p \in \gamma$. Другие рациональные
    числа в множество $\gamma$ не входят. Если $\alpha = 0^* \vee \beta = 0^*$,
    то $\gamma$ по определению состоит только из чисел $p \in \Q, p < 0$. 
    Тогда $\gamma$ - сечение.  
\end{theorem}
\begin{proof}
    Поскольку $(\forall p \in \Q, p < 0) \Rightarrow p \in \gamma$, то $\gamma$
    непусто; если $\alpha = 0^* \vee \beta = 0^*$, то $(\forall q > 0) 
    \Rightarrow q \notin \gamma$, в этом случае $\gamma \neq \Q$; если 
    $\alpha > 0^* \wedge \beta > 0^*$, то пусть 
    $u \notin \alpha, v \notin \beta$, тогда $u > 0 \wedge v > 0$ и 
    $(\forall s \in \alpha, s \geq 0 \wedge \forall t \in \beta, t \geq 0) 
    \Rightarrow (s < u \wedge t < v) \Rightarrow st < uv$, т.е. 
    $uv \notin \gamma$, т.е. всегда $\gamma \neq \Q$.  
    Если $\alpha = 0^* \vee \beta = 0^*$, то из определения $\gamma$ в этом
    случае следуют условия 2 и 3; если $\alpha > 0^* \wedge \beta > 0^*$, то 
    пусть $p \in \gamma, p > 0, 0 \leq q < p$; Пусть $p = st, s > 0, t > 0, 
    s \in \alpha, t \in \beta$. Если $q = 0$, то $0 = 0 * t, 0 \in \alpha,
    0 \in \gamma$; если $q > 0$, то $\frac{q}{p} \cdot t < t$, поэтому
    $\frac{q}{p} \cdot t \in \beta$, тогда $s \cdot \frac{q}{p} \cdot t \in \gamma$,
    но $s \cdot \frac{q}{p} \cdot t = \frac{q}{p}st = \frac{q}{p} \cdot p = q$, т.е.
    $q \in \gamma$; если $p = st, s > 0, t > 0$, то возьмем $s_1 > s, s_1 \in 
    \alpha$, тогда $s_1t \in \gamma и s_1t > p$. 
\end{proof}

\begin{definition}
    Пусть $\alpha, \beta \in \R$. Полагаем
    \begin{equation*}
        \alpha\beta \stackrel{def}{=} 
        \begin{cases}
            -(|\alpha||\beta|) & \alpha < 0^*, \beta \geq 0^* \\
            -(|\alpha||\beta|) & \alpha \geq 0^*, \beta < 0^* \\
            (|\alpha||\beta|) & \alpha < 0^*, \beta < 0^*
        \end{cases}
    \end{equation*}
\end{definition}
\begin{theorem}
    Справедливы следующие свойства: 
    \begin{enumerate}
        \item $\alpha\beta = \beta\alpha$; 
        \item  $(\alpha\beta)\gamma = \alpha(\beta\gamma)$;
        \item $\alpha(\beta + \gamma) = \alpha\beta + \alpha\gamma$;
        \item $\alpha \cdot 0^* = 0^*$;
        \item $\alpha\beta = 0^* \Leftrightarrow \alpha = 0^* \vee \beta = 0^*$;
        \item $\alpha \cdot 1^* = \alpha$;
        \item $\alpha < \beta, \gamma > 0^* \Rightarrow \alpha\gamma < \beta\gamma$
    \end{enumerate}
\end{theorem}
\begin{proof}
    Следуем из определения суммы и произведения и из соотвествующих свойств
    рациональных чисел.
\end{proof}

\begin{theorem}
    Если $\alpha \neq 0^*$, то $\forall \beta \in \R \exists! : 
    \alpha\gamma = \beta$.
\end{theorem}
\begin{proof}
    Аналогично доказательству существования и единственности $-\alpha$ и 
    $\beta - \alpha$; в данном случае вначале проверяем, что $\exists \delta :
    \alpha\delta = 1^*$, затем полагают $\gamma = \beta\delta$.
\end{proof}

\begin{designation}
    $\gamma = \frac{\beta}{\alpha}, \delta = \frac{1}{\alpha}, \gamma 
    \text{ - частное вещественных чисел} \beta, \alpha$, $\delta$ - обратное к 
    $\alpha$ число. Отметим также связь действий над сечениями и рациональными 
    числами. 
\end{designation}
\begin{theorem}
    Пусть $p, q \in \Q$. Тогда $p^* + q^* = (p + q)^*, (pq)^* = p^*q^*, 
    p^* < q^* \Leftrightarrow p < q$.
\end{theorem}
\begin{proof}
    Доказывается аналогично предыдущим теоремам.
\end{proof}

\begin{theorem}
    О плотности рациональных сечений в $\R$.
    Пусть $\alpha, \beta \in \R, \alpha < \beta$. Тогда $\exists r^*, r \in \Q :
    \alpha < r^* < \beta$.
\end{theorem}
\begin{proof}
    $(\alpha < \beta) \Rightarrow \exists p \in \Q : p \in \beta, 
    p \notin \alpha$. Выберем $r > p, r \in \beta$. Тогда в силу $r \notin r^*$
    имеем $r^* < \beta$; поскольку $p \notin \alpha, p < r$, то $p \in r^*$, 
    поэтому $\alpha < r^*$.
\end{proof}

\section{Теоема Дедекинда, супремумы и инфимумы.}
\begin{definition}
    Будем говорить, что в множестве вещественных чисел $\R$ определено сечение,
    если имеются множества $A \subset \R \wedge B \subset \R$ со следующими
    свойствами.
    \begin{enumerate}
        \item $A \neq 0^*, B \neq 0^*, A \neq \R, B \neq \R$
        \item $A \cup B = \R$
        \item $A \cap B = \emptyset$
        \item Если $\alpha \in A, \beta \in B$, то $\alpha < \beta$
    \end{enumerate}

    При этом множество $A$ называется нижним классом сечения, и числа 
    $\alpha \in A$ называются нижними числами, а множество $B$ называется 
    верхним классом сечения, и числа $\beta \in B$ называются верхними числами
    сечения.
\end{definition}
\begin{theorem}
    Теорема Дедекинда. Пусть имееется сечение $(A, B)$ множества $\R$. Тогда 
    существует единственное число $\gamma : \alpha \leq \gamma \forall
    \alpha \in A \wedge \gamma \leq \beta \forall \beta \in B$. При этом
    реализуется только одна возможность: либо в $A$ и $\gamma \in A$ и является
    максимальным числом, либо $\gamma \in B$ и $\gamma$ является минимальным
    числом в $B$.   
\end{theorem}
\begin{proof}
    Прежде всего проверим, что $(\gamma \in A) \Rightarrow \gamma$ -
    максимальное число в $A$, в $B$ нет минимального или $(\gamma \in B)
    \Rightarrow \gamma$ - минимальное в $B$, в $A$ нет максимального.
    Проверим первое из этих утверждений, второе доказывается аналогично.
    Мы пока предполагаем, что число (или числа) $\gamma$, удволетворяющие
    заключению теоремы, существуют. Предположим, что, наряду с $\gamma \in A$,
    $\gamma$ - максимальное в $A$, $\exists \gamma_1 \in \beta$, $\gamma_1$ - 
    минимальное число в $B$. Тогда условие 4 $\Rightarrow$ $\gamma < \gamma_1$.
    По теореме о плотности рациональных сечений $\exists r^* : \gamma < r^* <
    \gamma_1$. Тогда по предположению о минимальности $\gamma_1$ в $B$ имеем
    $r^* \notin B$, но в силу условий 2 и 3 тогда получаем, что $r^* \in
    \R \setminus B = A$, т.к. $\R \setminus B = A$. В силу преположения о
    максимальности $\gamma$ в $A$ из $r^* > \gamma \Rightarrow r^* \notin A$,
    т.е. $r^* \in \R \setminus A = B$, т.е. $r^* \in A \cap B$, что 
    противоречит условию 3 сечения. \\
    Таким образом,  $\gamma_1$ не существует и в $B$ нет 
    наименьшего числа. \\
    Докажем теперь, что $\gamma$ существует. (Оперируем с сечениями 
    рациональных чисел, другой трактовки у нас пока нет).
    Полагаем $\gamma \overset{def}{=}$ \{множество всех рациональных чисел
    $p$ т.ч. для какого-то $\alpha \in A, p \in \alpha$\}. По-другому это 
    определение можно записать так: 
    \begin{equation*}
        \gamma = \bigcap_{\alpha \in A}\alpha
    \end{equation*}
    Поскольку $\alpha$ - множество, и мы рассматриваем объединение указанных
    объектов. Проверим, что $\gamma$ - сечение $\Q$. 
    Поскольку $A \neq \emptyset$, тогда $\exists p \in \alpha, p \in \gamma$,
    т.е. $\gamma \neq \emptyset$. Далее $B \neq \emptyset$, поэтому 
    $\exists \beta \in B, \beta \notin A \wedge \beta > \alpha \forall
    \alpha \in A$ по условию 4 сечения $\R$. Возьмем $q \in \Q$, $q \notin 
    \beta$, тогда $q \notin \alpha$, если $q \in A$, поскольку в противном 
    случае, если бы $q \in \alpha$, то $q^* < \alpha, \alpha < \beta, 
    q^* < \beta$, но $q \notin \beta \Rightarrow q^* \geq \beta$ - противоречие.
    Таким образом, $q \notin \alpha \forall \alpha \in A$, т.е. $q \notin 
    \gamma, \gamma \neq \Q$ - проверено условие 1 сечения $\Q$.
    Если $p \in \gamma \wedge q < p$, то по определению $\gamma$ $\exists 
    \alpha \in A : p \in \alpha$, но тогда и $q \in \alpha$, т.е. $q \in \gamma$ -
    проверено условие 2 сечения $\Q$.
    Если $p \in \gamma$, то $\exists \alpha \in A: p \in \alpha$, тогда 
    $\exists q > p, q \in \alpha$, т.е. $q \in \gamma$ - проверено условие 3
    сечения $\Q$.
    Таким образом, $\gamma$ - сечение $\Q$.
    Из определения $\gamma$ следует, что $\alpha \subset \gamma \forall
    \alpha \in A$, т.е. $\alpha \leq \gamma \forall \alpha \in A$.
    Докажем, что $\forall \beta \in B \beta \geq \gamma$.
    Предположим, что это не так, тогда $\exists \beta_0 \in B : \beta_0 < 
    \gamma$. Тогда $\exists p \in \Q : p \in \gamma$, но $p \notin \beta_0$.
    Но если $p \in \gamma$, то $\exists \alpha_0 \in A : p \in \alpha_0$, что 
    влечёт $\beta_0 < \alpha_0$, а это противоречит условию 4 сечения $\R$.
\end{proof}

\begin{definition}
    Пусть $E \subset \R, E \neq \emptyset$. Множество $E$ называется 
    ограниченным сверху, если $\exists b \in \R : \forall a \in E$ выполнено
    $a \leq b$; число $b$ называют верхней границей множества $E$; множество
    $E$ называют ограниченным снизу, если $\exists c \in \R : \forall 
    a \in E$, выполнено $a \geq c$; число $c$ называют нижней границей 
    множества $E$. \\
    Множество $E$ называют ограниченным, если оно ограниченно 
    и снизу, и сверху. \\
    Число $b_0$ называют точной верхней границей $E$ или супремумом $E$,
    обозначается $b = sup E$, если $b_0$ - верхняя граница $E$ и 
    $\forall b_1 < b_0$ $b_1$ не является верхней границей $E$. \\
    Число $c_0$ называется точной нижней границей $E$ или инфимумом $E$, 
    обозначается $c_0 = inf E$, если $c_0$ - нижняя граница $E$ и 
    $\forall c_1 > c_0$ $c_1$ не является нижней границей $E$. 
\end{definition}

\begin{theorem}
    О существовании супремума и инфимума.
    Пусть $E \subset \R, E \neq \emptyset$, $E$ - ограничено сверху.
    Тогда $\exists sup E$.
    Пусть $E \subset \R, E \neq \emptyset$, $E$ - ограничено снизу.
    Тогда $\exists inf E$.
\end{theorem}
\begin{proof}
    Докажем существование супремума, существование инфимума доказывается 
    аналогично. Определим сечение $(A, B)$ в $\R$ следующим образом: 
    множество $A$ состоит из всех $\alpha \in \R : \exists x \in E \wedge
    x > \alpha$, $B = \R \setminus A$. 
    Из определения $A$ следует, что $\forall \alpha \in A$ число $\alpha$
    не является верхней границей $E$. Поскольку $A$ ограничено сверху, то 
    $A \neq \R$, а $E \neq \emptyset \Rightarrow A \neq \emptyset$, т.к. 
    если $x_0 \in E$, то $x_0 - 1^* < x_0, x_0 - 1^* \in A$. \\
    Если $\beta \in B$, то, по определению $A$, $\nexists x \in E : x > \beta$,
    ибо иначе было бы $\beta \in A$, т.е. $\forall x \in E$ имеем 
    $x \leq \beta$, т.е. $\beta$ - верхняя граница $E$ для $\forall \beta \in B$.
    Условия 1-3 сечения $\R$ для $(A, B)$ проверены. \\
    Проверим условие 4. Пусть $\alpha \in A, \beta \in B$. Тогда 
    $\exists x_0 \in E : x_0 > \alpha$, но $\beta \geq x_0$, т.е. 
    $\beta > \alpha$ - условие 4 проверено. \\
    Пусть $\gamma \in \R$ - число, определяемое сечением $(A, B)$ по теореме
    Дедекинда. Проверим, что $\gamma \notin A$. Если бы $\gamma \in A$, то 
    $\exists x_1 \in E : x_1 > \gamma$. Выберем $r^* : \gamma < r^* < x_1$,
    тогда $r^* \in A$ по определению $A$, $r^* > \gamma$, т.е. $\gamma$ - 
    не наибольшее число в $A$, что противоречит теореме Дедекинда. 
    Следовательно, $\gamma \in B$, т.е. $\gamma$ - верхняя граница $E$ и 
    $\gamma$ - наименьшее число в $B$ по теореме Дедекинда, т.е. 
    $\forall \gamma_1 < \gamma$, $\gamma_1$ не является элементом $B$, т.е. 
    $\gamma_1 \in A$, тогда $\exists x_2 \in E : x_2 > x_1$, т.е. $\gamma_1$ -
    не верхняя граница $E$. \\
    Таким образом $\gamma$ - точная верхняя граница $E$.
\end{proof}


\section{Десятичные дроби, определение корня, степени,
экспонент и логарифма}
\begin{definition}
    Пусть $E \neq \emptyset, G \neq \emptyset$ - множества. Отображением
    $F: E \rightarrow G$ будем называть правило, по котором $\forall x \in E$
    сопоставляется $F(x) \in G$. Оторбражение задается набором $E, G, F$.
    Далее в качестве синонима слова "отображение" будем использовать слово
    "функция" или "$G$-значная функция, определенная на $E$".
    Отображение $F$ называется инъективным или инъекцией, если из $x, y \in E,
    x \neq y \Rightarrow F(x) \neq F(y)$;
    отображение $F$ называется сюръективным или сюръекцией, если $\forall q 
    \in G \exists x \in E : F(x) = q$;
    отображение $F$ называется биективным или биекцией, если оно одновременно
    инъективно и сюръективно. Мы будем обозначать биекцию так: 
    $F: E \longleftrightarrow G$.
\end{definition}
\begin{definition}
    Множество $X$ называется счетным, если существует биекция 
    $F: \N \longleftrightarrow X$, $\N$ - множество натуральных чисел.
    Исторически счетные множества обозначаются так: $F(n) = x_n, x_n \in X,
    n \in \N$. Последовательностью будем называть любое отображение 
    $f: \N \rightarrow Y$, при этом обычно ее обозначают $y_1, y_2, \ldots
    y_n$, или $\{y_n\}_{n-1}^{\infty}$. Говорят при этом, что определена 
    последовательность множества $Y$.
\end{definition}

\begin{theorem}
    Объединение конечного или счетного множества попарно дизъюнктных 
    счетных множеств счетно.
\end{theorem}
\begin{proof}
    Пусть $X_k = \left\{a_{k1}, a_{k2}, \ldots\right\}$, где $k = 1,\ldots, n$
    или $k = 1,2, \ldots$; укажем способ нумерации $\bigcup_{k=1}^nX_k$ или 
    $\bigcup_{k \in \N}X_k$. Учтем, что если $k \neq 1$, то $a_{ki} \neq a_{kj}$,
    а если $k = 1$, но $i \neq j$, то $a_{ki} \neq a_{kj}$. В случае
    $k = 1,\ldots, n$ запишем элементы $a_{kj}$ в таблицу:
    \begin{center}
        \begin{tabular}{l l l l l}
            $a_{11}$ & $a_{12}$ & $\ldots$ & $a_{1s}$ & $\ldots$ \\
            $a_{21}$ & $a_{22}$ & $\ldots$ & $a_{2s}$ & $\ldots$ \\
            $\ldots$ & $\ldots$ & $\ldots$ & $\ldots$ & $\ldots$ \\
            $a_{n1}$ & $a_{n2}$ & $\ldots$ & $a_{ns}$ & $\ldots$ \\
        \end{tabular}
    \end{center}
    и будем "перенумеровывать" получившееся объединение $\bigcup_{k=1}^nX_k$
    змейкой: $a_{11} \rightarrow a_{21} \rightarrow \ldots
    \rightarrow a_{n1} \rightarrow a_{n2} \rightarrow a_{n-1 2} 
    \rightarrow \ldots$. \\
    В случае $\bigcup_{k \in \N}X_k$ укажем перенумерацию: $a_{11} \rightarrow
    a_{21} \rightarrow a_{12} \rightarrow a_{13} \rightarrow a_{22} 
    \rightarrow a_{31} \rightarrow \ldots$. \\
    При этом в каждом из двух случаев получаем перенумерацию всего объединения.
\end{proof}

\begin{assertion}
    Пусть $X$ - счетное множество, $Y \subset X$, $Y$ - бесконечное множество
    (т.е. в нем не конечное множество элементов). Тогда $Y$ - счетное.
\end{assertion}
\begin{proof}
    Пусть $X = \left\{x_1, x_2, \ldots\right\} \forall y \in Y
    \exists n: y = x_n$.
    Пусть $k \overset{def}{=} \left\{n \in \N: \exists y \in Y: y = x_n\right\}$.
    Пусть $n_1$ - минимальное число в множества $K$, тогда $\exists y \in Y:
    y = x_{n_1}$, положим $1 \longleftrightarrow y$, т.е. присвоим $y$ номер $1$,
    считаем его $y_1$. Пусть $K_1 = K \setminus {n_1}$, тогда $K \neq \emptyset$,
    т.к. $K$ бесконечное множество, пусть $n_2 \in K_1$ - минимальное число;
    $\exists y_2 \in Y : y_2 = x_{n_2}$, положим $K_2 = K_1 \setminus {n_2}$,
    и т.д. Таким образом мы построим биекцию $\N \longleftrightarrow Y$.
\end{proof}
\begin{corollary}
    Пусть $Y = \bigcup_{k = 1}^nX_k$ или $Y = \bigcup_{k \in \N}X_k$, где 
    $X_k$ - счетны, но необязательно дизъюнктны. Тогда $Y$ счетно.
\end{corollary}
\begin{proof}
    Слудует из теоремы и утверждения.
\end{proof}

\subsection{Сопоставление вещественным числам десятичных разложений}
Будем рассматривать только $\alpha > 0^*$, случай $\alpha < 0^*$ получается
добавлением знака "$-$". \\
Прежде всего биективно сопоставляем любому натуральному числу $n$ сечения 
$n^*$, для $n$ считаем известным его десятичное представление. Для 
$n^* < x < (n+1)*, n \geq 0, n \in \Z$, положим $\alpha = x - n^*$, тогда
$0^* < \alpha < 1^*$. Определим
\begin{multline*}
    E \overset{def}{=} 
    \left\{ 
        \left( 
            \frac{a_1}{10} + \frac{a_2}{10^2} + \ldots + \frac{a_k}{10^k}
            \right)^* : 0 \leq a_j \leq 9, 1 \leq j \leq k, k \geq 1, \right. \\
        \left. \left(
            \frac{a_1}{10} + \frac{a_2}{10^2} + \ldots + \frac{a_k}{10^k}
        \right)^* \leq \alpha
    \right\}
\end{multline*}

\begin{theorem}
    $\alpha = supE$
\end{theorem}
\begin{proof}
    Определение $E$ показывает, что множество $E \neq \emptyset$, т.к. 
    $0^* \in E$, и ограничено сверху, поэтому $\exists supE \overset{def}{=} 
    \beta, \beta \leq \alpha$, поскольку $\alpha$ - верхняя граница $E$.
    Предположим, что $\alpha > \beta$. Тогда $\exists r_1 \in \Q : \beta <
    r_1^* < \alpha \wedge \exists r_2 \in \Q : r_1^* < r_2^* < \alpha$.
    Пусть $t = r_2^* - r_1^*$. Выберем $k_0 \in \N$ так, чтобы $k^* > \frac{1}{t}$,
    тогда $\left((10^{k_0})^* > k_0^* > \frac{1}{t}\right) \Leftrightarrow 
    \left(\frac{1}{10^{k_0}}\right)^* < \left(\frac{1}{k_0}\right)^* < t$.
    По определению числа $\beta$ $\exists q = \frac{a_1}{10} + \frac{a_2}{10^2} +
    \ldots + \frac{a_{k_1}}{10^{k_1}}$, т.ч. $q^* > \beta - \frac{1}{10^{k_0}}
    \Leftrightarrow \beta < q^* + \frac{1}{10^{k_0}}$. Считаем, что $k_1 \geq k_0$,
    если это не так, то полагаем $a_{k_1 + 1} = \ldots = a_{k_0} = 0$, что 
    даёт возможность этому предположению. Считаем также, что $k_0$ выбрано так,
    чтобы $\alpha + \left(\frac{1}{10^{k_0}}\right) < 1$. \\
    Если бы выполнялось соотношение $q^* + \left(\frac{1}{10^{k_0}}\right)^* > 
    \alpha, q^* \leq \beta$, что влечёт неравенство
    \begin{equation*}
        \left(q^* + \left(\frac{1}{10^{k_0}}\right)^*\right) - q^* > \alpha 
        -\beta > r_2^* - r_1^* = t
    \end{equation*}
    т.е. $\left(\frac{1}{10^{k_0}}\right)^* > t$, что противоречит выбору $k_0$,
    т.е. $\alpha = \beta$.
\end{proof}

Пользуясь теоремой, сопоставим $\alpha, 0 < \alpha < 1$, последовательность
$\left\{a_k\right\}_{k = 1}^{\infty}, a_k \in \{0, 1, \ldots, 9\}$

\begin{designation}
    В дальнейшем полагаем 
    \begin{gather*}
        [a, b] \overset{def}{=} {x \in \R : a \leq x \leq b} \\
        (a, b) \overset{def}{=} {x \in \R: a < x < b} \\
        [a, b) \overset{def}{=} {x \in \R: a \leq x < b} \\
        (a, b] \overset{def}{=} {x \in \R: a < x \leq b} 
    \end{gather*}
\end{designation}

Новым по сравнению со школой является исопльзование построенных по Дедекинду
вещественных чисел.

Определим теперь числа $a_k(\alpha) \in \{0, 1, \ldots, 9\}$ и $P_k(\alpha) 
\in \Q$. Из $0 < \alpha < 1$ следует, что $\exists$ единственный промежуток
вида $\left[0, \frac{1}{10}\right)^*, \left[\frac{1}{10}, \frac{2}{10}\right)^*,
\ldots \left[\frac{9}{10}, 1\right)^*$, которому принадлежит $\alpha$. 
Пусть $\alpha \in \left[ \left(\frac{a_1(\alpha)}{10}\right)^*, 
\left(\frac{a_1(\alpha) + 1}{10}\right)^* \right)$, что дает $a_1(\alpha)$ и
пусть $P_1(\alpha) \overset{def}{=} \frac{a_1(\alpha)}{10}$. Далее $\exists$
единственный промежуток $\left[ \frac{a_1(\alpha)}{10} + 
\frac{a_2(\alpha)}{10^2}, \frac{a_1(\alpha)}{10} + \frac{a_2(\alpha) + 1}{10^2}
\right)$, т.ч. $\alpha \in \left[ \left(\frac{a_1(\alpha)}{10}\right)^* + 
\left(\frac{a_2(\alpha)}{10^2}\right)^*, \left(\frac{a_1(\alpha)}{10}\right)^* + 
\left(\frac{a_2(\alpha) + 1}{10^2}\right)^*
\right)$, что задает $a_2(\alpha)$, и полагаем $p_2(\alpha) = 
\frac{a_1(\alpha)}{10} + \frac{a_2(\alpha)}{10^2}$. Если уже определили 
$a_1(\alpha), \ldots, a_k(\alpha)$, и 
$P_k(\alpha) = \frac{a_1(\alpha)}{10} + \frac{a_2(\alpha)}{10^2} + \ldots +
\frac{a_k(\alpha)}{10^k}$, то существует единственный промежуток
\begin{equation*}
    \left[ 
        P_k(\alpha) + \frac{a_{k+1}(\alpha)}{10^{k+1}},
        P_k(\alpha) + \frac{a_{k+1}(\alpha) + 1}{10^{k+1}}
    \right)
\end{equation*}
такой что 
\begin{equation*}
    \alpha \in 
    \left[ 
        P_k(\alpha) + \left(\frac{a_{k+1}(\alpha)}{10^{k+1}}\right)^*,
        \left(P_k(\alpha)\right)^* 
        + \left(\frac{a_{k+1}(\alpha) + 1}{10^{k+1}}\right)^*
    \right)
\end{equation*}
тогда полагаем $P_{k + 1}(\alpha) = P_{k}(\alpha) + 
\frac{a_{k+1}(\alpha) + 1}{10^{k+1}}$.

Из построения следует, что $\forall k = 1,2, \ldots$ выполнено 
$P_k^*(\alpha) \leq \alpha < P_k^*(\alpha) + \left(\frac{1}{10^k}\right)^*$.
Положим $E_0 \overset{def}{=} \left\{r^* \in \Q : \exists
k \in \N : r^* = \left(P_k^*(\alpha)\right)^* \right\}$

\begin{assertion}
    $\alpha = supE_0$
\end{assertion}
\begin{proof}
    Из определений $E$ и $E_0$ следует, что $E_0 < E$, поэтому 
    $\alpha_0 \overset{def}{=} supE_0 \leq supE = \alpha$. Тогда имеем 
    неравенства, справедливые $\forall k \in \N$:
    \begin{equation}\label{ae0}
        P_k^*(\alpha) \leq \alpha_0 \leq \alpha < P_k^*(\alpha) + 
        \left(\frac{1}{10^k}\right)^* \Rightarrow 
        0 \leq \alpha - \alpha_0 \leq \left(\frac{1}{10^k}\right)^*
    \end{equation}
    Если бы $\alpha - \alpha_0 > 0$, то можно найти $k_0 : \alpha - \alpha_0 >
    \left(\frac{1}{10^{k_0}}\right)^*$, что противоречит соотношению \ref{ae0} 
    при $k \geq k_0$.
\end{proof}

В результате предыдущих рассуждений построено отображение $A(\alpha): \alpha 
\rightarrow a_1(\alpha), a_2(\alpha), \ldots$ из промежутка $(0; 1)$ в 
множество последовательностей, состоящих из элементов $0, 1, \ldots, 9$, 
что можно трактовать, как бесконечную десятичную дробь.

\begin{assertion}
    Отображение $A(\alpha)$ инъективно.
\end{assertion}
\begin{proof}
    Пусть $0 < \alpha_1 < \alpha_2 < 1, \alpha_1,\alpha_2 \in \R$. Выберем 
    минимальное $k_0$ такое, что $\alpha_2 - \alpha_1 \geq 
    \left(\frac{1}{10^{k_0}}\right)^*$. Если $k_0 = 1$, то $a_1(\alpha_1) <
    a_1(\alpha_2)$, поэтому $A(\alpha_1) \neq A(\alpha_2)$.
    Если $k_0 > 1$ и $\exists j, 1 \leq j \leq k_0 - 1 : a_j(\alpha_1) \neq 
    a_j(\alpha_2)$, то $A(\alpha_1) \neq A(\alpha_2)$; если же 
    $a_j(\alpha_1) = a_j(\alpha_2), 1 \leq j \leq k_0 - 1$, то 
    $a_{k_0}(\alpha_1) < a_{k_0}(\alpha_2)$ и $A(\alpha_1) \neq A(\alpha_2)$.
\end{proof}

Начиная с этого момента будем трактовать вещественное число также как $\pm$ 
(натуральное число + бесконечная десятичная дробь), поэтому не будем ставить 
знак у рациональных чисел. Важно заметить все привычне свойства вещественных 
чисел строго определены и обоснованы, если их определяют как сечения.
Пока речь шла только об арифметических действиях и неравенствах, в которых 
вещественные числа участвуют.

\subsection{Существование корня из вещественного числа}
\begin{theorem}
    Пусть $x > 0, n \geq 2, n \in \N$. Тогда $\exists! a > 0 : a^n = x$.
\end{theorem}
\begin{proof}
    Проверим единственность числа $a$, если оно существует. Пусть $a_0^n = x,
    a_0 > 0$, тогда $0 = a^n - a_0^n = (a - a_0)*A$, где $A = (a^{n-1} + 
    a^{n-2}a_0 + \ldots + a_0^n)$.
    Поскольку $A > 0$, то $(a - a_0) = 0 * \frac{1}{A} = 0, a = a_0$.
\end{proof}

\begin{definition}
    Полагаем $0! \overset{def}{=} 1, 1! \overset{def}{=} 1, n! \overset{def}{=}
    1 \cdot 2 \cdot \ldots \cdot n$, $C_n^m \overset{def}{=} 
    \frac{n!}{(n - m)!m!}, 0 \leq m \leq n$; или $C_n^m = C_n^{n - m}, C_n^0 = 1,
    C_n^1 = n$. 
    
    Бином Ньютона: пусть $n \geq 2; a, b \in \R$, тогда $(a + b)^n = 
    a^n + C_n^1a^{n-1}b + \ldots + C_n^{n-1}ab^{n-1} + b^n$.
    
    Пусть $E = \{t \in \R: t > 0, t^n \leq x\}$. Если $t_0 \overset{def}{=} 
    \frac{x}{1 + x}$, то $t_0 > 0, t_0 < 1, t_0^n = t_0^{n-1} * t_0 < 1^{n-1}
    \cdot t_0 < x$, т.е. $E \neq \emptyset$; если $t_1 = 1 + x$, то $t_1 > 1$,
    $t_1^n = t_1^{n-1} * t_1 > 1^{n-1} \cdot t_1 > x$, поэтому $t_1 \notin E$
    и является верхней границей $E$, т.е. $\exists supE$. Утверждаетсяя, что 
    $a = supE$. Предположим, что $b \overset{def}{=} supE, b^n < x$. 
    Выберем $0 < h < 1$ и также
    \begin{equation} \label{h}
        h < \frac{x - b^n}{(1 + b)^n - b^n}
    \end{equation} 
    Тогда 
    \begin{multline*}
        (b + h)^n = b^n + C_n^1b^{n-1}h + \ldots + C_n^{n-1}bh^{n-1} + h^n < \\
        b^n + C_n^1b^{n-1}h + \ldots + C_n^{n-1}bh + h = 
        b^n + h(C_n^1b^{n-1} + \ldots + C_n^{n-1}b + h) = \\
        b^n + h\left((1 + b)^n - 1\right) \overset{\ref{h}}{<} 
        b^n + x - b^n = x
    \end{multline*}
    т.е. $b + h \in E$, что противоречит тому, что $b - верхняя граница E$.
    Предположим, что $b^n > x$. Выберем $0 < v < 1, v < b$ и 
    \begin{equation}\label{v}
        v < \frac{b^n - x}{(1 + b)^n - b^n}
    \end{equation}
    Тогда
    \begin{multline*}
        (b - v)^n = b^n - C_n^1b^{n-1}v + C_n^2b^{n-2}v^2 - \ldots + 
        (-1)^{n-1}C_n^{n-1}bv^{n-1} + (-1)^{n}v^n = \\ b^n - 
        v(C_n^1b^{n-1} + C_n^2b^{n-2}v^1 - \ldots + 
        (-1)^{n-1}C_n^{n-1}bv^{n-1} + (-1)^{n}v^{n-1}) \geq \\
        b^n - v(C_n^1b^{n-1} + C_n^2b^{n-2}v^1 + \ldots + 
        C_n^{n-1}bv^{n-1} + v^{n-1}) > \\
        b^n - v(C_n^1b^{n-1} + C_n^2b^{n-2} + \ldots + C_n^{n-1}b + 1) = 
        b^n - v((1 + b)^n - b^n) \overset{\ref{v}}{>} b^n - (b^n - x) = x
    \end{multline*}
    т.е. $b - v$ - верхняя граница $E$, что противоречит тому, что $b = supE$.
    Итак, $b^n = x, b = a$.
\end{definition}

Далее приводится определение степени и логарифма без доказательств.

\begin{definition}
    $a^r, a > 0, r \in \Q$: если $r = \frac{p}{q}, q \in \N, p \in \Z$, то 
    полагаем $a^r \overset{def}{=} \left(a^{\frac{1}{q}}\right)^p$. При $r > 0$
    полагаем $0^r = 0$.
\end{definition}

\begin{definition}
    $a^\alpha, a > 1, \alpha \in \R : E = \left\{a^r : r \in \Q, r \leq \alpha
    \right\}$, тогда $a^\alpha = supE$. $1^\alpha = 1 \forall \alpha \in \R$.
    Если $0 < a < 1$, то $a^\alpha \overset{def}{=} 
    \left(\frac{1}{a}\right)^{-\alpha}$
\end{definition}

\begin{definition}
    $log_ab, a > 0, a \neq 1, b > 0$: если $a > 1$, то $E = \left\{
        x \in \R: a^x \leq b \right\}$, тогда $log_ab = supE$; если 
        $0 < a < 1$, то $log_ab \overset{def}{=} -log_{\frac{1}{a}}b$. 
\end{definition}

\begin{theorem}
    Для выражений $a^\alpha, log_ab$ справедливы все равнее встречающиеся в 
    школьном курсе утверждения.
\end{theorem}

\end{document}
