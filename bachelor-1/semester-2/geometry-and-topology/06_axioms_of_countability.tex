% !TeX root = ./main.tex
\documentclass[main]{subfiles}
\begin{document}
\chapter{Аксиомы счетности}
\section{Сепарабельность}
\begin{definition}
    $(X, \Omega)$ -- топологическое пространство.
    Говорят, что $X$ обладает второй аксиомой счетности, если у $X$ есть счетная база.
\end{definition}

\begin{definition}
    $(X, \Omega)$ -- топологическое пространство.
    $A \subset X$ называется всюду плотным в $X$, если $\Cl A =  X$
\end{definition}

\begin{definition}
    $X$ называется сепарабельным, если существует счетное всюду плотное множество в $X$.
\end{definition}

\begin{theorem}
    Из второй аксиомы счетности следует сепарабельность
\end{theorem}
\begin{proof}
    $\{U_i\}_{i \in I}^\infty$ -- счетная база.
    $x_i \in U_i \implies \{x_i\}_{i=1}^\infty$ -- счетное всюду плотное.
    Тогда $\Cl \{x_i\}_{i=1}^\infty = X$?

    Допустим противное:
    $y \in \Ex \{x_i\}_{i=1}^\infty$, значит $\Ex \{x_i\}_{i=1}^\infty = \bigcup_j U_{i_j} \ni x_{i_j}$.
    Внешность множества $\{x_i\}_{i=1}^\infty$ содержит $x_{i_j}$ -- противоречие.
\end{proof}
\begin{remark}
    Вторая аксиома счетности и сепарабельность -- топологические свойства.
\end{remark}

Здесь и далее в этой главе под словом <<счетное>> подразумевается <<не более чем счетное>>.

\begin{example}
    $X$ НБЧС, тогда $X$ сепарабельно.
\end{example}
\begin{example}
    $X$ -- антидискретное, тогда вторая аксиома счетности и сепарабельность есть.
\end{example}
\begin{example}
    $X$ -- дискретное:
    \begin{enumerate}
        \item $X$ счетное, тогда есть вторая аксиома счетности: база -- одноточечные подмножества
        \item $X$ более чем счетное, нет ни сепарабельности, ни второй аксиомы счетности.
              $\Cl A=A$ в дискретной топологии.
    \end{enumerate}
\end{example}
\begin{example}
    На $\R^n$ со стандартной топологией, есть вторая аксиома счетности и сепарабельность

    Рассмотрим $\mathfrak{B} = \{B(x, \epsilon): x, \epsilon > 0 \in \Q\}$.
    Это счетная база.
    Возьмем  $y_0 \in B(x_0, \epsilon)$, где $x_0, \epsilon$ не обязательно рациональные.
    $\rho\coloneqq \rho(x_0, y_0)$, тогда существует $z_0$ с рациональными координатами:
    $\rho(z_0, y_0) < \frac{\epsilon - \rho}{2}$, выберем $r\in \Q_+Ж \rho(z_0, y_0)< r < \frac{\epsilon- \rho}{2}$.
    Рассмотрим $B(z_0, r)$ такой что $y_0$ принадлежит ему.
    $B(z_0, r) \subset B(x_0, \epsilon)$.

    Сепарабельность: множество точек с рациональными координатами -- счетное всюду плотное.
\end{example}

\begin{remark}
    НЕ любое метрическое пространство обладает второй аксиомой счетности или сепарабельностью.

    Пусть $X$ континуальное, $\rho(x,y) = 1$ если $x \neq y$, тогда порождается дискретная топология.
\end{remark}

\begin{example}
    $X = \R$ с топологией Зариского (замкнутые, значит конечные).
    $X$ сепарабельно (любое бесконечное множество всюду плотно).
    Второй аксиомы счетности нет.

    Предположим, что она есть: $\{U_i\}_{i \in I}^\infty$ -- счетная база.
    $U_i = X \setminus \{x_{i_1}, ..., x_{i_{n_i}}\}$, тогда $\bigcup_{i=1}^\infty \{x_{i_1}, ..., x_{i_{n_i}}\}$ счетно.
    А $\R$ несчетно, тогда $\exists y \in U_i\ \forall i$.
    $U = X \setminus \{y\}$, $y \not \in U \neq \bigcup_j U_j \ni y$, значит счетной базы нет.
\end{example}

\begin{theorem}[Линделёфа]\label{axOfCount:lindelof}
    Если на $X$ есть вторая аксиома счетности, тогда из любого открытого покрытия $X$ можно выбрать НБЧС подпокрытие.
\end{theorem}
\begin{proof}
    $X = \bigcup_{i \in I} U_i, \mathfrak{B} = \{B_i\}_{i=1}^\infty$  -- счетная база.
    $\forall i\ U_i = \bigcup_{j} B_{j}$

    %% Неверное рассуждение?
    % Составим бесконечный двудольный граф: I доля -- множества $U_i$, II доля -- множества $B_j$,
    % проводим ребро если $U_i \supset B_j$.

    % $\forall B_j$ выберем один $U_i$ (если есть): $U_i \supset B_j$.
    % Выбрали НБЧС множество $U_i$.
    % Они покрывают всё $X$. (не подходит)

    $\{U_i\}$ вполне упорядочены (по теореме Цермело так можно).
    Рассмотрим $U_{i_1}$, отметим все $B_j \subset U_{i_1}$.
    Пусть $x_2 \not\in U_{i_1} \implies \exists U_{i_2} \ni x_2$
    тогда $U_{i_2} = \bigcup_j B_j$, отметим все такие $B_j$.
    На этом шаге мы отметили как минимум одно новое $B_j$.

    Продолжаем: $x_3 \not\in U_{i_1} \cup U_{i_2} \implies x_3 \in U_3 = \bigcup_j B_j$,
    отметили новое $B_j$.

    Таких шагов нельзя сделать более чем счетное количество.
    Таких  $U_{i_k}$  НБЧС количество, после которых новую точку,
    не входящую в их объединение, нельзя выбрать.
\end{proof}

\section{Секвенциальная компактность}
\begin{definition}
    $\{x_n\}_{n=1}^\infty$ -- последовательность в $X$.
    Говорим, что $x_0 \in \lim_{n \to \infty} x_n$ если
    $(\forall \epsilon >0\ \exists N \in \N : \forall n > N$ выполнено $\rho(x_n, x_0) < \epsilon$ или $x_n \in B(x_0, \epsilon))$

    $\forall U_{x_0}$ окрестность $\exists N \in \N: \forall n > \N \ x_n \in U_{x_0}$
\end{definition}

\begin{example}
    $\R$ с топологией Зариского.
    Пусть $x_i \neq x_j \implies \forall x_0\ x_n \to x_0$
    \[\forall U_{x_0} = X \setminus \{a_1, a_2, ..., a_n\}\ \exists N: \forall n > N\ x_n \neq a_n \implies x_n \in U_{x_0}\]
\end{example}
\begin{example}
    $\R$ с топологией типа Зариского: замкнутые = НБЧС.
    Если $x_i \neq x_j$, то $\not\exists x_0: x_n \to x_0$.
    Возьмем любой $x_0$ (считаем, что $x_0 \neq x_n$, иначе начнем последовательность с $x_{n+1}$)
    \[U_{x_0} = \R \setminus \{x_1, x_2, ..., \} \text{ -- открыто}\]
    $U_{x_0}$ не содержит ни одного члена последовательности.
\end{example}
\begin{remark}
    Если $X$ хаусдорфово, тогда предел не более чем единственный.
\end{remark}
\begin{proof}
    Допустим $x_0$ и $\tilde{x}_0$ -- пределы $x_n$.
    \[U_{x_0} \cap U_{\tilde{x}_0} = \varnothing\]
    Тогда с некоторого места:
    все $x_n \in U_{x_0}$ и  все $x_n \in U_{\tilde{x}_0}$.
    Но это противоречит с хаусдорфовостью.
\end{proof}

\begin{definition}
    $X$ называется секвенциально компактным пространством, если
    $\{x_i\}_{i = 1}^\infty \subset X\ \exists x_{n_k} \xrightarrow[k \to \infty]{} x_0$
    (из любой подпоследовательности можно выбрать сходящуюся).
\end{definition}
\begin{definition}
    $X$ обладает первой аксиомой счетности если $\forall x_0 \in X$, существует счетная база окрестностей $x_0$,
    т.е. $\exists \{B_{x_0, i}\}_{i = 1}^\infty: x_0 \in B_{x_0, i} \ x_0 \in \forall U$ открытому

    $\exists B_i: x_0 \in B_{x_0, i} \subset U \implies \{B_{x_0, i}\}_{i, x_0}$ -- база топологии.

    Это обобщение $B(x_0, \epsilon)$.
\end{definition}

\begin{remark}
    $X$ -- метрическое пространство, тогда $X$ обладает первой аксиомой счетности.
    $B(x_0, \epsilon)$, где $\epsilon \in \Q_+$.
\end{remark}

\begin{example}
    $\R$ с топологией Зариского не обладает первой аксиомой счетности.

    Допустим: есть счетное $\{U_{x_0, i}\}\ \forall x_0$.
    Рассмотрим $U_{x_0, 1}, U_{x_0, 2}$ и так далее, каждое из них НЕ содержит счетное число точек, в итоге счетный набор точек НЕ содержится в каком-то из этих множеств.
    Значит $\exists y \in U_{x_0, i}\ \forall i$.
    Возьмем $U = \R \setminus \{y\}$ -- окрестность $x_0$.
    $\not\exists U_{x_0, i} \subset U$ т.к. $U \not\ni y$.
\end{example}
\begin{remark}
    Из второй аксиомы счетности следует первая аксиома счетности.
\end{remark}

\begin{definition}
    $a$ называется точкой накопления, если для любой $U_a$ выполнено: $U_a \cap A$ -- бесконечно.
\end{definition}
\begin{remark}
    Точка накопления не обязательно лежит в $A$.
\end{remark}

[Примечание редактора: для следующих теорем большое доказательство будет разбито на несколько блоков для простоты восприятия]
\begin{theorem}\label{axOfCount:compactRel}
    Для утверждений:
    \begin{enumerate}
        \item $X$ компактно
        \item $A \subset X: |A| = \infty \implies \exists a$ -- точка накопления $A$.
        \item $X$ секвенциально компактно
        \item $\forall F_1 \supset F_2 \supset ...$ и $F_i \neq \varnothing$ -- замкнутое, тогда $\bigcap_{i = 1}^\infty F_i \neq \varnothing$
    \end{enumerate}
    выполнено:
    \begin{center}
        \input{figures/proof_scheme.pdf_tex}
    \end{center}
\end{theorem}
\begin{proof}
    Из 1 в 2:

    Допустим противное: любая $a \in X$ -- не точка накопления.
    Тогда $\exists U_a: U_a \cap A$ -- конечна.

    Соберем все $\{U_a\}_{a \in X}$ -- открытое покрытие $X$,
    значит $\exists U_{a_1},..., U_{a_n}$ -- конечное подпокрытие.

    Каждое $U_{a_i} \cap A$ -- конечное, тогда $\bigcup_{i=1} (U_{a_i} \cap A)$ -- конечное.
    Но это объединение есть $A$ -- противоречие.
\end{proof}
\begin{proof}
    Из 2 в 3:

    Хотим для любой последовательности иметь сходящуюся подпоследовательность.
    $A$ -- множество членов последовательности.

    Если $A$ конечно, то какой-то член повторяется бесконечное количество раз, его возьмем как подпоследовательность.

    Пусть $A$ бесконечное, тогда возьмем $x_0$ -- точка накопления.
    По первой аксиоме счетности: существует $\{U_i\}_{i=1}^\infty$ -- счетная база окрестностей $x_0$ и считаем, что $U_1 \supset U_2 \supset U_3 \supset ...$

    Пусть $V_1, V_2, ...$ -- какая-то счетная база окрестностей.
    Тогда $U_1 \coloneqq V_1$, $U_2 \coloneqq V_1 \cap V_2$, $U_3 \coloneqq V_1 \cap V_2 \cap V_3$ и т.д.

    $|U_i \cap A| = \infty$, значит выберем $a_i \in U_i \cap A$ так, чтобы все $a_i$ различны и номер $a_i$ в последовательности больше номеров предыдущих выбранных.
    Тогда $a_i \to x_0$. Почему?

    $\forall U$ -- окрестность $x_0$ $\exists U_n \subset U, U_{n+1} \subset U, U_{n+2} \subset U...$
    Рассмотрим $a_n \in U_n \subset U$, $a_{n+1} \in U_{n+1} \subset U$ и т.д.
    $\forall k\ge n\ a_k \in U \implies \lim_{n \to \infty} a_n = x_0$.
\end{proof}
\begin{proof}
    Из 3 в 4:

    $F_1 \supset F_2 \supset F_3 \supset ...$ -- замкнутые, $F_i \neq \varnothing$.
    $F_i \neq F_{i+1}$ (иначе сократим).
    Хотим $\bigcap F_i \neq \varnothing$.

    Выберем $x_n \in F_n \setminus F_{n-1}$.
    Отсюда $\{x_n\}$ -- последовательность.
    Тогда $\exists x_{n_k} \to x_0$.
    Утверждение: $x_0 \in \bigcap_{i=1}^\infty F_i$. Покажем, что $x_0 \in F_i\ \forall i$.

    Допустим: $x_0 \not\in F_m$.
    Выберем $U_m \coloneqq X \setminus F_m$ -- открытое.
    $x_0 \in U_m$ значит $\exists N: \forall k \ge N\ x_{n_k} \in U_m$.
    Все $x_{n_k} \not\in F_m$.

    НО если $n_k \ge m$, то $x_{n_k} \in F_{n_k} \subset F_m$. Противоречие.
\end{proof}
\begin{proof}
    Из 4 в 1:

    $X$ удовлетворяет второй аксиоме счетности.
    Пусть $\{U_i\}$  -- любое открытое покрытие $X$.
    Считаем, что $\{U_i\}$ счетно (по \ref{axOfCount:lindelof}), т.е. $\{U_i\} = \{U_1, U_2, ...\}$.

    Построим замкнутые множества:
    $V_1 = U_1, V_2 = U_1 \cup U_2, ..., V_n \coloneqq \bigcup_{i=1}^n U_i$.
    $V_1 \subset V_2 \subset V_3 \subset ...$ -- открытые.

    Тогда $F_i \coloneqq X \setminus V_i$ -- замкнутые.
    $F_1 \supset F_2 \supset F_3 \supset ...$
    Почему $F_i \neq \varnothing$?

    Если $F_k = \varnothing \implies V_k = X$.
    $U_1 \cup ... \cup U_k = X$ -- победа.

    Иначе по (3) $\bigcap_{k=1}^\infty F_k \neq \varnothing \implies \bigcup_{k=1}^\infty V_k \neq X$.
    Тогда и $\bigcup_{k=1}^\infty U_k \neq X$ значит $\{U_k\}$ не покрытие,
    НО изначально брали покрытие -- противоречие.
\end{proof}

\section{Компактность в метрических пространствах}
\begin{remark}
    $A \subset \R^n$ компактно $\Leftrightarrow A$ -- замкнуто и ограниченно.
    Если $A \subset (M, \rho)$ -- не обязательно.
\end{remark}
\begin{example}
    Есть $\R$ с дискретной топологией. $\rho(x,y) = 1$, если $x \neq y$.

    $\forall U$ -- замкнуто и ограничено. $B(x_0, 2) = X \supset U$, значит $U$ -- ограничено,
    $U$ замкнуто, т.к. любое множество замкнуто в дискретной топологии.
    Но если $U$ бесконечное множество, тогда $U$ не компактно.
\end{example}

\begin{definition}
    $(M, \rho)$ -- метрическое пространство.
    $\{x_n\}$ -- последовательность.
    $\{x_n\}$ называется фундаментальной, если
    \[\forall \epsilon >0\ \exists N(\epsilon): \text{ если } \forall n, k \ge N \text{ то }  \rho(x_n, x_k) < \epsilon\]
\end{definition}
\begin{definition}
    Пространство $M$ называется полным, если любая фундаментальная последовательность сходится.
\end{definition}
\begin{example}
    $\Q$  -- не полное, $\R$ -- полное.
\end{example}
\begin{theorem}[из курса матанализа]
    Следующие определения равносильны:
    \begin{enumerate}
        \item $[a_1; b_1] \supset [a_2; b_2] \supset ... \implies \exists x_0 \in \bigcap_{n=1}^\infty [a_n; b_n]$
        \item Любая фундаментальная последовательность сходится
    \end{enumerate}
\end{theorem}

\begin{definition}
    Пусть $\epsilon >0$, тогда $\{x_i\}_{i \in I}$ называете $\epsilon$-сетью пространства $M$,
    если $\forall y \in M\ \exists x_i: \rho(x_i, y) < \epsilon$.

    Переформулируем: $\{B(x_i, \epsilon)\}_{i \in I}$ -- покрытие $M$.
\end{definition}

Задача: Есть код из $n$ бинарных символов.
При передаче портится не более чем $k$ символов.
Сколько различных кодов можно передать?

Пусть $K_1, ..., K_l$ -- коды, которые можем передать.
Это означает $\rho(K_i, K_j) \ge 2 k$ (расстояние -- количество отличающихся бит).
Если набор $K_1, ..., K_l$ -- максимальный, тогда $\not \exists K_{l+1} : \rho(K_{l+1}; K_i) \ge 2k \implies \{K_i\}$ -- $2k$-сеть.

\begin{definition}
    Если $\forall \epsilon\ \exists$ конечная $\epsilon$-сеть, то $M$ называется вполне ограниченным.
\end{definition}

\begin{proposition}
    Если $M$ вполне ограниченно, то $M$ удовлетворяет второй аксиоме счетности.
\end{proposition}
\begin{proof}
    Пусть $\epsilon_k = \frac{1}{k}$, $X$ -- множество точек, входящих в какую либо из $\epsilon_k$-сетей.
    Значит $X$ -- счетное множество, как счетное объединение счетных множеств.
    $\{B\left(x_k, \frac{1}{l}\right) : x_k \in X, l \in \N\}$ -- база.

    Пусть $U \subset M$ -- открытое, $y_0 \in U$ -- точка.
    Докажем: $\exists B\left(x_k, \frac{1}{l}\right)\subset U$, $y_0$ лежит в этом шаре.

    $\exists B(y_0, \epsilon) \subset U$ (т.к. $U$ открыто).
    Скажем, что$\frac{1}{l} < \frac{\epsilon}{2}$,
    тогда $\exists x_n: \rho(x_n, y_0) < \frac{1}{l}, B\left(x_n, \frac{1}{l}\right)$ подходит.
\end{proof}
\begin{remark}
    Первая аксиома счетности выполняется в любом метрическом пространстве.
\end{remark}

\begin{theorem}
    $(M, \rho)$ -- метрическое пространство. Следующие определения равносильны:
    \begin{enumerate}
        \item $M$ компактно
        \item $M$ секвенциально компактно
        \item $M$ полное и вполне ограниченное
    \end{enumerate}
\end{theorem}
План доказательства: $1 \implies 2 \implies 3 \implies 2$, $2$ \& $3 \implies 1$.
\begin{proof}
    $1 \implies 2$

    $M$ -- метрическое пространство, значит первая аксиома счетности выполнена, тогда из компактности следует секвенциальная компактность по \ref{axOfCount:compactRel}.
    [далее прямой ссылки на теорему при применении нет]
\end{proof}
\begin{proof}
    $2 \implies 3$

    Полнота:

    Допустим, что $\{x_n\}$ -- фундаментальная, но не сходящаяся последовательность.
    $M$ -- секвенциально компактно, тогда существует $\{x_{n_k}\} : x_{n_k} \to x_0$,
    тогда и $x_n \to x_0$.
    \[\forall \epsilon >0: \exists N: \text{ если } n_k > N \text{ то } \rho(x_{n_k}, x_0) < \epsilon/2\]
    Если $l > N$, то $\rho(x_{n_k}, x_l) < \epsilon/2$, значит $\rho(x_l, x_0) < \epsilon$ и $x_n$ сходится.

    Вполне ограниченность:

    Допустим $\exists \epsilon$, что нет конечной $\epsilon$-сети.
    Тогда \[\exists x_1, x_2, ...:\rho(x_n, x_k) > \epsilon.\]

    Выберем $x_1$, $\exists x_2: \rho(x_1, x_2) \ge \epsilon$,
    $\exists x_3: \rho(x_3, x_1) \ge \epsilon$ и $\rho(x_3, x_2) \ge \epsilon$ и т.д.
    $\{x_i\}$ -- последовательность. У нее нет сходящейся подпоследовательности.
\end{proof}
\begin{proof}
    $3 \implies 2$

    Пусть $\{x_n\}$ -- последовательность, хотим выбрать фундаментальную подпоследовательность.
    (из-за полноты она будет сходящейся)
    Возьмем $\epsilon_1 = 1$, тогда существует конечная $\epsilon_1$-сеть.

    Пусть $y_1$ точка из конечной $\epsilon_1$-сети.
    Тогда в $B(y_1, 1)$  есть бесконечно много $x_i$.
    Выберем один из них: $x_{n_1}$.
    В следующих выборках будем брать только из $x_i$, входящих в $B(y_1, 1)$.
    Пусть $\epsilon_2 = 1/2$. Существует конечная $\epsilon_2$-сеть.
    В $B(y_2, 1/2)$ есть бесконечно много $x_i$.
    Выберем один из них: $x_{n_2}$.
    Пусть $\epsilon_3 = 1/3$ и т.д.

    Получим $\{x_{n_k}\} : x_{n_k} \in \bigcap_{i=1}^k B(y_k, \epsilon_i)$.
    Тогда $|x_{n_k} - x_{n_l}| < 2 \cdot \max\left\{\frac{1}{k}, \frac{1}{l}\right\}$.
    Пусть $k < l$, тогда $x_{n_k}, x_{n_l} \in B\left(y_k, \frac{1}{k}\right)$.
    Отсюда $\rho (x_{n_k}, x_{n_l}) < \frac{2}{k}$ и это означает, что $\{x_{n_k}\}$ -- фундаментальная.
\end{proof}
\begin{proof}
    $2$ \& $3 \implies 1$

    Из вполне ограниченности следует вторая аксиома счетности.
    Секвенциальная компактность со второй аксиомой счетности дает компактность.
\end{proof}
\end{document}