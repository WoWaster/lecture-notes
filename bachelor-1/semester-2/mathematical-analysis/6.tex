\chapter{Числовые ряды}
\begin{definition}
Пусть $\{a_n\}_{n-1}^{\infty}$ - произвольная последовательность. Рядом называется символ $\sum_{n=1}^{\infty}{a_n} \ \ (1)$ Частичной суммой ряда(1) называется выражение $S_n = a_1+ \dots a_n, n\geq 1$.
\end{definition}
Говорим, что ряд(1) сходится, если $\exists \lim\limits_{n \to \infty}{S_n} = S,$ $S \in \mathbb{R}$. В этом случае говорят, что S является суммой ряда (1) и пишут $\sum_{n=1}^{\infty}{a_n} = S \ \ (2)$
Если предела $S_n$ не существует или этот предел бесконечен, то говорят, что ряд (1) сходится, и в этом случае символу (1) не присваивают числового значения
\section{ Необзодимый признак сходимости ряда}
\begin{theorem}
	Пусть ряд (1) сходится, Тогда $a_n \underset{n\to \infty}{\to}0$
\end{theorem}
\begin{proof}
	Пусть $n \geq 2$. Тогда $S_n = a_1 + ... + a_{n-1}+a_n = S_{n-1} +a_n$, тогда $a_n = S_n-S_{n-1} \ \ (3)$
	Пусть, в соотвествии с (2), $S_n \underset{n\to \infty}{\to} S;$ тогда и $S_{n-1} \underset{n\to \infty}{\to} S$, поэтому $(3) \Rightarrow \lim\limits_{n \to \infty}{a_n} = \lim\limits_{n \to \infty}{S_n} - \lim\limits_{n \to \infty}{S_{n-1}} = S - S = 0,$ что и доказывает теорему. 
\end{proof}
\begin{remark}
	Признак сходимости является необходимым, но не достаточным
\end{remark}
\section{Критерий Коши сходимости ряда}
\begin{theorem} Для того, чтобы ряд (1) сходился, необходимо и достаточно, чобы $\forall \epsilon > 0 \ \exists N$ т.ч. $\forall m> n> N$ выполнялось $|\sum_{k = n+1}^{m}{a_k}| < \epsilon \ \ (4)$
\end{theorem}
\begin{proof}
	По критерию Коши существования конечного предела последовательности для того, что последовательность $\{S_k\}_{k =1}^{\infty}$ имела конечный предел, необходимо и достаточно, чтобы $\forall \epsilon > 0 \ \exists N$ т.ч. $\forall m > n> N$ выполнена $|S_m - S_n| < \epsilon \ \ (5)$ Поскольку $S_m-S_n = a_{n+1}+...+a_m$, то $(5) \Rightarrow (4)$
\end{proof}
\section{Ряды с неотрицательными слагаемыми}
Пусть $a_n \geq 0,n\geq 1, (6)$ тогда $S_{n+1}-S_n=a_n \geq 0$, поэтому по теореме о пределе монотонной последовательности существует $\lim\limits_{n \to \infty}{S_n} \in \overline{\mathbb{R}}$ Для сходимости ряда (1) с условиями $a_n \geq 0$ справедлив следующий критерий
\begin{theorem}
Для того, чтобы сходился ряд (1) при условии (6), необходимо и достаточно, чтобы $\exists M > 0$ т.ч. $\forall n $ выполнялось $S_n \leq M$
\end{theorem}
\begin{proof}
	Необходимость. Пусть $S_n \underset{n \to \infty}{\to} S$, $S \in \mathbb{R}$ Тогда по теореме об ограниченности последовательности, имеющей конечый предел, $\exists M $ т.ч. $|S_n| = S_n \leq M$, что доказывает необходимость.$\\$
	Достаточность. Пусть $S_n \leq M \ \forall n$, тогда по теореме о пределе монотонной последовательности $\exists \lim\limits_{n \to \infty}{S_n} \leq M$. Теорема доказана.
\end{proof}
\section{Признаки сравнения рядов с неотрицательными слагаемыми}
\begin{theorem}
	Пусть $a_n \geq 0, b_n \geq 0, n\geq 1$ Предположим, что $a_n \leq c b_n, c>0 \ \ (6) \\$
	(A) Предположим, что ряд $\sum_{n=1}^{\infty}{b_n}$ сходится. Тогда ряд $\sum_{n=1}^{\infty}{a_n}$ сходится и выполнено $\sum_{n=1}^{\infty}{a_n} \leq c $  $\sum_{n=1}^{\infty}{b_n}$ \ \ (7)
	(B) Предположим,что ряд $\sum_{n=1}^{\infty}{a_n}$ расходится. Тогда ряд $\sum_{n=1}^{\infty}{b_n}$ расходится.
\end{theorem}
\begin{proof} 
	Доказательство (A). Пусть $S_n = a_1 + ... + a_n, T_n = b_1 + ... + b_n$ Тогда условие (6) влечёт $S_n \leq cb_1 + ... + cb_n = cT_n \ \ \ (8) \\ $
	Пусть $T =\sum_{n=1}^{\infty}{b_n}$, тогда $T_n \leq T$ по теореме о пределе монотонной последовательности, и (18) $\Rightarrow$ $S_n \leq cT_n \leq cT \ \ (9)$ Из (9) следует, что $\exists \lim\limits_{n \to \infty}{S_n} = S$ и $S \leq cT$ Часть (А) доказана
	Доказательство (B). Предположим, что ряд $\sum_{n=1}^{\infty}{b_n}$ сходится. Тогда мы находимся в условии части (А), и ряд $\sum_{n=1}^{\infty}{a_n}$ должен сходится, что противоречит предположению. Часть (В) доказано.
\end{proof}
\section{Признак Коши сходимости ряда}
\begin{theorem}
	Пусть $a_n \geq 0, q = \overline{\lim\limits_{n \to \infty}{}}{\sqrt[n]{a_n}}$ Тогда ряд $\sum_{n=1}^{\infty}{a_n}$ сходится, если q < 1;
	ряд $\sum_{n=1}^{\infty}{a_n}$ расходится, если q > 1 $\\$
	В случае q = 1 признак не даёт определённого утверждения.
\end{theorem}
\begin{proof}
	Пусть $q < 1, r = \frac{1+q}{2}, q = \frac{1-q}{2}$. По свойствам верхних пределов $\exists N$ т.ч. $\forall n > N$ выполнено $\sqrt[n]{a_n} < q + \epsilon = r$ (10) Тогда $(10) \Rightarrow a_n < r^n, n> N$. Положим $c_0 = \underset{1 \leq k \leq N}{max}a_kr^{-k}$, $c = max(c_0,1) \ \ (11)$
	$
		\left.
	  \begin{array}{ccc}$
	Тогда (11) $\Rightarrow a_k \leq c_0r^k \leq cr^k, 1\leq k \leq N$
	$a_k \leq r^k\leq cr^k, k > N  
  \end{array}
	\right\}$ (12) \\
	Заметим, что $r + r^2+...+r^n = \frac{r-r^{n+1}}{1-r} \underset{n \to \infty}{\to} \frac{r}{1-r},$ поэтому ряд $\sum_{n=1}^{\infty}{r_n}$ сходится и (12) влечёт сходимость ряда $\sum_{n=1}^{\infty}{a_n}$ по признаку сравнения рядов. Пусть q > 1, $\epsilon_0 = q -1$. По свойству верхних пределов $\exists \{n_k\}_{k=1}^{\infty}$ т.ч. $\sqrt[n_k]{a_{n_k}} > q - \epsilon_0 = 1$ Поэтому $a_{n_k} > 1 = 1,$ поэтому свойство $a_n \underset{n \to \infty}{\to} 0$ не выполнено, следовательно, ряд $\sum_{n=1}^{\infty}{a_n}$ расходится по необходимому признаку сходимости ряда
\end{proof}
\section{Признак Даламбера сходимости ряда}
\begin{theorem}
	Пусть $a_n > 0;$ предположим, что $\exists \lim\limits_{n \to \infty}{\frac{a_{n+1}}{a_n}} = q$ Тогда, если q < 1, то ряд $\sum_{n=1}^{\infty}{a_n}$ сходится; если q > 1, то 
	этот ряд расходится, если q = 1, то нет определённого ответа
\end{theorem}
\begin{proof}
	Пусть $q < 1, r = \frac{1+r}{2}, \epsilon = \frac{1-q}{2}$ Тогда $\exists N$ т.ч. $\forall n > N$ выполнено $\frac{a_{n+1}}{a_n} < q + \epsilon = r, \ \ (13)$ а также $\frac{a_n}{a_{n-1}}<r,\frac{a_{n-1}}{a_{n-2}} < r$, ... $\frac{a_{N+2}}{a_{N+1}} < r \ \ (14)$ Перемножим неравенства (13) и (14): $\frac{a_{n+1}}{a_n} \cdot \frac{a_n}{a_{n-1}}...\frac{a_{N+2}}{a_{N+1}} < \underbrace{n - N}{ r...r} = r^{n-N} \ \ (15)$ т.е. $\frac{a_{n+1}}{a_{N+1}} < r^{-N-1}\cdot r^{n+1}, a_{n+1} < a_{N+1}r^{-N-1}\cdot r^{n+1} \ \ (16)$ Положим $c = \underset{1 \leq k \leq N+ 1}{max}a_kr^{-k},$ тогда при $k \leq N +1$ имеем $a_k \leq cr^k$, при $n > N +1$ из (16) следует $a_{n+1} \leq cr^{n+1}$, т.е. $\forall n \geq 1$ выполнено $a_n \leq cr^n$, и ряд $\sum_{n = 1}^{\infty}{a_n}$ сходится по признаку сравнения рядов. Пусть $q > 1, \epsilon_0 = q -1$ Тогда $\exists N_0$ т.ч. $\forall n > N_0$ выполнено $\frac{a_{n+1}}{a_n} > q - \epsilon_0 = 1$, т.е. $\frac{a_n}{a_{n-1}} > 1,$ ... $\frac{a_{N+2}}{a_{N+1}} > 1$, поэтому $a_{n+1} > a_n >...>a_{N+1} > 0,$ поэтому неверно, что $a_n \underset{n \to \infty}{\to} 0$, т.е. ряд расходится по необходимому признаку сходимости ряда
\end{proof}
\section{Интегральный признак сходимости ряда}
\begin{theorem}
	Пусть $f:[1,\infty]\to \mathbb{R}, f(x) \geq 0, f(x)$ убывает. Тогда ряд $\sum_{n = 1}^{\infty}{f(n)}$ и несобственный интеграл $\int\limits_{}^{\infty}{f(x)}dx$ сходится или расходится одновременно.
\end{theorem}
\begin{proof}
	Пусть $x \in [n,n+1],$ тогда условие монотонности влечёт $f(n) \geq f(x) \geq f(n+1)$, поэтому $\int\limits_{n}^{n+1}{f(n)}dx \geq \int\limits_{n}^{n+1}{f(x)}dx \geq \int\limits_{n}^{n+1}{f(n+1)}dx$, т.е. $1 \cdot f(n) \geq \int\limits_{n}^{n+1}{f(x)}dx \geq 1 \cdot f(n+1) \ \ (17)$
	(17) $\Rightarrow f(1) + ... + f(n) \geq \int\limits_{1}^{2}{f(x)}dx + \int\limits_{2}^{3}{f(x)}dx +...+ \int\limits_{n}^{n+1}{f(x)}dx \geq f(2) + ... + f(n+1)$,
	$f(1) + ... f(n) \geq \int\limits_{1}^{n+1}{f(x)}dx \geq f(2) + ... + f(n+1) \ \ (18) $
	Предположим, что $\int\limits_{1}^{\infty}{f(x)}dx$ сходится Тогда $\int\limits_{1}^{n+1}{f(x)}dx \leq \int\limits_{1}^{\infty}{f(x)}dx \ \ \ (19)$ в силу $f(x) \geq 0$ и (18) и (19) влекут $f(1)+f(2)+...+f(n+1) \leq f(1) + \int\limits_{1}^{\infty}{f(x)}dx \ \ (20)$
	Тогда $(20)  \Rightarrow$ ряд $\sum_{n=1}^{\infty}{f(x)}$ сходится Предположим, что ряд $\sum_{n=1}^{\infty}{f(n)}$ сходится. Тогда $(18) \Rightarrow \int\limits_{1}^{n+1}{f(x)}dx \leq f(1)+...+f(n) \leq \sum_{k=1}^{\infty}{f(k)} \ \ (21)$ Пусть A >1, выберем n так,чтобы n+1 > A, тогда (21) $\Rightarrow$ $\int\limits_{1}^{A}{f(x)}dx \leq \int\limits_{1}^{n+1}{f(x)}dx \leq \sum_{k=1}^{\infty}{f(k)} \ \  (22)$ Из (22) следует, что $\int\limits_1^{\infty}{f(x)}$ сходится.
\end{proof}
\begin{example}
	Рассмотрим $\sum_{n=1}^{\infty}{\frac{1}{n^p}}\ \ (23)$ По теорема этот ряд сходится одновременно с $\int\limits_{1}^{\infty}{\frac{dx}{x^p}}$. По примеру интеграл сходится при p > 1 и расходится при $p \leq 1$, поэтому ряд (23) сходится при p > 1 и расходится при $p \leq 1$. В частности, при p = 1 ряд $\sum_{n=1}^{\infty}{\frac{1}{n}}$ расходится
\end{example}
\section{Ряды со слагаемыми произвольных знаков}
\begin{definition}
	Будем говорить, что ряд $\sum_{n=1}^{\infty}{a_n} (1)$ абсолютно сходится, если сходится ряд $\sum_{n=1}^{\infty}{|a_n|}$ (2) Если ряд (1) сходится, а ряд (2) 
	расходится, то говорим, что ряд (1) сходится не абсолютно
\end{definition} 
\begin{theorem}
	Если ряд(1) абсолютно сходится, то он сходится
\end{theorem}
\begin{proof}
	Применим Критерий Коши к ряду (1) и к ряду (2). Возьмем $\forall \epsilon > 0$ По условию, ряд (2) сходится, поэтому $\exists N$ т.ч. $\forall m > n > N$ выполнено:$\\$
	 $|\sum_{k=n+1}^{m}{a_k}|=\sum_{k=n+1}^{m}{|a_k|} \ (3) \\$
	 При этом (3) $\Rightarrow$ $|\sum_{k=n+1}^{m}{a_k}| \leq  \sum_{k=n+1}^{m}{|a_k|} < \epsilon \ \ (4) \\$ (4) означает, что признакКоши применим к ряду (1), т.е. ряд(1) сходится. 
\end{proof}
\section{Перестановки рядов}
\begin{definition}
	Пусть $\alpha: \mathbb{N} \to \mathbb{N}$ - биекция множества натуральных чисел на себя; перестановкой ряда $\sum_{n=1}^{\infty}{a_n}$ будем называть ряд $\sum_{n=1}^{\infty}{b_n}$, где $b_n = a_{\alpha(n)}$. Если $\alpha^{-1}$- обратное к $\alpha$ отображение, то ряд $\sum_{n=1}^{\infty}{a_n}$ является перестановкой ряда $\sum_{n=1}^{\infty}{b_n}$, поскольку $a_n = b_{\alpha^{-1}(n)}$
\end{definition}
\begin{theorem}
	Пусть ряд $\sum_{n=1}^{\infty}{a_n}$ абсолютно сходится, $\sum_{n=1}^{\infty}{b_n}$ - перестановка этого ряда. Тогда $\sum_{n=1}^{\infty}{a_n}=\sum_{n=1}^{\infty}{b_n} \ \ (5)$
\end{theorem}
\begin{proof}
	Предположим в начале, что $a_n \geq 0$, тогда и $b_n \geq 0 \forall n$. Абсолютная сходимость ряда $\sum_{n=1}^{\infty}{a_n}$  в данном случае означает просто сходимость этого ряда, пусть $S = \sum_{n=1}^{\infty}{a_n}$. Возьмем $\forall N \in \mathbb{N}$ и пусть $A(N) = \underset{1\leq n\leq N}{max}\alpha(n)$. Тогда $\sum_{k=1}^{N}{b_k} = \sum_{n=1}^{N}{a_{\alpha(n)}}\leq \sum_{n=1}^{N}{a_n} \leq \sum_{n=1}^{\infty}{a_n} = S \ \ (6)$ Первое неравенство в (6) следует из предположения $a_k \geq 0 \ \forall k$ и того,что среди чисел $\alpha(n), 1\leq n \leq N$, могли быть не все числа от 1 до N. Из(6) следует, что $\sum_{k=1}^{\infty}{b_k} \leq S \ \ (7) \\$
	Пусть $T = \sum_{k=1}^{\infty}{b_k}$. Поскольку ряд $\sum_{n=1}^{\infty}{a_n}$ является перестановкой ряда $\sum_{k=1}^{\infty}{b_k}$, то по уже проведенным рассуждениям с переменной ролей рядов $\sum_{n=1}^{\infty}{a_n}$ и $\sum_{k=1}^{\infty}{b_k}$, то, она логично (7), получим $\sum_{n=1}^{\infty}{a_n} \leq T \ \ (8)$ из (7) и (8) получаем, что $S = T$, что и требуется  предположении $a_n \geq 0$. Пусть теперь $a_n$ могут иметь произвольный знак.$\\$ Положим
	$a^+ =
	\begin{cases}
		\text{a, если a} \geq 0 \\
		0, \text{если a} < 0 
	\end{cases}$
	$a^- =
	\begin{cases}
		|a| \text{,если a} \leq 0 \\
		0, \text{если a} > 0 
	\end{cases} \\$
	Тогда $a = a^+ - a^-, |a| = a^+ + a^-.$ Поскольку ряд $\sum_{n=1}^{\infty}{a_n}$ абсолютно сходится, то $S \underset{\text{по опр}}{=} \sum_{n=1}^{\infty}{|a_n|}=\sum_{n=1}^{\infty}{a_{n}^{+} + a_{n}^{-}} \in \mathbb{R} (9)$
	Поскольку $a_n^+ \leq |a_n|, a_{n}^{-} \leq |a_n|$, то (9) по признака мравнения рядов с неотрицателльными слагаемыми влечёт, что сходядтся ряды $\sum_{n=1}^{\infty}{a_{n}^{+}}$ и $\sum_{n=1}^{\infty}{a_{n}^{-}}$. Отметим также свойства сходящихся рядов, следующих из свойств пределов последовательностей: если сходятся ряды $\sum_{n=1}^{\infty}{p_n}$ и $\sum_{n=1}^{\infty}{q_n}$, то сходится ряд $\sum_{k=1}^{\infty}{p_n+q_n}$ и $\sum_{n=1}^{\infty}{p_n + q_n}=\sum_{n=1}^{\infty}{p_n}+ \sum_{n=1}^{\infty}{q_n}$; если сходится ряд $\sum_{n=1}^{\infty}{p_n}$ и $c \in \mathbb{R}$, то сходится ряд $\sum_{n=1}^{\infty}{c\cdot p_n}$ и $\sum_{n=1}^{\infty}{c\cdot p_n}= c\cdot \sum_{n=1}^{\infty}{p_n}$
	Применим эти свйоства в данном случае. Пусть $\alpha: \mathbb{N} \to \mathbb{N}$- биекция, $b_n = a_{\alpha(n)}$, тогда $b_{n}^{+} = a_{\alpha(n)^+}, b_{n}^{-} = a_{\alpha(n)}^{-}$, и по свойству установленному для рядов с неотрицательными слагаемыми, $\sum_{n=1}^{\infty}{b_n^+}= \sum_{n=1}^{\infty}{(b_n^+ - b_n^-)}=\sum_{n=1}^{\infty}{b_n^+}-\sum_{n=1}^{\infty}{b_n^-}=\sum_{n=1}^{\infty}{a_n^+} - \sum_{n=1}^{\infty}{a_n^-}=\sum_{n=1}^{\infty}{a_n^+-a_n^-}=\sum_{n=1}^{\infty}{a_n}$, что и требуется
\end{proof}
\section{Теорема Римана.} Пусть ряд $\sum_{n = 1}^{ \infty}{a_n}$ неабсолютно сходится, $S \in \mathbb{R}$ - любое число, тогда $\exists$ биекция $\alpha: \mathbb{N} \to \mathbb{N}$, такая, что для $b_n = a_{\alpha(n)}$ имеем $\sum_{n=1}^{\infty}{b_n} = S$.
\begin{proof}
Примем без доказательства
\end{proof}
\section{Признак Лейбница}
Знакопеременным рядом будем называть ряд вида $\sum_{n=1}^{\infty}{(-1)^{n-1}a_n},$ где $a_n > 0$
\begin{theorem}
	Пусть $a_n \to 0, a_n > 0, a_n$ монотонно убывает. Тогда ряд $\sum_{n=1}^{\infty}{(-1)^{n-1}a_n}$ сходится
\end{theorem}

\begin{proof}
	Пусть $S_m = \sum_{n=1}^{m}{(-1)^{n-1}a_n}$ Тогда в силу условия $a_n \geq a_{n+1}$ получим $S_{2m} = (a_1-a_2)+(a_3-a_4)+...+...(a_{2m-1}-a_{2m}) \leq (a_1-a_2)+(a_3 - a_4)+...+ (a_{2m-1}-a_{2m})+(a_{2m+1}-a_{2m+2}) = S_{2m+2}$ \ \ (10) $\\$
	$S_{2m-1} = a_1 - (a_2-a_3) - ... - (a_{2m-2}-a_{2m-1})\geq a_1 - (a_2 - a_3)-...-(a_{2m-2}-a_{2m+1})-(a_{2m}-a_{2m+1}) = S_{2m+1} \ \ (11)$
	В силу условия $a_n > 0$ получаем $S_{2m+1} = S_{2m} + a_{2m+1} > S_{2m} \ \ (12)$ Из (10)- (12) находим, что $0 \leq a_1-a_2 = S_2 \leq ... \leq S_{2m-2}\leq S_{2m} < S_{2m+1} \leq S_{2m-1} \leq ... \leq S_1  =a_1 \ \ (13)$ Из (13) находим, что $\exists \lim\limits_{m \to \infty}{S_{2m}}\overset{\text{по опр}}{=} S' \leq a_1, $и$\exists \lim\limits_{m \to \infty}{S_{2m+1}}= S'' \geq S'$   Далее, в силу $a_n \to 0$ имеем $S'' -S'' = \lim\limits_{m \to \infty}{S_{2m+1}}-\lim\limits_{m \to \infty}{S_{2m}}=\lim\limits_{m \to \infty}{S_{2m+1}-S_{2m}}= \lim\limits_{m \to \infty}{a_{2m+1}} = 0$, т.к. S' = S'' =S и $\exists \lim\limits_{n \to \infty}{S_{n}} = S,$ что и требуется
\end{proof}
\begin{example}
Рассмотрим ряд $\sum_{n = 1}^{\infty}{(-1)^{n-1}\cdot \frac{1}{n}} \ \ (14)$ По признаку Лейбница этот ряд сходится, то ряд $\sum_{n = 1}^{\infty}{|(-1)^{n-1}\cdot \frac{1}{n}|} = \sum_{n = 1}^{\infty}{\frac{1}{n}}$ расходится. Таким образом, ряд (14) сходится неабсолютно
\end{example}
\section{Признак Абеля.} 
\begin{theorem}
	Рассмотрим ряд $\sum_{n = 1}^{\infty}{ a_n b_n} \ \ (15)$ и пусть ряд $\sum_{n = 1}^{\infty}{ a_n} \ \ (16)$ сходится, а $b_n$ монотонна и $\exists M > 0$ т.ч. $|b_n|\leq M \ \forall n$ Тогда ряд (15) сходится
\end{theorem}
\begin{proof}
	Положим $A_n =a_n, A_{n+1}= a_n + a_{n+1},...$ Применим преобразование, называемое преобразованием Абеля: поскольку $a_k = A_k - A_{k-1}, k \geq n=1$, то $a_{n+1}b_{n+1}+...+a_{m}b_m = (A_{n+1}-A_n)b_{n+1}+(A_{n+2}-A_{n+1})b_{n+2}+...+(A_m - A_{m-1})b_m=-A_nb_{n+1}+A_{n+1}(b_{n+1}-b_{n+2})+...A_{m-1}(b_{m-1}-b_m) + A_mb_m \ \ (17)$ Выберем $\epsilon > 0$ В силу сходимости ряда  (16) $\exists N$ т.ч. $\forall m \geq n > N$ выполнено $|\sum_{k = n}^{m}{ a_k}|<\epsilon \ \ \ (18) \\$
	Из (17) и (18) получаем $|a_{n+1}b_{n+1}+...+a_mb_m| \leq |A_nb_{n+1}|+|\sum_{k= n+1}^{m-1}{A_k(b_k-b_{k+1})}| + |A_mb_m| < \epsilon \cdot M + \sum_{k = n + 1}^{m-1}{|A_k||b_k-b_{k+1}|} + \epsilon \cdot M \leq 2\epsilon M + \epsilon \sum_{k =n + 1}^{m-1}{|b_k - b_{k+1}|} = 2\epsilon M + \epsilon |\sum_{k = n + 1}^{m-1}{(b_k - b_{k+1})}| = 2\epsilon M + \epsilon |b_{n+1}-b_m| \leq 2\epsilon M + \epsilon(|b_{n+1}|+|b_{m}|) \leq 4 \epsilon M \ \ (19) \\$
	В равенстве (19) использовалась монотонность $b_n$ В силу произвольности $\epsilon > 0$ из (19) по критерию Коши следует сходимость ряда (15). 
\end{proof}
\section{Признак Дирихле}
\begin{theorem}
Пусть $\exists L > 0$ т.ч. $\forall n$ выполнено $|\sum_{k= 1}^{n}{a_k}| \leq L$ и пусть $b_n$ монотонна и $b_n \underset{n \to \infty}{\to} 0$. Тогда ряд (15) сходится.
\end{theorem}
\begin{proof}
	Возьмем $\forall \epsilon < 0$, выберем N так, чтобы $\forall n > N$ выполнялось $|b_n| < \epsilon$ Положим опять $A_n = a_n, A_{n+1} = a_n + a_{n+1},...A_m = a_n+...+a_m$. В силу условия имеем при $k \geq n: |A_k|= |\sum_{r = 1}^{k}{a_r} - \sum_{r = 1}^{n-1}{a_r}| \leq |\sum_{r = 1}^{k}{a_r}|+ |\sum_{r = 1}^{n-1}{a_r}| \leq 2L \ (20) \\$ 
	Применяя преобразования Абеля (17) и используя условие $|b_k| < \epsilon, k > n,$ и (20), получим $|a_{n+1}b_{n+1}+...+a_mb_m|\leq |A_nb_{n+1}| + \sum_{k = n+1}^{m-1}{|A_k||b_k - b_{k+1}|+|A_mb_m|<2\epsilon L + 2L \sum_{k = n+1}^{m-1}{|b_k-b_{k+1}|}} + 2L \epsilon = 4\epsilon L + 2L|\sum_{k = n+1}^{m-1}{(b_k-b_{k+1})}| = 4\epsilon L + 2L|b_{n+1}-b_m| < 4\epsilon L + 2L \cdot 2 \epsilon = 8\epsilon L \ \ \ (21) \\$
	В силу произвольности $\epsilon > 0$ применение критерия Коши к (21) показывает, что ряд (15) сходится.
\end{proof}