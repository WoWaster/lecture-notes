\chapter{Интеграл Римана-Стильтьеса}
$\\$ \textbf{!!! Все рассматриваемые функции ограничение}
$\\$ Разбиением промежутка [a,b] будем называть множество $x_{0},...x_{n} \ n \geq 1  \ \ a = x_{0} < ... < x_{n} = b$ Обозначание: \underline{P}(position) $\Delta x_{i} := x_{i+1} -x_{i} $ $m_{i} = inf\ f(x)\ x \in [x_{i},x_{i+1}]$ \ \ $M_{i} = sup\ f(x) \ x \in [x_{i},x_{i+1}] m_{i}\geq M_{i}$
$\\$ $U(f,\underline{P}) = \sum_{i = 0}^{n-1}M_{i}\Delta x_{i}$
$\\$ $L(f,\underline{P})= \sum_{i = 0}^{n-1}m_{i}\Delta x_{i} $ \ \ \ \ \ \ \ \ \ \ \ (*)
$\\$ $U(f,\underline{P})\geq L(f,\underline{P})\ \forall p $
$\\$ \begin{definition} Верхний интеграл Римана: $\int\limits_{a}^{\overline{b}}{f}dx = inf\ U(f,\underline{P})$
$\\$ Нижний интеграл Римана: $\int\limits_{\underline{a}}^{b}{f}dx = sup\ L(f,\underline{P})$ будет доказано, что $\int\limits_{a}^{\overline{b}}{f}dx \geq \int\limits_{\underline{a}}^{b}{f}dx$
\end{definition}
\begin{definition} Функция f интегрируема по Риману на [a,b] если $\int\limits_{\underline{a}}^{b}{f}dx = \int\limits_{a}^{\overline{b}}{f}dx =: \int\limits_{a}^{b}{f}dx$ \ \ $f \in \mathbb{R}$
\end{definition}
$\\$ $[a,b] , \underline{P} = \bigcup_{j = 0}^{n}{x_{j}}$ \ \ $a = x_{0}<x_{1}<...<x_{n}=b$
$\\$ f - функция, опред. на [a,b] $\alpha : \alpha(x)\leq \alpha(y) \ \forall x<y \ \ (1)$
$\\$ $\Delta x_{i}= x_{i+1}-x_{i} \ m_{i} = \underset{x \in [x_{i},x_{i+1}]}{inf\ f(x)};\ \  M_{i} = \underset{x \in [x_{i},x_{i+1}]}{sup\ f(x)} \ \ (2)$
$\\$ Верхняя сумма Римана-стильтьеса: U(f,$\alpha,\underline{P})=\sum^{n-1}_{i =0}M_{i}(\alpha(x_{i+1})-\alpha(x_{i})) \ \ \ (3)$
$\\$ Нижняя сумма: $L(f,\alpha,\underline{P}) = \sum_{i = 0}^{n-1}m_{i}(\alpha(x_{i+1})-\alpha(x_{i})) \ \ \ (4)$
$\\$ $\Delta\alpha(x_{i})=\alpha(x_{i+1})-\alpha(x_{i}) \ \ (5) $
$\\$ Если $\alpha(x)\equiv x$, то см. предыдущие опр-я (*)
\section{Свойства сумм Римана-Стильтьеса}
$\\$ 
\begin{enumerate}
	\item$ (1) \Rightarrow \Delta\alpha(x_{i}) \geq 0 \ m_{i}\leq M_{i} \Rightarrow \sum_{i=0}^{n-1}m_{i}\Delta\alpha(x_{i})\leq \sum_{i =0}^{n-1}M_{	i}\Delta\alpha(x_{i})
	   \\ \underline{L(f,\alpha,\underline{P}) \leq U(f,\alpha,\underline{P})}$
	\item $\exists m,M : \forall x\ m\leq f(x) \leq M \Rightarrow m \leq m_{i}; \ \ M_{i}\leq M \forall i (8)
		\\ m\sum_{i=0}^{n-1}(\alpha(x_{i+1})-\alpha(x_{i}))=m(\alpha(b)-\alpha(a)) = \sum_{i= 0}^{n-1}m_{i}\Delta\alpha(x_{i})\leq \sum_{i=1}^{n-1}M\Delta\alpha(x_{0})=M(\alpha(b)-\alpha(a)) \ \ \ (9)
		\\$
	\item \underline{P} \ \ $x_{i_{0}}< x^{*} <x_{i_{0}\neq 1} $
	$\\$ $\underline{P^{*}}=\underline{P}\cup {x^{*}} $, тогда $L(f,\alpha,\underline{P})\leq L(f,\alpha,\underline{P^{*}}) \ \ (10)$
	$\\$ $U(f,\alpha,\underline{P^*}) \leq U(f,\alpha,\underline{P}) \ \ (11)$
	$\\$ \underline{оказательство.} $m_{i}, M_{i}\ \  i \neq i_{0}\  m'_{i_{0}}:= \underset{x \in[x_{i_{0}},x^{*}]}{f(x)} \ \ m''_{i_{0}}:=\underset{x \in [x^{*},x_{i_{0}}]}{inf\ f(x)}$
	$\\$ $L(f,\alpha,\underline{P^{*}})-L(f,\alpha,\underline{P})=m'_{i_{0}}(\alpha(x^{*})-\alpha(x_{i_{0}}))+m^{n}_{i_{0}}(\alpha(x_{i_{0}\neq 1})-\alpha(x^*)) - m_{i_{0}}(\alpha(x_{i_{0}\neq 1})-\alpha(x_{i_{0}})) \textbf{=}$
	$\\$ $\alpha(x_{i_{0}\neq 1})-\alpha(x_{i_{0}})= (\alpha(x_{i_{0}\neq 1})-\alpha(x^*))+(\alpha(x^*)-\alpha(x_{i_{0}}))$
	$\\$ \textbf{=} $(m'_{i_{0}}-m_{i_{0}})(\alpha(x^*)-\alpha(x_{i_{0}}))+(m''_{i_{0}}-m_{i_{0}})(\alpha(x_{i_{0}+1})-\alpha(x^*)) \geq 0$ т.к $m_{i_{0}}\geq m'_{i} \ \ m_{i_{0}}\leq m''_{i_{0}} \ \ (12)  \qed $
\end{enumerate}
$\\$ \begin{remark} $\alpha(x) \equiv x$, то частным случаемм будет: $L(f,\underline{P})\leq L(f,\underline{P^*})$ \ \ $U(f,\underline{P} \geq U(f,\underline{P^*})$
\end{remark}
$\\$ \begin{definition} $P_{2}$ - измельчение разбиения $P_{1}$, если все точки $P_{1}$ содрж. в $P_{2} \ \ (P_{1}\subset P_{2})$ 
\end{definition}
$\\$ $L(f,\alpha,P_{1}) \leq L(f,\alpha,P_{2})$ \ \ (13)
$\\$ $I(f,\alpha,P_{1}) \geq U(f,\alpha,P_{2})$ \ \ (14) (док-во: см. п. 3)
$\\$ 4! \ $P_{1}, P_{2}$- произвольные разбиения, тогда справедливо неравенство: $L(f,\alpha,P_{1})\leq U(f,\alpha,P_{2}) \ \ (15)$
$\\$ \begin{proof}
$\\$ $P = P_{1}\cup P_{2}$ не огр. общности $P_{1}neq P_{2}$, тогда P - измельчение как $P_{1}$, так и $P_{2}$, тогда
$\\$ $L(f,\alpha,P_{1})\leq L(f,\alpha,P)\leq U(f,\alpha,P)\leq U(f,\alpha,P_{2}) \ $
\end{proof}
$\\$ \begin{remark} $L(f,P_{1})\leq U(f,P_{2}) $ Терминолог. замечание: Иногда данные суммы наз.суммами Дарбу(так было бы исторически правильнее)
\end{remark}
\section{Определение интеграла Римана-Стильтьесса}
$\\$ В соотнош. (15) фиксируем разбиение $P_{2}$, тогда $-\infty < \underset{P_{1}}{sup\ L(f,\alpha,P_{1})}\leq U(f,\alpha,P_{2}) \ \ (16)$
$\\$ \begin{definition}Нижний интеграл Римана-Стильтьеса 
$\\$ $\underline{\int{}}fd\alpha := \underset{P_{1}}{sup\ L(f,\alpha,P_{1})}$(17)
$\\$ $-\infty< \underline{\int{}}fd\alpha \leq U(f,\alpha,P_{2}) < +\infty$  (17')
$\\$ (17') $\Rightarrow \underline{\int{}}d\alpha \leq \underset{P_{2}}{inf\ U(f,\alpha,P_{2}) < +\infty}  \ \ (18)$
\end{definition}
$\\$ \begin{definition}Верхний интеграл Римана-Стильтьеса
$\\$ $\overline{\int{}}fd\alpha:= \underset{P_{2}}{inf\ U(f,\alpha,P_{2})} \ \ \ (19)$
$\\$ (18),(19) $\Rightarrow \underline{\int{}}fd\alpha \leq \overline{\int{}}fd\alpha \ \ (20)$
\end{definition}
$\\$ \begin{theorem} $\forall $ огранич. ф-ции f и  $\forall$ возраст. ф-ции $\alpha \exists$ нижний и верхний интегралы Римана-Стильтьеса и между ними справедл. соотн. (20) 
\end{theorem}
$\\$ \begin{remark}$\alpha(x)\equiv x$, то $\int\limits_{\underline{a}}^{b}{f}dx \leq \int\limits_{a}^{\underline{b}}{f}dx$ \ \ \ (20')
\end{remark}
$\\$ \begin{definition} f интегрируема по Риману -стильбтьесу с весом $\alpha$ на [a,b], если $\underline{\int{}}fd\alpha = \overline{\int{}}fd\alpha $ \ \ (21) $\stackrel{def}{\Leftrightarrow} f \in R(\alpha) $
$\\$ $\int\limits_{a}^{b}{fd\alpha}:= \underline{\int{}}fd\alpha = \overline{\int{}}fd\alpha \ \ \ (22)$ 
\end{definition}
\section {Критерий интегрируемости по Риману-Стильтьесу с весом $\alpha$}
$\\$ \begin{theorem}$f \in R(\alpha) \Leftrightarrow \forall \epsilon \ > 0$ $\exists$ разбиение P: $U(f,\alpha,P)-L(f,\alpha,P)<\epsilon$ \ \ (23)
\end{theorem}
$\\$ \begin{proof}(Достаточность)
$\\$ $L(f,\alpha,P) \leq \underline{\int{}}fd\alpha \leq \overline{\int{}}fd\alpha \leq U(f,\alpha,P) \ \ \Rightarrow 0 \underset{(20)}{\leq} \overline{\int{}}fd\alpha - \underline{int{}}fd\alpha \leq U(f,\alpha,P)-L(f,\alpha,P) \underset{(23)}{\epsilon} \forall \epsilon \ \ (24) \ \ \Rightarrow \overline{\int{}}fd\alpha = \underline{\int{}}fd\alpha$
$\\$ (Необходимость) f - инт. по Риману- Стиотитесу с весом $\alpha$ Фиксируем $\epsilon > 0$
$\\$ $\exists P_{1}: U(f,\alpha,P_{1} < \overline{\int{}}fd\alpha + \frac{\epsilon}{2} \ \ (25)$
$\\$ $\exists P_{2}: L(f,\alpha,P_{2} > \underline{\int{}}fd\alpha - \frac{\epsilon}{2} \ \ (26)$
$\\ P = P_{1} \cup P_{2}$ Eсли $P_{1}\neq P_{2}$, то P- измельчение $P_{1}$ и $P_{2}$ (Если $P_{1} = P_{2}$, то все просто) По сл. из свойства 3
$\\$ $L(f,\alpha,P_{2}) \leq L(f,\alpha,P) \leq \underline{\int{}}fd\alpha \leq \overline{\int{}}fd\alpha \leq U(f,\alpha,P)\leq U(f,\alpha,P_{1}) \ \  (27)$
$\\$ $\underline{\int{}}fd\alpha - \frac{\epsilon}{2} < L(f,\alpha,P) \leq  \int{}fd\alpha \leq U(f,\alpha,P) < \overline{\int{}}fd\alpha + \frac{\epsilon}{2}$
$\\$ $U(f,\alpha,P) - L(f,\alpha,P) < (\overline{\int{}}fdx + \frac{\epsilon}{2})-(\underline{\int{}}fdx-\frac{\epsilon}{2}) < \epsilon$
\end{proof}
$\\$ \begin{remark} $\alpha(x) = x: f \in R \Leftrightarrow\  \forall \epsilon > 0\ \exists P : U(f,P) - L(f,P) < \epsilon \ \ (28)$
$\\ $ Кроме того $L(f,\alpha,P) \leq \int{f}d\alpha \leq U(f,\alpha,P) \ \ (29)$ 
$\\ $ В частности$L(f, P) \leq \int{f}dx \leq U(f,P)$ \ \ \ (30)
\end{remark}
\section{Достаточные признаки интегрируемости функции по Риману-Стильтесу}
$\\$ f опред. на [a,b], $\alpha$ - монотонная функция
$\\$ \begin{assertion} Если $f \in C([a,b])$, то $f \in R(\alpha)$
$\\$ Пусть P - разбиение [a,b]: a = $x_{0}<x_{1}<...<x_{n}= b $
$\\$ Диаметром разбиения будем называть $\mu(P) = \underset{0\leq i\leq n-1}{max}\Delta x_{i}$; Докажем более сильное утверждение
\end{assertion}
$\\$ \begin{assertion} Пусть $f \in C([a,b]),$тогда $\forall \epsilon > 0 \ \exists \delta > 0: \forall $ разбиения P $\mu(P)<\delta$ и $\forall t_{i}\int[x_{i},x_{i+1}]$
$\\$ |$\sum_{i =0}^{n-1}f(t_{i})\Delta\alpha(x_{i}) - \int\limits_{a}^{b}{fd\alpha}$| < $\epsilon$ \ \ \ \ (1)
\end{assertion}
$\\$ \begin{proof} P произв. разбиение $\exists t'_{i}\int[x_{i},x_{i+1}] \\
(2) \ \ \  f(t'_{i}) = \underset{t\in[x_{i},x_{i+1}]}{min}f(t)$ (по 2-ой Т. Вейерштрасса)
$\\$ $\exists t''_{i} \in [x_{i},x_{i+1}] $
$\\$ (3) \ \ \ $f(t''_{i}) = \underset{t \in [x_{i},x_{i+1}]}{max}f(t) $
$\\$
$\left.
  \begin{array}{ccc}
    (2),(3) \Rightarrow L(f,\alpha,P)=\sum_{i=0}^{n-1}f(t'_{i})\Delta\alpha(x_{i}) \\
    \ \ \ \ \ \ \ \ \ \ \ \ \ \ \ U(f,\alpha,P)=\sum_{i=0}^{n-1}f(t''_{i})\Delta\alpha(x_{i})
  \end{array}
\right\}$ (4)
$\\$ (4) $\Rightarrow U(f,\alpha,P) -L(f,\alpha,P) = \sum_{i =0}^{n-1}(f(t''_{i})-f(t'_{i}))\Delta\alpha(x_{i})) \ \ (5) $ не огр. общности $\alpha(x) \neq const$
$\\$ Применим Т.Кантора: Пусть $\sigma: \sigma(\alpha(b)-\alpha(a)) < \epsilon $ \ \ \ (6)
$\\$ Выберем $\delta:\ \forall x \in [a,b],\ \forall y_{1},y_{2} \in [x,x+\delta] \ \ |f(y_{1})-f(y_{2}) |<\sigma \ \ \ (7) $ Наложим ограничение на P: $\mu(P)<\delta \ \ (8)$
$\\$ (8) $\Rightarrow$ $\forall i \ \ |t''_{i}-t'_{i}|\leq \Delta x_{i} \leq \mu(P) < \delta \ \ (9)$
$\\$ (5),(7) $\Rightarrow$ $U(f,\alpha,P) - L(f,\alpha,P) \leq \sum_{i =0}^{n-1}\sigma\Delta\alpha(x_{i}) = \sigma(\alpha(b)-\alpha(a))<\epsilon \Rightarrow$ f интегрируема по Риману-Стильтьесу с весом $\alpha$ 
$\\$ Докажем соотношение 1: Пусть $t_{i} \in [x_{i},x_{i+1}] f(t'_{i})\leq f(t_{i}) \leq f(t''_{i})
\overset{(4)}{\Rightarrow} 
\left.
  \begin{array}{ccc}
      \sum_{i=0}^{n-1}f(t_{i})\Delta\alpha(x_{i}) \leq U(f,\alpha,P) \\
    \sum_{i=0}^{n-1}f(t_{i})\Delta\alpha(x_{i}) \geq L(f,\alpha,P)
  \end{array}
\right\}$ (10)
$\\$ $L(f,\alpha,P) \leq \int\limits_{a}^{b}{fd\alpha} \leq U(f,\alpha,P) \ \ \ (11)$
$\\$ сравнивая (10) и (11) получим:
$\\$ $\sum_{i=0}^{n-1}f(t_{i})\Delta\alpha(x_{i}) - \int\limits_{a}^{b}{fd\alpha}|\leq U(f,\alpha,P)-L(f,\alpha,P) < \epsilon$
\end{proof}
$\\$ \begin{remark} Если f непрерывна на [a,b], то она интегрируема по Риману
\end{remark}
$\\$\begin{assertion}]Пусть $\alpha \in C([a,b])$ и монотонна на [a,b] f - монотонна и определена на [a,b] (без требования непрерывности), тогда $f \in R(\alpha)$
\end{assertion}
$\\$ \begin{proof} Не огр. общ. $\alpha \neq const$  Пусть P-разбиение a =$x_{0}<x_{1}<...<x_{n} $=b
$\\$ $\alpha(x_{1})-\alpha(x_{0}) = \frac{\alpha(b)-\alpha(a)}{n} \ \ (12)$
$\\$ $k<n-2 \ \ \ \ \alpha(x_{k+1})-\alpha(x_{k}) = \frac{\alpha(b)-\alpha(a)}{n} \  \ \ \ (13)$
$\\$ $\alpha(x_{n})-\alpha(x_{n+1}) = \frac{\alpha(b)-\alpha(a)}{n} \ \ \ (14)$
$\\$ $(12) \Leftrightarrow \alpha(x_{1} = \alpha(a)+\frac{\alpha(b)-\alpha(a)}{n}) \ \ \ \alpha(a)<\alpha(x_{1})\alpha(b)$ .............
$\\$ Пусть P определено в (12)-(14)  $\forall t \in [x_{i},x_{i+1}] $ $f(x_{i})\leq f(t)\leq f(x_{i+1}) \ \ \ (15) $
$\\$ (15) $\Rightarrow
\left.
  \begin{array}{ccc}
      L(f,\alpha,P) = \sum_{i=0}^{n-1}f(x_{i})\Delta\alpha(x_{i})
    U(f,\alpha,P) = \sum_{i=0}^{n-1}f(x_{i+1})\Delta\alpha(x_{i})
  \end{array}
\right\}$
$\\$ $ U(f,\alpha,P)-L(f,\alpha,P) = \sum_{i=0}^{n-1}(f(x_{i+1})-f(x_{i}))\Delta\alpha(x_{i}) \textbf{=}$
$\\$ (12)-(14) $\Rightarrow \Delta\alpha(x_{i}) = \frac{\alpha(b)-\alpha(a)}{n} \ \ \ (16)$
$\\$ $\textbf{=} \frac{\alpha(b)-\alpha(a)}{n}\sum_{i = 0}^{n-1}(f(x_{i+1})-f(x_{i}))=\frac{(\alpha(b)-\alpha(a))(f(b)-f(a))}{n \epsilon \ \ (17)}$
\end{proof}
$\\$ \begin{corollary}[Замечание] f - монотонна $\Rightarrow$ f интегрируема по Риману на промежутке [a,b]
\end{corollary}
\section{Свойства интеграла Римана-Стильтьеса}
\begin{enumerate}
	\item $f_{1},f_{2} \in R(\alpha)\Rightarrow f_{1}+f_{2} \in R(\alpha) \\
	\int{(f_{1}+f_{2})}d\alpha=\int{f_{1}}d\alpha+\int{f_{2}}d\alpha$
	\item $f\in R(\alpha) \Rightarrow cf \in R(\alpha) \\
	\int{(cf)}d\alpha = c\int{f}d\alpha$
	\item $f_{1},f_{2}\in R(\alpha)$ и $f_{1}(x)\leq f_{2}(x) \forall x \in [a,b] \Rightarrow \int{f_{1}}d\alpha \leq \int{f_{2}}d\alpha$
	\item $f \in R(\alpha[a,b])$ - означ, что f интегр. на промежутке [a,b] $\forall c \in (a,b) \ \ f\in R(\alpha[a,c]), \ f\in R(\alpha[c,b]) \\
	\int\limits_{a}^{b}{f}d\alpha = \int\limits_{a}^{c}{f}d\alpha + \int\limits_{c}^{b}{f}d\alpha$
	\item $f \in R(\alpha,[a,b])$ и |f(x)|$\leq M$ $\forall x\in [a,b] \Rightarrow |\int{f}d\alpha|\leq M(\alpha(b)-\alpha(a))$
	\item $f \in R(\alpha_{1}), f\in R(\alpha_{2}), f \in R(\alpha_{1}+\alpha_{2}) \ \ \int{f}d(\alpha_{1}+\alpha_{2}) = \int{f}d(\alpha_{1})+\int{f}d(\alpha_{2})$
	\item $f \in R(\alpha), c\geq 0 \Rightarrow f\in R(c\alpha)$ и $\int{f}d(c\alpha)=c\int{f}d\alpha$
\end{enumerate}

$\\$\begin{lemma} Пусть $f_{1},f_{2} $ определены и огранич. на [x,y], тогда справедливы неравенства:
$\\$ $ \underset{t \in [x,y]}{inf}f_{1}(t) + \underset{t \in [x,y]}{inf}f_{2}(t) \leq \underset{t \in [x,y]}{inf}(f_{1}(t)+ f_{2}(t)) \ \ \ (18)$
$\\$ $ \underset{t \in [x,y]}{sup}f_{1}(t) + \underset{t \in [x,y]}{sup}f_{2}(t) \geq \underset{t \in [x,y]}{sup}(f_{1}(t)+ f_{2}(t)) \ \ \ (19)$
\end{lemma}
$\\$ \begin{proof} докажем (18)
$\\$ Пусть $\eta > 0$ $\exists t_{0} \in [x,y]: f_{1}(t_{0}) + f_{2}(t_{0}) < \underset{t\in [x,y]}{inf}(f_{1}(t)+f_{2}(t)) + \eta  \ \ \ (20)$
$\\$ $f_{1}(t_{0}) \geq \underset{t\in [x,y]}{inf}(f_{1}(t)) \ \ \ (21)$
$\\$ $f_{2}(t_{0}) \geq \underset{t\in [x,y]}{inf}(f_{2}(t)) \ \ \ (22)$
$\\$ (20)-(22) $\Rightarrow inf f_{1}+inf f_{2} \leq inf (f_{1}+f_{2} + \eta ) \qed$ 
$\\$ Применим Лемму для промеж $[x_{i},x_{i+1}]$
$\\$ (18),(19) $\Rightarrow$ $L(f_{1},\alpha,P) + L(f_{2},\alpha,P) \leq  L(f_{1}+f_{2} ,\alpha,P) \leq U(f_{1}+f_{2} ,\alpha,P) \leq U(f_{1},\alpha,P) + U(f_{2},\alpha,P) \ \ \ (24)$
$\\$ Выберем разбиение $P_{1}$ : 
$\\$ $\overset{(1)}{\leq} U(f,\alpha,P_{1})- L(f_{1},\alpha,P_{1}) < \frac{\epsilon}{2} \ \ \ (25) P_{2}:$
$\\$ $\overset{(2)}{\leq} U(f,\alpha,P_{2}) -L(f_{1},\alpha,P_{2}) < \frac{\epsilon}{2}$ \ \ \ (26)
$\\$ Пусть P = $P_{1} \cup P_{2}, P-$ измельчение $P_{1}$ и $P_{2}$, тогда
$\\$ $ U(f_{1},\alpha,P) - L(f_{1},\alpha,P) \overset{(1)}{\leq}$
$\\$ $ U(f_{2},\alpha,P) - L(f_{1},\alpha,P) \overset{(2)}{\leq}$
$\\$ (24)-(26) $U(f_{1}+f_{2},\alpha,P)-L(f_{1}+f_{2},\alpha,P)\leq (U(f_{1},\alpha,P)-L(f_{1},\alpha,P))+(U(f_{2},\alpha,P)-L(f_{2},\alpha,P)) < \frac{\epsilon}{2} + \frac{\epsilon}{2} = \epsilon$
$\\$ $f_{1} \in R(\alpha) f_{2} \in R(\alpha) \Rightarrow f_{1}+f_{2} \in R(\alpha)$
$\\$ $\forall \epsilon > 0$ Выберем разбиения $P_{1},P_{2}$; 
$\\$ $
\left.
  \begin{array}{ccc}
      L(f_{1},\alpha,P_{1})-\frac{\epsilon}{2} < \int{f_{1}}d\alpha < L(f_{1},\alpha,P_{1}) + \frac{\epsilon}{2} \\
    U(f_{2},\alpha,P_{2})-\frac{\epsilon}{2} < \int{f_{2}}d\alpha < L(f_{2},\alpha,P_{2}) + \frac{\epsilon}{2}
  \end{array}
\right\}$ (27)
$\\$ Пусть P = $P_{1}\cup P_{2}$
$\\$ (26),(27) $\Rightarrow \int{f_{1}}d\alpha + \int{f_{2}}d\alpha - \epsilon < \int{f_{1}+f_{2}}d\alpha < \int{f_{1}}d\alpha + \int{f_{2}}d\alpha + \epsilon \ \ \ (28)$ т.к. $L(f_{1}+f_{2},\alpha,P)\leq \int{f_{1}+f_{2}}d\alpha \leq U(f_{1}+f_{2},\alpha,P)$ см. (24)
(28) $\Rightarrow \int{(f_{1}+f_{2})}d\alpha = \int{f_{1}}d\alpha + \int{f_{2}}d\alpha \qed$
\end{proof}
\begin{theorem} Пусть $f \in R(\alpha)$ (f опред. на [a,b]) $m \leq f(x)\leq M\ \forall x \in [a,b]$ Пусть $\varphi \in C([m,M])$, тогда $\varphi(f(x))\in R(\alpha)$
\end{theorem}
 \begin{corollary}
\begin{enumerate}
	\item $f \in R(\alpha) \Rightarrow f^2 \in R(\alpha)$
	\item $f,g \in R(\alpha) \Rightarrow fg = \frac{1}{4}((f+g)^2-(f-g)^2)\in R(\alpha) $
	\item $f\in R(\alpha) \Rightarrow |f| \in R(\alpha)$
\end{enumerate}
\end{corollary}
\begin{remark} $f \in R(\alpha) |\int{f}d\alpha|\leq \int{|f|}d\alpha$
\end{remark}
\begin{proof}[Замечания]
$\\$ Пусть $k \in {-1,1}: \ \ k\int{f}d\alpha = |\int{f}d\alpha| \ \ |\int{f}|d\alpha = k\int{f}d\alpha = \int{(kf)}d\alpha \leq \int{|f|}d\alpha \qed$
\end{proof}
\section{Интеграл Римана- Стильтьеса как предел интегральных сумм.}
$\\$ $f \in C([a,b]) \Rightarrow f\in R(\alpha) $ было доказано: $\forall \epsilon > 0 \ \exists \delta > 0 : \forall P (P = \bigcup_{i=0}^{n}{x_{i}}), \mu(P) < \delta  \forall t_{i} \in [x_{i},x_{i+1}]$
$\\$ $|\int{f}d\alpha - \sum_{i =0}^{n-1}f(t_{i})\Delta\alpha(x_{i})|< \epsilon$  
\begin{definition}Пусть даны фунции f, $\alpha$ (монот. возр) разбиение P: $x_{0}<x_{1}<...<x_{n}$ набор T: $t_{0}\leq t_{1} \leq ... \leq t_{n} \  \ \ x_{i}\leq t_{i} \leq x_{i+1} \ \ (1)$
$\\$ Интегральной суммой Римана- Стильтьеса будем называть : $S(f,\alpha,P,T):= \sum_{i =0}^{n-1}f(t_{i})\Delta\alpha(x_{i}) \ \ \ (2)$
$\\$ Ясно, что $|\int{f}d\alpha - S(f,\alpha,P,T)|<\epsilon$
\end{definition}
\begin{definition}Пусть f опр. на [a,b], $\alpha$ монот. возр. будем говорить, что $\exists$ предел интегральных сумм Римана-Стильтьесса ф-ции f с вессом $\alpha$, если  $\exists I > 0:\ \forall \epsilon > 0 \ \exists \delta > 0:\ \forall P\ (\mu(P)<\delta)$ и $\forall$ набора точек T выполнено неравенство $|I - S(f,\alpha,P,T)|<\epsilon \ \ (3)$
$\\$ I- предел интегральных сумм Римана- Стильтьеса функции f с весом $\alpha$
$\\$ $I:= \lim\limits_{\mu(P) \to 0}S(f,\alpha,P,T) \ \ \ (4)$
\end{definition}
\section{Утверждения о пределах интегральных сумм Римана- Стильтьеса}
\begin{theorem}$f \in C([a,b]) \ \ \int{f}d\alpha = \lim\limits_{\mu(P) \to 0}S(f,\alpha,P,T)$
\end{theorem}
\begin{theorem} f - ограничена на [a,b] $\alpha$ - возраст.
$\\$ $\exists I = \lim\limits_{\mu(P) \to 0}S(f,\alpha,P,T) \ \ \ (5) $
$\\$ (5) $\Rightarrow f \in R(\alpha)$ и I = $\int{f}d\alpha \ \ \ (6) $
\end{theorem}
\begin{theorem} Пусть $f \in R(\alpha)$ Пусть $\alpha$ непрерывна на [a,b] (7), тогда $\exists$ предел интегральных сумм и $\int{f}d\alpha = \lim\limits_{\mu(P) \to 0}S(f,\alpha,P,T) \ \ \ (8) $
$\\$ \underline{Замечание.} Пусть $\alpha(x) \equiv x$ $f\in R([a,b])$ - функция интегр. по Риману.$ \int\limits_{a}^{b}{f}dx = \lim\limits_{\mu(P) \to 0}S(f,\alpha,P,T) \ \ \ (8')$
\end{theorem}
\section{ интеграл римана с переменным верхним пределом}
$\\$ Пусть $f \in R([a,b]) \Rightarrow f\in R([a,x]) \ \forall x\in (a,b)$
$\\$ Определим новую ф-цию $\forall x \in (a,b]:$ Ф(x) = $\int\limits_{a}^{x}{f}dt \ \ (9)$ - интеграл Римана с перем. верхним пределом
$\\$ x = a: Ф(a):= 0 (10)
$\\$ \begin{theorem}\ \ \  1) Ф $\in C([a,b])$ \\ 2) $\forall x_{0}\in(a,b):$ непр. в $x_{0}$ \\ $\exists$ Ф'($x_{0}$)=f($x_{0}$) \ \ (11)
\end{theorem}
\begin{proof} 1) $\exists M : \forall x \ \ |f(x)|\leq M \ \ (12)$
$\\$ рассмотрим $x_{1},x_{2}: a\leq x_{1}< x_{2}\leq b] $
$\\$ Ф($x_{2}$) = $\int\limits_{a}^{x_{2}}{f}dt = \int\limits_{a}^{x_{1}}{f}dt+\int\limits_{x_{1}}^{x_{2}}{f}dt =$Ф($x_{1}$)+$\int\limits_{x_{1}}^{x_{2}}{f}dt$
$\\$ Ф($x_{2}$)-Ф($x_{1}$) = $\int\limits_{x_{1}}^{x_{2}}{f}dt \ \ \ (13)$
$\\$ (12),(13) $\Rightarrow$ |Ф($x_{2}$)-Ф($x_{1}$)| = $|\int\limits_{x_{1}}^{x_{2}}{f}dt| \leq \int\limits_{x_{1}}^{x_{2}}{|f|}dt \leq \int\limits_{x_{1}}^{x_{2}}{M} = M(x_{2}-x_{1}) \ \ \ (14)$
$\\$ (14)  $\Rightarrow$ Ф $\in C([a,b])$
$\\$ 2) $\forall \epsilon > 0\ \exists \delta > 0 : \forall x\in (x_{0}-\delta,x_{0}+\delta) \ \ |f(x)-f(x_{0})|<\epsilon \ \ \ \ (15)$
$\\$ Рассмотрим $\frac{\text{Ф}(x_{0}+h)-\text{Ф}(x_{0})}{n}$ \ \ \ \ Пусть $-\delta<h<0 \ \ (16)$
$\\$ (13) $\Rightarrow$ Ф($x_{0}$)-Ф($x_{0}+h$)=$\int\limits_{x_{0}+h}^{x}{f}dt=\int\limits_{x_{0}+h}^{x_{0}}{(f-f(x_{0})+f(x_{0}))}dt = \int\limits_{x_{0}+h}^{x_{0}}{(f-f(x_{0}))}dt+\int\limits_{x_{0}+h}^{x_{0}}{f(x_{0})}dt$
$\\$ $\int\limits_{x_{0}+h}^{x_{0}}{f-f(x_{0})}dt + f(x_{0})\cdot (-h) \ \ \ (17) \Rightarrow\ \frac{\text{Ф}(x_{0}+h)-\text{Ф}(x_{0})}{n} = f(x_{0}) - \frac{1}{n}\int\limits_{x_{0}+h}^{x_{0}}{(f-f(x_{0}))}dt \ \ \ (18)$
$\\$ (15) \ \ $\Rightarrow$ $|-\frac{1}{n}\int\limits_{x_{0}+h}^{x_{0}}{(f-f(x_{0}))}dt|\leq \frac{1}{|n|}\int\limits_{x_{0}+h}^{x_{0}}{|f-f(x_{0})|}dt \leq \frac{1}{|n|}\int\limits_{x_{0}+h}^{x_{0}}{\epsilon dt}= \epsilon \ \ \ (19)$   
$\\$ (18),(19) $\Rightarrow$ $|\frac{\text{Ф}(x_{0}+h)-\text{Ф}(x_{0})}{n} - f(x_{0})|\leq \epsilon \ \ \ (20) \Rightarrow (11)$, что верно для $\forall h : 0 0<|n|< \delta$ (доказ. аналог)
\end{proof}
\begin{corollary} $f \in C([a,b]) \Rightarrow f \in R([a,b]) \ \ \text{Ф}(x) = \int\limits_{a}^{b}{f}dt$ $\forall x_{0} \in (a,b) \ \ \exists \text{Ф'}(x_{0}) = f(x_{0}) \ \ \text{Ф} \in C([a,b])$
\end{corollary}
\section{Формула Ньютона-Лейбница}
\begin{theorem} $f \in R([a,b])$ Ф'(x) = f(x) Ф $\in C([a,b])$, тогда $\int\limits_{a}^{b}{f}dx$ = Ф(b) - Ф(a) \ \ (1)
\end{theorem}
$\\$ \begin{proof} $\int\limits_{a}^{b}{f}dx = \lim\limits_{\mu(P) \to 0}S(f,x,P,T) \ \  \ (2)$
$\\$ (2) означает: $\forall \epsilon > 0 \ \exists \delta > 0: \forall P\ \mu(P)<\delta \ \ P = \bigcup_{i=0}^{n}{x_{i}} \ \ \forall t_{i} \in [x_{i}, x_{i+1}] \ \ 0\leq l \leq n-1$
$\\$ $|\int\limits_{a}^{b}{f}dx - \sum_{i = 0}^{n-1}f(t_{i})\Delta x_{i}|< \epsilon \ \ (3)$
$\\$ Ф(b) - Ф(a) = $\sum_{i =0}^{n-1}(\text{Ф}(x_{i+1})-\text{Ф}(x_{i})) \textbf{=}$
$\\$ По т. Лагранжа $\forall i \exists t_{i} \in (x_{i},x_{i+1}): \text{Ф}(x_{i+1})-\text{Ф}(x_{i}) = \text{Ф}'(t_{i})\Delta x_{i} \ \ \ (4)$
$\\$ $\textbf{=} \sum_{i =0}^{n-1}\text{Ф}'(t_{i})\Delta x_{i} \underset{(1)}{=} \sum^{n-1}_{i = 0}f(t_{i})\Delta x_{i} = S(f,x,P,T)$
$\\$ В силу произвольности набора T  соотношении 3:
$\\$ $|\int\limits_{a}^{b}{f}dx -(\text{Ф}(a)-\text{Ф}(b))|<\epsilon \ \forall \epsilon > 0 \ \ \  (7) \Rightarrow \int\limits_{a}^{b}{f}dx$ = Ф(b)-Ф(а)
\end{proof} 
\section{Интеграл Римана-Стильтьеса как интеграл римана}
\begin{theorem}$f \in R([a,b]) \ \alpha \in C([a,b])$
$\\$ $\forall x \in (a,b) \ \exists \alpha'(x): \alpha'\in R([a,b])$, тогда f $\in R(\alpha)$ и $\int\limits_{a}^{b}{f}d\alpha = \int\limits_{a}^{b}{f\alpha'}dx$
\end{theorem}
$\\$ \begin{proof} Если доказать, что $\lim\limits_{\mu(P) \to 0}{S(f,\alpha,P,T)} = I_{0} \ \ (9)$, то из этого будет следовать, что $f \in R(\alpha)$ и $\int\limits_{a}^{b}{f}d\alpha = I_{0} \ \ (10)$. Ясно, что $f\cdot \alpha' \in R([a,b])$ Положим $I:= \int\limits_{a}^{b}{f}\alpha'dx \ \ (11)$
$\\$ $\forall \epsilon > 0\ \exists \delta > 0:\ \forall P, \mu(P)< \delta \ \forall T$ выполняется: $|\int\limits_{a}^{b}{f\alpha'}dx - \sum_{i =0}^{n-1}{f(t_{i})\alpha'(t_{i})\Delta x_{i}}| < \epsilon \ \ (12)$
$\\$ Докажем, что $|I - \sum_{i = 0}^{n-1}f(t_{i})\Delta\alpha(x_{i})|$ меньше выраж, связанного с $\epsilon$
$\\$ $\Delta \alpha(x_{i}) = \alpha(x_{i+1}) -\alpha(x_{i})$
$\\$ по т. Лагранжа $\exists s_{i} \in (x_{i},x_{i+1}) :  \ \ \ \Delta \alpha(x_{i}) = \alpha(x_{i+1})-\alpha(x_{i}) = \alpha'(s_{i})\Delta x_{i} \ \ (13)$
$\\$ $|I - \sum_{i =0}^{ n -1}f(t_{i})\Delta\alpha(x_{i})| \underset{(13)}{=} |I - \sum_{i = 0}^{n-1}f(t-{i})\alpha'(s_{i})\Delta x_{i}| \leq |I - \sum_{i = 0}^{n -1}f(t_{i})\alpha'(t_{i})\Delta x_{i}|+| \sum_{i = 0}^{n-1}f(t_{i})(\alpha'(t_{i})-\alpha'(s_{i}))\Delta x_{i}| \underset{(11),(12)}{<} \epsilon + |\sum_{i = 0}^{n-1}f(t_{i})(\alpha'(s_{i})-\alpha'(t_{i}))\Delta x_{i}| \ \ (14)$
$\\$ $\exists M :\ \forall x \in [a,b] \ \ |f(x)|\leq M \ \ (15)$
$\\$ (15) $\Rightarrow |\sum_{i = 0}^{n-1}f(t_{i})(\alpha'(s_{i})-\alpha'(t_{i}))\Delta x_{i}| \  \leq \ \sum_{i = 0}^{n-1}|f(t_{i})||\alpha'(s_{i})-\alpha'(t_{i})|\Delta x_{i}\ \leq\ M\cdot\sum_{i = 0}^{n-1}|\alpha'(s_{i})-\alpha'(t_{i})|\Delta x_{i} \ \ \ (16)$
$\\$ $s_{i}, t_{i} \in [x_{i},x_{i+1}]$ $\tau^*_{i}, \tau_{i}$ - точки $s_{i}$ и $t_{i}$, переименнованные так, что $\alpha'(\tau^{*}_{i}) \geq \alpha'(\tau_{i}) \ \ \ (17)$
$\\$ $|\alpha'(s_{i})-\alpha'(t_{i})| = \alpha'(\tau^{*}_{i})-\alpha'(\tau_{i}) \ \  \ (18)$
$\\$ (18) $ \Rightarrow \sum_{i = 0}^{n-1}|\alpha'(s_{i})-\alpha'(t_{i})|\Delta x_{i} = \sum_{i = 0}^{n-1}\alpha'(\tau^*_{i})\Delta x_{i} - \sum_{i = 0}^{n-1}\alpha'(\tau_{i})\Delta x_{i} = S(\alpha',x,p,\mathbb{T}^*) -S(\alpha',x,P,\mathbb{T}) \ \ \ \ (19) $, где $\mathbb{T}^*=\bigcup_{i = 0}^{n-1}{\tau_{i}^*}\ \ \  \mathbb{T} = \bigcup_{i = 0}^{n-1}{\tau_{i}}$ т.к. $\alpha \in R([a,b])$, то $\forall \epsilon > 0 \ \exists \sigma_{1} > 0:\ \forall P, \mu(P) < \sigma$ и $\forall \overset{\wedge}{\mathbb{T}}$ и $\forall \overset{\wedge}{\mathbb{T}} \ \ \ |\int\limits_{a}^{b}\alpha'dx - S(\alpha',x,P,\overset{\wedge}{\mathbb{T}})| < \epsilon \ \ (20)$ Положим $\overset{\wedge}{\mathbb{T}}:= \mathbb{T}$, а затем $\overset{\wedge}{\mathbb{T}}:= \mathbb{T}^*$ Получим:
$\\$ $|S(\alpha',x,P,\mathbb{T}^*)-S(\alpha,x,P,\mathbb{T})|\leq|S(\alpha',x,P,\mathbb{T}^*)-\int\limits_{a}^{b}{\alpha'}dx|+|\int\limits_{a}^{b}{\alpha'}dx-S(\alpha',x,P,\mathbb{T})|< \epsilon + \epsilon = 2\epsilon$ \ \  \ (21)
$\\$ Пусть $\delta_{2}:= min(\delta,\delta_{1})$
$\\$ (21) $\Rightarrow$ $M\sum_{i = 0}^{n-1}|\alpha'(s_{i})-\alpha'(t_{i})|\Delta x_{i} < 2\epsilon M \ \ (M> 0) \ \ (22)$
$\\$ $(16),(22) \Rightarrow |\sum_{i = 0}^{ n-1}f(t_{i})(\alpha'(s_{i})-\alpha'(t_{i}))\Delta x_{i}|< 2M\epsilon \ \ \ (23)$
$\\$ $ (14),(23) \Rightarrow |I - \sum_{i = 0}^{n-1}f(t_{i})\Delta\alpha(t_{i})|<\epsilon + 2M\epsilon$ = $(2M+1)\epsilon \ \ \ (24)$
$\\$ Это неравенство верно для $\forall$ разбиения, диаметр которого меньше $\delta_{2} (\mu(P)<\delta_{2})$
$\\$ $(24) \Rightarrow I = \lim\limits_{\mu(P) \to 0}{S(f,\alpha,P,T)} \ \ (25)$
$\\$ По теорему (25) $\Rightarrow f \in R(\alpha)$ и $I = \int\limits_{a}^{b}{f}d\alpha \ \ (26)$ Сравнивая (11) и (26) получим доказываемую формулу
\end{proof}
