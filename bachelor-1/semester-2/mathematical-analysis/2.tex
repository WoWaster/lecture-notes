\chapter{Неопределенный интеграл}

$\\$ \begin{definition} Пусть I - открытое множество f определена на I, будем называть функцию F определенную на I первообразной функции f, если $\forall x \in I F'(x)= f(x) \ \ \ (1)$
\end{definition}
\begin{definition} $f \in C(I)$, тогда у нее $\exists$о меньшей мере одна первообразная (см п. 3.7.)
\end{definition}
\section{Теорема о структуре множества первообразных}
$\\$ \begin{theorem} I = (a,c) ((a,$\infty$),$(-\infty,b),\mathbb{R}$) $f \in C(I)$
$\\ \exists F'_{0}(x) = f(x) \forall x \in I$
$\\ F'(x) = f(x)\  \ \exists c_{0}: F(x) = F_{0}(x) +c_{0} \ \ (2)$
\end{theorem}
\begin{proof} Пусть $\varphi(x) = F(x) - F(x_{0}) \forall x \in I$
$\\$ $\varphi(x) = c_{0} \ \ (4)$
$\\$ (4)$\Rightarrow$(2)
\end{proof}
$\\ \forall C \ F_{1}(x) = F_{0}+C \ \ F_{1}'(x) = F_{0}'(x) + C' = f(x)$
$\\$ \begin{definition} Пусть f определена на I, у которой $\exists$ хотя бы одна первообразная. Множество всех первообразных f обозначается символов: $\int{f(x)}dx$
$\\$ если $\int{f(x)}dx = \int{g(x)}dx$, то это понимается как совпадение двух множеств
$\\$ Пусть $a \in \mathbb{R}$ $a \cdot \int{f(x)}dx$ - каждый элемент множества, умножается на а 
$\\$ 0 $\cdot \int{f(x)}dx = \{0\}$
$\\$ $f_{1},...,f_{m}$ опред I \ \ $\int{f_{1}(x)}dx+...+\int{f_{m}(x)}dx$ - множество полученное из всех вомзожных сумм первообразных
\end{definition}
\section{Свойства неопределенного интеграла}
\begin{property}
$\\ $ (1) g - функция, определена на $\forall x\in I \ \exists g'(x): \int{g'(x)}dx = \{g(x)+C\}:= g(x) + C$, $C \in \mathbb{R}$
\end{property}
\begin{property}
$\\$ (2) Если соотв. интегралы существуют, то $\int{f_{1}(x)}dx+...+\int{f_{m}(x)}dx = \int{(f_{1}(x)+...+f_{m}(x))}dx$
\end{property}
\begin{proof}
$\int{f_{1}(x)}dx+...+\int{f_{m}(x)}dx = (F_{1}(x)+c_{1})+...+(F_{m}(x)+c_{m}) =(F_{1}(x)+...+F_{m}(x)) + (c_{1}+...+c_{m})$ \qed \ \ \ (1)
\end{proof}
\begin{property}
$\\$ $a \neq 0$ a$\int{f(x)}dx = \int{(a\cdot f(x))}dx$
\end{property}
\begin{proof}
$\\$ $a(F(x)+C) = aF(x) + aC = aF(x) + C_{1}$, $C_{1}$ - произв. пост. $\qed$
\end{proof}
\begin{property}
$\\$ $0 \cdot \int{f(x)}dx = 0$
\end{property}
\begin{property}
$\\$ $a \neq 0$ f, ax+b - функции, с соответсвующими областями значений и определения, тогда $\int{f(ax+b)}dx = \frac{1}{a}F(ax+b) + C$
\end{property}
\begin{property}[Интегрирование по частям]
$\\$  $f,g \in C(I)$ (I- промеж, луч или $\mathbb{R}) \forall x \in I \exists f'(x), \exists g'(x)$ и $\exists \int{f'(x)g(x)}dx$\  $\int{f(x)g'(x)}dx$, тогда
$\\$ $\int{f'(x)g(x)}dx = f(x)g(x) - \int{f(x)g'(x)}dx \ \  \ (2)$ 
\end{property}
\begin{proof} Пусть $U'(x) = f'(x)g(x)$ \ $V'(x)_ = f(x)g'(x)$
$\\$ (U(x) + V(x))' = f'(x)g(x) + f(x)g'(x) = (f(x)$\cdot$g(x))' \ \ (3)
$\\$ (3) $\Rightarrow \exists c_{0}: U(x) + V(x) = f(x)g(x) = c_{0}$ \ \ (4)
$\\$ $\int{f'(x)g(x)}dx = U(x) + c_{1}$ \ \ (5')
$\\$ f(x)g(x) - V(x)- $c_{2} = U(x) + c_{0} - c_{2} \ \ (5'')$
$\\$ (5'),(5'') $\Rightarrow (2)$
\end{proof}
\begin{property}[Формула замены переменной]
$\\$ I множество(промеж, луч, ось) $\omega(x) \omega' \in C(I)$ J - множество (того же типа, что I) $\forall x \in I \omega(x)\in J$ f - функция: F'(u) = f(u), тогда $\int{f(\omega(x))\omega'(x)}dx = F(\omega(x)) + C$
\end{property}
\begin{proof}
Очевидно. 
 \end{proof}
\section{Таблица основных неопределенных интегралов}
$\\$ $(1) \int{0\cdot}dx = C $
$\\$ $(2) \int{1\cdot dx} = \int{dx} = x + C $
$\\$ $(3) r \neq -1 (r \in \mathbb{N}, I:=\mathbb{R}; r \in \mathbb{Z},r<0,I:=\mathbb{R}\backslash{0},r \in \mathbb{R}\backslash \mathbb{Z},I:=(0,\infty)) \int{x^2}dx = \frac{x^{r+1}}{r + 1} + C$
$\\$ $(4) x> 0 (\ln{x})'= \frac{1}{x}, x<0 (\ln{-x})' =\frac{1}{x} \ (\ln{|x|})' = \frac{1}{x} \int{\frac{dx}{x}} = \ln{|x|} + C$
$\\$ $(5) \int{e^x}dx = e^x + C$
$\\$ $(6) a > 0, a \neq 1 (a^x)'=a^x\cdot \ln{a}\ \ \int{a^x}dx = \frac{a^x}{\ln{a}}+ C$
$\\$ $(7) \int{\cos{x}}dx = \sin{x} + C$
$\\$ $(8) \int{\sin{x}}dx = -\cos{x}+ C$
$\\$ $(9) \int{\frac{dx}{\cos^2{x}}} = \tg{x} + C, x \in \mathbb{R}\backslash\{\frac{\pi}{2} + \pi k\}$
$\\$ $(10) \int{\frac{dx}{\sin^2{x}}} = -\cot{x} + C , x \in \mathbb{R}\backslash \pi k$
$\\$ $(11) \int{\frac{dx}{1+x^2}} = \arctan{x} + C =  -\arcctg{x} + C_{1}$
$\\$ $(12) \int{\frac{dx}{\sqrt[]{1-x^2}}} = \arcsin{x} + C , x\in (-1,1)$
$\\$ $(13) \int{\ln{x}}dx,\ \ x>0\ g(x) = \ln{x}\ f(x) = x\ f(x) = x\ f'(x) \equiv 1$
$\\$ $\int{x'\ln{x}}dx = x\ln{x} - \int{x(\ln{x})'}dx = x\ln{x} - x + C$
$\\$ Примеры интегралов не приводящихся к конечным комбинациям элементарных функций: $\int{\frac{dx}{\ln{x}}}dx,\int{\frac{e^x}{x}}dx,\int{\frac{\sin{x}}{x}}dx$ и т.д.

\section{Интегрирование рациональных выражений }

$\\$ \begin{definition} P(x), Q(X)$\neq $0 - многочлены рац. функций $R(x) = \frac{P(x)}{Q(x)} deg\P(x) < deg(Q)$
$\\$ Простейшими дробями будем называть выраж. вида
$\\$ (6) $\frac{b}{(x-a)^n} \ \ a,b \in \mathfrak{R}, b\neq 0, n \geq 1$
$\\$ (7) $\frac{px+q}{(x^2+cx+d)^{2m}} \ \ |p|+|q| > 0 \ m \geq 1 \ \ x^2+cx+d$ - неприводим над $\mathbb{R}$
\end{definition}
$\\$ $\frac{P(x)}{Q(x)}$ = $\sum^{N}_{j = 1}r_{j}(x) \ \ \ (8)$
$\\$ (8) $\Rightarrow \int{\frac{P(X)}{Q(x)}}dx =  \sum_{j = 1}^{N}\int{r_{j}(x)}dx \ \ (9)$
$\\$ Случай (6) $\int{\frac{a}{(bx+c)^n}}dx = a\int{\frac{dx}{(bx+c)^n}} = a\int{(bx+c)^{-n}}dx =
  	\begin{cases}
  	\frac{a}{b}\cdot\frac{1}{1-n}(bx+c)^{-n+1} +  C, n\neq 1 
  	\\ \frac{a}{b}\cdot \ln{|bx+c| + C_{2}, n= 1}
 	 \end{cases}$
$\\$ Случай (7) $x^2 + cx + d = (x+ \frac{c}{2})^2 + d - \frac{c^2}{4} = (x+ \frac{c}{2})^2+v^2$ \ \ $d - \frac{c^2}{4} = v^2 \  N>0$
$\\$ $\int{\frac{px+q}{(x^2+cx+d)^m}}dx = \int{\frac{p(x+\frac{c}{2})+q-\frac{pc}{2}}{((x+\frac{c}{2})^2+v^2)^m}}dx$  = $\int{\frac{px_{1}+q_{1}}{(x^2_{1} + v^2)^m}}dx_{1} = \underbrace{p\int{\frac{x_{1}}{(x^2_{1}+v^2)^m}}}_{1} + q_{1}\int{\frac{dx_{1}}{(x^2_{1}+v^2)^m}}$, где $q_{1}:= q - \frac{pc}{2}, x_{1}:= x + \frac{c}{2}$
$\\$ Рассмотрим 1 $\int{\frac{x_{1}}{(x^2_{1} + v^2)^m}} = \frac{1}{2}\int{\frac{2x_{1}dx_{1}}{(x^2_{1} +v^2)^m}} =\frac{1}{2}\int{\frac{(x^2{1})'dx_{1}'}{(x^2_{1} +v^2)^m}} = \frac{1}{2} \int{\frac{dx_{2}}{(x^2_{1} +v^2)^m}} = \frac{1}{2}\cdot 
\begin{cases}
	\frac{1}{1-m}(x_{2} + v^2)^{-m+1}+ C,\  m\neq 1 \\
	\ln{|x_{2}+ v^2|} + C_{1}, \ m = 1
\end{cases}$
$\\$ $\int{\frac{dx_{1}}{(x^2_{1} + v^2)^m}} = \int{\frac{vdx_{3}}{(v^2x_{3}+v^2)^m}} = v^{1-2m}\int{\frac{dx_{3}}{(x^2_{3} + 1)^m}}dx \ \ (10)$, где $x_{1} = vx_{3}, x'_{1} = v , v > 0$
$\\$ Рассмотрим $\int{\frac{dy}{y^2+1}} = \arctan{y} + c$ Пусть Ф$_{1}(y):= \arctan{y}$ ... Ф$_{m}(y):= \int{\frac{dy}{(y^2+1)^m}}$
$\\$ Найдем Ф$_{m+1} - ?$
$\\$ $\int{\frac{dy}{(y^2+1)^m}} = \int{\frac{y'dy}{(y^2+1)^m}} = y\cdot \frac{1}{(y^2+1)^m} - \int{y\cdot(\frac{1}{(y^2+1)^m})'}dy = \frac{y}{(y^2+1)^m} = 2m \cdot \int{y\cdot y}dy = \frac{y}{(y^2+1)^m} + 2m\int{\frac{dy}{(y^2+1)^m}} - 2m \int{\frac{dy}{(y^2+1)^{m+1}}} \Rightarrow \int{\frac{dy}{(y^2+1)^{m+1}}} = \frac{1}{2m} \cdot \frac{y}{(y^2+1)^m + \frac{2m-1}{2m}}\int{\frac{dy}{(y^2+1)^m}} \ \ \ (12) $
$\\$ (12) $\Rightarrow$ Ф$_{m+1}(y):=\frac{1}{2m}\cdot \frac{y}{(y^2+1)^m} + \frac{2m-1}{2m}$ Ф$_{m}(y) \ \  (13)$
$\\$ \begin{center} *\ \  *\ \  * \end{center}
$R(x)  = \frac{P(x)}{Q(x)}, deg\ P(x), deg\ P(x) < deg\ Q(x)$ тогда $\int{R(x)}dx$ явл. конечной комбинацией элементарных функций.
$\\$ Если p(x), Q(x) $\neq 0$ - многочлены, то P(x) = r(x)Q(x) + $p_{1}(x) \ \ \ (14)$ $deg\ r(x) \geq 0$ $deg\ p_{1}(x) < deg Q(x)$
$\\$ Тогда $\frac{P}{Q} = r + \frac{p_{1}}{Q}$ $\int{\frac{P(x)}{Q(x)}}dx = \int{r(x)}dx  +\int{\frac{p_{1}(x)}{Q(x)}}dx$
$\\$ \begin{theorem} Если R(x) - любая рациональная функций то $\int{R(x)}$ может быть выражен в виде конечной комбинации элементарных функций
\end{theorem}
\begin{definition} Рациональная функцией R(U,V) будем называть выражение R(U,V) = $\frac{P(U,V)}{Q(U,V)}$, где P, Q - многочлены (Q$\neq$ 0)
\end{definition}
\begin{theorem}$\int{R(\cos{x},\sin{x})}dx$ представим в виде конечной комбинации элементарных функций.
\end{theorem}
\begin{proof} $\tan{\frac{x}{2}} = y \ \ (1) \ \ \ x = 2\arctan{y} \ \ \ (5)$
$\\$ $\cos{x} = 2\cos^2{\frac{x}{2}} - 1$ \ \ $\tan^2{\frac{x}{2}}+1 = \frac{\sin^2{\frac{x}{2}}}{\cos^2{\frac{x}{2}}}+ 1 = \frac{1}{\cos^2{\frac{x}{2}}} \ \ (2)$
$\\$ $\cos{x} = \frac{2}{y^2+1}- 1 = \frac{1-y^2}{1+y^2}\  \  \ (3)$
$\\$ $\sin{x} = 2\sin{\frac{x}{2}}\cos{\frac{x}{2}} = 2\frac{\sin{\frac{x}{2}}}{\cos{\frac{x}{2}}}\cdot \cos^2{\frac{x}{2}} = \frac{2y}{y^2+1} \ \ (4)$
$\\$ $y'(x) = (\tan{\frac{x}{2}})' = \frac{1+y^2}{2}$ $x'(y) = \frac{2}{y^2+1} \  \ \ (6)$
$\\$ $\int{R(\cos{x},\sin{x})}dx = \int{R(\frac{1-y^2}{1+y^2},\frac{2y}{1+y^2})\cdot\frac{2}{1+y^2}}dy \ \ (7)$
$\\$ $R_{1}(y) = R(\frac{1-y^2}{1+y^2},\frac{2y}{1+y^2})\cdot \frac{2}{1+y^2}$ - рациональная функция
\end{proof}
\section{Интегралы некоторых иррациональных функций}
$\\$ R(U,V) n > 1, $a_{1}b_{2} - a_{2}b_{1} \neq 0$
$\\$ $\int{R((\frac{a_{1}x+b_{1}}{a_{2}x+b_{2}})^{\frac{1}{n}},x)}dx \ \  \ (9)$
$\\$ В частности вместо $\frac{a_{1}x+b_{1}}{a_{2}x+b_{2}})^{\frac{1}{n}}$ может находиться $(a_{1}x+ b_{1})^{\frac{1}{n}}, a_{1} \neq 0$; $(\frac{a_{1}x+b_{1}}{x})^{\frac{1}{n}}, b_{1}\neq 0; x^{\frac{1}{n}} $
$\\$ \begin{theorem} Интеграл (9) - конечная комбинация элементарных функций
\end{theorem}
$\\$ \begin{proof} (10); $(\frac{a_{1}x+b_{1}}{a_{2}x+b_{2}})^{\frac{1}{n}} = y; \ \frac{a_{1}x+ b_{1}}{a_{2}x+b_{2}} = y^n \ \ \ (11)$
$\\$ $x = \frac{b_{2}}{a_{1}-a_{2}y^{n}} \ \ \ (12)$ Очевидно x'(y) - рац. функция
$\\$ (10)-(12) $\Rightarrow$ $\int{R((\frac{a_{1}x+b_{1}}{a_{2}x+b_{2}})^{\frac{1}{n}},x)}dx$= $\int{R(y,\frac{b_{2}y^n - b_{1}}{a_{1}-a_{2}y^n})}\cdot x'(y)dy$
$\\$ $R(y,\frac{b_{2}y^n-b_{1}}{a_{1}-a_{2}y^n})x'(y) = R_{2}(y)$
$\\$ (13) $\Rightarrow$ $\int{R((\frac{a_{1}x+b_{1}}{a_{2}x+b_{2}})^{\frac{1}{n}},x)}dx = \int{R_{2}(y)dy \ \ (14)}$ 
\end{proof}
\section{Интегралы от биномиальных дифференциалов}
$\\$ $\int{x^m(a+bx^n)^p}dx$-  интеграл от бин. дифф. $a,b \neq 0, m,n,p \in \mathbb{Q}$ 
$\\$  Рассмотрим случаи
\begin{enumerate}
	\item p - целое $p \neq 0$ , m = $\frac{M}{L}, n = \frac{N}{L}$, где $L\in \mathbb{N},M,N \in \mathbb{Z}$
	$\\$ $x^m(a+bx^m)^p = (x^{\frac{1}{L}})^M(a+b(x^{\frac{1}{L}})^N)^p$
	$\\$ $R(u,v):= U^M(a+bU^N)^P$, что явл. рац. функцией 
	\item p - не целое Пусть $x^n=y,x = y^{\frac{1}{n}}, x'(y) = \frac{1}{n}y^{\frac{1}{n}-1}$
	$\\$ $\int{x^m(a+bx^n)^p}dx = \frac{1}{n}\int{y^{\frac{m}{n}}\cdot(a+by)^py^{\frac{1}{n}-1}}dy = \frac{1}{n}\int{y^{\frac{m+1}{n}-1}(a+by)^p}dy$
	\item[2.а] $\frac{m+1}{n}$ - целое \ (15) Пусть $p = \frac{r}{q}, q \in \mathbb{N}, r \in \mathbb{Z}$
	$\\$ $R(u,v) = (a+bu)^r\cdot v^{\frac{m+1}{n}-1}$ (15) $\Rightarrow$ R - рациональная фукция от u,v 
	$\\$ $\frac{1}{n}\int{y^{\frac{m+1}{n}-1}(a+by)^p}dy = \int{R((a+by)^{\frac{1}{q}},y)}dy$
	\item[2.б]  $\frac{m+1}{n}$ - не целое 
	$\\ \frac{1}{n}\int{y^{\frac{m+1}{n}-1}(a+by)^p} = \frac{1}{n}\int{y^{\frac{m+1}{n}-1 + p}(\frac{a+by}{y})^p}dy \ \ (16)$, где $\frac{m+1}{n} + p $ - целое $\qed$
\end{enumerate} 
$\\$ \begin{theorem}[Теорема(Чебышева).] Других случаев, при которых рац. дифференциал может быть выражен в виде кон. комбинаций нет
\end{theorem}
\section{ Подстановка Эйлера}
$\\$ $R(u,v)$ -  рац. выраж.
$\\$ $\sqrt[]{ax^2+bx+c}, a \neq 0$
$\\$ $\int{R(\sqrt[]{ax^2+bx+c},x)}dx$ - требуется привести интеграл к известнымфункциям
$\\$ $\RNumb{1}$-ая подстановка (a>0)
$\\$ $\sqrt[]{ax^2+bx+c}=\sqrt[]{a}\cdot x + t \ \ (1)$
$\\$ $ax^2+bx+c = ax^2+2\sqrt{a}xt+t^2$
$\\$ $x = \frac{t^2-c}{b-2\sqrt{a}t} \ \ (2) \ \ ; \ \sqrt{ax^2+bx+c} = \sqrt{a}\cdot \frac{t^2-c}{b-2\sqrt{a}t} + t \ \ (3)$
$\\$ $x'(t)$ - рац.функция, тогда $\int{R(\sqrt{ax^2+bx+c},x)}dx = \int{R(\sqrt{a}\frac{t^2-c}{b-2\sqrt{a}t}+t,\frac{t^2-c}{b-2\sqrt{a}t})(\frac{t^2-c}{b-2\sqrt{a}t})'}dt= \int{R_{1}(t)}dt$
$\\$ $\RNumb{2}$-ая подстановка (c>0)
$\\$ $\sqrt{ax^2+bx+c} = \sqrt{c} + tx \ \ (4)$
$\\$ $ax^2+bx+c = c + 2\sqrt{c}tx+t^2x^2$
$\\$ $x \neq 0 : ax+ b = 2\sqrt{c}t + t^2x$
$\\$ $x = \frac{2\sqrt{c}t -b}{a-t^2} \ \  (5) \ \ ; \sqrt{ax^2+bx+c} = \sqrt{c}+t\cdot \frac{2\sqrt{c}t-b}{a-t^2} \ \ (6) $, тогда $\int{R(\sqrt{ax^2+bx+c},x)}dx = \int{R(\sqrt{c} + t\cdot \frac{2\sqrt{c}t-b}{a-t^2},t\frac{2\sqrt{c}t-b}{a-t^2})(\frac{a\sqrt{c}t-b}{a-t^2})'}dt = \int{R_{2}(t)}dt$
$\\$ $\RNumb{3}$-ая подстановка
$\\$ Пусть $\gamma,\delta \in \mathbb{R}$ - корни $ax^2+bx+c, \gamma \neq \delta$
$\\$ $\sqrt{ax^2+bx+c} = t(x-\gamma) \ \ \ (7)$
$\\$ $ax^2+bx+c = t^2(x-\gamma)^2$
$\\$ $a(x-\gamma)(x-\delta) = t^2(x-\gamma)^2$ \ \ $a(x-\gamma)(x-\delta) = t^2(x-\gamma)^2$ $x \neq \gamma, \ \ \ x \neq \delta$
$\\$ $ax-a\delta = t^2x-t^2\gamma$
$\\$ $x = \frac{a\delta - t^2\gamma}{a-t^2} \ \ (8); \ \sqrt{ax^2+bx+c} = t \ \ (\frac{a\delta-t^2\gamma}{a-t^2}-\gamma)$ и т.д
$\\$ Если a<0, c<0< нет 2 разл. корней, то $ax^2+bx+c \leq 0$ и о корне из этого выражения речи быть не может 
\section{Подстанока Абеля}
$\\$ $m \geq 1, a\neq 0\  \ int{\frac{dx}{(ax^2+bx+c)^{\frac{m+1}{w}}}}$ \ \ \ $f(x):= ax^2+bx+c$
$\\$ t = ($\sqrt{f(x)}$)' = $\frac{f'(x)}{2\sqrt{f(x)}}=\frac{ax+\frac{b}{2}}{\sqrt{f(x)}} \ \ (9)$
$\\$ (9) $\Rightarrow t^2f(x)=a^2x^2+abx+\frac{b^2}{4}$
$\\$ $t^2ax^2+4t^2bx+t^2c=a^2x^2+abx+\frac{b^2}{4}$
$\\$ $4t^2ax^2+4t^2bx+4t^2c=4a^2x^2+4abx+b^2 \ \ (10') $
$\\$ $4af(x) =4a^2x^2+4abx+4ac \ \ (10'')) $
$\\$ Вычтем из (10') (10''): $4t^2f(x)-4af(x)=b^2-4ac$ \ \ $f(x)=\frac{b^2-4ac}{4(t^2-a)}$
$\\$ $\frac{1}{(f(x))^{\frac{2m+1}{2}}}=(\frac{4}{b^2-4ac})^{\frac{2m+1}{2}}(t^2-a)^{\frac{2m+1}{2}} \ \ (11)$
$\\$ $ t= \frac{f'(x)}{2\sqrt{f(x)}}; \\ \ t\sqrt{f(x)}=\frac{1}{2}f'(x)=ax+\frac{b}{2} \ \ (12)$
$\\$ (12) $\Rightarrow t'_{x}\sqrt{f(x)}+t(\sqrt{f(x)})'_{x} = a \ \ (13)$ 
$\\$ (13) $\Rightarrow t'_{x}\sqrt{f(x)}+t^2=a \ \ t'_{x}=\frac{a-t^2}{\sqrt{f(x)}}$
$\\$ $\int{\frac{dx}{(ax^2+bx+c)^{\frac{2m+1}{2}}}}=\int{\frac{1}{(f(x))^m}\cdot \frac{dx}{\sqrt{f(x)}}}$ \textbf{=}
$\\$ $\frac{1}{(f(x))^m} = (\frac{a}{b^2-4ac})^m(t^2-a)^m  \ \ (14)$
$\\$ \textbf{=} $\int{(\frac{4}{b^2-4ac})^m(t^2-a)^m \cdot \frac{t'_{x}}{-t^2+a}}dx = (\frac{4}{b^2-4ac})^m\int{(t^2-a)^{m-1}t'_{x}}dx = -(\frac{4}{b^4-4ac})^m\int{(t^2-a)^{m-1}}dt$
