\chapter{Исследование функций}
\section{Критерий постоянства функции}
\begin{theorem}\label{c}
$\\$ Пусть f определена на (a,b) $\forall$x $\in$ (a,b) $\exists$ f'(x) для того, чтобы f была постоянной  ($f(x)\equiv c_{0}$) необходимо и достаточно, чтобы $\forall$x $\in$ (a,b), f'(x)=0
\end{theorem}




\begin{proof}
$x_{0} \in (a,b), x \neq x_{0}$
Применим к промежуткам с концами $x, x_{0}$ теорему Лагранжа
$f(x) - f(x_{0}) = f'(c)(x-x_{0}) = 0 f(x) = f(x_{0})$
\end{proof}

\begin{corollary}
$\\$ На (a,b) определены функции g, h $\forall$$x_{0}$ $\in$ (a,b) $\exists$ g'($x_{0}$) = h'($x_{0}$) $\exists$ c: $\forall$x $\in$ (a,b) g(x) = h(x) + $c_{0}$
\end{corollary}
\begin{proof}
Пусть f(x) = g(x) - h(x) $\forall$x $\in$ (a,b) f'(x) = g'(x) - h'(x) = 0
$\\$ пр Предыдущей теореме f(x) $\equiv \ c_{0}$ $\forall$x $\in$ (a,b) 
\end{proof}
\section{Критерий возрастания и убывания функции}
\begin{theorem}
$\\$ Пусть f,g - определены на (a,b) $\forall$x $\in$ (a,b) $\exists$ f'(x), g'(x)
$\\$ \RNumb{1}. \ 1) Для того, чтобы f была монотонной возрастающей на (a,b) $\Leftrightarrow$ $\forall$x $\in$ (a,b) f'(x) $\geq$ 0\ \ \ \ (1)
\par 2) Для того, чтобы g была монотонной убывающей на (a,b) $\Leftrightarrow$ $\forall$x $\in$ (a,b) g'(x) $\leq$ 0\ \ \ \ (2)
$\\$ \RNumb{2}. 3) Для того, чтобы f строго возр. $\Leftrightarrow$ выполнялось св-во (1) и $\nexists$ ($\alpha , \beta$) $\subset$ (a,b) : $\forall$x $\in$ ($\alpha , \beta$), f'(x) = 0\ \ \ \ (3)
\par 4) Для того, чтобы g была строго убыв. $\Leftrightarrow$ выполнялось (2) и $\nexists$($\gamma, \delta$) $\subset$ (a,b) : $\forall$x $\in$($\gamma, \delta$) g'(x) = 0\ \ \ \ (4)
\end{theorem}

\begin{proof}\RNumb{1} \\
(Необх) $x_{0}\  \in (a,b)$ Пусть h > 0 тогда f($x_{0}+h$ - $f(x_{0})\ \geq 0$
$\frac{f(x_{0}+h)-f(x_{0})}{h} \geq 0\ \ \ \ (5)$
$\\$ Переходя к пределу: $\lim\limits_{h\to +0}\frac{f(x_{0}+h)-f(x_{0})}{h} \geq$ 0 или $f'(x_{0}) \geq 0$
$\\$ (Достаточность.) Пусть выполнено (1) Пусть$ x_{1},x_{2} : b >x_{1} > x_{2} > a $ по теореме Лагранжа $\exists$ c $\in (x_{2},x_{1}): f(x_{1})-f(x_{2}) = f(c)(x_{1}-x_{2})$ 
\end{proof}

\begin{proof}\RNumb{2} \\
(Необх) ¿ Если бы $\exists$ ($\alpha , \beta$) на котором выполн (3)
$\\$ Применяя критерий пост. ф$-$ций $\forall$x $\in$ ($\alpha , \beta$) f(x) = f($\frac{\alpha + \beta}{2}$) 
$\alpha , \beta$ \ \ \ \  \ \ \ \ \ ?!
$\\$ (Достаточность)
Пусть $\nexists (\alpha,\beta) \subset (a,b) : \forall x \ \in (\alpha,\beta)$ f'(x) = 0
$\\$ Пусит функция НЕ строго возрастает $\Rightarrow \exists x_{1},x_{2} : a<x_{1}<x_{2}<b$ $f(x_{1} = f(x_{2})$\ \ \ \ (6)
$\\$ $\forall x \in (x_{1},x_{2}) $ $f(x_{1}) \leq f(x) \leq f(x_{2})$ \ \ \ \ (7)
$\\$ (6),(7) $\Rightarrow \forall x \in (x_{1},x_{2})$ $f(x) \equiv f(x_{1}) \Rightarrow f'(x) = 0$ \ \ \ \ ?! 
\end{proof}

\section{Приложение доказанных теорем}

$\\$ Известно: $\sin{x} < x< \tg{x}$ (0<x<$\frac{\pi}{2}$)
$\\$ Можно доказать: $\sin{x} > \frac{2}{\pi}x $\ \ \ \ (8)
\begin{proof}
Пусть g(x) = $\frac{\sin{x}}{x} $, g'(x) = $\frac{x\cos{x}-\sin{x}}{x^2}\ =\ \frac{\cos{x}}{x^2}(x\ -\ \tg{x}) < 0 \Rightarrow g(x)$ строго монотонно убывает на $(0,\frac{\pi}{2})$
$\\$ $\forall x_{0}\ \in (0,\frac{\pi}{2})$ Пусть $x_{1}, x_{2}: x_{0} < x_{1} < x_{2} < x_{2} < \frac{\pi}{2}$
$\\$ $\frac{\sin{x_{0}}}{x_{0}} > \frac{\sin{x_{1}}}{x_{1}} > \frac{\sin{x_{2}}}{x_{2}}$ \ \ \ \ (9)
$\\$ x $\rightarrow \frac{\pi}{2}-0$, $g(\frac{\pi}{2})= \frac{2}{\pi}$
$\\$ $\frac{\sin{x_{0}}}{x_{0}} > \frac{\sin{x_{1}}}{x_{1}} \geq \lim\limits_{x_{2} \to \frac{\pi}{2}-0}\frac{sin{x_{2}}}{x_{2}} = \frac{2}{\pi} \Rightarrow (8)$
\end{proof}
\section{Нахождение локальных экстремумов}
$\\$ \begin{definition}{Опр.} E$\subset \mathbb{R}$ и $x_{0} \ \in$ E, $x_{0}$ - точка сгущения Е, f,g определены на E(a)
$\\$a) $x_{0}$ называется точкой локального максимума для f если $\exists$ окрестность $\omega_{1} \ni x_{0} : \forall x \in \omega_{1}\cap\ E\ f(x)\leq f(x_{0})$
$\\$б) $x_{0}$ называется точкой локального минимума для g, если $\exists$ окрестность $\omega_{2} \ni x_{0} : \forall x \in \omega_{2}\cap\ E\ g(x)\leq g(x_{0})$
$\\$в) $x_{0}$ называется точкой экстремума, если она либо локальный максимум, либо локальный минимум
$\\$г) $x_{0}$ называется точкой строго локального максимума для f, если $\exists$ окрестность $\omega_{1} \ni x_{0} : \forall x \in \omega_{1}\cap\ E,\ x\neq x_{0} :\ f(x) < f(x_{0})$
$\\$д)называется точкой строго локального минимума для g, если $\exists$ окрестность $\omega_{2} \ni x_{0} : \forall x \in \omega_{2}\cap\ E,\ x\neq x_{0} :\ g(x) < g(x_{0})$
$\\$е) $x_{0}$ - точка строго локального экстремума функции, если она либо строго локальный минимум, либо максимум 
\end{definition}
\section{Необходимый признак локального экстремума. Теорема Ферма}
$\\$ \begin{theorem} f определена на (a,b) $\forall x_{0}\in(a,b)\ \exists f'(x_{0})$ Если в точке $x_{0}$ локальный экстремум, то f'($x_{0}$) = 0 
\end{theorem}
\section{Достаточный признак локального экстремума}
$\\$ \begin{theorem} f определена на (a,b)
$\\$ \RNumb{1} На (a,b) $\exists$ производные порядка до 2n-1(включительно) n $\geq$ 1 и для $x_{0}$ $\in$ (a,b) $\exists f^{(2n)}(x_{0}) $
$\\$ $f'(x_{0})$ = ... = $f^{(2n-1)}(x_{0}) = 0$\ \ \ \ (10)
$\\$ $f^{(2n)}(x_{0}) \neq 0$ \ \ \ \ (11)
$\\$ Тогда в точке $x_{0}$ достигается строгий локальный 
экстремум. При этом если $f^{(2n)}(x) > 0$, то достигается строгий локальный максимум, а если $f^{(2n)}(x) < 0$ - строгий локальный минмиум 

$\\$ \RNumb{2} $\forall 1 \leq k \leq 2n$ $\forall x \in (a,b) \exists f^{n}(x)$,
$\exists f^(2n+1)(x_{0})$ 
$\\$  $f'(x_{0})$ = ... = $f^{(2n)}(x_{0}) = 0$ \ \ \ \ (12)
$\\$ $f^{(2n+1)}(x_{0}) \neq 0$ \ \ \ \ (13) 
$\\$ Тогда в точке $x_{0}$ экстремумов нет
\end{theorem}
$\\$ \begin{proof} \RNumb{1}
$\\$ Применим формулу тейлора с остатком Пеано:
$\\$ $x \neq x_{0}$
$\\$ $f(x) = f(x_{0}) + \sum_{k = 1}^{2n}\frac{f^{n}(x_{0})}{k!}(x-x_{0})^{k} + r(x)$ \ \ \ \ (14)
$\\$ $\frac{r(x)}{(x-x_{0})^{2n}} \xrightarrow[x \to x_{0}]{} 0$ \ \ \ \ (15)
$\\$ (10),(14) $\Rightarrow f(x) = f(x_{0})$ + $\frac{f^{(2n)}(x_{0})}{(2n)!}(x-x_{0})^{2n} + r(x) = f(x_{0}) + \frac{f^{(2n)}(x_{0})}{(2n)!}(x-x_{0})^{2n}\underbrace{(1 + \frac{(2n)!}{f^{(2n)}(x)}\cdot \frac{r(x)}{(x-x_{0})^{2n}})}_{t(x)} $
$\\$ (15) $\Rightarrow$ $\exists$ окрестность $\omega \ni$ $x_{0} : \forall x \in \omega\  t(x) > \frac{1}{2}$ \ \ \ \ (16)
$\\$ Знак f(x) - $f(x_{0})$ зависит только от знака производной $f^{(2n)(x_{0})}$
\end{proof}
$\\$ \begin{proof} \RNumb{2}
$\\$ Применим формулу Тейлора порядка 2n+1 :
$\\$ f(x)=f($x_{0}$) + $\frac{f^{(2n+1)}(x_{0})}{(2n+1)!}(x-x_{0})^{2n+1} + r_{1}(x)$ = $f(x_{0}) + \frac{f^{(2n+1)(x_{0})}}{(2n+1)!}(x-x_{0})^{2n+1}(\underbrace{1+\frac{(2n+1)!}{f^{(2n+1)}(x_{0})}\cdot \frac{r(x)}{(x-x_{0})^{2n+1}}}_{t_{1}(x)})$\ \ (18)
$\frac{r_{1}(x)}{(x-x_{0})^{2n+1}} \xrightarrow[x\to x_{0}]{}$ 0 \ \ \ \ (17)
$\\$ (17) $\Rightarrow$ $\exists \omega_{1} \ni x_{0} : \forall x \in \omega_{1}\ t_{1}(x) > \frac{1}{2}$ \ \ \ \ (19)
$\\$ если $x_{1} < x_{0} < x_{2}$  $x_{1}, x_{2}$ $\in$ $\omega$, то $f^{(2n+1)}(x_0) (x_{1} - x_{2})^{2n+1}$ и 
$\\$ $f^{(2n+1)}(x_{0})(x_{2}-x_{0})^{2n+1}$  имеют разные знаки
\end{proof}
\section{Исследвание выпуклости и вогнутости функций}
$\\$ Через I будем обозначать любое из множеств: [a,b], [a,b), (a,b], (a,b), a<b, (a,$\infty$), [a,$\infty$), (-$\infty$,b), (-$\infty$,b], (-$\infty$,$\infty$)При чем, если $x_{1}$, ...$x_{n} \in I$ n$\geq$ 2 (1)
$\\$ $p_{1},...p_{n} : p_{j} > 0, p_{1} + ... + p_{n} = 1$ \ \ \ \ (3)
$\\$ (1),(3) $\Rightarrow$ $p_{1}x_{1} + ... + p_{n}x_{n} \in I$ - Важное замечание!
$\\$ \begin{definition} I-множество, f $\in C(I)$ функция f выпукла, если $\forall x_{1},x_{2} \in I$, $\forall p_{1},p_{2} > 0$, $p_{1} + p_{2} = 1$
$\\$ $f(p_{1}x_{1}+ p_{2}x_{2}) \leq p_{1}f(x_{1}) + p_{2}f(x_{2})$ \ \ \ \ (5)
$\\$ Пусть h $\in$ $C(I)$
$\\$ вогнута на I, если при тех же обозначениях $h(p_{1}x_{1}+p_{2}x_{2}) \geq p_{1}h(x_{1})+p_{2}h(x_{1})$
\end{definition}
\section{Свойства выпуклых и вогнутых функций }
$\\$ Обозначения:
$\\$ выпуклость f $\in$ cnv (I)
$\\$ вогнутость f $\in$ cnc (I)
\begin{property} 
$\\$
\begin{enumerate}
	\item  f $\in$ cnv(I) $\Rightarrow$ (-f) $\in$ cnc(I)
	$\\$h $\in$ cnc(I) $\Rightarrow$ (-h) $\in$ cnv(I)
	\item $c_{0}$ >0
	$\\$ f $\in$ cnv(I) $\Rightarrow$ $c_{0}f \in cnv(I)$
	$\\$ h $\in$ cnc(I) $\Rightarrow$ $c_{0}h \in cnc(I)$
	\item $f_{1},...,f_{m}\ \in\ cnv(I) \Rightarrow \sum_{i = 1}^{m}f_{i} \in cnv(I)$
	$\\$ $h_{1},...,h_{n}\ \in\ cnc(I) \Rightarrow \sum_{i = 1}^{n}h_{i} \in cnc(I)$
\end{enumerate}
\end{property}
\section{Неравенство Йенсена}
$\\$ \begin{theorem} Пусть $f \in cnv(I)$,
$\\$$n \geq 2$, 
$\\$$x_{1},...,x_{n} \in I$, 
$\\$$p_{1},...,p_{n} > 0$, 
$\\$$p_{1},...,p_{n} = 1$
$\\$ $f(p_{1}x_{1}+...+p_{n}x_{n}) \geq p_{1}f(x_{1})+...+p_{n}f(x_{n})$\ \ \ \ (6)
$\\$ Пусть $h \in cnc(I)$ При тех же обозначенияъ $h(p_{1}x_{1}+...+p_{n}x_{n}) \geq p_{1}h(x_{1})+...+p_{n}h(x_{n})$ \ \ \ \ (7)
\end{theorem}
$\\$ \begin{proof} Докажем ждя cnv-функций(см свойство 1)
$\\$ Доказательство проведем индукцией по n, если n = 2 - тривиально.
Пусть верно для $n \geq 2$ докажем для n+1 точек $x_{1},...x_{n+1}$ $\in I$
$\\$ $p_{1},...,p_{n+1} > 0$, $\sum_{i = 1}^{n+1}p_{i} = 1$
$\\$ Пусть $\widetilde{p_{n}} = p_{n} +p_{n+1} > 0$
$\\$ $p_{1}+...+p_{n-1}+\widetilde{p_{n}} = p_{n} +p_{n+1} > 0$ = $p_{1}+...+p_{n+1}$ = 1 \ \ \ \ (8)
$\\$ Пусть $\widetilde{x_{n}}= \frac{1}{\widetilde{p_{n}}}(p_{n}x_{n}+p_{n+1}x_{n+1})$ \ \ \ \ (9)
$\\$ $\widetilde{x_{n}} = \frac{p_{n}}{\widetilde{p_{n}}}x_{n} + \frac{p_{n+1}}{\widetilde{p_{n}}}x_{n+1}$, но $\frac{p_{n}}{\widetilde{p_{n}}}+\frac{p_{n+1}}{\widetilde{p_{n}}}=\frac{\widetilde{p_{n}}}{\widetilde{p_{n}}} = 1$ \ \ \ \ (10) $\\$ $\Rightarrow \in I$ \ \ \ \ (11)(см замечания в начале лекции)
$\\$ (9) $\Leftrightarrow$ $\widetilde{p_{n}}\widetilde{x_{n}} = p_{n}x_{n}+p_{n+1}x_{n+1}$ Тогда по (9),(8) и инд. предп. получаем:
$f(p_{1}x_{1}+...+p_{n-1}x_{n-1}+\widetilde{p_{n}}\widetilde{x_{n}}) \leq p_{1}f(x_{1})+...+p_{n-1}f(x_{n-1})+\widetilde{p_{n}}f(\widetilde{x_{n}})$ \ \ \ \ (12)
$\\$ $f(\widetilde{x_{n}}) = f(\frac{p_{n}}{\widetilde{p_{n}}}x_{n} + \frac{p_{n+1}}{\widetilde{p_{n}}}x_{n+1}) \stackrel{(10),(11)}{\leq} \frac{p_{n}}{\widetilde{p_{n}}}f(x_{n}) + \frac{p_{n+1}}{\widetilde{p_{n}}}f(x_{n+1}) \Rightarrow \widetilde{p_{n}}\cdot f(\widetilde{x_{n}})\leq p_{n}f(x_{n}) + p_{n+1}f(x_{n+1})$\ \ \ \ (13)
$\\$ (8), (12), (13) $\Rightarrow (6)$
\end{proof}
\section{Преобразование условий выпуклости и вогнутости}
$\\$ рассмотрим мн-во I, $x_{1},x_{2} \in I; x_{1} < x_{2} $
$\\$ Пусть x: $x_{1} < x < x_{2}$ Очевидно $x\ \in\ I$
$\\$ $\frac{x_{2}-x}{x_{2}-x_{1}} > 0, \frac{x-x_{1}]}{x_{2}-x_{1}} >0, \frac{x_{2}-x}{x_{2}-x_{1}}+ \frac{x-x_{1}}{x_{2}-x_{1}} = 1$

$\\$ $\frac{x_{2}-x}{x_{2}-x_{1}}x_{1} + \frac{x-x_{1}}{x_{2}-x_{1}}x_{2} = x$ \ \ \ \ (14)
$\\$ (14) $\Rightarrow f(x) \leq \frac{x_{2}-x}{x_2-x_{1}}f(x_{1})+\frac{x-x_{1}}{x_{2}-x_{1}}f(x_{2})$ \ \ \ \ (15)
$\\$ h(x) $\geq \frac{x_{2}-x_{1}}{x_{2}-x_{1}}h(x_{1}) + \frac{x-x_{1}}{x_{2}-x_{1}}h(x_{2})$ \ \ \ \ (16)
$\\$ \begin{assertion} Если f выпукла на I, то справ. (15) \end{assertion}
$\\$ Если f выпукла на I, то справ. (16)
$\\$ \begin{corollary} Пусть g - функция опред. на I: g$\in cnv(I)$ и одновременно $g \in cnc(I)$ \ \ \ \ \ (17)
$\\$ Тогда $\exists c_{1}, c_{2} : g(x) = c_{1}x+c_{2}$(g линейная функция) \ \ \ \ (18) \end{corollary}

$\\$ \begin{theorem}
Пусть g - вып и вогнута. Пусть $x_{1},x_{2} \in I: x_{1} < x_{2}$ рассмотрим x:$x_{1} < x < x_{2}$
$\\$ (15),(17) $\Rightarrow$ $g(x) \leq \frac{x_{2}-x}{x_{2}-x_{1}}g(x_{1})+ \frac{x-x_{1}}{x_{2}-x_{1}}g(x_{2})$ \ \ \ \ (19)
$\\$ (16),(17) $\Rightarrow g(x) \geq \frac{x_{2}-x_{1}}{x_{2}-x_{1}}g(x_{1}) + \frac{x-x_{1}}{x_{2}-x_{1}}g(x_{2})$ \ \ \ (20)
$\\$ (19),(20) $\Rightarrow g(x) = \frac{x_{2}-x}{x_{2}-x_{1}} + \frac{x-x_{1}}{x_{2}-x_{1}}g(x_{2}) = c_{1}(x_{1},x_{2})x+c_{2}(x_{1},x_{2})$ \ \  \ \ (21)
\end{theorem}
$\\$ Упр $c_{1}(x_{1},x_{2}),c_{2}(x_{1},x_{2})$ не зависят от $x_{1}$ и $x_{2}$

$\\$ Единственными функциями, которые одновременно выпуклы и вогнуты являются линейные функции


\section{Выпуклость/Вогнутость и свойства производной функции}
$\\$ \begin{theorem}$f \in C(I) \forall x \in I \exists f'(x)\ \ f \in cnv(I) \Leftrightarrow f'(x)$ монотонно возрасает \ \ \ (22)
$\\ h \in C(I) \forall x \in I \exists h'(x)$
$\\$ $f \in cnc(I) \Leftrightarrow f'(x)$ монотонно убывает \ \ \ \ (23)
\end{theorem}
$\\$ \begin{proof} Докажем только (22) (см свйоство 1)
$\\$ (Необходимость) $x_{1} < x_{2}; x_{1},x_{2} \in I$
$\\$ (15): $f(x) \leq \frac{x_{2}-x}{x_{2}-x_{1}}f(x_{1}) + \frac{x-x_{1}}{x_{2}-x_{1}}f(x_{2})$
$\\$ $f(x)-f(x_{1}) \leq \frac{x_{2}-x}{x_{2}-x_{1}} + \frac{x-x_{1}}{x_{2}-x_{1}}f(x_{2}) - (\frac{x_{2}-x}{x_{2}-x_{1}}f(x_{1}) +\frac{x_{2}-x}{x_{2}-x_{1}}f(x_{1}))\ =\  \frac{x-x_{1}}{x_{2}-x_{1}}(f(x_{2}-f(x_{1})))$ Учитывая x-$x_{1} > 0$ получим:
$\\$ $\frac{f(x)-f(x_{1})}{x-x_{1}} \leq \frac{f(x_{2})-f(x_{1})}{x_{2}-x_{1}}\ \ \ (24) $
$\\$ Пусть x =  $x_{1}+h, h>0 $
$\\$ $\frac{f(x_{1}+h)-f(x_{1})}{h} \leq \frac{f(x_{2})-f(x_{1})}{x_{2}-x_{1}} $ Переходя к пределу при $h \longrightarrow 0$
$\\$ $f'(x_{1}) \leq \frac{f(x_{2})-f(x_{1})}{x_{2}-x_{1}}\ \ \ (25) $
$\\$ (15) : -f(x) $\geq -\frac{f(x_{2})-f(x_{1}))}{x_{2}-x_{1}}f(x_{1}) - \frac{x-x_{1}}{x_{2}-x_{1}}f(x_{2}) $
$\\$ $f(x_{1}) -f(x) \geq (\frac{x_{2}-x}{x_{2}-x_{1}}f(x_{2})+\frac{x-x_{1}}{x_{2}-x_{1}}f(x_{2})) - \frac{x_{2}-x}{x_{2}-x_{1}}f(x_{1}) - \frac{x-x_{1}}{x_{2}-x_{1}}f(x_{2}) = \frac{x_{2}-x}{x_{2}-x_{1}}(f(x_{2})-f(x_{1})) $ $x_{2} > 0$
$\\$ $\frac{f(x_{2}-f(x))}{x_{2}-x} \geq \frac{f(x_{2})-f(x_{1})}{x_{2}-x_{1}}$
$\\$ Пусть x = $x_{2} - h$, h > 0
$\\$ (26): $\frac{f(x_{2})-f(x_{2}-h)}{h} \geq \frac{f(x_{2})-f(x_{1})}{x_{2}-x_{1}}$
$\\$ $ - \frac{f(x_{2}-h)-f(x_{2})}{h} \geq \frac{f(x_{2})-f(x_{1})}{x_{2}-x_{1}} \ \ \ (27)$
$\\$ При $h \rightarrow 0 f'(x_{0}) \geq \frac{f(x_{2})-f(x_{1})}{x_{2}-x_{1}} \ \ \ \ (28) $
$\\$ (25),(28) $\Rightarrow f'(x_{1})\leq f'(x_{2})$
$\\$ (Достаточность) (?) $f \in cnv(I ) \  x_{1}< x<x_{2} \Leftrightarrow f(x) \leq \frac{x_{2}-x}{x_{2}-x_{1}}f(x_{1}) + \frac{x-x_{1}}{x_{2}-x_{1}}f(x_{2})$ (см п."преобразование условий выпуклости и вогнутости") 
$\\$ $(x_{2}-x_{1})f(x) \leq (x_{2}-x)f(x_{1}) + (x-x_{1})f(x_{2})$
$\\$ $(x_{2}-x)f(x) - (x_{2}-x)f(x_{1})\leq (x-x_{1})(f(x_{2})-f(x)) $
$\\$ $(x_{2}-x)(f(x)-f(x_{1}))\leq (x-x_{1})(f(x_{2})-f(x))\ \ \ (2) $
$\\$ Применим теорему Лагранжа на $[x_{1},x]$ и $[x,x_{2}] :$
$\\$ $\exists c_{1} \in (x_{1},x) :\ f(x)-f(x_{1}) = f'(c_{1})(x-x_{1}) \ \ \ (3)$
$\\$ $\exists c_{2} \in (x,x_{2}) :\ f(x_{1})-f(x) = f'(c_{2})(x_{2}-x) \ \ \ (4)$
$\\$ (3) $\Rightarrow$ $(x_{2}-x)(f(x)-f(x_{1}))=f'(c_{1})(x_{2}-x)(x-x_{1})$ \ \ \ (5)
$\\$ (4) $\Rightarrow$ $(x-x_{1})(f(x_{2}-f(x)))=f'(c_{2})(x-x_{1})(x_{2}-x)$ \ \ \ (6)
$\\$ $c_{1} < x < c_{2}  f'(c_{1})\leq f'(c_{2}) \ \  (7)$ \ \ (5)-(7) $\Rightarrow (2) $
\end{proof}
\begin{corollary} $f \in C(I) \forall x \in I \exists f''(x)$ тогда $f \in cnv(I) \Leftrightarrow f''(x) \geq 0 \forall x \in I \ \ \ (8)$
$\\$ $f \in cnc (I) \Leftrightarrow f''(x) \leq 0\ \forall x \in I \ \ \ (9)$
\end{corollary}
\begin{remark} I - множество, f определена на I $x_{0} \in I$ - внутрення точка $\forall x \in I\ \exists f''(x) \ x_{0} $ называется точкой перегиба для f, если $\exists \alpha,\beta \in I $ \ $\alpha < x_{0} < \beta :$ на  $[\alpha,x_{0}]$  f выпукла (вогнута), $[x_{0},\beta]$ f вогнута (выпукла)
$\\$ Если $x_{0}$ - точка перегиба, то $f''(x_{0}) = 0$
\end{remark}
$\\$
\begin{proof} $f\in cnv([\alpha,x_{0}]),f \in cnc([x_{0},\beta]) $ Пусть $f''(x_{0}) = (f')'(x_{0})\neq 0$ тогда, например f''$ 
(x_{0}) >0$ Рассмотрим $ [x_{0},\beta] $, где $f(x_{0} \leq 0 ?! )$
\end{proof}
\section{Приложения признака выпуклости и вогнутости}
$\\$ (1) $f(x) = \ln{x}\ \ x>0$ I = (0,$\infty$) $f'(x) = \frac{1}{x}, f^{n}(x) = -\frac{1}{x^2}<0 \forall x \Rightarrow f(x) \in cnc(I)$
$\\$ Пусть $x_{1}...x_{n} > 0 n\geq 2$ $p_{1} = ... = p_{n} = \frac{1}{n}$ 
$\\$ по неравенсиву Йенсена:
$\\$ $\ln{(\frac{x_{1}+...x_{n}]}{n})}\geq \frac{1}{n}\ln{x_{1}}+...+\frac{1}{n}\ln{x_{n}} = \ln{(x_{1}...x_{n})^{\frac{1}{n}}}$ 
$\\$ Потенцируя, получим: $\frac{1}{n}\sum_{i = 1}^{n}x_{i} = (x_{1}...x_{n})^{\frac{1}{n}}$
$\\$ (2) I = (0,$\infty$)
$\\$ $f(x) = x^k,k>1\ f'(x) = kx^{k-1}\ f''(x) = k(k-1)x^{k-2}>0$
$\\$ Пусть $x_{1}...x_{n} > 0;\ q_{1}...q_{n} >0\ Q = q_{1}+...+q_{n}\ p_{1} = \frac{q_{1}}{Q},...,p_{n} = \frac{q_{2}}{Q}$
$\\$ Применим неравнство Йенсена $(p_{1}x_{1}+...+p_{n}x_{n})^k\leq p_{1}x_{1}^k+...+p_{n}x_{n}^k$
$\\$ $(q_{1}x_{1}+...+q_{n}x_{n})^k \leq Q^k(p_{1}x_{1}^k+...+p_{n}x_{n}^k) = Q^{k-1}(q_{1}x_{1}^k+...+q_{n}x_{n}^k)$
$\\$ Пусть $a_{j}>0$, $b_{j} > 0$, $1 \leq j \leq n$: 
$\\$ $q_{j}x_{j} = a_{j}b_{j}\ \ \ \ (12)$
$\\$ $q_{j}x_{j}^k = a_{j}^k \ \  \ \ (13) $
$\\$ $x_{j}^{k-1} = \frac{a_{j}^{k-1}}{b_{j}^{-1}} \ \ \  (14)$
$\\$ (14) $\Rightarrow x_{j} = a_{j} \cdot b_{j}^{-\frac{1}{k-1}} \ \ \ \ (15)$ $x_{j}^k = a_{j}^k \cdot b_{j}^{-\frac{k}{k-1} }$
$\\$ $q_{j} = \frac{a_{j}^k}{a_{j}^{k}\cdot b_{j}^{-\frac{k}{k-1}}} = b_{j}^{\frac{k}{k-1}} \ \ \ (16)$
$\\$ $\Rightarrow Q = \sum_{i = 1}^{n}b_{j}^{\frac{k}{k-1}} \ \ \  (17) $
$\\$ (11),(12),(13),(17) $\Rightarrow (a_{1}b_{1}+...+a_{n}b_{n})^k \leq (b_{1}^{\frac{k}{k-1}}+...+b_{n}^{\frac{k}{k-1}})(a_{1}^k+...+a_{n}^k)$
$\\$ \section{Неравенство Гельдера}
$\\$ $a_{1}b_{1}+...+a_{n}b_{n} \leq (a_{1}^k+...+a_{n}^k)^{\frac{1}{k}}(b_{1}^{\frac{k}{k-1}}+...+b_{n}^{\frac{k}{k-1}})^{\frac{k-1}{k}}$
$\\$ k = 2 - Неравенство Кошиб-Буняковского-Шварца
$\\$ Пусть $\frac{k}{k-1} = k', \frac{1}{k}+\frac{1}{k'} = \frac{1}{k}+ \frac{k-1}{k} = 1$
$\\$ k,k' - сопряженные показатели в неравенстве Гельдера
$\\$ $a_{1}b_{1}+...+a_{n}b_{n} \leq (a_{1}^k+...+a_{n}^k)^{\frac{1}{k}}(b_{1}^{k'}+...+b_{n}^{k'}){\frac{1}{k'}} $












